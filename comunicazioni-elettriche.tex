% !TEX encoding = UTF-8 Unicode
% !TEX TS-program = pdflatex
% !TEX root = comunicazioni-elettriche.tex
% !TeX spellcheck = it_IT
\documentclass[10pt,a4paper,onecolumn,titlepage,twoside,openright,final]{book}
% lingua
\usepackage[utf8]{inputenc}
\usepackage[T1]{fontenc}
\usepackage{lmodern}
\usepackage[italian]{babel}
\usepackage[babel]{csquotes}

\usepackage[mark]{gitinfo2}

% ambienti math
\usepackage{amsmath}
\usepackage{amsthm}
\usepackage{amsfonts}
\usepackage{amssymb}
\usepackage{thmtools}

% unità di misura
\usepackage[binary-units]{siunitx}
\sisetup{per-mode=symbol,inter-unit-product=\ensuremath{{}\cdot{}}}
\DeclareSIUnit\dBm{dBm}

% indici acronimi bibliografia
\usepackage{makeidx}
%\usepackage[backend=biber]{biblatex}
\usepackage{multicol}
\usepackage{acronym}

% diagrammi circuiti elettrici
\usepackage[siunitx]{circuitikz}

% grafica
\usepackage{graphicx}
\usepackage{xcolor}
\usepackage{pgfplots,pgfplotstable}
\pgfplotsset{compat=1.12,/tikz/prefix=plots/}
\usetikzlibrary{math,matrix,chains}
\usetikzlibrary{scopes,positioning,fit,intersections}
\usetikzlibrary{angles,shapes,arrows,patterns,fadings}
\usetikzlibrary{decorations.pathreplacing,decorations.pathmorphing,decorations.markings,decorations.shapes}
\usepgfplotslibrary{fillbetween,patchplots}
\pgfplotsset{every linear axis/.append style={axis lines=middle,no markers,enlargelimits}}
\pgfplotsset{trig format plots=rad}
\pgfplotsset{/pgfplots/colormap={graywhite}{gray=(0.75) gray=(1.0)}}
%\usetikzlibrary{external}
%\tikzexternalize[prefix=plots/]

% icone creative commons
\usepackage{ccicons}

% float and figure
\usepackage{float}
\usepackage{subfig}
\usepackage{caption}
\captionsetup{tableposition=top,figureposition=bottom,font=small,format=hang}
\usepackage{booktabs}
\usepackage{tablefootnote}
\renewcommand{\thefootnote}{\fnsymbol{footnote}}

% hyperlink
\usepackage{hyperref}
\hypersetup{
	pdfauthor={Marco Salvatore Vanadia},
	pdftitle={Appunti di Comunicazioni Elettriche},
	pdfsubject={Modulo Teoria dei Segnali e Modulo Fondamenti di Telecomunicazioni},
	pdfencoding=auto,
	psdextra,
	colorlinks,
	linkcolor={black},
	citecolor={blue!50!black},
	urlcolor={blue!80!black}
}


% definizioni nuovi stili teoremi
\theoremstyle{definition}
\newtheorem{definizione}{Definizione}[chapter]
\newtheorem{esempio}{Esempio}[chapter]
\newtheorem{esercizio}{Esercizio}[chapter]
\newtheorem{nota}{Nota}[chapter]
\newcommand{\keyword}[2][]{\textsc{#2}\index{#1}}

% operatori matematici
\DeclareMathOperator{\Sen}{sen}
\DeclareMathOperator{\Cos}{cos}
\DeclareMathOperator{\Sinc}{sinc}
\DeclareMathOperator{\senh}{senh}
\DeclareMathOperator{\tg}{tg}
\DeclareMathOperator{\arctg}{arctg}
\DeclareMathOperator{\arcsen}{arcsen}
\DeclareMathOperator{\Arg}{Arg}
\DeclareMathOperator{\Log}{log_{10}}
\DeclareMathOperator{\Exp}{e}
\DeclareMathOperator{\Esp}{\mathsf{E}}
\DeclareMathOperator{\impulse}{\delta}
\DeclareMathOperator{\sgn}{sgn}
\DeclareMathOperator{\step}{u}
\DeclareMathOperator{\ramp}{r}
\DeclareMathOperator{\pramp}{p}
\DeclareMathOperator{\Rect}{rect}
\DeclareMathOperator{\Laplace}{\mathcal{L}}
\DeclareMathOperator{\Fourier}{\mathcal{F}}
\DeclareMathOperator{\Erf}{erf}
\DeclareMathOperator{\Erfc}{erfc}

\DeclareMathOperator{\Realpart}{Re}
\DeclareMathOperator{\Imaginarypart}{Im}
\DeclareMathOperator{\Imaginary}{\mathit{j}\!}
%\DeclareMathOperator{\Imaginary}{\jmath}


% nuovi comandi
\newcommand{\eqname}[1]{\tag*{#1}}
\newcommand*{\f}[3][]{\,{#2}^{#1}\!\left({#3}\right)}
\newcommand*{\erf}[1]{\f{\Erf}{#1}}
\newcommand*{\erfc}[1]{\f{\Erfc}{#1}}
\newcommand*{\sen}[2][]{\f[#1]{\Sen}{#2}}
\renewcommand*{\cos}[2][]{\f[#1]{\Cos}{#2}}
\newcommand*{\sinc}[1]{\f{\Sinc}{#1}}
\renewcommand*{\P}[1]{\f{P}{#1}}
\newcommand*{\rect}[1]{\f{\Rect}{#1}}
\newcommand*{\E}[1]{\,{\Esp}\!\left[{#1}\right]}
\newcommand*{\e}[1]{\Exp^{#1}}
\newcommand*{\abs}[1]{\left|{#1}\right|}
\newcommand*{\norma}[1]{\lVert{#1}\rVert}
\newcommand*{\trasp}[1]{{#1}^\mathsf{T}}
\newcommand*{\conj}[1]{{#1}^\ast}
\renewcommand{\Re}[1]{\Realpart\!\left\lbrace{#1}\right\rbrace}
\renewcommand{\Im}[1]{\Imaginarypart\!\left\lbrace{#1}\right\rbrace}
\renewcommand{\imath}{\Imaginary}

\newcommand*{\tc}{\;\;\textrm{t.c.}\;\;}
\newcommand*{\intd}[4]{\int\limits_{#1}^{#2}{#3}{\,\diff{#4}}}
\newcommand*{\intinf}[2]{\int\limits_{-\infty}^{+\infty}{#1}{\,\diff{#2}}}
%\newcommand*{\restrict}[2]{{#1}\bigg|_{#2}}
\newcommand*{\restrict}[2]{\left.{#1}\right\rvert_{#2}}
\newcommand*{\bound}[3]{\left.\left[#3\right]\right\rvert_{#1}^{#2}}

\newcommand*\diff{\mathop{}\!\mathrm{d}}
\providecommand*{\deriv}[3][]{\dfrac{\diff^{#1}#2}{\diff #3^{#1}}}
\providecommand*{\pderiv}[3][]{\frac{\partial^{#1}#2}{\partial #3^{#1}}}
\providecommand*{\Deriv}[2][]{\mathop{}\!\mathrm{D}^{#1}#2}

\providecommand*{\fourier}[1]{\Fourier\!\left\lbrace{#1}\right\rbrace}

% alternativa a \bar e \overline
\newcommand{\overbar}[1]{\mkern 1.5mu\overline{\mkern-1.5mu#1\mkern-1.5mu}\mkern 1.5mu}
\newcommand{\vect}[1]{{\overbar{{#1}}}}

% simboli per insiemi numerici
\newcommand{\numberset}{\mathbb}
\newcommand{\N}{\numberset{N}}
\newcommand{\Z}{\numberset{Z}}
\newcommand{\R}{\numberset{R}}
%\newcommand{\Q}{\numberset{Q}}
%\newcommand{\C}{\numberset{C}}

% sostituzione simboli greci americani con europei
\renewcommand{\rho}{\varrho}
\renewcommand{\phi}{\varphi}
\renewcommand{\theta}{\vartheta}
\renewcommand{\epsilon}{\varepsilon}

% definizione blocchi diagrammi
\tikzstyle{block} = [draw, thick, fill=white, rectangle, minimum height=3em, minimum width=4em]
\tikzstyle{sum} = 	[draw, thick, fill=white, circle, minimum width=5mm]
\tikzstyle{mult} =	[draw, thick, fill=white, circle, minimum width=5mm, cross]
\tikzstyle{fitted} =	[draw, thick, dotted, inner sep=4mm, rounded corners]

\tikzstyle{passabasso}=	[block, path picture={
	\draw [<->] (-16pt, 12pt)--(-16pt,-12pt)--(18pt,-12pt);
	\draw (-16pt,5pt)-|(10pt,-12pt);
}]
\tikzstyle{passabanda}=	[block, path picture={
	\draw [<->] (-16pt, 12pt)--(-16pt,-12pt)--(18pt,-12pt);
	\draw (-8pt,-12pt)--(-8pt,5pt)-|(10pt,-12pt);
}]
\tikzstyle{decisore}=	[block, path picture={
	\draw [->] (-16pt, 0pt)--(16pt,0pt);
	\draw [->] (0pt,-12pt)--(0pt,12pt);
	\draw [thick] (-16pt,-8pt)--(0pt,-8pt)|-(16pt,8pt);
}]
\tikzstyle{campionatore} = [block, draw=none, minimum width=3em, path picture={
	\draw(path picture bounding box.west)-- ++(30:12mm);
	\draw[<-](path picture bounding box.west)++(8mm,0)arc(0:60:8mm);
}]
\tikzstyle{sumpic}=	[draw, thick, circle, node distance=1cm, path picture={
	\draw[black](path picture bounding box.south)--(path picture bounding box.north)(path picture bounding box.west)--(path picture bounding box.east);
}]
\tikzstyle{cross} = [path picture={
	\draw[black](path picture bounding box.south east)--(path picture bounding box.north west)(path picture bounding box.south west)--(path picture bounding box.north east);
}]
\tikzstyle{clock} = [path picture={
	\draw[black](path picture bounding box.south east)--(path picture bounding box.east)--(path picture bounding box.center)--(path picture bounding box.south)--(path picture bounding box.south west)--(path picture bounding box.west);
}]
\tikzstyle{dot}=	[thick,decorate,decoration={markings,mark connection node=dot,
	mark=at position 0.5 with{%
		\node[transform shape,inner sep=0pt](dot){}[fill]circle(2pt) node[below]{#1};},
	mark=at position 1 with{%
		\arrow{latex}}
}]
\tikzstyle{section}=[thick,decorate,decoration={markings,
	mark connection node=sect,
	mark=at position 0.5 with {%
		\node (sect) [transform shape,rotate=-90,minimum width=20pt,minimum height=3pt,draw=none] {};
		\draw [thin,dashed] ($(sect.north)+(0,-20pt)$)node[below]{#1} -- ($(sect.north)+(0,20pt)$);
	},
	mark=at position 1 with{%
		\arrow{latex'}}
}]
\tikzset{
	arcnode/.style 2 args={
	decoration={
		raise=#1,
		markings,
		mark=at position 0.5 with {
			\node[inner sep=0] {#2};
		}
	},
	postaction={decorate}
}}
% picture diagrammi
\tikzset{
	pics/carc/.style args={#1:#2:#3:#4}{
		code={
			\draw[pic actions,arcnode={-10pt}{\tiny #4}] (#1:#3) arc(#1:#2:#3);
		}
	}
}
\tikzset{
	antenna/.pic={
	\draw[thick](0,0)--(.5,0)arc(180:160:4)(.5,0)arc(180:200:4)(.5,.25)--(1,0)--(.5,-.25);
	\draw[decorate,decoration={expanding waves,angle=30}](1,0)--(3,0);
}}
\tikzset{
	antennarx/.pic={
		\draw[thick](0,0)--(.5,0)arc(180:160:4)(.5,0)arc(180:200:4)(.5,.25)--(1,0)--(.5,-.25);
		\draw[decorate,decoration={expanding waves,angle=30,amplitude=30}](3,0)--(1,0);
}}
\tikzset{antennabase/.pic={\pic{antenna}--++(0,-.5)--++(-.5,-1)++(.5,1)--++(.5,-1)++(-.5,0)node[below]{#1};}}

% definizione funzione pgfplots
\newcommand{\eval}[1]{\pgfmathparse{#1}\pgfmathresult}
% gradino #1=t #2=t0
\pgfmathdeclarefunction{grad}{2}{%
	\pgfmathparse{(#1)<=(#2))?0.:1.}%
}

% impulso rettangolare singolo #1=t #2=T
\pgfmathdeclarefunction{imp}{2}{%
	\pgfmathparse{(abs(#1)<=(#2/2))?1.:0.}%
}
% impulso triangolare #1=t #2=T
\pgfmathdeclarefunction{tri}{2}{%
	\pgfmathparse{(abs(#1)<=(#2))?1-abs(#1)/#2:0.}%
}
% impulso rettangolare #1=t #2=a #3=b
\pgfmathdeclarefunction{rect}{3}{%
	\pgfmathparse{(#1<#2)||(#1>#3)?0.:1.}%
}
% rampa #1=t #2=tmin #3=tmax
\pgfmathdeclarefunction{ramp}{3}{%
	\pgfmathparse{(#1<#2)?0.0:((#1>#3)?1.0:(#1-#2)/(#3-#2))}%
}
% impulso seno cardinale #1=f #2=T
\pgfmathdeclarefunction{sinc}{2}{%
	\pgfmathparse{sin(pi*(#2)*(#1))/(pi*(#2)*(#1))}%
}
% gauss #1=x #2=\mu #3=\sigma^2
\pgfmathdeclarefunction{gauss}{3}{%
	\pgfmathparse{1/(#3*sqrt(2*pi))*exp(-((#1-#2)^2)/(2*#3^2))}%
}
% noise
\pgfmathdeclarefunction{noise}{2}{%
	\pgfmathparse{#1+rand*#2}%
}
% Nyquist
\pgfmathdeclarefunction{nyquist}{4}{%
	\pgfmathparse{(#1<#4)?0.0:((#1>#2)?1.0:(.5*(1+cos((pi*(abs(#1)-#4)/(2*#3))))))}%
}

% random decoration
\pgfdeclaredecoration{random}{step}
{
	\state{step}[width=+\pgfdecorationsegmentlength]
	{ 	\pgfmathsetmacro{\delta}{rand*\pgfdecorationsegmentamplitude}
		\pgfmathsetmacro{\deltax}{\delta*cos(90+\pgfdecoratedangle)}
		\pgfmathsetmacro{\deltay}{\delta*sin(90+\pgfdecoratedangle)}
		\pgfpathlineto{\pgfpoint{\pgfdecorationsegmentlength-\deltax}{\deltay}}
	}
	\state{final}
	{
		\pgfpathlineto{\pgfpointdecoratedpathlast}
	}
}
% clear page
\makeatletter
\def\cleardoublepage{\clearpage\if@twoside \ifodd\c@page\else
	\hbox{}
	\vspace*{\fill}
	\begin{center}
	\end{center}
	\vspace{\fill}
	\thispagestyle{empty}
	\newpage
	\if@twocolumn\hbox{}\newpage\fi\fi\fi}
\makeatother

\pgfdeclarepatternformonly[/tikz/radius,\thickness,\size]{rings}
{\pgfpoint{-0.5*\size}{-0.5*\size}}
{\pgfpoint{0.5*\size}{0.5*\size}}
{\pgfpoint{\size}{\size}}
{
	\pgfsetlinewidth{\thickness}
	\pgfpathcircle\pgfpointorigin{\pgfkeysvalueof{/tikz/radius}}
	\pgfusepath{stroke}
}
\newdimen\thickness
\tikzset{
	radius/.initial=4pt,
	size/.store in=\size, size=20pt,
	thickness/.code={\thickness=#1},
	thickness=0.75pt
}

% formato pagina
\usepackage[width=15.00cm, left=3.00cm, height=23.00cm]{geometry}

% frontespizio Poliba
\usepackage{polibatitle}
\title{Appunti di \\ Comunicazioni Elettriche}
\subtitle{Modulo Teoria dei Segnali\\Modulo Fondamenti di Telecomunicazioni}
\author{Marco Salvatore \textsc{Vanad\`{i}a}}
\authoremail{marco.vanadia@gmail.com}
\authorURL{https://it.linkedin.com/in/marcovanadia}
\professor{Prof. Ing. L.A. Grieco}
\department{Dipartimento di Ingegneria Elettrica e dell'Informazione}
\version{\gitRel}
\versiondate{\gitAuthorDate}


\makeindex

\begin{document}
\frontmatter
\maketitle
\newpage

% dedica
\null\vspace{\stretch{1}}
\begin{flushright}a Giulia\end{flushright}
\vspace{\stretch{2}}\null
\vfill
Il presente documento è rilasciato sotto licenza \ccLogo \textbf{Creative Commons 3.0 by-sa-nc} \ccbyncsa.

\'{E} consentita la creazione di opere derivate, traduzioni, adattamenti, totali o parziali, fatta salva l'attribuzione dell'autore originale e il mantenimento della licenza.

Per collaborare alla stesura e alla correzione dei sorgenti è possibile iscriversi come contributori al repository GitHub all'indirizzo \url{https://github.com/mova77/poliba-comunicazioni-elettriche}

Bari, novembre 2016 \\ Marco Salvatore \textsc{Vanadìa}\\Politecnico di Bari
\cleardoublepage\clearpage{\pagestyle{empty}\cleardoublepage}

% indice contenuti
\tableofcontents

\mainmatter
\part{Teoria dei Segnali}
\chapter{Segnali elementari}
\section{Tipi di segnali}
Si definisce \textsc{segnale} una grandezza fisica variabile cui è associata una informazione.

\`{E} possibile classificare i segnali secondo vari criteri.

La funzione che definisce il segnale può avere dominio continuo, con la cardinalità dei numeri reali, per i segnali a \textsc{tempo continuo}, (ad es. s. analogici). La funzione che definisce il segnale può essere definita come una successione numerica, con la cardinalità dei numeri naturali, per i segnali \textsc{tempo discreti}.

Il valore della ampiezza assunta dal segnale può assumere valori continui per \textsc{segnali reali}, ad esempio la tensione ai capi di un bipolo, o discreti per \textsc{segnali numerici}, ad esempio un segnale binario 0 e 1.

Un segnale può ripetersi ad intervalli regolari risultando un segnale \textsc{periodico}\index{segnale!periodico}: $T\in\R, T>0$ si ha $s(t)$ periodico se $\forall n\in\Z\colon s(t)=s(t+n T)$
o non ripetersi come segnale \keyword[segnale!aperiodico]{aperiodico}, ad esempio un qualunque segnale di durata finita.

\subsection{Energia di un segnale}
Ad un segnale è associata la sua energia e potenza pertanto si possono avere segnali ad \keyword[segnale!di energia]{energia finita} se \begin{equation}\label{eq:segnale_energia}E_s=\intinf{\abs{s(t)}^2}{t}<+\infty \qquad E_s=\sum_{n=-\infty}^{+\infty}\abs{s(n)}^2 <+\infty\end{equation}
Un segnale periodico è un esempio di segnale che non ha energia finita infatti anche se l'energia nel periodo è finita $\intd{-T/2}{T/2}{\abs{s(t)}^2}{t}<+\infty$ non è finito l'integrale su $\R$.

\subsection{Potenza di un segnale}
Un segnale ha \keyword[segnale!di potenza]{potenza finita} quando \begin{equation}\label{eq:segnale_potenza}P_s=\lim\limits_{T\to+\infty}{\frac{1}{T}\intd{-\frac{T}{2}}{\frac{T}{2}}{\abs{s(t)}^2}{t}<+\infty}   \qquad  P_s=\lim\limits_{N\to+\infty}{\frac{1}{2N+1}\sum_{n=-N}^{N}{\abs{s(n)}^2}<+\infty}\end{equation}
Per i segnali ad energia finita la potenza è nulla.

Si parla di \textsc{segnali di potenza} per i segnali di energia infinita ma potenza finita.

\subsection{Segnale reale, pari e dispari}
Un \keyword[segnale!reale]{segnale reale} assume valori reali. Un segnale \textsc{complesso} può assumere valori definiti in modulo e fase, o equivalentemente in parte reale e parte immaginaria
\[s_c(t)=s_R(t)+\imath s_I(t)\]

Si hanno inoltre segnali \keyword[segnale!pari]{segnale pari} $s(t)=s(-t)$ e segnali \keyword[segnale!dispari]{segnale dispari} $s(t)=-s(-t)$.

\`{E} possibile estrarre la parte pari e quella dispari di un segnale
\[\begin{cases}
s_P(t)=\frac{1}{2}[s(t)+s(-t)] \\
s_D(t)=\frac{1}{2}[s(t)-s(-t)]
\end{cases}\]

\section{Operazioni sui segnali}
\subsection{Traslazione}
$s(t) \to s(t-t_0)$ traslo l'origine del segnale in $t_0$. Se $t_0>0$ segnale ritardato, se $t_0<0$ è anticipato.
\subsection{Ribaltamento}
$s(t) \to s(-t)$ ribalto l'asse dei tempi (variabile indipendente) rispetto all'asse delle ordinate.
\subsection{Scala}
$s(t)\to s(a t), a\in\R$ scalo l'asse dei tempi, restringo il segnale originale per $a>1$, lo espando con $0<a<1$.
Le operazioni di scalatura e ribaltamento non sono commutative con la traslazione. L'ordine delle operazioni cambia il risultato.
	\subsection{Convoluzione}\index{segnali!convoluzione}
La convoluzione di due segnali, definita come
\[y(t)=x(t)\ast  h(t)=\intinf{x(\tau)h(t-\tau)}{\tau}\]
\subsubsection{Proprietà commutativa $x(t)\ast h(t)= h(t)\ast x(t)$}
\begin{proof}[Dim.]
$x(t)\ast h(t)=\intinf{x(\tau)h(t-\tau)}{\tau}=\intinf{-x(t-\alpha)h(\alpha)}{\alpha}=h(t)\ast x(t)$ dove si è effettuata la sostituzione $\alpha=t-\tau$, $\diff\tau=-\diff\alpha$
\end{proof}
\subsubsection{Proprietà associativa $[x(t)\ast y(t)]\ast h(t)= x(t)\ast [h(t)\ast y(t)]$}
\subsubsection{Prop. distributiva rispetto alla somma $[x(t)+y(t)]\ast h(t)= x(t)\ast h(t)+y(t)\ast h(t)$}

\section{Segnali elementari}

\subsection{Gradino unitario}\index{segnale!gradino}
\[ \step(t)=\begin{cases}
1 & t>0 \\
0 & t<0
\end{cases}
\]
\subsection{Rampa}\index{segnale!rampa}
\[ \ramp(t)=t\step(t) \]
\subsection{Rampa parabolica}\index{segnale!rampa parabolica}
\[ \pramp(t)=\intinf{\ramp(\tau)}{\tau}=\frac{t^2}{2}\step(t)
\]

\begin{figure}
\centering
\subfloat[Gradino $\step(t)$]{
\begin{tikzpicture}[scale=.6]
\begin{axis}[axis lines=middle,no markers,enlargelimits,xtick={-1,0,1},ytick={0,1}]
\addplot [very thick]coordinates {(-1,0)(0,0)(0,1)(1,1)};
\addplot [dashed]coordinates {(1,1)(1.2,1)};
\end{axis}\end{tikzpicture}} \qquad
\subfloat[Rampa $\ramp(t)$]{
\begin{tikzpicture}[scale=.6]
\begin{axis}[axis lines=middle,no markers,enlargelimits,xtick={-1,0,1},ytick={0,1}]
\addplot [very thick]coordinates {(-1,0)(0,0)(1,1)};
\addplot [dashed]coordinates {(1,1)(1.2,1.2)};
\end{axis}\end{tikzpicture}} \qquad
\subfloat[Rampa parabolica $\pramp(t)$]{
\begin{tikzpicture}[scale=.6]
\begin{axis}[axis lines=middle,no markers,enlargelimits,xtick={-1,0,1},ytick={0,1}]
\addplot [very thick,domain=-1:1] {x<0?0:x^2};
\addplot [dashed,domain=1:1.2] {x^2};
\end{axis}\end{tikzpicture}}
\caption{Segnali elementari}\label{fig:segn_el}
\end{figure}

\subsection{Segnale rettangolare e onda quadra}\index{segnale!rettangolare}
\[ \rect{\frac{t}{\tau}}=\begin{cases}1 & \abs{t} < \frac{\tau}{2} \\
0 & \abs{t} > \frac{\tau}{2} \end{cases} \]

\`{E} un segnale di energia finita pari a $\tau$. Per $\tau=1$ si ha l'onda quadra.
Il segnale è discontinuo in $\pm\frac{T}{2}$ ma si può estendere per continuità definendo $s(t_0)=\frac{1}{2}[s(t_0^-)+s(t_0^+)]$

Si costruisce il segnale periodico di periodo $T$ come somma di infiniti segnali rettangolari traslati
\[ \mathrm{sq}(t)=\sum_{n=-\infty}^{+\infty} \rect{\frac{t-n T}{\tau}}, T>\tau \]

Se $\tau=\frac{T}{2}$ il tempo in cui il segnale è diverso da zero, ovvero il \emph{duty cycle} $t/\tau$  è del 50\%, con un valor medio $1/2$. In generale il valor medio è $\frac{1}{T}\intd{-T/2}{T/2}{sq(\xi)}{\xi}=\frac{\tau}{T}$
\begin{figure}
\begin{center}\begin{tikzpicture}[scale=.6]
\begin{scope}\begin{axis}[axis lines=middle,no markers,enlargelimits,xtick={-1,-.5,0,.5,1},xticklabels={$-1$,$-\frac{\tau}{2}$,$0$,$\frac{\tau}{2}$,$1$},ytick={0,1}]
\addplot [very thick,samples=200,domain=-1:1]  {abs(x)<.5?1:0};
\end{axis}\end{scope}
\begin{scope}[xshift=7cm]\begin{axis}[axis lines=middle,no markers,enlargelimits,xtick={-2,-1,-.5,0,.5,1,2},xticklabels={$-2$,$-1$,$-\frac{1}{2}$,0,$\frac{1}{2}$,$1$,$2$},ytick={0,1}]
\def\_tau{.5}
\def\T{1}
\foreach \n in {-3,-2,-1,0,1,2,3}
\addplot [very thick,samples=200,domain=\T*\n-\T/2:\T*\n+\T/2]  {abs(x-\n*\T)<\_tau/2?1:0};
\end{axis}\end{scope}
\begin{scope}[xshift=14cm]\begin{axis}[axis lines=middle,no markers,enlargelimits,xtick={-2,-1,0,1,2},ytick={0,1}]
\def\tau{.25}
\def\T{1}
\foreach \n in {-3,-2,-1,0,1,2,3}
\addplot [very thick,samples=200,domain=\T*\n-\T/2:\T*\n+\T/2]  {abs(x-\n*\T)<\tau/2?1:0};
\end{axis}\end{scope}
\end{tikzpicture}
\end{center}
\caption{Segnale rettangolare e onde quadre con \emph{duty cycle} del 50\% e 25\%}
\end{figure}

\subsection{Delta di Dirac}\index{segnale!impulso}
Il segnale rettangolare $\frac{1}{T} \rect{\frac{1}{T}}$ di base T e altezza 1/T ha area unitaria. Portando al limite $T\to 0$ il rettangolo diventa un impulso. Non una funzione in senso proprio ma una distribuzione integrabile chiamata \keyword[segnale!delta di Dirac|see {impulso}]{delta di Dirac}: \[\delta(t)=\lim\limits_{T\to 0}\frac{1}{T}\rect{\frac{1}{T}}\]

\subsubsection{Proprietà dell'impulso}
\begin{enumerate}
\item il segnale impulso ha area unitaria \[ \intinf{\impulse(t)}{t}=1 \]
\item il segnale impulso è funzione pari \[ \impulse(t)=\impulse(-t) \]
\item estrazione di un campione da un segnale $s(t)$ con un impulso in $t=\tau$ \[ \intinf{s(t)\impulse(t-\tau)}{t}=s(\tau) \]
è equivalente ad un impulso in $\tau$ di area $s(\tau)$
\begin{align*}
 s(t)\impulse(t-\tau)=s(\tau)\impulse(t-\tau) &\implies \\ & \intinf{s(t)\impulse(t-\tau)}{t} =\intinf{s(\tau)\impulse(t-\tau)}{t} = s(\tau)\intinf{\impulse(t-\tau)}{t} = s(\tau)
 \end{align*}

\item rappresentazione di un segnale come somma di infiniti impulsi
\[s(t)=\intinf{s(\tau)\impulse(t-\tau)}{\tau}=s(t)\ast\impulse(t)\]
\item derivata dell'impulso (\textsc{doppietto}) $\impulse'(t)$
\[ \intinf{s(t)\impulse'(t-\tau)}{t}=-s'(\tau) \]
\begin{proof}[Dim.]
applico l'integrazione per parti $\int{u \diff v}=u v-\int{v\diff u}$ con $u=s(t) ,\, \diff u=s'(t)\diff t ,\, \diff v=\impulse'(t-\tau)\diff t ,\, v=\impulse(t-\tau)$  a
\[\intinf{s(t)\impulse'(t-\tau)}{t}= \restrict{s(t)\impulse(t-\tau)}{-\infty}^{+\infty} -\intinf{\impulse(t-\tau)s'(t)}{t}=-s'(\tau)  \]
\end{proof}
\item impulso è derivata del gradino
\[\intinf{\impulse(\tau)}{\tau}=\step(t)  \to \impulse(t)=\deriv{\step(t)}{t} \]
\item scala
\[\impulse(a t+b)=\intinf{s(t)\impulse(a t+b)}{t}\]
applicando le sostituzioni $\begin{cases}x=a t+b \\ t=\frac{x-b}{a}\end{cases}$
\[\intinf{\f{s}{\frac{x-b}{a}}\impulse(x)}{\frac{x}{\abs{a}}}=\frac{1}{\abs{a}}\f{s}{-\frac{b}{a}}\]
\[\intinf{s(t)\frac{1}{\abs{a}}\f{\impulse}{t+\frac{b}{a}}}{t}=\frac{1}{\abs{a}}\f{s}{-\frac{b}{a}}\]
\[\implies\impulse(a t+b)=\frac{1}{\abs{a}}\f{\impulse}{t+\frac{b}{a}} \]

\end{enumerate}

\subsection{Segnale sinusoidale}\index{segnale!sinusoidale}
Il segnale sinusoidale di ampiezza $A$, pulsazione angolare $\omega=2\pi f$, periodo $T=\frac{2\pi}{\omega}$, frequenza $f=\frac{1}{T}$, fase iniziale $\phi$
\[s(t)=A\sen{2\pi f t+\phi} \]

Potenza media
\[P_m=\frac{1}{T}\intd{-T/2}{T/2}{A^2\Sen^2(2\pi f t+\phi)}{t}= \frac{A^2}{2}\]

infatti essendo $\Sen^2(x)=\frac{1}{2}-\frac{1}{2}\cos{2x}$
\[\begin{split}P_m&=\frac{1}{T}\intd{-T/2}{T/2}{\left(\frac{A^2}{2}-\frac{A^2}{2}\cos{4\pi f t+2\phi}\right)}{t}=\\
&=\restrict{\frac{A^2}{2 T}}{-T/2}^{T/2}-\frac{A^2}{2 T}\underbrace{\intd{-T/2}{T/2}{\cos{4\pi f t+2\phi}}{t}}_{=0}=\frac{A^2}{2}\end{split}\]

Potenza di picco \[P_p=\max\limits_{t} A^2\Sen^2(2\pi f t+\phi)=A^2\]

Fattore di picco \[\frac{P_p}{P_m}=2\]

\subsection{Segnale seno cardinale}\index{segnale!seno cardinale}
\[\sinc{t}=\frac{\sen{\pi t}}{\pi t}\]

\begin{figure}[!ht]
\centering
\subfloat[][$\sinc{t}=\frac{\sen{\pi t}}{\pi t}$]
{\begin{tikzpicture}[scale=.8]
\begin{axis}[axis lines=middle,no markers,enlargelimits,xscale=1.5,xtick={0,1,2,3,4,5,6},ytick={0,1}]
\addplot [very thick,domain=-6:6,samples=100] {sin(pi*x)/(pi*x)};
\end{axis}\end{tikzpicture}} \qquad
\subfloat[][$\sinc{\frac{t}{T}}=\frac{\sen{\frac{\pi t}{T}}}{\frac{\pi t}{T}}$]
{\begin{tikzpicture}[scale=.8]
\begin{axis}[axis lines=middle,no markers,enlargelimits,xscale=1.5,xtick={-9.424,-6.283,-3.141,0,3.141,6.283,9.424},ytick={0,1},xticklabels={$-3T$,$-2T$,$-T$,$0$,$T$,$2T$,$3T$}]
\addplot [very thick,domain=-3.1*pi:3.1*pi,samples=100] {sin(x)/x};
\end{axis}\end{tikzpicture}}
\caption{Segnale seno cardinale}
\label{fig:sinc}
\end{figure}

\chapter{Serie e trasformata di Fourier}

\section{Serie di Fourier esponenziale}
Dato un segnale periodico $s(t)$ di frequenza $f_0=\frac{1}{T}$ e periodo $T$ tale che $s(t)=s(t+k T)$, è possibile esprimerlo come somma di infinite sinusoidi pesate di frequenza multipla di $f_0$, $f_k=k f_0=\frac{k}{T}$.

\begin{definizione}
Dato il segnale periodico $s(t)=s(t+k T)$ si definisce la sua rappresentazione in \textsc{serie di Fourier} in forma di esponenziali complessi:
\begin{equation}\label{eq:serie_fourier}\index{serie!di Fourier}
s(t)=\sum_{k=-\infty}^{+\infty} c_k \e{\imath 2\pi k f_0 t}
\end{equation}
\end{definizione}
Le sinusoidi pesate sono espresse in forma di esponenziali complessi ricordando le formule di Eulero
\begin{equation}\e{\imath x}=\cos{x}+\imath\sen{x}\qquad\cos{x}=\frac{\e{\imath x}+\e{-\imath x}}{2}\qquad\sen{x}=\frac{\e{\imath x}-\e{-\imath x}}{2\imath}\end{equation}
I coefficienti $c_k$ costituiscono il peso o contributo della sinusoide a frequenza $f_k$ e si calcolano come
\begin{equation}\label{eq:serie_fourier_coef}\index{serie!di Fourier!coefficienti}
c_k=\frac{1}{T}\intd{-\frac{T}{2}}{\frac{T}{2}}{s(t)\e{-\imath 2\pi k f_0 t}}{t}
\end{equation}
Si può verificare sostituendo $s(t)$
\[c_k=\frac{1}{T}\intd{-\frac{T}{2}}{\frac{T}{2}}{ \left[\sum_{n=-\infty}^{+\infty} c_n \e{\imath 2\pi n f_0 t}\right] \e{-\imath 2\pi k f_0 t}}{t} = \frac{1}{T}\intd{-\frac{T}{2}}{\frac{T}{2}}{ \sum_{n=-\infty}^{+\infty} c_n \e{\imath 2\pi (n-k) f_0 t} }{t} \]
ipotizzando la convergenza della serie $s(t)$ ed essendo gli operatori $\sum$ e $\int$ lineari posso invertirne l'ordine
\[\begin{split}=&\frac{1}{T}\sum_{n=-\infty}^{+\infty} c_n \intd{-\frac{T}{2}}{\frac{T}{2}}{\e{\imath 2\pi(n-k)f_0 t}}{t}
=\frac{1}{T}\sum_{n=-\infty}^{+\infty} c_n \bound{-\frac{T}{2}}{\frac{T}{2}}{\frac{\e{\imath 2\pi(n-k)\frac{1}{T}t}}{\imath 2\pi(n-k)\frac{1}{T} }}\\
=&\frac{1}{T}\sum_{n=-\infty}^{+\infty} c_n T \frac{\e{\imath \pi(n-k)}-\e{-\imath\pi(n-k)}}{2\imath\pi(n-k)}
=\sum_{n=-\infty}^{+\infty} c_n \frac{\sen{\pi(n-k)}}{\pi(n-k)}
=\sum_{n=-\infty}^{+\infty} c_n \sinc{n-k} = c_k
\end{split}\]
essendo definito il $\sinc{x}=\frac{\sen{\pi x}}{\pi x}$, si ha che sinusoidi a frequenza diversa risultano ortogonali tra loro ovvero $\sinc{n-k}=\begin{cases}
1 & n=k \\ 0 & n\neq k\end{cases}$, l'integrale sul periodo del prodotto da contributo nullo e la sommatoria per $n$ si riduce al solo contributo per $n=k$.

I coefficienti $c_k$ sono numeri complessi $c_k=\abs{c_k}\e{\imath\theta_k}$ con ampiezza e fase associati alla armonica di frequenza $f_k$.

Il coefficiente $c_0$ corrispondente alla frequenza nulla si dice \textsc{componente continua} del segnale ed è pari al valor medio del segnale periodico in un periodo:
\begin{equation}c_0=\frac{1}{T}\intd{-\frac{T}{2}}{\frac{T}{2}}{s(t)}{t} \end{equation}

\begin{nota}Nei sistemi reali è sempre presente rumore per cui i termini $c_k$ ad un certo punto non danno contributi utili ad alte frequenze quindi posso sommare un numero finito di termini $c_0 + c_1\e{\imath 2\pi f_0 t}+c_{-1}\e{-\imath 2\pi f_0 t}+\dots\approx s(t)$.\end{nota}

\section{Condizioni di esistenza Dirichlet}\index{serie!di Fourier!condizioni Dirichlet}
Perché converga lo sviluppo in serie di Fourier del segnale $s(t)$ sono sufficienti le seguenti condizioni di esistenza:
\begin{enumerate}
\item $s(t)$ sia assolutamente integrabile in un periodo: $\intd{0}{T}{\abs{(s(t)}}{t}<+\infty$
\item $s(t)$ abbia nel periodo un numero finito di massimi e minimi
\item $s(t)$ abbia nel periodo un numero finito di discontinuità di I specie\footnote{Esistono finiti $\lim\limits_{t\to t_0+}s(t)\neq\lim\limits_{t\to t_0-}s(t)$, discontinuità I specie eliminabile $s(t_0)=\frac{s(t_0^+)-s(t_0^-)}{2}$}
\end{enumerate}
\begin{nota}Le condizioni sono molto restrittive ai fini pratici, anche segnali elementari come l'impulso e il gradino non ammettono sviluppo in serie di Fourier sotto tali condizioni.\end{nota}

\section{Serie di Fourier di sinusoidi}
Lo sviluppo in serie di Fourier può essere espresso come serie somma di seni e coseni o sinusoidi
\[\begin{split} s(t)&=\sum_{k=0}^{+\infty}\left[ a_k \sen{2\pi\frac{k}{T}t} + b_k \cos{2\pi\frac{k}{T}t}\right] \\
&=\sum_{k=0}^{+\infty}A_k\left[\sen{2\pi\frac{k}{T}t+\phi_k}\right] =\sum_{k=0}^{+\infty}A_k\left[\sen{2\pi\frac{k}{T}t}\Cos\phi_k +\cos{2\pi\frac{k}{T}t}\Sen\phi_k \right]
\end{split}\]
dove $a_k,b_k \in\R,\,\begin{cases}a_k=A_k\Cos\phi_k \\b_k=A_k\Sen\phi_k \end{cases}\implies\begin{cases}A_k^2=a_k^2+b_k^2\\ \tg\phi_k=\frac{b_k}{a_k} \end{cases}\implies\begin{cases}A_k=\sqrt{a_k^2+b_k^2}\\ \phi_k=\arctg\frac{b_k}{a_k} (+\pi\text{ se }a_k<0) \end{cases}$

I coefficienti $a_k$ e $b_k$ si calcolano come
\[a_k=\frac{2}{T}\intd{-\frac{T}{2}}{+\frac{T}{2}}{s(t)\sen{2\pi\frac{k}{T}t}}{t} \qquad b_k=\frac{2}{T}\intd{-\frac{T}{2}}{+\frac{T}{2}}{s(t)\cos{2\pi\frac{k}{T}t}}{t}\]
che per $k=0$ si semplificano nei coefficienti $a_0=0 \quad b_0=\frac{1}{T}\intd{0}{T}{s(t)}{t}$

\section{Equivalenza tra serie di Fourier}
Si dimostra che le serie di Fourier espresse come somma di esponenziali complessi e come somma di sinusoidi sono equivalenti.

\begin{proof}[Dim.]
\[s(t)=\sum_{k=-\infty}^{+\infty} c_k \e{\imath 2\pi\frac{k}{T}t} = \sum_{k=0}^{+\infty}\left[ a_k \sen{2\pi\frac{k}{T}t} + b_k \cos{2\pi\frac{k}{T}t}\right]
\]

Si vede che $\qquad b_0=c_0 \qquad b_k=c_k+c_{-k} \qquad a_k=\imath (c_k-c_{-k})$

Infatti si può scrivere
\[\begin{split}\sum_{k=-\infty}^{+\infty} & c_k \e{\imath 2\pi\frac{k}{T}t} = c_0+ \sum_{k=1}^{+\infty} \left[ c_k \e{\imath 2\pi\frac{k}{T}t} + c_{-k} \e{-\imath 2\pi\frac{k}{T}t} \right] = \\
&= c_0 + \sum_{k=1}^{+\infty} {\left\lbrace c_k \left[\cos{2\pi\frac{k}{T}t}+\imath\sen{2\pi\frac{k}{T}t}\right]
+c_{-k}\left[\cos{2\pi\frac{k}{T}t}-\imath\sen{2\pi\frac{k}{T}t}\right] \right\rbrace}=\\
&= c_0 + \sum_{k=1}^{+\infty} {\left[(c_k+c_{-k}) \cos{2\pi\frac{k}{T}t}+
(c_k-c_{-k})\imath\sen{2\pi\frac{k}{T}t}\right] }\end{split}\]
\end{proof}

\[b_k=c_k+c_{-k}=\frac{1}{T}\intd{-\frac{T}{2}}{\frac{T}{2}}{s(t)\left[\e{\imath 2\pi\frac{k}{T}t}+\e{-\imath 2\pi\frac{k}{T}t}\right]}{t}=\frac{2}{T}\intd{-\frac{T}{2}}{\frac{T}{2}}{s(t)\cos{2\pi\frac{k}{T}t}}{t} \]
\[a_k=\imath(c_k-c_{-k})=\imath\frac{1}{T}\intd{-\frac{T}{2}}{\frac{T}{2}}{s(t)\left[\e{\imath 2\pi\frac{k}{T}t}-\e{-\imath 2\pi\frac{k}{T}t}\right]}{t}=\frac{2}{T}\intd{-\frac{T}{2}}{\frac{T}{2}}{s(t)\sen{2\pi\frac{k}{T}t}}{t} \]

\begin{nota}L'informazione del segnale è contenuta nei pesi $c_k$ o equivalentemente $a_k$ e $b_k$.
\end{nota}

\section{Simmetrie coefficienti}
Per un segnale reale $s(t)\in\R$ la parte immaginaria deve essere nulla ovvero i termini $c_k=\Re c_k + \imath \Im c_k$ devono essere tali che $c_k \e{\imath 2\pi\frac{k}{T}t}+ c_{-k}\e{-\imath 2\pi\frac{k}{T}t}\in\R,\forall k$
\[c_k+c_{-k}\in\R \implies \Im c_k+\Im c_{-k}=0 \implies \Im c_k=-\Im c_{-k}\]
\[c_k-c_{-k} \text{ immaginario puro } \implies \Re c_k -\Re c_{-k}=0 \implies \Re c_k=\Re c_{-k} \]

Per un segnale reale quindi i coefficienti sono complessi e coniugati $c_k=\conj{c_{-k}}$ (\textsc{simmetria Hermitiana})

Per un segnale reale e pari si ha $a_k=0\,\forall k \implies c_k=c_{-k}$

Per un segnale reale e dispari si ha $b_k=0\,\forall k \implies c_k=-c_{-k}$

\section{Esempio serie di Fourier di un'onda quadra}
\begin{esempio}
Segnale onda quadra di periodo T durata $\tau$ in fig.\ref*{fig:onda_quadra} \[s(t)=\sum_{n=-\infty}^{+\infty}\rect{\frac{t-n T}{\tau}}\]

\begin{figure}[ht]
\begin{center}\begin{tikzpicture}
	\begin{axis}[width=12cm, height=4cm, axis lines=middle,no markers,enlargelimits,xtick={-1,-.5,0,.5,1,1.5,2,2.5,3},xticklabels={$-\frac{T}{2}$,$-\frac{\tau}{2}$,$0$,$\frac{\tau}{2}$,$\frac{T}{2}$,$T\!\!-\!\frac{\tau}{2}$,$T$,$T\!\!+\!\frac{\tau}{2}$},ytick={0,1}]
	\addplot [very thick,samples=200,domain=-1:1]  {abs(x)<.5?1:0};
	\addplot [very thick,samples=200,domain=1:3]  {abs(x-2)<.5?1:0};
	\end{axis}
	\end{tikzpicture}
\end{center}
\caption{Onda quadra}\label{fig:onda_quadra}
\end{figure}
Il segnale è reale pari ($c_k=c_{-k}$), i coefficienti della serie di Fourier si calcolano applicando la formula \ref{eq:serie_fourier_coef}:
\[\begin{split}c_k&=\frac{1}{T}\intd{-\frac{T}{2}}{\frac{T}{2}}{s(t)\e{-\imath 2\pi\frac{k}{T}t}}{t}
=\frac{1}{T}\intd{-\frac{T}{2}}{\frac{T}{2}}{\e{-\imath 2\pi\frac{k}{T}t}}{t}
=\frac{1}{T}\bound{-\frac{\tau}{2}}{\frac{\tau}{2}}{ \frac{\e{-\imath 2\pi\frac{k}{T}t}}{-\imath 2\pi\frac{k}{T}}}=\\
&=\frac{1}{\pi k} \frac{-\e{-\imath 2\pi\frac{k}{T}\frac{\tau}{2}}+\e{\imath 2\pi\frac{k}{T}\frac{\tau}{2}}}{\imath 2}
=\frac{\sen{\pi \tau \frac{k}{T}}}{\pi k}
=\frac{\tau}{T}\frac{\sen{\pi k \frac{\tau}{T}}}{\pi k \frac{\tau}{T}}
=\frac{\tau}{T}\sinc{k \frac{\tau}{T}}
\end{split}\]

\begin{figure}[h!]
\centering
{\begin{tikzpicture}[scale=.6]
	\begin{axis}[width=10cm,axis lines=middle,no markers,enlargelimits,xscale=1.5,xtick={-3,-1,0,1,3},xticklabels={$-T$,$-\tau$,0,$\tau$,$T$},ytick={0,1},yticklabels={0,$\frac{\tau}{T}$}]
	\addplot [dashed,domain=-7:7,samples=100] {sin(pi*x/3)/(pi*x/3)};
	\addplot+[only marks,samples at={-3,-2,-1,0.001,1,2,3}]
	{sin(pi*x/3)/(pi*x/3)};
	\end{axis}\end{tikzpicture}}
\caption{Valori coefficienti $c_k=\frac{\tau}{T}\sinc{\frac{t}{T}}$}
\label{fig:sinc_onda_quadra}
\end{figure}
\end{esempio}

\section{Proprietà della serie di Fourier}
\subsection{Linearità}L'integrale è un operatore lineare pertanto una combinazione lineare di segnali trasforma linearmente i coefficienti della serie
\[ y(t)=a s_1(t)+ b s_2(t)\]
\[ c_{y k}= a c_{1 k} + b c_{2 k} \]

\subsection{Traslazione nel tempo}
Un segnale traslato nel tempo $s(t-\tau)$, ritardato o anticipato di un tempo $\tau$, vede i coefficienti modificati nella fase $c_r^{'}=c_k \e{-\imath 2\pi\frac{k}{T}\tau}$

\begin{proof}[Dim.]
\[ \intd{-\frac{T}{2}}{\frac{T}{2}}{s(t-\tau)\e{-\imath 2\pi\frac{k}{T}t}}{t} \]
effettuando il cambio di variabile $t-\tau=x$, $t=x+\tau$, $\diff t=\diff x$
\[\intd{-\frac{T}{2}}{\frac{T}{2}}{s(x)\e{-\imath 2\pi\frac{k}{T}x}\e{-\imath 2\pi\frac{k}{T}\tau}}{x} =
\e{-\imath 2\pi\frac{k}{T}\tau} \intd{-\frac{T}{2}}{\frac{T}{2}}{s(x)\e{-\imath 2\pi\frac{k}{T}x}}{x}\]
si nota anche che data la periodicità per $\tilde{\tau}=\tau+N T$
\[\e{-\imath 2\pi\frac{k}{T}\tilde{\tau}}=
\e{-\imath 2\pi\frac{k}{T}\tau} \underbrace{\e{-\imath 2\pi\frac{k}{T}N T}}_{1}=\e{-\imath 2\pi\frac{k}{T}\tau}
\]
\end{proof}

\section{Potenza segnale periodico (Teo. Parseval)}
La potenza del segnale periodico è contenuta nei suoi toni armonici:
\begin{equation}\label{eq:teo_Parseval}\index{Teorema!Parseval}
P=\sum_{k=-\infty}^{+\infty}{\abs{c_k}^2}
\end{equation}

\begin{proof}[Dim.]
\[\begin{split}P&= \frac{1}{T}\intd{-\frac{T}{2}}{\frac{T}{2}}{\abs{s(t)}^2}{t}=\frac{1}{T}\intd{-\frac{T}{2}}{\frac{T}{2}}{ \left(\sum_{k=-\infty}^{+\infty}{c_k\e{\imath 2\pi\frac{k}{T}t}}\right) \left(\sum_{n=-\infty}^{+\infty}{c_n\e{\imath 2\pi\frac{n}{T}t}}\right)}{t}= \\
&=\frac{1}{T}\intd{-\frac{T}{2}}{\frac{T}{2}}{ \left(\sum_{k=-\infty}^{+\infty}{c_k\e{\imath 2\pi\frac{k}{T}t}}\right) \left(\sum_{n=-\infty}^{+\infty}{\conj{c_n}\e{-\imath 2\pi\frac{n}{T}t}}\right)}{t}= \\
\intertext{il prodotto di armoniche ortogonali da contributo non nullo solo per $n=k$, pertanto}
&=\frac{1}{T}\intd{-\frac{T}{2}}{\frac{T}{2}}{ \left(\sum_{k=-\infty}^{+\infty}{c_k \conj{c_k} \e{\imath 2\pi\frac{k}{T}t} \e{-\imath 2\pi\frac{k}{T}t}}\right)}{t}= \\
&=\frac{1}{T}\intd{-\frac{T}{2}}{\frac{T}{2}}{\sum_{k=-\infty}^{+\infty}{\abs{c_k}^2}}{t}=
\frac{1}{T}\sum_{k=-\infty}^{+\infty}{\abs{c_k}^2}\intd{-\frac{T}{2}}{\frac{T}{2}}{}{t}=\sum_{k=-\infty}^{+\infty}{\abs{c_k}^2}
\end{split}\]
\end{proof}

\section{Trasformata di Fourier}
Per segnali non periodici la potenza del segnale non è concentrata nei moti armonici ma è distribuita con continuità a tutte le frequenze dello spettro.

Il segnale aperiodico è rappresentabile come l'\textsc{integrale di Fourier}
\begin{equation}
s(t)=\intinf{S(f)\e{\imath 2\pi f t}}{f}
\end{equation}

La funzione complessa $S(f)$, nella variabile continua $f$, rappresenta la \textsc{trasformata di Fourie}r del segnale $s(t)$
\begin{equation}
S(f)=\intinf{s(t)\e{-\imath 2\pi f t}}{t}
\end{equation}
Il modulo e la fase della grandezza complessa definiscono
\[\abs{S(f)} \text{ spettro di ampiezza} \quad \angle S(f) \text{ spettro di fase}\]

\subsection{Condizioni di esistenza di Dirichlet}
I criteri di Dirichlet condizioni sufficienti per l'esistenza della trasformata di Fourier di un segnale $s(t)$:
\begin{enumerate}
\item $\intinf{\abs{s(t)}}{t}<\infty$ (assolutamente integrabile)
\item numero finito di discontinuità (tutte di I specie)
\item numero finito di max e min
\end{enumerate}
\begin{nota}Le condizioni non sono necessarie: alcune funzioni pur non soddisfacendo le tre condizioni sono trasformabili.\end{nota}
\begin{esempio}
Il segnale aperiodico di durata limitata $\tau$ ha trasformata di Fourier
\[s(t)=\rect{\frac{t}{\tau}}\quad\fourier{\rect{\frac{t}{\tau}}}=\tau\sinc{f\tau} \]

\begin{figure}[h!]
\centering
\subfloat[][$s(t)=\rect{\frac{t}{\tau}}$]
{\begin{tikzpicture}[scale=.6]
	\begin{axis}[axis lines=middle,no markers,enlargelimits,xscale=1.5,xtick={-.5,0,.5},xticklabels={$-\frac{\tau}{2}$,$0$,$\frac{\tau}{2}$},ytick={0,1}]
	\addplot [very thick,samples=200,domain=-1:1]  {abs(x)<.5?1:0};
	\end{axis}
\end{tikzpicture}}\qquad
\subfloat[][$S(f)=\tau\sinc{f\tau}$]
{\begin{tikzpicture}[scale=.6]
	\begin{axis}[axis lines=middle,no markers,enlargelimits,xscale=1.5,xtick={-9.424,-6.283,-3.141,0,3.141,6.283,9.424},ytick={0,1},xticklabels={$-3\tau$,$-2\tau$,$-\tau$,$0$,$\tau$,$2\tau$,$3\tau$},yticklabels={$0$,$\tau$}]
	\addplot [very thick,domain=-3.5*pi:3.5*pi,samples=100] {sin(x)/x};
	\end{axis}\end{tikzpicture}}
\end{figure}

\[\begin{split}S(f)&=\intinf{s(t)\e{-\imath 2\pi f t}}{t}
=\intd{-\frac{\tau}{2}}{\frac{\tau}{2}}{1\cdot\e{-\imath 2\pi f t}}{t}
=\bound{-\frac{\tau}{2}}{\frac{\tau}{2}}{\frac{\e{-\imath 2\pi f t}}{-\imath 2\pi f}}=\\
&=\frac{-\e{-\imath 2\pi f\frac{\tau}{2}}+\e{\imath 2\pi f\frac{\tau}{2}}}{\imath 2\pi f}
=\tau\frac{\sen{\pi f\tau}}{\pi f\tau}
=\tau\sinc{f\tau}
\end{split}\]
\end{esempio}

\section{Proprietà trasformata di Fourier}
\subsection{Simmetria}
La trasformata di un segnale reale $s(t)$ gode di simmetria hermitiana $\conj{S}(f)=S(-f)$
\begin{equation}\conj{S}(f)=\intinf{s(t)\e{+\imath 2\pi f t}}{t}=S(-f)\end{equation}
Espresso nella parte reale e immaginaria
\[S(f)=S_R(f)+\imath S_I(f)=\intinf{s(t)\cos{2\pi f t}}{t}-\imath\intinf{s(t)\sen{2\pi f t}}{t}\]
Se $s(t)$ è reale pari, la trasformata è reale pari: \[\conj{S}(f)=\conj{[S_R(f)]}=S(-f)\] \[S_R(f)=S_R(-f)\]
Se $s(t)$ è reale dispari, la trasformata è immaginaria pura dispari:
\[\conj{S}(f)=\conj{[+\imath S_I(f)]}=-\imath S_I(f)=\imath S_I(-f)\] \[S_I(f)=-S_I(-f)\]

\subsection{Linearità}
La trasformata di Fourier di combinazione lineare di segnali gode di linearità
\begin{equation}
\fourier{a x(t)+ b y(t)}= a X(f)+ b Y(f)
\end{equation}

\begin{proof}[Dim.]
Applicando la definizione di trasformata e per la linearità dell'operatore integrale
\[ \intinf{\left[a x(t)+ b y(t)\right]\e{-\imath 2\pi f t}}{t}= a\intinf{x(t)\e{-\imath 2\pi f t}}{t} + b\intinf{y(t)\e{-\imath 2\pi f t}}{t}= a X(f) + b Y(f) \]
\end{proof}

\subsection{Traslazione nel tempo}
La trasformata di Fourier di un segnale anticipato o ritardato nel tempo modifica lo spettro di fase del segnale
\begin{equation}
\fourier{s(t-\tau)}= S(f)\e{-\imath 2\pi f \tau}
\label{eq:trasf_Fourier_trasl}
\end{equation}
\begin{proof}[Dim.]
Applicando un cambio di variabile $\alpha=t-\tau$ e la definizione di trasformata
\[\intinf{s(t-\tau)\e{-\imath 2\pi f t}}{t}=
\intinf{s(\alpha)\e{-\imath 2\pi f\alpha}\e{-\imath 2\pi f\tau}}{\alpha}=S(f)\e{-\imath 2\pi f\tau}\]
\end{proof}

\subsection{Ribaltamento}
La trasformata di Fourier di un segnale invertito nel tempo ha lo spettro del segnale ribaltato nelle frequenze
\begin{equation}\fourier{s(-t)}=S(-f)\end{equation}
\begin{proof}[Dim.]
Applicando un cambio di variabile $\alpha=-t$ e la definizione di trasformata
\[\intinf{s(-t)\e{-\imath 2\pi f t}}{t}=
\intinf{s(\alpha)\e{-\imath 2\pi (-f)\alpha}}{\alpha}=S(-f)\]
\end{proof}

\subsection{Cambiamento di scala temporale}
La trasformata di Fourier di un segnale compresso o dilatato nella scala dei tempi dilata o comprime lo spettro delle frequenze del segnale
\begin{equation}\label{eq:trasf_Fourier_scala}
\fourier{s(a t)}=\frac{1}{\abs{a}}\,\f{S}{\frac{f}{a}}
\end{equation}
\begin{proof}[Dim.] Applicando un cambio di variabile $\alpha=a t, t=\frac{\alpha}{a}, \diff t=\frac{\diff\alpha}{a}$ e la definizione di trasformata
\[\intinf{s(a t)\e{-\imath 2\pi f t}}{t}=
\int_{-\infty}^{\infty}{s(\alpha)\e{-\imath 2\pi\frac{f}{a}\alpha}}{\frac{\diff\alpha}{\abs{a}}}=\frac{1}{\abs{a}}\,\f{S}{\frac{f}{a}}\]
\end{proof}

\subsection{Derivazione}
La trasformata di Fourier della derivata di un segnale
\begin{equation}
\fourier{s'(t)}=\imath 2\pi f\,S(f)
\end{equation}
\begin{proof}[Dim.] Si ha infatti che per un segnale derivabile $s(t)$
\[s(t)=\intinf{S(f)\e{\imath 2\pi f t}}{f}\implies s'(t)=\deriv{}{t}\intinf{S(f)\e{\imath 2\pi f t}}{f}=
\intinf{ \underbrace{\imath 2\pi f S(f)}_{\fourier{s'(t)}}\e{\imath 2\pi f t}}{f} \]
\end{proof}

\begin{nota}L'operatore derivata temporale di un segnale si comporta come un filtro passa alto perché nelle frequenze si ha il prodotto con il fattore $\imath 2\pi f$ che essendo proporzionale alla frequenza $f$ esalta in modulo le alte frequenze, mentre attenua le frequenze per $f\to 0$. La fase viene modificata di $\pm\frac{\pi}{2}$ a seconda del segno di $f$.\end{nota}

\subsection{Convoluzione nel tempo}
La trasformata di Fourier del prodotto di convoluzione di due segnali
\begin{equation}
\fourier{x(t)\ast y(t)}=X(f)\cdot H(f)
\label{eq:trasf_Fourier_conv}
\end{equation}
\begin{proof}[Dim.] Applicando la trasformata di Fourier al prodotto di convoluzione e l'eq.\ref{eq:trasf_Fourier_trasl}
\[\begin{split}& \intinf{ \left[\intinf{x(\tau)h(t-\tau)}{\tau}\right] \e{-\imath 2\pi f t}}{t} =
\intinf{ x(\tau) \left[\intinf{h(t-\tau)\e{-\imath 2\pi f t}}{t}\right]}{\tau}=\\
=& \intinf{ x(\tau) H(f)\e{-\imath 2\pi f \tau}}{\tau} = X(f) H(f) \end{split}\]
\end{proof}

\subsection{Dualità}
Nota la trasformata di Fourier $S(f)$ di un segnale $s(t)$ si ha che la trasformata di Fourier del segnale temporale $S(t)$ è per dualità $s(-f)$
\begin{equation}
\begin{split}
s(t) &\overset{\Fourier}{\to} S(f) \\
S(t) &\overset{\Fourier}{\to} s(-f)
\end{split}
\end{equation}
\begin{proof}[Dim.]
Nella tra il segnale e la sua trasformata $s(t)=\intinf{S(f)\e{+\imath 2\pi f t}}{f}$
si ha scambiando formalmente le variabili $t$ e $f$ \[s(f)=\intinf{S(t)\e{+\imath 2\pi f t}}{t}\]
cambiando variabile $f$ con $-f$
\[\fourier{S(t)}=\intinf{S(t)\e{-\imath 2\pi f t}}{t}=s(-f)\]
\end{proof}

\begin{esempio}
\begin{equation}
\begin{split}
\rect{t} &\overset{\Fourier}{\to} \sinc{f}\\
\sinc{t} &\overset{\Fourier}{\to} \rect{-f}=\rect{f}
\end{split}
\end{equation}
La proprietà di dualità consente di ottenere velocemente risultati che richiederebbero molti calcoli applicando la definizione.
\end{esempio}

\begin{esempio}\label{es:trasf_Fourier_conv}

\begin{figure}[h!]
\centering
\subfloat[][$s(t)=\rect{\frac{t}{T}}\ast\rect{\frac{t}{T}}$]
{\begin{tikzpicture}[scale=.67]
\begin{axis}[axis lines=middle,no markers,enlargelimits,xscale=1.2,xtick={-1,0,1},ytick={0,1},xticklabels={$-T$,0,$T$},yticklabels={0,$T$}]
\addplot [very thick]coordinates {(-2,0)(-1,0)(0,1)(1,0)(2,0)};
\end{axis}\end{tikzpicture}}\qquad
\subfloat[][$S(f)=T^2\Sinc^2(f T)$] {
\begin{tikzpicture}[scale=.7]
\begin{axis}[axis lines=middle,no markers,enlargelimits,xscale=1.5,xtick={-9.424,-6.283,-3.141,0,3.141,6.283,9.424},ytick={0,1},xticklabels={$-\frac{3}{T}$,$-\frac{2}{T}$,$-\frac{1}{T}$,$0$,$\frac{1}{T}$,$\frac{2}{T}$,$\frac{3}{T}$},yticklabels={$0$,$T^2$}]
\addplot [very thick,domain=-3.5*pi:3.5*pi,samples=100] { (sin(x)/x)^2 };
\end{axis}\end{tikzpicture}
}
\caption{Esempio \ref{es:trasf_Fourier_conv} convoluzione segnali rettangolari e trasformata}
\end{figure}

\[s(t)=\rect{\frac{t}{T}}\ast\rect{\frac{t}{T}}\]
Per la prop. di scala $\rect{\frac{t}{T}}\overset{\Fourier}{\to}T\sinc{f T}$ e per la prop. di convoluzione si ha
\[S(f)=T^2\Sinc^2(f T)\]
\end{esempio}

\subsection{Traslazione in frequenza}
L'anti-trasformata di Fourier del segnale traslato in frequenza $S(f-f_0)\overset{\Fourier^{-1}}{\rightarrow}s(t)\e{\imath 2\pi f_0 t}$

\begin{proof}[Dim.]
Applicando l'anti trasformata di Fourier allo spettro traslato in frequenza, con opportuno cambio di variabile $\alpha=f-f_0, f=f_0+\alpha, \diff\alpha=\diff f$
\[\intinf{S(f-f_0)\e{\imath 2\pi f t}}{f}=
\intinf{S(\alpha)\e{\imath 2\pi\alpha t}\e{\imath 2\pi f_0 t}}{\alpha}=\e{\imath 2\pi f_0 t} s(t) \]
\end{proof}

\subsection{Prodotto nel tempo $\to$ convoluzione in frequenza}
Dati due segnali $x(t)$ e $y(t)$ con le loro trasformate di Fourier $X(f)$ e $Y(f)$. La trasformata del segnale prodotto
\begin{equation}
\begin{split}
z(t) = x(t) y(t) &\overset{\Fourier}{\rightarrow} Z(f) = X(f)\ast Y(f)
\end{split}
\end{equation}

\begin{proof}[Dim.] Applicando la definizione
\[\begin{split}Z(f)&=\intd{t=-\infty}{+\infty}{z(t)\e{-\imath 2\pi f t}}{t}=\intd{t=-\infty}{+\infty}{x(t)y(t)\e{-\imath 2\pi f t}}{t}=\\
\intertext{sostituendo a $x(t)$ la sua espressione come integrale di Fourier}
&=\intd{t=-\infty}{+\infty}{\left[\intd{\nu=-\infty}{+\infty}{X(\nu)\e{\imath 2\pi\nu t}}{\nu}\right]y(t)\e{-\imath 2\pi f t}}{t}=\\
\intertext{invertendo l'ordine di integrazione}
&=\intd{\nu=-\infty}{+\infty}{X(\nu)\left[\intd{t=-\infty}{+\infty}{y(t)\e{-\imath 2\pi(f-\nu)t}}{t}\right]}{\nu}=\\
\intertext{si ha nella parentesi quadra la trasformata di $y(t)$ calcolata alla frequenza $f-\nu$}
&=\intd{\nu=-\infty}{+\infty}{X(\nu)Y(f-\nu)}{\nu} = X(f)\ast Y(f)
\end{split}\]
\end{proof}

\begin{nota}
Dato un segnale generico è possibile estrarre una finestra di durata limitata moltiplicando il segnale per un $\Rect$. Per la proprietà della trasformata del prodotto risulta che un segnale di durata limitata ha sempre banda a tutte le frequenze perché il $\Rect$ ha spettro infinito.
\end{nota}

\subsection{Teorema di Parseval}
Per i segnali di energia finita si può sempre calcolare la trasformata di Fourier. Per il teorema di Parseval
\begin{equation}E_s=\intinf{\abs{s(t)}^2}{t}=\intinf{\abs{S(f)}^2}{f}\label{eq:parseval}\end{equation}
che da indicazioni su come è distribuita l'energia del segnale alle varie frequenze. Per l'integrale di Fourier $s(t)=\intinf{S(f)\e{\imath 2\pi f t}}{t}$ il segnale $s(t)$ è la somma di infinite sinusoidi pesate da $S(f)$: che è anche l'integrale dei contributi infinitesimi di energia alle varie frequenze dello spettro.
\begin{nota}\`{E} importante notare che per la trasformata di Fourier cambiare l'energia ad una frequenza $f_0$ non comporta cambiamenti ad altre frequenze.\end{nota}
\begin{proof}[Dim.]
\[\begin{split}\intinf{\abs{s(t)}^2}{t}&=\intinf{s(t)\conj{s}(t)}{t}
=\intinf{s(t)\conj{\left[\intinf{S(f)\e{\imath 2\pi f t}}{f}\right]}}{t}=\\
&=\intinf{s(t)\intinf{\conj{S}(f)\e{-\imath 2\pi f t}}{f}}{t}=\intinf{\conj{S}(f)\intinf{s(t)\e{-\imath 2\pi f t}}{t}}{f}=\\
&=\intinf{\conj{S}(f)S(f)}{f}=\intinf{\abs{S(f)}^2}{f}\end{split}\]
\end{proof}

\begin{nota}Tutte le proprietà della trasformata di Fourier valgono sotto le condizioni di Dirichlet. Tali condizioni sono molto stringenti ai fini pratici, anche segnali elementari come l'impulso e il gradino non ammettono trasformata di Fourier. Le condizioni saranno superate con la trasformata generalizzata.\end{nota}

\subsection{Trasformata funzione generalizzata $\delta(t)$}
Per calcolare la trasformata di Fourier della funzione generalizzata delta di Dirac $\delta(t)$ si applica la definizione tenendo conto della proprietà campionatrice della $\delta(t)$
\begin{equation}
\fourier{\delta(t)}=\intinf{\delta(t)\e{-\imath 2\pi f t}}{t}=\restrict{\e{-\imath 2\pi f t}}{t=0}=1
\end{equation}
Da questo risultato per la funzione generalizzata e per il teorema di dualità è possibile calcolare la trasformata di Fourier del segnale ad energia infinita come il segnale costante
\begin{equation}
\fourier{1}=\delta(-f)=\delta(f)
\end{equation}
\[\fourier{c}=c\,\delta(f)\]

\subsection{Trasformata funzione generalizzata $\step(t)$}
La trasformata di Fourier della funzione gradino unitario ideale $\step(t)$ non esiste. Bisogna far ricorso alla trasformata con le funzioni generalizzate, definendo il gradino come
\[\step(t)=\frac{1}{2}+\frac{1}{2}\sgn(t)\]
\begin{equation}
\fourier{\step(t)}=\frac{1}{2}\delta(f)+\frac{1}{\imath 2\pi f}
\end{equation}
\begin{figure}[h!]\centering
\begin{tikzpicture}[scale=.6]
\begin{axis}[axis lines=middle,no markers,enlargelimits,xscale=1.5,xtick={-1,0,1},ytick={0,.5,1},yticklabels={$0$,$\frac{1}{2}$,$1$}]
\addplot [very thick]coordinates {(-1,0)(0,0)};
\addplot [only marks, samples at={0}]coordinates {(0,.5)};
\addplot [very thick]coordinates {(0,1)(1,1)};
\end{axis}\end{tikzpicture}
\caption{Funzione $\sgn(t)$}
\end{figure}

\begin{proof}[Dim.]
La trasformata della funzione $\sgn(t)$ si ottiene per dualità della trasformata della funzione $\frac{1}{t}\overset{\Fourier}{\to}-\imath\pi\sgn(f)$
\[\begin{split}&\fourier{\frac{1}{t}}=\intinf{\frac{1}{t}\e{-\imath 2\pi f t}}{t}= \underbrace{\intinf{\frac{1}{t}\cos{2\pi f t}}{t}}_{\stackrel{=0}{1/t \text{ dispari}}} -\imath\intinf{\frac{1}{t}=\sen{2\pi f t}}{t}=\\
&=-\imath 2\pi f\intinf{\frac{\sen{2\pi f t}}{2\pi f t}}{t}=-\imath 2\pi f\intinf{\sinc{2 f t}}{t}= -\imath\frac{2\pi f}{2\abs{f}} = -\imath\pi\sgn(f)
\end{split} \]
L'integrale nel tempo da $-\infty$ a $+\infty$ di un segnale è pari alla trasformata del segnale calcolata per $f=0$. La trasformata del $\sinc{t}\overset{\Fourier}{\to}\rect{f}$ per la prop. di dualità, la trasformata di $\sinc{2\alpha t}\overset{\Fourier}{\to}\frac{1}{2\abs{\alpha}}\rect{\frac{f}{2\alpha}}$ per la prop. di scala, calcolata per $f=0$ il $\rect{0}=1$, da cui
$\intinf{\sinc{2 f t}}{t}=\frac{1}{2\abs{f}}$.

Da $s(t)=\frac{1}{t}\overset{\Fourier}{\to} -\imath\pi\sgn(f)$, per dualità $S(t)\overset{\Fourier}{\to} s(-f)$ da cui \[\fourier{\sgn(t)}=\frac{1}{-\imath\pi}\frac{1}{-f}=\frac{1}{\imath\pi f}\] da cui il risultato per la trasformata generalizzata del gradino \[\fourier{\step(t)}]=\fourier{\frac{1}{2}+\frac{1}{2}\sgn(t)}=\frac{1}{2}\delta(f)+\frac{1}{\imath 2\pi f}\]
\end{proof}

\subsection{Teorema integrazione completo}\index{Teorema!integrazione completo}
Per un segnale definito come integrale di convoluzione con un gradino
\[y(t)=\intd{-\infty}{t}{x(\tau)}{\tau}=\intinf{x(\tau)\step(t-\tau)}{\tau}=x(t)\ast\step(t)\]
si può determinare la trasformata generalizzata
\begin{equation}
Y(f)=X(f) U(f)=\frac{1}{2} X(0) \delta(f) + \frac{X(f)}{\imath 2\pi f}
\end{equation}
che tiene conto nel primo termine della componente continua del segnale.

\subsection{Trasformate funzioni seno e coseno}
Dalla funzione generalizzata $\delta(t)\overset{\Fourier}{\rightarrow}1$ si hanno per il teo. traslazione del tempo e per il teo. dualità le trasformate di Fourier delle funzioni seno e coseno
\[\begin{split}\delta(t-t_0)&\overset{\Fourier}{\to}\e{-\imath 2\pi f t_0}\\
\e{\imath 2\pi f_0 t}&\overset{\Fourier}{\to}\delta(f-f_0)\end{split}\]
\begin{equation}
\cos{2\pi f_0 t}=\frac{\e{\imath 2\pi f_0 t}+\e{-\imath 2\pi f_0 t}}{2}\overset{\Fourier}{\to}\frac{\delta(f-f_0)+\delta(f+f_0)}{2}\end{equation}
\begin{equation}\sen{2\pi f_0 t}=\frac{\e{\imath 2\pi f_0 t}-\e{-\imath 2\pi f_0 t}}{2\imath}\overset{\Fourier}{\to}\frac{\delta(f-f_0)-\delta(f+f_0)}{2\imath}\end{equation}

\subsection{Modulazione}
Per il teorema trasformata del prodotto e la proprietà campionatrice della funzione generalizzata $\delta$ si ha il risultato notevole per cui la trasformata di un segnale modulato risulta
\begin{equation}\label{eq:fourier_modulazione_coseno}
s(t) \cos{2\pi f_0 t}\overset{\Fourier}{\to}\frac{1}{2}\left[S(f-f_0)+S(f+f_0)\right]
\end{equation}
\begin{equation}
s(t) \sen{2\pi f_0 t}\overset{\Fourier}{\to}\frac{1}{2\imath}\left[S(f-f_0)-S(f+f_0)\right]
\end{equation}

\begin{proof}[Dim.]
Si ha infatti \[\fourier{s(t)\cos{2\pi f_0 t}}=S(f)\ast\frac{1}{2}\left[\delta(f-f_0)+\delta(f+f_0)\right]\] e che \[S(f)\ast\delta(f-f_0)=\intinf{S(\nu)\delta(f-f_0-\nu)}{\nu}=S(f-f_0)\]
\end{proof}

\begin{figure}[h!]
\centering
\subfloat[][$S(f)$]
{\begin{tikzpicture}[scale=.6]
	\begin{axis}[axis lines=middle,no markers,enlargelimits,xscale=1.5,xtick={-1,0,1},ytick={1}]
	\addplot [very thick]coordinates {(-2,0)(-1,0)(0,1)(1,0)(2,0)};
	\end{axis}\end{tikzpicture}}\qquad\subfloat[][$\frac{1}{2}\left\lbrace S(f-f_0)+S(f+f_0)\right\rbrace$] {
	\begin{tikzpicture}[scale=.6]
	\begin{axis}[axis lines=middle,no markers,enlargelimits,xscale=1.5,xtick={-2,2},ytick={1},xticklabels={$-f_0$,$f_0$},yticklabels={$\frac{1}{2}$}]
	\addplot [very thick]coordinates {(-4,0)(-3,0)(-2,1)(-1,0)(1,0)(2,1)(3,0)(4,0)};
	\end{axis}\end{tikzpicture}
}
\caption{Esempio modulazione}
\end{figure}

\subsection{Trasformata segnali periodici}
Un segnale periodico $s(t)$ è la somma di infinite repliche di un segnale ristretto $s_T(t)$ ad un periodo $\left[-\frac{T}{2},\frac{T}{2}\right]$ e per la definizione di serie di Fourier somma di esponenziali complessi pesati da coefficienti
\begin{equation}
s(t)=\sum_{n=-\infty}^{+\infty}{s_T(t-n T)}=\sum_{k=-\infty}^{+\infty}{c_k\e{\imath 2\pi\frac{k}{T}t}}
\end{equation}

Definita la trasformata di Fourier del segnale ristretto $S_T(f)=\fourier{s_T(t)}$ si ha che i coefficienti 	\[c_k=\frac{1}{T}\intd{-\frac{T}{2}}{\frac{T}{2}}{s(t)\e{-\imath 2\pi\frac{k}{T}t}}{t}=\frac{1}{T}\intinf{s_T(t)\e{-\imath 2\pi\frac{k}{T}t}}{t}=\frac{1}{T}\f{S_T}{\frac{k}{T}}\]
quindi
\[s(t)=\frac{1}{T}\sum_{k=-\infty}^{+\infty}{\f{S_T}{\frac{k}{T}}\e{\imath 2\pi\frac{k}{T}t}}\]
\begin{equation}
S(f)=\frac{1}{T}\sum_{k=-\infty}^{+\infty}{\f{S_T}{\frac{k}{T}}\f{\delta}{f-\frac{k}{T}}}
\end{equation}
La trasformata di Fourier di un segnale periodico è la somma di infiniti impulsi a frequenza multipla della fondamentale.

\clearpage
\section{Esempi ed esercizi}
\begin{esempio}
La trasformata di un treno di impulsi è un treno di impulsi in frequenza
\[s(t)=\sum_{n=-\infty}^{+\infty}{\delta(t-n T)} \qquad S(f)=\frac{1}{T}\sum_{k=-\infty}^{+\infty}{\f{\delta}{f-\frac{k}{T}}}\]
\begin{nota}
Esempio importante per digitalizzazione e campionamento (par.\ref{sec:campionamento}).
\end{nota}

\begin{figure}[h!]
\centering
\subfloat[][$s(t)$]{\begin{tikzpicture}
	\draw [-latex] (-4,0)--(4,0);
	\draw [-latex] (0,0)--(0,2);
	\foreach \i in {-3,...,3}
		\draw [very thick,-latex] (\i,0)--(\i,1);
	\foreach \i in {-3,...,3}
		\draw (\i,0)--(\i,-1mm);
	\node at(-3,0) [below] {$-3T$};
	\node at(-2,0) [below] {$-2T$};
	\node at(-1,0) [below] {$-T$};
	\node at(1,0) [below] {$T$};
	\node at(2,0) [below] {$2T$};
	\node at(3,0) [below] {$3T$};
	\node [right] at (0,2) {$s(t)$};
	\end{tikzpicture}
}\qquad\subfloat[][$S(f)$]{\begin{tikzpicture}
\draw [-latex] (-3,0)--(3,0);
\draw [-latex] (0,0)--(0,2);
\foreach \i in {-5,...,-1} \draw [very thick,-latex] (\i/2,0)--(\i/2,1);
\foreach \i in {1,...,5} \draw [very thick,-latex] (\i/2,0)--(\i/2,1);
\foreach \i in {-5,...,-1} \draw (\i/2,0)--(\i/2,-1mm);
\foreach \i in {1,...,5} \draw (\i/2,0)--(\i/2,-1mm);
\node at(-1.5,0) [below] {$-\frac{3}{T}$};
\node at(-1,0) [below] {$-\frac{2}{T}$};
\node at(-.5,0) [below] {$-\frac{1}{T}$};
\node at(.5,0) [below] {$\frac{1}{T}$};
\node at(1,0) [below] {$\frac{2}{T}$};
\node at(1.5,0) [below] {$\frac{3}{T}$};
\node [right] at (0,2) {$S(F)$};
\end{tikzpicture}}
\caption{Treno di impulsi nel tempo e in frequenza}
\end{figure}
\end{esempio}

\begin{esercizio}
Determinare la serie di Fourier del segnale onda triangolare in figura
\begin{figure}[h!]
\begin{center}\begin{tikzpicture}
\begin{axis}[width=12cm, height=4cm, axis lines=middle,no markers,enlargelimits,xtick={-2,-1,0,1,2,3},xticklabels={$-T$,$-\frac{T}{2}$,$0$,$\frac{\tau}{2}$,$T$,$\frac{3}{2}T$},ytick={-1,0,1},yticklabels={$-A$,0,$A$}]
\addplot [very thick,domain=-2:2]coordinates {(-2,1)(-1,-1)(0,1)(1,-1)(2,1)(3,-1)};
\end{axis}
\end{tikzpicture}
\end{center}
\caption{Onda triangolare}\label{fig:onda_triangolare}
\end{figure}

\[s(t)=\sum_{k=-\infty}^{+\infty}{c_k \e{\imath 2\pi\frac{k}{T}t}} =c_0+\sum_{k=1}^{+\infty}{a_k\sen{2\pi\frac{k}{T}t}+b_k\cos{2\pi\frac{k}{T}t}}\]

Il segnale ha media nulla $c_0=0$. Il segnale è pari $a_k=0$.
\[\begin{split}b_k&=\frac{2}{T}\intd{-\frac{T}{2}}{\frac{T}{2}}{s(t)\cos{2\pi\frac{k}{T}t}}{t}=\frac{4}{T}\intd{0}{\frac{T}{2}}{s(t)\cos{2\pi\frac{k}{T}t}}{t}=\\
&=\frac{4}{T}\intd{0}{\frac{T}{2}}{\left(A-\frac{2 A}{T/2}\right)\cos{2\pi\frac{k}{T}t}}{t}=\\
&=\frac{4A}{T}\bound{0}{\frac{T}{2}}{\frac{\sen{2\pi\frac{k}{T}t}}{2\pi\frac{k}{T}}} -\frac{16A}{T^2}\intd{0}{\frac{T}{2}}{t\cos{2\pi\frac{k}{T}t}}{t}=\\
\intertext{per parti $\int u\diff v= u v-\int v\diff u$ con $u=t$, $\diff u=\diff t$, $\diff v=\cos{2\pi\frac{k}{T}t}\diff t$, $v=\frac{\sen{2\pi\frac{k}{T}t}}{2\pi\frac{k}{T}}$}
&=2A \frac{\sen{\pi k}}{\pi k}-\frac{16A}{T^2}{\left\lbrace \bound{0}{\frac{T}{2}}{t\frac{\sen{2\pi\frac{k}{T}t}}{2\pi\frac{k}{T}}}-\intd{0}{\frac{T}{2}}{\frac{\sen{2\pi\frac{k}{T}t}}{2\pi\frac{k}{T}}}{t}\right\rbrace}=\\
&=2A \frac{\sen{\pi k}}{\pi k}-\frac{16A}{T^2}{\left\lbrace\frac{T}{2}\frac{\sen{\pi k}}{2\pi\frac{k}{T}}-\bound{0}{\frac{T}{2}}{\frac{\cos{2\pi\frac{k}{T}t}}{\left(2\pi\frac{k}{T}\right)^2}}\right\rbrace}=\\
&=2A \frac{\sen{\pi k}}{\pi k}-\frac{16A}{T^2}{\left\lbrace\frac{T^2}{4}\frac{\sen{\pi k}}{\pi k}-\frac{\cos{\pi k}}{\left(2\pi\frac{k}{T}\right)^2}-\frac{1}{\left(2\pi\frac{k}{T}\right)^2}\right\rbrace}=\\
&=2A \frac{\sen{\pi k}}{\pi k}-4A{\left\lbrace\frac{\sen{\pi k}}{\pi k}-\frac{\cos{\pi k}}{\left(\pi k\right)^2}-\frac{1}{\left(\pi k\right)^2}\right\rbrace}=\\
\intertext{dove $\sen{\pi k}=0$ per $k=1,\dots+\infty$, $\cos{\pi k}=+1$ per k dispari, $-1$ per k pari}
&=\frac{4A}{(\pi k)^2}\left[1-\cos{\pi k}\right] = \frac{4A}{(\pi k)^2}\left[1-(-1)^k\right]
\end{split}\]
\end{esercizio}

\begin{esercizio}Calcolare energia e intensità spettrale di energia del segnale
\[x(t)=A\rect{\frac{t}{2T}}\]
Il segnale ha trasformata $\quad X(f)=A 2 T \sinc{2 f T}$\\
Il segnale ha energia
\[E_x=\intinf{\abs{x(t)}^2}{t}=2 A^2 T\]
Il segnale ha spettro di energia
\[\abs{X(f)}^2=(2 A T)^2\Sinc^2{2 f T}\]
\end{esercizio}
\begin{esercizio}
Calcolare energia e intensità spettrale di energia del segnale
\[y(t)=D\e{-\alpha t}\step(t)\]
Il segnale ha trasformata
\[\begin{split}Y(f)&=\intinf{ D\e{-\alpha t}\e{-\imath 2\pi f t}}{t}=D\intd{0}{+\infty}{\e{-(\alpha+\imath 2\pi f)t}}{t}=\\
&=\bound{0}{+\infty}{-\frac{D}{\alpha+\imath 2\pi f}\e{-(\alpha+\imath 2\pi f)t}}=\frac{D}{\alpha+\imath 2\pi f}\end{split}\]
Il segnale ha energia
\[E_y=\intinf{\abs{y(t)}^2}{t}=\intd{0}{+\infty}{D^2\e{-2\alpha t}}{t}=\bound{0}{+\infty}{\frac{D^2}{-2\alpha}\e{-2\alpha}}=\frac{D^2}{2\alpha}\]
Il segnale ha spettro di energia
\[\abs{Y(f)}=\frac{D^2}{\abs{\alpha+\imath 2\pi f}^2}=\frac{D^2}{\alpha^2+(2\pi f)^2}\]
\end{esercizio}
\begin{esercizio}
Calcolare la trasformata di Fourier del segnale
\[x(t)=\f{\Cos^2}{2\pi\frac{t}{T_0}}\]

ricordando che $\Cos^2 x=\frac{1}{2}+\frac{1}{2}\cos{2x}$ si ottiene
\[x(t)=\frac{1}{2}+\frac{1}{2}\cos{4\pi\frac{t}{T_0}}\]
\[X(f)=\frac{1}{2}\delta(f)+\frac{\f{\delta}{f-\frac{2}{T_0}}+\f{\delta}{f+\frac{2}{T_0}}}{4}\]
Si hanno tre righe spettrali: la componente continua, è assente la prima armonica $\frac{1}{T_0}$, si hanno due righe a frequenza doppia della fondamentale.
\begin{figure}[h!]
\centering\begin{tikzpicture}
\draw [-latex] (-3,0)--(3,0);
\draw [-latex] (0,0)--(0,2.5);
\draw [very thick,-latex] (-2,0)--(-2,1);
\draw [very thick,-latex] (0,0)--(0,2);
\draw [very thick,-latex] (2,0)--(2,1);
\draw (-2,0) -- (-2,-1mm) node [below] {$-\frac{2}{T_0}$};
\draw (-1,0) -- (-1,-1mm) node [below] {$-\frac{1}{T_0}$};
\draw (0,0) -- (0,-1mm) node [below] {$0$};
\draw (1,0) -- (1,-1mm) node [below] {$\frac{1}{T_0}$};
\draw (2,0) -- (2,-1mm) node [below] {$\frac{2}{T_0}$};
\node [right] at (0,2.5) {$X(f)$};
\end{tikzpicture}
\end{figure}
\end{esercizio}

%Il capitolo 3 è stato revisionato dal Prof. Grieco il 12/01/2017.
\chapter{Sistemi lineari tempo-invarianti}
\label{ch:teoriasegnali-capitolo3}
Un sistema fisico di elaborazione di un segnale può essere modellato come una black box che riceve in ingresso uno o più segnali e restituisce in uscita uno o più segnali.

\begin{figure}[!ht]
	\centering
	\begin{tikzpicture}[node distance=2cm]
		\node [block] (system) at (0,0) {S};
		\node [left of=system](input) {$s(t)$};
		\node [right of=system] (output) {$r(t)$};
		\draw [-latex] (input) -- (system);
		\draw [-latex] (system) -- (output);
	\end{tikzpicture}
\end{figure}

\section{Classificazione dei sistemi}
Si classificano i sistemi per le seguenti proprietà:
\begin{description}
\item[Memoria:] Un sistema \emph{privo di memoria} ha uscita $r(t)=f[s(t)]$ che dipende solo dal valore di s in $t$ altrimenti è \emph{dotato di memoria}.

\item[Causalità:] Un sistema è causale se l'uscita $r(t_0)$ dipende dai valori assunti da $s(t)$ per $t\leq t_0$, \emph{anti-causale}, l'uscita è influenzata da valori futuri dell'ingresso.

\item[Tempo invarianza:] Un sistema tempo-invariante presenta lo stesso segnale di uscita in risposta ad un ingresso ritardato
\[
	\begin{split}
		s(t)&\to r(t)\\
		s(t-\tau)&\to r(t-\tau)
	\end{split}
\]

\item[Invertibilità:] Un sistema inverso può riportare la risposta $r(t)$ di un sistema al segnale originale $s(t)$
\begin{figure}[!ht]
	\centering
	\begin{tikzpicture}[node distance=2cm,>=latex']
		\node [block,node distance=3cm] (system) at (0,0) {S};
		\node [left of=system](input) {$s(t)$};
		\node [block, right of=system, node distance=3cm] (inverse) {$S^{-1}$};
		\node [right of=inverse] (output) {$s(t)$};
		\draw [->] (input) -- (system);
		\draw [->] (system) -- (inverse) node[pos=.5,below] {$r(t)$};
		\draw [->] (inverse) -- (output);
	\end{tikzpicture}
\end{figure}

\item[Linearità:] Un sistema per cui vale il principio di sovrapposizione degli effetti, per cui dati $s_1(t)\to r_1(t)$ e $s_2(t)\to r_2(t)$, alla combinazione lineare degli effetti corrisponde la combinazione lineare delle uscite
\[
	a\cdot s_1(t)+ b\cdot  s_2(t)\to a\cdot r_1(t)+b\cdot r_2(t)
\]

\item[Stabilità:] Un sistema \textsc{BIBO}-stabile ha una uscita limitata in risposta ad un ingresso limitato
\[
	\abs{s(t)}<\infty\;\exists R_\text{max}<\infty\ni'\forall t\abs{r(t)}<R_\text{max}
\]
\end{description}

\section{Risposta all'impulso}
In un sistema lineare vale il principio di sovrapposizione degli effetti. Qualunque segnale in ingresso  si può rappresentare come somma di infiniti impulsi
\[
	s(t)=s(t)\ast\delta(t)=\intinf{s(\tau)\delta(t-\tau)}{\tau}
\]

Si può calcolare la risposta del sistema a qualunque ingresso $s(t)$ conoscendo la risposta all'impulso $\delta(t)\to h(t)$ o più in generale in funzione di $t$ e da un istante iniziale $\tau$: $\delta(t-\tau)\to h(t,\tau)$ come
\[
	r(t)=\intinf{s(\tau)h(t,\tau)}{\tau}
\]

\begin{figure}[!ht]
	\centering
	\begin{tikzpicture}[node distance=2cm,>=latex']
		\node [block] (system) at (0,0) {S};
		\node [left of=system](input) {$\delta(t-\tau)$};
		\node [right of=system] (output) {$h(t,\tau)$};
		\draw [->] (input) -- (system);
		\draw [->] (system) -- (output);
	\end{tikzpicture}
\end{figure}

Se il sistema è \textsc{LTI} lineare tempo invariante $h(t,\tau)=h(t-\tau)$, per cui si ha il risultato notevole che qualunque risposta del sistema si ottiene come convoluzione del segnale di ingresso con la risposta all'impulso del sistema
\begin{equation}
	r(t)=\intinf{s(\tau)h(t-\tau)}{\tau}=s(t)\ast h(t)
\end{equation}

\section{Sistemi Lineari Tempo Invarianti}
Per i sistemi LTI la risposta in frequenza del sistema ha trasformata di Fourier
\begin{equation}
\begin{split}
	r(t)&=s(t)\ast h(t)\\R(f)&=S(f)\cdot H(f)
\end{split}
\end{equation}

Si tratta di sistemi privi di memoria per cui l'uscita dipende solo dal segnale in ingresso in $t$, $r(t)=f[s(t)]$, il che implica una risposta all'impulso $h(t)=k$ costante.

Sono sistemi causali in cui $r(t_0)$ dipende dai valori $s(t)$ per $t<t_0$.

La stabilità BIBO per sistemi LTI implica l'assoluta integrabilità della risposta all'impulso (rispetta le condizioni di Dirichlet)
\[
	\text{LTI BIBO}\iff \intinf{\abs{h(t)}}{t}<\infty
\]

\`E possibile ottenere il sistema inverso che restituisca il segnale in ingresso, che abbia una risposta all'impulso tale che $h(t)\ast g(t)=\delta(t)$ ovvero
\[
	H(f)\cdot G(f)=1 \qquad G(f)=\frac{1}{H(f)}
\]

Un sistema LTI con in ingresso un segnale di energia ha in uscita un segnale di energia
\[
	E_s=\intinf{\abs{S(f)}^2}{f} \qquad E_r=\intinf{\abs{S(f)}^2 \abs{H(f)}^2}{f}
\]
\begin{nota}
	Proprietà importante dei sistemi LTI per i filtri lineari in cui interessa dimensionare soprattutto la banda passante e l'energia / densità spettrale di energia.
\end{nota}

\section{Filtri ideali}
\subsection{Filtro passa basso ideale}\index{filtro!passa basso}
Il filtro passa basso ideale elimina tutte le frequenze non comprese nella banda passante $f\in[-B,B]$.

\begin{figure}[!ht]
	\centering
	\begin{tikzpicture}[xscale=.8,yscale=.6]
		\begin{axis}[axis lines=middle,no markers,enlargelimits,xscale=2,xtick={-.5,0,.5},xticklabels={$-B$,$0$,$B$},ytick={0,1},xlabel=$f$,ylabel=$H_\text{LP}(f)$]
		\addplot [very thick,samples=100,domain=-1:1]  {abs(x)<.5?1:0};
		\addplot [dashed,samples=11,domain=-1:1]  {abs(x)<.5?1:0};
		\end{axis}
	\end{tikzpicture}
\end{figure}

\begin{equation}
\label{eq:filtro_passa_basso_ideale}
	H_\text{LP}(f)=\rect{\frac{f}{2B}}\qquad h_\text{LP}(t)=2B\sinc{2 B t}
\end{equation}

\`E ideale perché per filtrare tutte le frequenze al di fuori dell'intervallo bisognerebbe conoscere il segnale nel tempo anche per valori futuri (filtro anti-causale non fisicamente realizzabile).

\begin{nota}
	Il filtro passa basso viene applicato sempre prima di altri filtri amplificatori per attenuare il rumore non desiderato che ha energia a tutte le frequenze.
\end{nota}

\subsection{Filtro passa alto ideale}
\index{filtro!passa alto}
Il filtro passa alto ideale fa passare tutte le frequenze al di fuori dell'intervallo $[-B,B]$. Il filtro reale si attenua alle più alte frequenze.

\begin{figure}[!ht]
	\centering
	\begin{tikzpicture}[xscale=.8,yscale=.6]
		\begin{axis}[axis lines=middle,no markers,enlargelimits,xscale=2,xtick={-.5,0,.5},xticklabels={$-B$,$0$,$B$},ytick={0,1},xlabel=$f$,ylabel=$H_\text{LP}(f)$]
		\addplot [very thick,samples=100,domain=-1.5:1.5]  {abs(x)>.5?1:0};
		\addplot [dashed,samples=11,domain=-1:1] {abs(x)>.5?1:0};
		\addplot [dashed,samples=11,domain=-2:-1] {x+2};
		\addplot [dashed,samples=11,domain=1:2] {-x+2};
		\end{axis}
	\end{tikzpicture}
\end{figure}
\begin{equation}
	H_\text{HP}(f)=1-H_\text{LP}(f)\qquad h_\text{HP}(t)=\delta(t)-2B\sinc(2 B t)
\end{equation}

\subsection{Filtro passa banda ideale}
\index{filtro!passa banda}
Il filtro passa banda ideale fa passare le frequenze attorno a $f_0$ tra $f_L$ e $f_H$ e attorno a $-f_0$ tra $-f_H$ e $-f_L$ con banda $B=f_H-f_L$.

\begin{figure}[!ht]
	\centering
	\begin{tikzpicture}[xscale=.8,yscale=.6]
		\begin{axis}[axis lines=middle,no markers,enlargelimits,xscale=2,xtick={-1.5,-1,-.5,0,.5,1,1.5},xticklabels={$-f_h$,$-f_0$,$-f_L$,$0$,$f_L$,$f_0$,$f_H$},ytick={0,1},xlabel=$f$,ylabel=$H_\text{BP}(f)$]
		\addplot [very thick,samples=100,domain=-2:0]  {abs(x+1)<.5?1:0};
		\addplot [very thick,samples=100,domain=0:2]  {abs(x-1)<.5?1:0};
		\addplot [dashed,samples=11,domain=-2:0]  {abs(x+1)<.5?1:0};
		\addplot [dashed,samples=11,domain=0:2]  {abs(x-1)<.5?1:0};
		\end{axis}
	\end{tikzpicture}
\end{figure}

Si definisce il \textsc{fattore di qualità}\index{fattore di qualità} $Q=\frac{f_0}{B}$, che risulta tanto migliore quanto più è stretta la banda attorno alla frequenza $f_0$.
\begin{equation}
	H_\text{BP}(f)=\rect{\frac{f-f_0}{B}}+\rect{\frac{f+f_0}{B}}
\end{equation}
\[
	h_\text{BP}(t)=B\sinc{B t}\e{\imath 2\pi f_0 t}+B\sinc{B t}\e{-\imath 2\pi f_0 t}=2 B\sinc{B t}\cos{2\pi f_0 t}
\]

\begin{figure}[!ht]
	\centering
	\begin{tikzpicture}
		\begin{axis}[axis lines=middle,no markers,enlargelimits,xscale=2,xtick={-9.424,-6.283,-3.141,0,3.141,6.283,9.424},ytick={0,2},xticklabels={$-\frac{3}{B}$,$-\frac{2}{B}$,$-\frac{1}{B}$,$0$,$\frac{1}{B}$,$\frac{2}{B}$,$\frac{3}{B}$},yticklabels={$0$,$2B$},ylabel={$h_\text{BP}(t)$}]
		\addplot [thick,domain=-3.5*pi:3.5*pi,samples=300] {(2*sin(x)/x)*cos(8*x)};
		\addplot [dashed,domain=-3.5*pi:3.5*pi,samples=100] {(2*sin(x)/x)};
		\end{axis}
	\end{tikzpicture}
	\caption{Risposta all'impulso del filtro passa banda ideale}
\end{figure}

\begin{esempio}
Esempio di filtro passa basso realizzato con circuito \textsc{RC}.

\begin{figure*}[!ht]
	\centering
	\begin{circuitikz}
		\draw (0,0)	to[open,v^>=${V_i(t)}$] (0,3)
			to[R, l=${R}$, *-] (3,3)
			to[C, l=${C}$,i>_=${i(t)}$] (3,0) to[short,-*] (0,0)
			(3,0) -- (4,0) to[open,v>=${V_u(t)}$, *-*] (4,3) -- (3,3)
			(1.5,1.5) node[scale=3]{$\circlearrowright$}
			(1.5,1.5) node{$i(t)$};
	\end{circuitikz}
	\caption{Filtro passa basso realizzato con circuito RC}
\end{figure*}

La tensione sul condensatore $v_c$ pari alla tensione di uscita $v_u$
\[
	v_u(t)=\frac{q(t)}{C}
\]
\[
	\deriv{v_u(t)}{t}=\frac{1}{C}i(t)
\]
L'equazione differenziale che descrive l'andamento delle tensioni
\[
	v_i(t)=R i(t)+ v_u(t)=R C \deriv{v_u(t)}{t} + v_u(t)
\]
Trasformata secondo Fourier
\[
	V_i(f)=R C \imath 2\pi f_o V_u(f) + V_u(f)
\]
Funzione di trasferimento ingresso-uscita
\[
	H(f)=\frac{V_i(f)}{V_u(f)}=\frac{1}{1+\imath 2\pi f R C}
\]
dove $f_T=\frac{1}{2\pi R C}$ rappresenta la frequenza di taglio.

\begin{figure}[!ht]
	\centering
	\begin{tikzpicture}[scale=.8]
		\begin{axis}[axis lines=middle,no markers,enlargelimits,xscale=1.5,xtick={.159},xticklabels={$f_T$},ytick={0,1},xlabel=$f$,ylabel=$\abs{H(f)}$]
		\addplot [very thick,samples=200,domain=-1:1]  {1/(sqrt(1+(2*pi*x)^2))};
		\addplot [dashed] coordinates {(-.159,.707)(.159,.707)(.159,0)};
		\end{axis}
	\end{tikzpicture}
	\caption{Modulo della funzione di trasferimento}
\end{figure}

La risposta in ampiezza espressa in decibel, come confronto tra potenze relative ad un riferimento
\[
	\abs{H(f)}_\text{dB}=10\Log\frac{\abs{H(f)}^2}{\abs{H(f_0)}^2}
\]
si calcola sulla potenza del segnale, ovvero alla sua risposta in ampiezza al quadrato
\[
	\abs{H(f)}^2=\frac{1}{1+(2\pi f R C)^2}
\]
calcolata rispetto al riferimento $f_0=0$ per cui $\abs{H(f_0)}^2=1$

Alla frequenza di taglio $f_T=\frac{1}{2\pi R C}$ si dimezza il modulo quadro ovvero si ha una attenuazione pari a
\[
	\abs{H\left(f_T\right)}_\text{dB}=10\Log\frac{1}{2}=-3\text{dB}
\]

\begin{figure}[!ht]
	\centering
	\begin{tikzpicture}[scale=.8]
		\begin{axis}[xmode=log,axis lines=middle,no markers,enlargelimits,xlabel=$f$,ylabel=$\abs{H(f)}_\text{dB}$]
		\addplot [very thick,samples=200,domain=.01:100] {-10*log10( 1+(2*pi*x*.1)^2)};
		\addplot [dashed] coordinates{(.01,-3)(100,-3)};
		\end{axis}
	\end{tikzpicture}
	\caption{Modulo in decibel della funzione di trasferimento}
\end{figure}
\end{esempio}

\section{Autocorrelazione per segnali ad energia finita}
Si definisce per segnali ad energia finita la funzione \textsc{autocorrelazione}\index{segnale!autocorrelazione}, grandezza che indica quanto un segnale $x(t)$ sia simile ad una sua replica ritardata di un tempo $\tau$
\begin{equation}
	R_x(\tau)=\intinf{x(t)\conj{x}(t-\tau)}{t}=x(\tau)\ast \conj{x}(-\tau)
\end{equation}
La trasformata di Fourier del prodotto di convoluzione, essendo $x(-t)\overset{\Fourier}{\to}X(-f)$ si ha $\conj{x}(t)\overset{\Fourier}{\to}\conj{X}(-f) \implies \conj{x}(-t)\overset{\Fourier}{\to}\conj{X}(f)$
\[
	x(\tau)\ast \conj{x}(-\tau)\overset{\Fourier}{\to}X(f)\conj{X}(f)
\]

La funzione di autocorrelazione è anche l'antitrasformata di Fourier dello spettro di energia del segnale
\begin{equation}
	R_x(\tau)=\intinf{\abs{X(f)}^2\e{\imath 2\pi f\tau}}{f}
\end{equation}

\textbf{Proprietà}
\begin{enumerate}
\item La funzione di autocorrelazione calcolata per $\tau=0$ coincide con l'energia del segnale
\[
	R_x(0)=\intinf{x^2(t)}{t}=E_x
\]

\item La funzione di autocorrelazione è una funzione pari
\[
	R_x(\tau)=R_\conj{x}(-\tau)
\]
\begin{proof}[Dim.]
Facilmente dimostrabile in frequenza
\begin{align*}
	R_x(\tau)&=\intinf{\abs{X(f)}^2\e{\imath 2\pi f\tau}}{f}\\
	R_x(-\tau)&=\intinf{\abs{X(f)}^2\e{-\imath 2\pi f\tau}}{f}\\
	R_\conj{x}(-\tau)&=\intinf{\abs{X(f)}^2\e{\imath 2\pi f\tau}}{f}
	\qedhere
\end{align*}
\end{proof}

\item Il massimo della funzione di autocorrelazione si ha per $\tau=0$
\[
	\abs{R_x(\tau)}\leq R_x(0)
\]
\begin{proof}[Dim.]
Applicando la diseguaglianza di Schwarz:
\[
	\abs{R_x(\tau)}^2=\abs{\intinf{x(t)x(t-\tau)}{t}}^2\leq\intinf{\abs{x(t)}^2}{t}\cdot\intinf{\abs{x(t-\tau)}^2}{t}=R_x^2(0)
	\qedhere
\]
\end{proof}
\end{enumerate}

\begin{nota}
	Se un segnale ad energia finita $x(t)$ viene applicato a un sistema LTI BIBO stabile allora anche l'uscita $y(t)$ è un segnale ad energia finita con spettro di energia $\abs{Y(f)}^2=\abs{H(f)}^2\abs{X(f)}^2$. Si noti inoltre che l'energia delle componenti di $y$ alle varie frequenze si ottiene sagomando l'energia delle componenti di $x$ attraverso $\abs{H(f)}^2$.
\end{nota}

\section{Cross correlazione per segnali ad energia finita}
Si definisce la \textsc{cross-correlazione}\index{segnali!cross-correlazione} tra due segnali $x(t)$ e $y(t)$ la misura della somiglianza tra due segnali tra cui intercorre un ritardo di un tempo $\tau$
\begin{equation}
	R_{xy}(\tau)=\intinf{x(t)\conj{y}(t-\tau)}{t}=x(\tau)\ast \conj{y}(-\tau)
\end{equation}
\begin{equation}
	R_{yx}(\tau)=\intinf{y(t)\conj{x}(t-\tau)}{t}=y(\tau)\ast \conj{x}(-\tau)
\end{equation}
Si può dimostrare che 
\[
	R_{xy}(\tau)=R_\conj{yx}(-\tau)
\]
infatti se $R_{yx}(\tau)=y(\tau)\ast \conj{x}(-\tau)\implies R_\conj{yx}(\tau)=\conj{y}(\tau)\ast x(-\tau)\implies R_\conj{yx}(-\tau)=\conj{y}(-\tau)\ast x(\tau)=R_{xy}$.

Due segnali si dicono \textsc{ortogonali}\index{segnali!ortogonali} se sono incorrelati qualunque sia il ritardo $\tau$
\[
	R_{xy}(\tau)=0\;\forall\tau
\]

\section{Autocorrelazione per segnali a potenza finita}
Per i segnali a potenza finita si è definita la potenza (eq.\ref{eq:segnale_potenza}) come
\[
	P_s=\lim\limits_{T\to+\infty}{\frac{1}{2T}\intd{-T}{T}{\abs{s(t)}^2}{t}}
\]
Tale quantità si può definire anche nel dominio delle frequenze, come integrale della densità spettrale di potenza del segnale. Estraendo una ristretta del segnale $s_T(t)=s(t)\rect{\frac{t}{2T}}$ all'intervallo $[-T,T]$ essa ha sicuramente energia finita, pertanto se ne potrà esprimere la trasformata di Fourier, $s_T(t)\overset{\Fourier}{\to}S_T(f)$. Per il teorema di Parseval (eq.\ref{eq:parseval}) l'energia è distribuita nello spettro
\[
	E_{s_T}=\intinf{\abs{s_T(t)}^2}{t}=\intinf{\abs{S_T(f)}^2}{f}
\]

Si può quindi definire la potenza come limite dell'energia della limitazione $s_T(t)$ rapportata al periodo, al tendere dell'intervallo della limitazione all'infinito
\[
	P_s=\lim\limits_{T\to+\infty}{\frac{1}{2T}\intd{-T}{T}{\abs{s_T(t)}^2}{t}}=\lim\limits_{T\to+\infty}{\frac{1}{2T}\intinf{\abs{S_T(f)}^2}{f}}
\]
Scambiando il limite con l'integrale si ha
\[
	P_s=\intinf{\lim\limits_{T\to+\infty}\frac{1}{2T}\abs{S_T(f)}^2}{f}
\]
la funzione integranda che si definisce \textsc{densità spettrale di potenza}\index{densità spettrale di potenza} o \textsc{spettro di potenza}
\begin{equation}
	S_P(f)=\lim\limits_{T\to+\infty}{\frac{1}{2T}\abs{S_T(f)}^2}=\lim\limits_{T\to+\infty}{\frac{1}{2T}\;S_T(f)S_\conj{T}(f)}
\end{equation}
La densità spettrale di potenza gode di proprietà simili a quelle della densità spettrale di energia, ovvero è una funzione pari per segnali reali, è sempre non negativa e il suo integrale su tutte le frequenze restituisce la potenza complessiva del segnale.
Inoltre come per i segnali ad energia finita, il passaggio di un segnale a potenza finita attraverso un sistema lineare tempo invariante restituisce in uscita un segnale sempre a potenza finita, la cui densità spettrale risulterà $S_{P_y}(f)=S_{P_x}(f)\cdot\abs{H(f)}^2$

Antitrasformando la funzione densità spettrale di potenza si ottiene
\[
	\Fourier^{-1}\left\lbrace{S_P(f)}\right\rbrace=\lim\limits_{T\to+\infty}{\frac{1}{2T}\;s_T(\tau)\ast s_\conj{T}(-\tau)}=\lim\limits_{T\to+\infty}{\frac{1}{2T}\intd{-T}{T}{s_T(t)s_\conj{T}(t-\tau)}{t}}
\]
che è la funzione di \textsc{autocorrelazione}\index{segnale!autocorrelazione} per segnali di potenza finita
\begin{equation}
	R_s(\tau)=\lim\limits_{T\to+\infty}{\frac{1}{2T}\intd{-T}{T}{s(t)\conj{s}(t-\tau)}{t}}
\end{equation}

\chapter{Teoria della probabilità}

\section{Concetti di base}
Si definisce \textsc{spazio campione $\Omega$} l'insieme di tutti i possibili risultati di un esperimento o eventi. Un \textsc{evento} è un qualsiasi sotto insieme di possibili risultati di un esperimento, sottoinsieme di $\Omega$.

Dato $A$, un evento, si definisce il suo complemento $\bar{A}$, ovvero tutto lo spazio campione meno A, $\bar{A}=\Omega\setminus A$

$A\cup B$ (unione di A e B) è il sottoinsieme degli eventi che appartengono ad A o a B

$A\cup\bar{A}=\Omega$ l'unione di A e il complemento di A costituisce l'intero spazio campione

$A\cap B$ (intersezione di A e B) è il sottoinsieme degli eventi che appartengono sia ad A che a B

$A\cap\bar{A}=\emptyset$ l'intersezione tra eventi complementari è l'insieme vuoto.

\section{Spazio di probabilità}
Una definizione assiomatica di uno spazio di probabilità per \textsc{esperimenti aleatori} è caratterizzata da
\begin{itemize}
\item uno spazio campione $\Omega$
\item un insieme $S$ degli eventi di interesse (non necessariamente $\Omega$)
\item una legge di probabilità $P(.)$ secondo Kolmogorov
\end{itemize}

\section{Legge di probabilità}
Una \textsc{legge di probabilità} secondo Kolmogorov mette in corrispondenza un evento $A\in S$ con un numero reale tale che
\begin{equation}P(A)\geq 0\end{equation}
\begin{equation}P(\Omega)=1\end{equation}
eventi mutualmente esclusivi \begin{equation}A\cap B=0\implies P(A\cup B)=P(A)+P(B)\end{equation}

\section{Modelli alternativi}
Modelli matematici alternativi di probabilità sono la definizione di Laplace secondo cui, per uno spazio campione finito e un esperimento simmetrico (esempio: lancio di un dado non truccato), la probabilità è calcolata come numero di eventi favorevoli fratto il numero di eventi possibili
\begin{equation}
P(A)=\frac{N_F(A)}{N_P(A)}
\end{equation}

Definizione di Van Mises o \emph{frequentista} per cui ripetendo $N$
volte un esperimento e contando il numero $N_A$ di volte in cui si verifica l'evento A, si ha che la \textsc{frequenza relativa} $\frac{N_A}{N}$ si approssima al valore della probabilità di A
\begin{equation}P(A)=\lim\limits_{N\to\infty}\frac{N_A}{N}
\end{equation}

La definizione frequentista rientra nella teoria assiomatica infatti
$P(A)=\frac{N_A}{N}\geq 0$, se $A=\Omega\implies N_A=N\implies P(A)=1$, inoltre
e $A\cap B=\emptyset$ allora $N_{A\cup B}=N_A+N_B\implies P(A\cup B)= \lim\limits_{N\to\infty}\frac{N_{A\cup B}}{N}=\lim\limits_{N\to\infty}\frac{N_A+N_B}{N}=P(A)+P(B)$

\section{Teoremi}
A partire da tali assiomi si derivano i seguenti teoremi
\begin{equation}P(\bar{A})=1-P(A)\end{equation}

\begin{proof}[Dim.] $A\cup\bar{A}=\Omega\implies P(A\cup\bar{A})=1$

$A\cap\bar{A}=0\implies P(A\cup\bar{A})=P(A)+P(\bar{A})$
\end{proof}

\begin{equation}P(\emptyset)=0\end{equation}

\begin{proof}[Dim.] $\bar{\Omega}=\emptyset\implies P(\emptyset)=P(\bar{\Omega})=1-P(\Omega)=0$
\end{proof}

\begin{equation}0\leq P(A)\leq 1\end{equation}

\begin{proof}[Dim.] Per il primo assioma $P(A)\geq 0$ e $P(\bar{A})\geq 0$ ma è anche $P(A)=1-P(\bar{A})$
\end{proof}

\subsection{Teorema probabilità congiunte}\index{Teorema!probabilità congiunte}
\begin{equation}P(A\cup B)=P(A)+P(B)-P(A\cap B)\end{equation}

\begin{proof}[Dim.]
\[A\cup B=A\cup(B\cap\bar{A})\implies P(A\cup B)=P(A)+P(B\cap\bar{A})\]

\[B=B\cap(A\cup\bar{A})=(B\cap A)\cup(B\cap\bar{A})\implies P(B)=P(B\cap A)+P(B\cap\bar{A})\]
insieme $\implies P(A\cup B)=P(A)+P(B)-P(B\cap A)$
\end{proof}

\subsection{Teorema probabilità condizionata}\index{Teorema!probabilità condizionata}
$P(A)$ è la probabilità che si verifichi l'evento A a priori.
$P(A|B)$ è la probabilità che si verifichi l'evento A dato che si sia verificato l'evento B, la prob. di A a posteriori:
\begin{equation}
P(A|B)=\frac{P(A\cap B)}{P(B)}
\end{equation}

Nel caso particolare che A e B siano \textsc{eventi indipendenti}, la probabilità che si verifichi A non è influenzata da B, pertanto \[P(A|B)=P(A)\qquad P(A\cap B)=P(A)P(B)\]

\subsection{Teorema di Bayes}\index{Teorema!di Bayes}
\begin{equation}
P(A|B)=\frac{P(B|A)P(A)}{P(B)}
\end{equation}
che si può verificare facilmente sostituendo alla prob. condizionata $P(B|A)=\frac{P(A\cap B)}{P(A)}$

\subsection{Teorema delle Probabilità Totali (Bayes)}\index{Teorema!delle Probabilità Totali (Bayes)}
Data una partizione dello spazio campione $\bigcup_i B_i=\Omega, B_i\cap B_j=\emptyset \;\forall i\neq j$
\begin{equation}
P(A)=\sum_i P(A|B_i)P(B_i)
\end{equation}

\begin{proof}[Dim.]
Si dimostra facilmente notando che $A=\bigcup_i B_i \cap A=\bigcup_i (A\cap B_i) \implies P(A)=\sum_i P(A\cap B_i)=\sum_i P(A|B_i)P(B_i)$
\end{proof}

\section{Esperimento aleatorio composto}
Si considerano due esperimenti aleatori differenti caratterizzati ognuno dal proprio spazio campione. Si può considerare un esperimento composto in cui si osservano i due eventi $A_1\subseteq\Omega_1$ e $A_2\subseteq\Omega_2$, la coppia $A=A_1\times A_2$ nel nuovo spazio campione $\Omega_1\times\Omega_2$.

I due eventi possono essere indipendenti, in tal caso la probabilità dell'evento composto è pari al prodotto delle probabilità $P(A)=P(A_1)\cdot P(A_2)$.

Se i due esperimenti sono in qualche modo legati, si influenzano, sarà necessario invece valutare il grado di correlazione dei due eventi.

Le medesime considerazioni fatte per due esperimenti si possono fare per la composizione di $N$ qualsiasi esperimenti o prove ripetute ed indipendenti in un esperimento composto.

\subsection{Prove binarie di Bernoulli}
Esperimento composto da $n$ prove ripetute di esperimenti uguali tra loro e indipendenti, ciascuno con due soli possibili esiti $\omega_0$ e $\omega_1$ con probabilità $P(\omega_0)=p$ e $P(\omega_1)=1-p$.
Esempio classico il lancio di $n$ monete o il lancio ripetuto di una stessa moneta con risultato la sequenza di testa o croce.

Si definisce l'evento $A=\lbrace\omega_0$ si presenta $k$ volte in $n$ esperimenti$\rbrace$, $0\leq k\leq n$

La \textsc{formula binomiale di Bernoulli} esprime la probabilità dell'evento A:
\begin{equation}
P(A)=\binom{n}{k} p^k (1-p)^{n-k}
\end{equation}

Il coefficiente binomiale
\begin{equation}\binom{n}{k}=\frac{n!}{k!(n-k)!}\end{equation}
tiene conto delle possibili \textsc{combinazioni} in cui $k$ elementi possono essere disposti in $n$ posizioni non distinguendoli per l'ordine. Si ottiene dal numero di \textsc{disposizioni} possibili di $k$ oggetti in $n$ posizioni distinguendo le posizioni $D_{n,k}=n\cdot(n-1)\dots(n-k+1)=\frac{n!}{(n-k)!}$ diviso il numero delle \textsc{permutazioni} $P_{n,k}=k!$ di $k$ oggetti in $n$ posizioni.

\section{Esempi esperimenti aleatori}
\begin{esempio}
Esperimento di estrazione di palline bianche e nere da 5 urne di 3 tipi:
\begin{itemize}
	\item $A_1$: 2 urne con 2 palline bianche 1 nera
	\item $A_2$: 1 urna con 10 palline nere
	\item $A_3$: 2 urne con 3 palline bianche 1 nera
\end{itemize}
Evento $E_1=\lbrace$ pallina estratta da urna a caso è bianca $\rbrace$
\[E_1=(A_1\cap W)\cup(A_2\cap W)\cup(A_3\cap W)\]
Per il teorema delle probabilità totali
\[P(E_1)=P(A_1\cap W)+P(A_2\cap W)+P(A_3\cap W)=\frac{4}{15}+0+\frac{3}{10}\]
dove per il teorema delle probabilità condizionate
\[\begin{split}P(A_1\cap W)&=P(W|A_1) P(A_1)=\frac{2}{3}\cdot\frac{2}{5}=\frac{4}{15}\\
P(A_2\cap W)&=P(W|A_2) P(A_2)=0\cdot\frac{1}{5}=0\\
P(A_3\cap W)&=P(W|A_3) P(A_3)=\frac{3}{4}\cdot\frac{2}{5}=\frac{3}{10}\end{split}\]

Evento $E_2=\lbrace$ pallina estratta a caso sia bianca da urna di tipo 3 $\rbrace$
\[E_2=(A_1\cap W)\cup(A_2\cap W)\cup(A_3\cap W)\]

Per il teorema di Bayes la probabilità condizionata è
\[P(E_2)=P(A_3|W)=\frac{P(W|A_3)P(A_3)}{P(W)}=\frac{\frac{3}{4}\cdot\frac{2}{5}}{\frac{17}{30}}=\frac{3}{4}\cdot\frac{2}{5}\cdot\frac{30}{17}=\frac{9}{17}\cong 0,529\]
\end{esempio}

\begin{esempio}
Esperimento aleatorio composto
\begin{itemize}
\item 1 scatola con 2000 articoli di cui 5\% grandi
\item 1 scatola con 500 articoli di cui 40\% grandi
\item 1 scatola con 1000 articoli di cui 10\% grandi
\item 1 scatola con 1000 articoli di cui 10\% grandi
\end{itemize}
Si scelga una scatola a caso e si estragga un articolo a caso e si valuti la probabilità $P(G)$ che sia estratto un articolo grande.

Per il teorema delle probabilità totali ho
\[P(G)=\sum_{i=1}^{4}P(G|S_i)P(S_i)\]

$P(S_i)=\frac{1}{4}$, $P(G|S_1)=0.05$, $P(G|S_2)=0.4$, $P(G|S_3)=0.1$, $P(G|S_4)=0.1$

\[P(G)=\frac{1}{4}(0.05+0.4+0.1+0.1)=\frac{1}{4}\cdot 0.65=0.1625\]
Nel caso sia estratto un articolo grande, quale è la probabilità che sia dalla scatola 3?

Per il teorema di Bayes
\[P(S_3|G)=\frac{P(G|S_3)P(S_3)}{P(G)}=0.1\cdot\frac{1}{4}\cdot\frac{1}{0.1625}\cong 0.154\]
\end{esempio}

\begin{esempio}
Esperimento aleatorio composto

Ho 6 scatole di 3 tipi diversi ognuna con 12 palline alcune bianche alcune nere
\begin{itemize}
	\item $A_1$ 1 scatola con 7 palline bianche 5 nere
	\item $A_2$ 2 scatola con 5 palline bianche 7 nere
	\item $A_3$ 3 scatola con 4 palline bianche 8 nere
\end{itemize}
Ad ogni estrazione la pallina viene rimessa nella scatola.

Qual è la probabilità dell'evento $E=\lbrace 2B+1N \rbrace$ ?

Si può scomporre l'evento in eventi disgiunti ed applicare il teorema delle probabilità totali
\[P(E)=\sum_{i=1}^{3}P(E|A_i)P(A_i)\]
dove data la ripartizione delle scatole $P(A_1)=\frac{1}{6}, P(A_2)=\frac{2}{6}=\frac{1}{3}, P(A_3)=\frac{3}{6}=\frac{1}{2}$
Le probabilità condizionate si calcolano come prove ripetute di Bernoulli binarie indipendenti sull'evento E:
\[\begin{split}
P(E|A_1)&=\binom{3}{2}P(B|A_1)^2 P(N|A_1)=3\left(\frac{7}{12}\right)^2\frac{5}{12}=\frac{5}{4}\left(\frac{7}{12}\right)^2\\
P(E|A_2)&=\binom{3}{2}P(B|A_2)^2 P(N|A_2)=3\left(\frac{5}{12}\right)^2\frac{7}{12}=\frac{7}{4}\left(\frac{5}{12}\right)^2\\
P(E|A_3)&=\binom{3}{2}P(B|A_3)^2 P(N|A_3)=3\left(\frac{4}{12}\right)^2\frac{8}{12}=\frac{2}{9}
\end{split}\]

\[\begin{split}P(E)&=P(E|A_1)P(A_1)+P(E|A_2)P(A_2)+P(E|A_3)P(A_3)=\\
&=\frac{1}{6}\cdot\frac{5}{4}\left(\frac{7}{12}\right)^2+\frac{1}{3}\cdot\frac{7}{4}\left(\frac{5}{12}\right)^2+\frac{1}{2}\cdot\frac{2}{9}\cong 0.28\end{split}\]

Qual è la probabilità che si verifichi l'evento $E=\lbrace 2B+1N\rbrace$ da scatole del tipo $A_3$?
Per il teorema di Bayes
\[P(A_3|E)=\frac{P(E|A_3)P(A_3)}{P(E)}=\frac{2}{9}\cdot\frac{1}{2}\cdot\frac{1}{0.28}\cong 0,39\]
\end{esempio}

\chapter{Variabili aleatorie continue e discrete}
\label{ch:teoriasegnali-capitolo5}
\section{Variabili aleatorie}
Si consideri un esperimento aleatorio avente spazio campione $\Omega$, una classe degli eventi $S$ e una legge di probabilità $P(\cdot)$. Si definisce la corrispondenza $X(\omega_i)$ che associa a ciascun risultato $\omega_i\in\Omega$ un unico numero reale $a\in\R$.

Se l'insieme dei risultati di un esperimento aleatorio per i quali è verificata la disuguaglianza $X(\omega)\leq a$ è un evento, $\forall a\in\R$, si dice che $X$ è una \textsc{variabile aleatoria}\index{variabile aleatoria}.

Tale definizione di variabile aleatoria modella un esperimento che ha un risultato numerico non prevedibile a priori ed è derivato da una operazione di misura. Ripetendo l'esperimento si otterrà un insieme di valori che rappresenta la variabile aleatoria.\\

\begin{figure}[ht]
	\centering
	\subfloat[]{\begin{tikzpicture}[scale=.7]\begin{axis}[axis lines=middle,no markers,enlargelimits,xtick={0},ytick={0},xlabel=\(x\),ylabel=\(F_X(x)\)]
		\addplot [very thick] gnuplot [raw gnuplot] {
			% from http://tex.stackexchange.com/a/341886
			set samples 50;
			cdfn(x,mu,sd) = 0.5 * ( 1 + erf( (x-mu)/sd/sqrt(2)) );
			pdfn(x,mu,sd) = 1/(sd*sqrt(2*pi)) * exp( -(x-mu)**2 / (2*sd**2) );
			tpdfn(x,mu,sd,a,b) = pdfn(x,mu,sd) / ( cdfn(b,mu,sd) - cdfn(a,mu,sd) );
			plot [x=-15:9] tpdfn(x,9,10,-10,10)
		};
	\end{axis}\end{tikzpicture}} \qquad
	\subfloat[]{\begin{tikzpicture}[scale=.7]\begin{axis}[axis lines=middle,no markers,enlargelimits,xtick={0},ytick={0},xlabel=\(x\),ylabel=\(f_X(x)\)]
		\addplot [very thick,samples=200,domain=-3:6] {1/(1*sqrt(2*pi))*exp(-((x-2)^2)/(2*1^2))};
	\end{axis}\end{tikzpicture}} \\
	\subfloat[]{\begin{tikzpicture}[scale=.7]\begin{axis}[axis lines=middle,no markers,enlargelimits,xtick={0},ytick={0},xlabel=\(x\),ylabel=\(F_X(x)\)]
		\addplot [very thick] coordinates {(0,0) (1,0) (1,.1) (2,.1) (2,.2) (3,.2) (3,.4) (4,.4) (4,.7) (5,.7) (5,.8) (6,.8) (6,.9) (7,.9) (7,.95) (8,.95) (8,.97) (9,.97) (9,1) (10,1)};
	\end{axis}\end{tikzpicture}} \qquad
	\subfloat[]{\begin{tikzpicture}[scale=.7]\begin{axis}[axis lines=middle,no markers,enlargelimits,xtick={0},ytick={0},xlabel=\(x\),ylabel=\(f_X(x)\)]
		\addplot [very thick,->] coordinates {(0,0) (0,.02)};
		\addplot [very thick,->] coordinates {(1,0) (1,.07)};
		\addplot [very thick,->] coordinates {(2,0) (2,.15)};
		\addplot [very thick,->] coordinates {(3,0) (3,.18)};
		\addplot [very thick,->] coordinates {(4,0) (4,.19)};
		\addplot [very thick,->] coordinates {(5,0) (5,.16)};
		\addplot [very thick,->] coordinates {(6,0) (6,.12)};
		\addplot [very thick,->] coordinates {(7,0) (7,.06)};
		\addplot [very thick,->] coordinates {(8,0) (8,.04)};
		\addplot [very thick,->] coordinates {(9,0) (9,.02)};
		\addplot [very thick,->] coordinates {(10,0) (10,.01)};
	\end{axis}\end{tikzpicture}} \\
	\subfloat[]{\begin{tikzpicture}[scale=.7]\begin{axis}[axis lines=middle,no markers,enlargelimits,xtick={0},xlabel=\(x\),ylabel=\(F_X(x)\)]
		\addplot [very thick] gnuplot [raw gnuplot] {
			% from http://tex.stackexchange.com/a/341886
			set samples 50;
			cdfn(x,mu,sd) = 0.5 * ( 1 + erf( (x-mu)/sd/sqrt(2)) );
			pdfn(x,mu,sd) = 1/(sd*sqrt(2*pi)) * exp( -(x-mu)**2 / (2*sd**2) );
			tpdfn(x,mu,sd,a,b) = pdfn(x,mu,sd) / ( cdfn(b,mu,sd) - cdfn(a,mu,sd) );
			plot [x=-15:2] tpdfn(x,9,10,-10,10);
			replot [x=2:9] tpdfn(x,9,10,-10,10)+0.05;
		};
		\addplot [very thick] coordinates {(2,0.092) (2,0.1425)};
	\end{axis}\end{tikzpicture}} \qquad
	\subfloat[]{\begin{tikzpicture}[scale=.7]\begin{axis}[axis lines=middle,no markers,enlargelimits,xtick={0},ytick={0},xlabel=\(x\),ylabel=\(f_X(x)\)]
		\addplot [very thick,samples=200,domain=-3:6] {1/(1*sqrt(2*pi))*exp(-((x-2)^2)/(2*1^2))};
		\addplot [very thick,->] coordinates {(2,0) (2,.42)};
	\end{axis}\end{tikzpicture}}
	\caption{Funzioni distribuzione e densità di probabilità per una variabile aleatoria continua, discreta e mista.}
\end{figure}

\section{Funzione distribuzione di probabilità}
Si vuole ora stabilire come trasferire la legge di probabilità sugli eventi alle variabili aleatorie che vi associa numeri sull'asse reale.

Si definisce \textsc{funzione distribuzione di probabilità}\index{variabile aleatoria!funzione distribuzione di probabilità} di una variabile aleatoria $X$ la funzione che associa per un fissato valore reale $x$ il valore della probabilità dell'evento $\lbrace X\leq x\rbrace$, ovvero ne descrive il suo comportamento statistico:
\begin{equation}
\label{eq:funz_dist_prob}
	F_X(x)=P(X\leq x)
\end{equation}

\paragraph{Proprietà funzione distribuzione di probabilità}
\begin{enumerate}
\item $0\leq F_X(x)\leq 1$
\item $\lim\limits_{x\to+\infty}F_X(x)=1$
\item $\lim\limits_{x\to-\infty}F_X(x)=0$
\item $F_X$ è monotona non decrescente:
\[
	x_1<x_2\implies F_X(x_1)\leq F_X(x_2)
\]
\item $F_X$ è continua a destra:
\[
	\lim\limits_{h\to 0^+}{F_X(x+h)}=F_X(x)
\]
\item se $F_X$ presenta discontinuità di prima specie in $\bar{x}$ la probabilità dell'evento $X=\bar{x}$ è pari alla differenza tra limite destro e sinistro
\[
	P(X=\bar{x})=\lim\limits_{h\to 0^+}{F_X(x+h)} - \lim\limits_{h\to 0^-}{F_X(x+h)}
\]
\item la probabilità dell'evento $a < X \leq b$:
\[
	P(a<X\leq b)=F_X(b)-F_X(a)
\]
\end{enumerate}

Le variabili aleatorie possono essere \emph{continue}, \emph{discrete} o \emph{miste} in base alla continuità della funzione distribuzione di probabilità $F_X(x)$.

Per una variabile aleatoria discreta la funzione di distribuzione di probabilità è continua a tratti: $F_X(x)=\sum_k P(X=x_k)\step{(x-x_k)}$, infatti la probabilità è concentrata in “masse” di probabilità $p_k=P(X=x_k)$ in corrispondenza dei punti di discontinuità $x_k$.

Una variabile aleatoria continua ha la funzione distribuzione di probabilità continua, ovvero la probabilità che la variabile aleatoria assuma un particolare valore $P(X=\bar{x})$ è infinitesima.

\section{Funzione densità di probabilità}
Si può dare una definizione alternativa di variabile aleatoria legata alla \textsc{funzione densità di probabilità}\index{variabile aleatoria!funzione densità di probabilità}
\begin{equation}
\label{eq:funz_dens_prob}
	f_X(x)=\deriv{F_X(x)}{x}
\end{equation}
dove la funzione di distribuzione di probabilità
\begin{equation}
	F_X(x)=\intd{-\infty}{x}{f_X(\alpha)}{\alpha}
\end{equation}

\paragraph{Proprietà funzione densità di probabilità}
\begin{enumerate}
\item $f_X(x)\geq 0$ perché $F_X(x)$ è monotona crescente
\item $P(a<X\leq b)=F_X(b)-F_X(a)=\intd{a}{b}{f_X(x)}{x}$
\item $\lim\limits_{x\to\pm\infty}f_X(x)=0$
\item $\intinf{f_X(x)}{x}=1$ in quanto rappresenta la probabilità di un evento certo
\item $P(x<X\leq x+\Delta x)=\intd{x}{x+\Delta x}{f_X(\alpha)}{\alpha}\cong f_X(x)\Delta x$
\end{enumerate}

Per una variabile aleatoria discreta la densità di probabilità è concentrata in impulsi
\begin{equation}
	f_X(x)=\sum_k P(X=x_k) \delta(x-x_k)
\end{equation}

Se la variabile aleatoria è continua la densità di probabilità è distribuita con continuità.
Se la variabile aleatoria è mista coesistono le due condizioni.

\section{Operazioni su variabili aleatorie}
Data una variabile aleatoria $X$ è possibile ottenere una nuova variabile aleatoria che apporti trasformazioni a $X$: $Y=g(X)$. Essa ha funzione distribuzione di probabilità
\begin{equation}
	F_Y(y)=P(Y\leq y)=P(g(X)\leq y)
\end{equation}
Nel dominio $D_Y=\left\lbrace x\mid g(x)\leq y\right\rbrace$ è possibile calcolare la funzione distribuzione di probabilità $F_Y(y)=\int_{D_Y}{f_X(x)\diff x}$ e la funzione densità di probabilità $f_Y(y)=\deriv{F_Y(y)}{y}$

Nel caso in cui la funzione $g(\cdot)$ sia monotona strettamente crescente o strettamente decrescente è possibile definire la sua inversa $g^{-1}(\cdot)$. Di conseguenza nel caso di $g(\cdot)$ strettamente crescente
\[
	F_Y(y)=P(Y\leq y)=P(g(X)\leq y)=P(X\leq g^{-1}(y))=F_X(g^{-1}(y))\implies
\]
\begin{equation}
	f_Y(y)=f_X(g^{-1}(y))\cdot\deriv{g^{-1}(y)}{y}=\restrict{\frac{f_X(x)}{g'(x)}}{x=g^{-1}(y)}
\end{equation}
Nel caso di funzione $g(\cdot)$ strettamente decrescente $g(x)\leq y\iff x>g^{-1}(y)$
\[F_Y(y)=P(Y\leq y)=P(g(X)\leq y)=P(X>g^{-1}(y))=1-F_X(g^{-1}(y))\implies
\]
\begin{equation}
	f_Y(y)=f_X(g^{-1}(y))\cdot-\deriv{g^{-1}(y)}{y}=\restrict{-\frac{f_X(x)}{g'(x)}}{x=g^{-1}(y)}
\end{equation}
In generale
\begin{equation}
	F_Y(y)=\int\limits_{D_Y}{\frac{f_X(x)}{\abs{g'(x)}}\diff x}
\end{equation}
Si osservi l'integrale per $g'(x)=0$ e anche per $f_X(x)=0$ quando sono costanti sia $F_X(x)$ che $g(x)$.

\section{Indici statistici di variabile aleatoria}
Il massimo di informazione che si può trarre da un esperimento aleatorio è la funzione densità di probabilità che caratterizza la variabile aleatoria.
Quando non si conosce questa funzione è comunque possibile determinare dei parametri statistici che permettono di descriverne alcune proprietà.\\

Il più importante parametro statistico è il \textsc{valor medio}\index{valor medio} $\mu_X$ di v.a. continua $X$:
\begin{equation}
\label{eq:v_a_media}
	\mu_X=\intinf{x f_X(x)}{x}
\end{equation}
Per variabile aleatoria discreta si ha $f_X(x)=\sum_k p_k\delta(x-x_k)$, il valor medio
\begin{equation}
\label{eq:v_a_discreta_media}
	\mu_X=\intinf{x f_X(x)}{x}=\sum_k p_k \intinf{x \delta(x-x_k)}{x}=\sum_k x_k p_k
\end{equation}

\`E possibile calcolare il valor medio $\mu_Y$ di funzione trasformata di v.a. $Y=g(X)$ introducendo l'\textsc{operatore aspettazione}\index{operatore!aspettazione}
\begin{equation}
	\E{g(X)}=\intinf{g(x)f_X(x)}{x}
\end{equation}
che nel caso della media assume la semplice relazione $\mu_X=\E{X}$.

L'operatore valor medio gode della proprietà di linearità, data dall'integrale, per cui
\begin{equation}
	\E{a\cdot g(X)+b\cdot h(X)}=a \E{g(X)}+b \E{h(X)}
\end{equation}

Per variabile aleatoria $Y=g(X)$ ottenuta da trasformazione si ha la notevole semplificazione di non dover calcolare la funzione densità di probabilità $f_Y$ nota $f_X$.
Per il \textsc{teorema del valor medio} si può calcolare direttamente il valor medio
\begin{equation}
\label{eq:teo_valor_medio}
	\mu_Y=\E{Y}=\E{g(X)}=\intinf{g(x) f_X(x)}{x}
\end{equation}

Due variabili aleatorie possono avere lo stesso valor medio pur avendo distribuzioni di probabilità differenti. Si quantifica questo indice statistico introducendo il parametro \textsc{varianza}\index{varianza} di variabile aleatoria continua definita come
\begin{equation}
\label{eq:v_a_varianza}
	\sigma^2_X=\E{(X-\mu_X)^2}=\intinf{(x-\mu_X)^2 f_X(x)}{x}
\end{equation}

La varianza di variabile aleatoria discreta
\begin{equation}
\label{eq:v_a_varianza_discreta}
	\sigma^2_X=\E{(X-\mu_X)^2}=\sum_k {(x-\mu_X)^2 p_k}
\end{equation}

Una misura di quanto sia dispersa la probabilità attorno alla media è data dalla \textsc{deviazione standard}\index{deviazione standard} $\sigma_X$, la radice quadrata della varianza.\\

Si definisce \textsc{momento di ordine $k$} l'aspettazione della potenza $k$ di v.a.
\begin{equation}
	\E{X^k}=\intinf{x^k f_X(x)}{x}
\end{equation}

Nel caso particolare $k=2$ si ha l'indice \textsc{valore quadratico medio}\index{valore quadratico medio} o \textsc{potenza statistica} di v.a.
\begin{equation}
\label{eq:v_a_valore_quad_medio}
	m^2_X=\E{X^2}=\intinf{x^2 f_X(x)}{x}
\end{equation}
\begin{nota}
	Il valor quadratico medio è associato alla potenza del segnale, un parametro importante per il dimensionamento di un sistema di telecomunicazioni.
\end{nota}
Varianza e potenza sono legate dalla relazione
\begin{equation}
	\begin{split}
		\sigma^2_X&=\E{(X-\mu_X)^2}=\E{X^2-2X\mu_X+\mu^2_X}=\\
		&=\E{X^2}-2\E{X}\mu_X+\mu^2_X=m^2_X-2\mu^2_X+\mu^2_X=\\
		&=m^2_X-\mu^2_X
	\end{split}
\end{equation}
quindi $\E{X^2}=m^2_X=\sigma^2_X+\mu^2_X$

\section{Funzione generatrice dei momenti}
Per il calcolo degli indici statistici di variabili aleatorie più complesse ci si avvale della \textsc{funzione generatrice dei momenti}\index{funzione generatrice dei momenti}, definita come aspettazione della variabile aleatoria trasformata $\e{t X}$:
\begin{equation}
	\begin{split}
		G_X(t)&=\E{\e{t X}}=\intinf{\e{t x}f_X(x)}{x}=\\
\intertext{sviluppando in serie di Taylor l'esponenziale}
		&=\E{1+x t+\frac{(x t)^2}{2}+\frac{(x t)^3}{3!}+\dots+\frac{(x t)^k}{k!}}=\sum_{k=0}^{+\infty}{\frac{\E{x^k}}{k!}t^k}
	\end{split}
\end{equation}

Si definiscono i momenti di ordine $k$ come derivata della funzione generatrice calcolata in $t=0$.

\begin{description}
\item[Momento del primo ordine (valor medio)]
\begin{equation}
	\restrict{\deriv{G_X(t)}{t}}{t=0}=\E{X}=\mu_X
\end{equation}
\item[Momento del secondo ordine (potenza)]
\begin{equation}
	\restrict{\deriv[2]{G_X(t)}{t}}{t=0}=\E{X^2}
\end{equation}
\item[Momendo di ordine $k$]
\begin{equation}
	\restrict{\deriv[k]{G_X(t)}{t}}{t=0}=\E{X^k}
\end{equation}
\end{description}

\section{Variabile aleatoria uniforme}
La \textsc{variabile aleatoria uniforme}\index{variabile aleatoria!uniforme} presenta una densità di probabilità costante nell'intervallo $[a,b]$ e zero altrove. Dovendo sottendere un'area unitaria si ha un rettangolo di altezza costante $1/(b-a)$, base $(b-a)$ centrato in $(b+a)/2$
\begin{equation}
	f_X(x)=\begin{cases}
		0&x\leq a\\ 
		\frac{1}{b-a}&a<x\leq b\\
		0&x>b
	\end{cases}
	\quad
	f_X(x)=\frac{1}{b-a}\rect{\frac{x-\frac{b+a}{2}}{b-a}}
\end{equation}
La funzione distribuzione essendone l'integrale ha la definizione di una rampa
\begin{equation}
	F_X(x)=\begin{cases}
		0&x\leq a\\
		\frac{x-a}{b-a}&a<x\leq b\\
		1&x>b
	\end{cases}
\end{equation}
\begin{figure}[!ht]
	\centering
	\subfloat[][$f_X(x)$]{
		\begin{tikzpicture}[scale=.7]
			\begin{axis}[axis lines=middle,no markers,enlargelimits,xscale=1.2,xtick={1,2,3},xticklabels={$a$,$\frac{b+a}{2}$,$b$},ytick={2},yticklabels={$\frac{1}{b-a}$},xlabel={$x$}]
			\addplot [very thick] coordinates{(-1,0)(1,0)(1,2)(3,2)(3,0)(4,0)};
			\end{axis}
		\end{tikzpicture}}\qquad
	\subfloat[][$F_X(x)$]{
		\begin{tikzpicture}[scale=.7]
			\begin{axis}[axis lines=middle,no markers,enlargelimits,xscale=1.2,xtick={1,2,3},xticklabels={$a$,$\frac{b+a}{2}$,$b$},ytick={1},xlabel={$x$}]
			\addplot [very thick] coordinates{(-1,0)(1,0)(3,1)(4,1)};
			\end{axis}
		\end{tikzpicture}}
	\caption{Variabile aleatoria uniforme}
\end{figure}

\begin{description}
\item[Valor medio]
\[\mu_X=\intd{a}{b}{\frac{1}{b-a}x}{x}=\frac{1}{b-a}\bound{a}{b}{\frac{x^2}{2}}=\frac{b+a}{2}
\]
\item[Valor quadratico medio] 
\[\E{X^2}=\intd{a}{b}{\frac{1}{b-a}x^2}{x}=\frac{1}{b-a}\bound{a}{b}{\frac{x^3}{3}}=\frac{1}{3}\frac{b^3-a^3}{b-a}=\frac{b^2+a b+a^2}{3}
\]
\item[Varianza]
\[\sigma^2_X=\E{X^2}-\mu^2_X=\frac{b^2+a b+a^2}{3}-\left(\frac{b+a}{2}\right)^2=\frac{(b-a)^2}{12}
\]
\item[Deviazione standard]
\[\sigma_X=\frac{b-a}{2\sqrt{3}}
\]
\item[Deviazione standard normalizzata]
\[\frac{\sigma_X}{\mu_X}=\frac{\frac{b-a}{2\sqrt{3}}}{\frac{b+a}{2}}=\frac{b-a}{b+a}\frac{1}{\sqrt{3}}
\]
\end{description}

\section{Variabile aleatoria esponenziale}
La \index{variabile aleatoria!esponenziale}\textsc{variabile aleatoria continua esponenziale} unilatera con parametro $\lambda>0$ ha una funzione densità di probabilità che parte con una discontinuità in $x=0$ e decresce esponenzialmente a zero. Utile a modellare i tempi di intercorrenza tra richieste di un servizio, problemi di affidabilità e calcolo del rischio. La variabile aleatoria esponenziale è una variabile aleatoria a coda lunga, dove la coda indica eventi rari trascurabili, caratteristica che aiuta a dimensionare i sistemi.
\begin{equation}
	f_X(x)=\lambda\e{-\lambda x}\step(x)
\end{equation}
\begin{equation}
	F_X(x)=\intd{-\infty}{x}{f_X(\alpha)}{\alpha}=\intd{-\infty}{x}{\lambda\e{-\lambda \alpha}}{\alpha}=1-\e{-\lambda x}
\end{equation}
\begin{figure}[!ht]
	\centering
	\subfloat[][$f_X(x)=\lambda\e{-\lambda x}$]{
		\begin{tikzpicture}[scale=.66]
			\begin{axis}[axis lines=middle,no markers,enlargelimits,xtick={0},ytick={0,2},yticklabels={0,$\lambda$},ylabel={$f_X(x)$}]
			\addplot [very thick,domain=-1:3,samples=200] {x>0?2*exp(-2*x):0};
			\end{axis}
		\end{tikzpicture}}\qquad
	\subfloat[][$F_X(x)=1-\e{-\lambda x}$]{
		\begin{tikzpicture}[scale=.66]
			\begin{axis}[axis lines=middle,no markers,enlargelimits,xtick={0},ytick={0,1},ylabel={$F_X(x)$}]
			\addplot [very thick,domain=-1:3,samples=100] {x>0?1-exp(-2*x):0};
			\end{axis}
		\end{tikzpicture}
	}
	\caption{Variabile aleatoria esponenziale}
\end{figure}

\begin{flushleft}
Calcolo degli indici statistici:
\end{flushleft}
\begin{description}
\item[Valor medio]
\begin{equation}
\begin{split}
	\mu_X&=\intinf{x f_X(x)}{x}=\intd{0}{+\infty}{x \lambda\e{-\lambda x}}{x}=\\
\intertext{integrando per parti $\int{u \diff v}=u v-\int{v\diff u}, u=\lambda x$, $\diff u=\lambda\diff x$, $\diff v=\e{-\lambda x}\diff x$, $v=-\frac{1}{\lambda}\e{-\lambda x}$}
	&=\bound{0}{+\infty}{-x\e{-\lambda x}}+\intd{0}{+\infty}{\e{-\lambda x}}{x}=\frac{1}{\lambda} 	\qquad\text{essendo} \intd{0}{+\infty}{\lambda\e{-\lambda x}}{x}=1
\end{split}
\end{equation}
\item[Potenza]
\begin{equation}
\begin{split}
	\E{X^2}&=\intinf{x^2 f_X(x)}{x}=\intd{0}{+\infty}{\lambda x^2\e{-\lambda x}}{x}=\\
\intertext{integrando per parti $u=x^2$, $\diff u=2x\diff x$, $\diff v=\lambda\e{-\lambda x}$, $v=-\e{-\lambda x}$}
	&=\bound{0}{+\infty}{-x^2\e{-\lambda x}}+\frac{2}{\lambda}\underbrace{\intd{0}{+\infty}{x \lambda\e{-\lambda x}}{x}}_{{\mu_X}={\frac{1}{\lambda}}}=\frac{2}{\lambda^2}
\end{split}
\end{equation}
\item[Varianza]
\[
	\sigma^2_X=\E{X^2}-\mu^2_X=\frac{2}{\lambda^2}-\frac{1}{\lambda^2}=\frac{1}{\lambda^2}
\]
\item[Deviazione standard]
\[
	\sigma_X=\frac{1}{\lambda}=\mu_X
\]
\item[Deviazione standard normalizzata dalla media] \[
	\frac{\sigma_X}{\mu_X}=1
\]
\end{description}

\section{Variabile aleatoria discreta di Poisson}
La \index{variabile aleatoria!di Poisson}\textsc{variabile aleatoria discreta di Poisson} modella le richieste di servizio o il conteggio per arrivi indipendenti in un intervallo di tempo $T$ e con un parametro di intensità $\Lambda>0$ in istanti di tempo discreti (valori interi non negativi). La funzione densità di probabilità (di massa) $f_X(x)=\sum_k p_k \delta(x-k)$ ha espressione
\begin{equation}
	f_X(x)=\sum_{k=0}^{+\infty}{\frac{(\Lambda T)^k}{k!}\e{-\Lambda T}\delta(x-k)}
\end{equation}
La funzione distribuzione di probabilità (di massa) $F_X(x)=\sum_k p_k \step(x-k)$ ha espressione
\begin{equation}
	F_X(x)=\sum_{k=0}^{+\infty}{\frac{(\Lambda T)^k}{k!}\e{-\Lambda T}\step(x-k)}
\end{equation}
con masse di probabilità $p_k=\frac{(\Lambda T)^k}{k!}\e{-\Lambda T}$.

\begin{flushleft}
Calcolo degli indici statistici:
\end{flushleft}
\begin{description}
\item[Valor medio]
\begin{equation}
\begin{split}
	\mu_X&=\sum_{k=0}^{+\infty}{k p_k}=\sum_{k=0}^{+\infty}{k \frac{(\Lambda T)^k}{k!}\e{-\Lambda T}}=\sum_{k=1}^{+\infty}{\frac{(\Lambda T)^k}{(k-1)!}\e{-\Lambda T}}=\\
	&=\Lambda T \e{-\Lambda T} \sum_{k=1}^{+\infty}{\frac{(\Lambda T)^{k-1}}{(k-1)!}}=\Lambda T \e{-\Lambda T}\sum_{z=0}^{+\infty}{\frac{(\Lambda T)^z}{z!}}=\Lambda T \e{-\Lambda T}\e{\Lambda T}=\\
	&=\Lambda T
\end{split}
\end{equation}
\item[Potenza]
\begin{equation}
\begin{split}
	\E{X^2}&=\sum_{k=0}^{+\infty}{k^2 p_k}=\sum_{k=0}^{+\infty}{k^2 \frac{(\Lambda T)^k}{k!}\e{-\Lambda T}}=\e{-\Lambda T}\sum_{k=0}^{+\infty}{k\frac{(\Lambda T)^k}{(k-1)!}}=\\
	&=\e{-\Lambda T}\sum_{k=1}^{+\infty}{k\frac{(\Lambda T)^{z-1}}{(k-1)!}\Lambda T}=
	\e{-\Lambda T}\sum_{k=1}^{+\infty}{(k-1+1)\frac{\Lambda T(\Lambda T)^{z-1}}{(k-1)!}}=\\
	&=\e{-\Lambda T}\left[\sum_{k=1}^{+\infty}{(k-1)\frac{\Lambda T(\Lambda T)^{z-1}}{(k-1)!}} + \sum_{k=1}^{+\infty}{\frac{\Lambda T(\Lambda T)^{z-1}}{(k-1)!}}\right]=\\
	&=\e{-\Lambda T}\left[\sum_{z=0}^{+\infty}{z\frac{\Lambda T(\Lambda T)^z}{z!}} + \Lambda T\e{\Lambda T}\right]=\\
	&=\e{-\Lambda T}\left[\sum_{z=1}^{+\infty}{\frac{\Lambda T(\Lambda T)^{z-1}}{(z-1)!}} + \Lambda T\e{\Lambda T}\right]=\\
	&=\e{-\Lambda T}\left[(\Lambda T)^2 \e{\Lambda T} + \Lambda T\e{\Lambda T}\right]=\\
	&=(\Lambda T)^2 + \Lambda T
\end{split}
\end{equation}
\end{description}

Gli stessi indici possono essere calcolati più agevolmente con la funzione generatrice dei momenti:
\begin{equation}
	G_X(t)=\sum_{k=0}^{+\infty}{\e{t x}p_k\delta(x-k)}=\sum_{k=0}^{+\infty}{\e{t k}\frac{(\Lambda T)^k}{k!}\e{-\Lambda T}}=\sum_{k=0}^{+\infty}{\frac{(\e{t}\Lambda T)^k}{k!}\e{-\Lambda T}}=\e{-\Lambda T}\e{\Lambda T\e{t}}
\end{equation}

\begin{description}
\item[Valor medio (momento del primo ordine)]
\begin{equation}
\begin{split}
	\mu_X&=\restrict{\deriv{G_X(t)}{t}}{t=0}=\restrict{\deriv{\e{-\Lambda T}\e{\Lambda T\e{t}}}{t}}{t=0}=\restrict{\e{-\Lambda T}{\Lambda T}\e{t}\e{\Lambda T\e{t}}}{t=0}=\Lambda T \e{-\Lambda T}\e{\Lambda T}=\\
	&=\Lambda T
\end{split}
\end{equation}

\item[Potenza (momento del secondo ordine)]
\begin{equation}
\begin{split}
	\E{X^2}&=\restrict{\deriv[2]{G_X(t)}{t}}{t=0}=
	\restrict{\deriv{\e{-\Lambda T}{\Lambda T}\e{t}\e{\e{t}\Lambda T}}{t}}{t=0}=
	\restrict{\e{-\Lambda T}{\Lambda T}\deriv{\e{t}\e{\e{t}\Lambda T}}{t}}{t=0}=\\
	&={\e{-\Lambda T}\Lambda T}\restrict{\left[\e{t}\e{\e{t}\Lambda T}+\e{t}\e{\e{t}\Lambda T}{\e{t}\Lambda T}\right]}{t=0}=\\
	&={\e{-\Lambda T}\Lambda T}[\e{\Lambda T}+\e{\Lambda T}{\Lambda T}]=\\
	&=(\Lambda T)^2+\Lambda T
\end{split}
\end{equation}
\end{description}

\section{Variabile aleatoria binomiale o di Bernoulli}
Variabile aleatoria discreta applicata ad esperimenti ripetuti di \index{variabile aleatoria!di Bernoulli}\textsc{Bernoulli}, indipendenti tra loro, che possono dare due soli possibili risultati, successo con probabilità $p$ e insuccesso con probabilità $1-p$. La variabile binomiale conta il numero di successi.
La funzione densità di probabilità (di massa) $f_X(x)=\sum_k p_k \impulse(x-k)$ e la funzione distribuzione di probabilità (di massa) $F_X(x)=\sum_k p_k \step(x-k)$ hanno massa di probabilità data dalla formula binomiale di Bernoulli:
\begin{equation}
	P(X=k)=\binom{n}{k} p^k (1-p)^{n-k}\quad k=0,\dots,n
\end{equation}
Funzione generatrice dei momenti
\begin{equation}
\begin{split}
	G_X(t)=\E{\e{t X}}&=\sum_{k=0}^{n}{\e{t k}\binom{n}{k}p^k(1-p)^{n-k}}=\\
	&=\sum_{k=0}^{n}{\binom{n}{k}(p\e{t})^k(1-p)^{n-k}}=\\
	&=(p\e{t}+1-p)^n
	\end{split}
\end{equation}
\begin{flushleft}
Calcolo degli indici statistici utilizzando la funzione generatrice dei momenti:
\end{flushleft}
\begin{description}
\item[Valor medio]
\begin{equation}
	\mu_X=\restrict{\deriv{G_X(t)}{t}}{t=0}=\restrict{n(p\e{t}+1-p)^{n-1}p\e{t}}{t=0}=n p
\end{equation}
\item[Potenza]
\begin{equation}
\begin{split}
	\E{X^2}&=\restrict{\deriv[2]{G_X(t)}{t}}{t=0}=\restrict{\deriv{np\e{t}(p\e{t}+1-p)^{n-1}}{t}}{t=0}=\\
	&=\restrict{np\e{t}(p\e{t}+1-p)^{n-1}+np\e{t}(n-1)(p\e{t}+1-p)^{n-2}p\e{t}}{t=0}=\\
	&=np+np^2(n-1)=\\
	&=np[p(n-1)+1]
\end{split}
\end{equation}
\item[Varianza]
\begin{equation}
\begin{split}
	\sigma^2_X&=\E{X^2}-\Esp^2[X]=np[p(n-1)+1]-n^2p^2=\\
	&=np+n^2p^2-np^2-n^2p^2=\\
	&=np(1-p)
\end{split}
\end{equation}
\end{description}

\section{Variabile aleatoria geometrica}
La \index{variabile aleatoria!geometrica}\textsc{variabile aleatoria geometrica} è una variabile aleatoria discreta che consente di calcolare la probabilità di ottenere il primo insuccesso dopo aver ottenuto $k$ successi consecutivi quando si effettuano degli esperimenti indipendenti e ripetuti di Bernoulli.
La variabile aleatoria geometrica ha pertanto masse di probabilità
\begin{equation}
	P(X=k)=p^k(1-p)\quad k=0,\dots,\infty
\end{equation}
La funzione generatrice dei momenti
\begin{equation}
\begin{split}
	G_X(t)=\E{\e{t X}}&=\sum_{k=0}^{\infty}{\e{t k}p^k(1-p)}=(1-p)\sum_{k=0}^{\infty}{(p\e{t})^k}=\\
	&=\frac{1-p}{1-p\e{t}}=(1-p)(1-p\e{t})^{-1}
\end{split}
\end{equation}
dove la serie geometrica $\sum_{k=0}^{\infty}{(p\e{t})^k}$ converge in un intorno di $t=0$ dove calcoliamo i momenti.

\begin{description}
\item[Valor medio]
\begin{equation}
	\mu_X=\restrict{\deriv{G_X(t)}{t}}{t=0}=\restrict{(1-p)(-1)(1-p\e{t})^{-2}(-p\e{t})}{t=0}=\frac{p}{1-p}
\end{equation}
\item[Potenza]
\begin{equation}
\begin{split}
	\E{X^2}&=\restrict{\deriv[2]{G_X(t)}{t}}{t=0}=\restrict{\deriv{p(1-p)\e{t}(1-p\e{t})^{-2}}{t}}{t=0}=\\
	&=p(1-p)\restrict{[\e{t}(1-p\e{t})^{-2}+\e{t}2(1-p\e{t})^{-3}p]}{t=0}=\\
	&=p(1-p)[(1-p)^{-2}+2p(1-p)^{-3}]=\\
	&=p(1-p)\frac{(1-p)+2p}{(1-p)^3}=\\
	&=\frac{p(1+p)}{(1-p)^2}
\end{split}
\end{equation}
\item[Varianza]
\begin{equation}
	\sigma^2_X=\E{X^2}-\Esp^2[X]=\frac{p(1+p)}{(1-p)^2}-\frac{p^2}{(1-p)^2}=\frac{p}{(1-p)^2}
\end{equation}
\item[Deviazione standard]
\begin{equation}
	\sigma_X=\frac{\sqrt{p}}{1-p}
\end{equation}
\end{description}

\section{Derivazione e significato delle variabili aleatorie esponenziale e di Poisson}
La variabile aleatoria esponenziale e quella discreta di \index{variabile aleatoria!di Poisson}\textsc{Poisson} modellano esperimenti di richieste di servizio o attesa di eventi indipendenti tra loro. Nell'intervallo di tempo fissato $T$ si vuole calcolare la probabilità che si verifichino un certo numero di eventi, quale sia il tempo di interarrivo, il tempo che intercorre tra due eventi successivi, e il tempo di attesa affinché si verifichi il primo evento a partire da un istante iniziale di riferimento.

\`E possibile derivare la distribuzione di probabilità della v.a. di Poisson studiando il fenomeno sotto alcune ipotesi.

Fissato il periodo $T$ lo si suddivide in $n$ intervallini di durata $\delta T=\frac{T}{n}$. Si prova a descrivere come v.a. di Bernoulli la possibilità che possano verificarsi $k$ eventi nell'intervallo finito $T$, al più un evento per ogni singolo intervallino $\delta T$.
\begin{equation}
\begin{cases}
	P(N(\delta T)=1)&=p\\
	P(N(\delta T)=0)&=1-p
\end{cases}
\end{equation}
\begin{equation}
	P(N(T)=k)=\binom{n}{k}p^k(1-p)^{n-k}
\end{equation}
Detta $\Lambda$ l'intensità del processo di Poisson, si ha che $\Lambda T=n p=\alpha$ è il numero medio di arrivi nell'intervallo di tempo $T$. Tale intensità è caratteristica del processo e non dipende dal numero di intervallini in cui si suddivide $T$. Portando al limite il numero di intervallini, $n\to\infty$ si ha

\begin{equation}
\begin{split}
	P(N(T)=k)&=\binom{n}{k}p^k(1-p)^{n-k}=\lim\limits_{n\to\infty}\binom{n}{k}\left(\frac{\alpha}{n}\right)^k\left(1-\frac{\alpha}{n}\right)^{n-k}=\\
	&=\lim\limits_{n\to\infty}\frac{n!}{k!(n-k)!}\left(\frac{\alpha}{n}\right)^k\left(1-\frac{\alpha}{n}\right)^{n-k}=\\
	&=\frac{\alpha^k}{k!}\lim\limits_{n\to\infty}\underbrace{\frac{n!}{n^k(n-k)!}}_{\to 1}\underbrace{\left(1-\frac{\alpha}{n}\right)^n}_{\to\e{-\alpha}} \underbrace{\left(1-\frac{\alpha}{n}\right)^{-k}}_{\to 1}=\\
	&=\frac{\alpha^k}{k!}\e{-\alpha}=\frac{(\Lambda T)^k}{k!}\e{-\Lambda T}
	\end{split}
\end{equation}

Con $T=1$ si ha la probabilità che nell'unità di tempo si verifichino $k$ eventi.

La probabilità che non capitino eventi vale $P(N(1)=0)=\e{-\Lambda}$.

Si vuole calcolare ora il \emph{tempo di attesa}, ovvero il tempo che bisogna attendere fino al primo evento a partire dall'istante iniziale di osservazione. La distribuzione di probabilità della variabile aleatoria tempo di attesa può essere espressa come probabilità che non sia capitato alcun evento fino al tempo $x$ ovvero $P(\tau>x)=\e{-\Lambda x}$
\begin{equation}
	F_\tau(x)=P(\tau\leq x)=1-P(\tau\geq x)=1-\e{-\Lambda x}
\end{equation}
\begin{equation}
	f_\tau(x)=\Lambda\e{-\Lambda x}
\end{equation}
che risulta essere una funzione di distribuzione di una variabile aleatoria esponenziale.

Si può vedere infatti che se fino ad un tempo $\tau$ non ho eventi il tempo di attesa per il successivo evento è sempre esponenziale, con lo stesso parametro $\Lambda$ (processo fisico privo di memoria):
\[
	P(X>\tau+t_0|X>\tau)=\frac{P(X>\tau+t_0,X>\tau)}{P(X>\tau)}=\frac{P(X>\tau+t_0)}{P(X>\tau)}=\frac{\e{-\Lambda(\tau+t_0)}}{\e{-\Lambda\tau}}=\e{-\Lambda t_0}
\]

Similmente a partire da un istante in cui è capitato un evento si voglia determinare qual è la probabilità che sia $\tau$ il tempo da attendere per l'evento successivo. Per eventi indipendenti il tempo di interarrivo ha distribuzione e densità di probabilità uguali a quelle calcolate per il tempo di attesa, ovvero di una variabile aleatoria esponenziale con parametro $\Lambda$ caratteristica di un sistema privo di memoria.

\begin{nota}
	Nei sistemi ingegneristici la v.a. esponenziale modella la “coda lunga” che tiene conto della probabilità di eventi rari che possono avvenire in un tempo infinito. Nella pratica si utilizza una variabile aleatoria esponenziale troncata, in cui si corregge con un fattore moltiplicativo la funzione densità per ottenere un'area unitaria.
\end{nota}

\begin{esempio}
Il circuito elettrico $RC$ alimentato dal generatore di tensione $V_i$ a partire dall'istante $t=0$ modella un resistore che ha un tempo di guasto aleatorio descritto da variabile aleatoria esponenziale con funzione densità di probabilità esponenziale monolatera
\[
	f_X(x)=\frac{1}{2\alpha}\e{-\frac{x}{2\alpha}}\step(x)\quad\alpha=RC
\]
\begin{figure*}[h]
	\centering
	\begin{circuitikz}[american voltages]
		\draw (0,0)	to[battery,v^>=${V_i}$] (0,3)
		to[switch,l=${t=0}$] (2,3)
		to[opening switch,l=${t=X}$] (4,3)
		to[R, l=${R}$] (6,3)
		to[C, l=${C}$,i>_=${i(t)}$] (6,0) to[short] (0,0)
		(6,0) -- (7,0) to[open,v>=${V_o(t)}$, *-*] (7,3) -- (6,3);
	\end{circuitikz}
\end{figure*}

Qual è la tensione di uscita $V_o(X)$ nel momento del guasto?

Per rispondere bisogna esprimere la funzione densità di probabilità $fV_(v)$ della v.a. $V_o(X)$ trasformata della v.a. tempo di guasto $X$. Il legame tra il tempo e la tensione sul condensatore è  $V_o(t)=V_i(1-\e{-\frac{t}{\alpha}})$.
Si ha così la variabile aleatoria trasformata $V(X)=V_i(1-\e{-\frac{x}{\alpha}})\step(X)$

Per il teorema fondamentale della trasformazione di variabile aleatoria la funzione densità di probabilità si calcola come
\[
	f_V(x)=\restrict{\frac{f_X(x)}{\abs{g'(x)}}}{x=g^{-1}(v)}
\]
dove $g(x)=v=V_i(1-\e{\frac{x}{\alpha}})$, $g'(x)=\frac{V_i}{\alpha}\e{-\frac{x}{\alpha}}$ e $x=g^{-1}(v)=-\alpha\f{\ln}{1-\frac{v}{V_i}}$, sostituendo la $f_X(x)$ si ha
\[
	f_V(v)=\restrict{\frac{\frac{1}{2\alpha}\e{-\frac{x}{2\alpha}}}{\abs{\frac{V_i}{\alpha}\e{-\frac{x}{\alpha}}}}}{x=g^{-1}(v)}=
	\frac{\frac{1}{2\alpha}\e{-\frac{-\alpha\f{\ln}{1-\frac{v}{V_i}}}{2\alpha}}}{\frac{V_i}{\alpha}\e{-\frac{-\alpha\f{\ln}{1-\frac{v}{V_i}}}{\alpha}}}=
	\frac{1}{2 V_I}\frac{\e{\ln\sqrt{\left(1-\frac{v}{V_i}\right)}}}{\e{\f{\ln}{1-\frac{v}{V_i}}}}=
	\frac{1}{2 V_i}\frac{1}{\sqrt{\left(1-\frac{v}{V_i}\right)}}
\]

\begin{figure}[h]
	\centering
	\subfloat[][$f_X(x)$]{
		\begin{tikzpicture}[scale=.8]
			\begin{axis}[axis lines=middle,no markers,enlargelimits,xtick={0},ytick={0,.5},yticklabels={0,$\frac{1}{2\alpha}$},xlabel=$x$,ylabel={$f_X(x)$}]
			\addplot [very thick,domain=-0.5:0,samples=2] {0};
			\addplot [very thick,domain=0:3,samples=200] {0.5*exp(-x)};
			\end{axis}
		\end{tikzpicture}}\qquad
	\subfloat[][$f_V(v)$]{
		\begin{tikzpicture}[scale=.8]
			\begin{axis}[axis lines=middle,no markers,enlargelimits,xtick={1},ytick={0,.5},xticklabels={$V_i$},yticklabels={0,$\frac{1}{2 V_i}$},xlabel=$v$,ylabel={$f_V(v)$}]
			\addplot [very thick,domain=-0.5:0,samples=2] {0};
			\addplot [very thick,domain=0:0.9,samples=200] {0.5/sqrt(1-x)};
			\addplot [very thick,domain=1:1.5,samples=2] {0};
			\end{axis}
		\end{tikzpicture}}
	\caption{Funzione densità di probabilità e sua trasformata}
\end{figure}

\end{esempio}

\section{Variabile aleatoria gaussiana o normale}
La \index{variabile aleatoria!gaussiana}\textsc{variabile aleatoria gaussiana} $X\in\mathcal{N}(\mu_X,\sigma^2_X)$ ha funzione densità di probabilità
\begin{equation}
\label{eq:densita_gaussiana}
	f_X(x)=\frac{1}{\sqrt{2\pi\sigma^2_X}}\;\e{-\tfrac{(x-\mu_X)^2}{2\sigma^2_X}}
\end{equation}
dove, come si dimostra, $\mu_X$ e $\sigma^2_X$ sono il valor medio e la varianza che caratterizzano la variabile aleatoria stessa.
Modella esperimenti che sono la risultante di tanti fenomeni elementari indipendenti.

La variabile aleatoria gaussiana $X_N\in\mathcal{N}(0,1)$ a valor medio nullo e varianza unitaria è detta \textsc{variabile normale standard}
\begin{equation}
\label{eq:densita_normale}
	f_{X_N}(x)=\frac{1}{\sqrt{2\pi}}\;\e{-\tfrac{x^2}{2}}
\end{equation}
Per il teorema fondamentale della trasformazione di variabile aleatoria si ha che la generica variabile gaussiana $X\in\mathcal{N}(\mu_X,\sigma^2_X)$ può essere espressa come trasformazione lineare della variabile aleatoria normale standard $X_N$:
\begin{equation}
	X=\sigma_X\cdot X_N+\mu_X
\end{equation}
\begin{equation}
	f_X(x)=\frac{1}{\sigma_X}\f{f_{X_N}}{\frac{x-\mu_X}{\sigma_X}}=\frac{1}{\sqrt{2\pi\sigma^2_X}}\;\e{-\tfrac{(x-\mu_X)^2}{2\sigma^2_X}}
\end{equation}


\begin{figure}[!ht]
\centering
\begin{tikzpicture}
\begin{axis}[axis x line=middle, axis y line=left,no markers,enlargelimits,xscale=2,xtick={-3,-2,-1,0,1,2,3},xticklabels={$\mu-3\sigma$,$\mu-2\sigma$,$\mu-\sigma$,$\mu$,$\mu+\sigma$,$\mu+2\sigma$,$\mu+3\sigma$},ytick={0.3989},yticklabels={$\frac{1}{\sqrt{2\pi\sigma^2}}$},xlabel=$x$,
after end axis/.code={\draw [thick,decoration={brace,mirror,raise=17pt},decorate]
	(axis cs:-1,0) -- node[below=17pt] {$68,3\%$} (axis cs:1,0);
	\draw [thick,decoration={brace,mirror,raise=35pt,amplitude=5pt},decorate]
	(axis cs:-2,0) -- node[below=37pt] {$95,4\%$} (axis cs:2,0);
	\draw [thick,decoration={brace,mirror,raise=52pt,,amplitude=10pt},decorate]
	(axis cs:-3,0) -- node[below=60pt] {$99,74\%$} (axis cs:3,0);
}
]
\pgfmathsetmacro\valueMu{gauss(0,0,1)}
\pgfmathsetmacro\valueA{gauss(1,0,1)}
\pgfmathsetmacro\valueB{gauss(2,0,1)}
\draw [dashed] (axis cs:0,0) -- (axis cs:0,\valueMu);
\draw [dashed] (axis cs:1,0) -- (axis cs:1,\valueA)
(axis cs:-1,0) -- (axis cs:-1,\valueA);
\draw [dashed] (axis cs:2,0) -- (axis cs:2,\valueB)
(axis cs:-2,0) -- (axis cs:-2,\valueB);
\addplot[very thick,smooth,domain=-4:4] {gauss(x,0,1)};
\end{axis}\end{tikzpicture}
\caption{Funzione densità di probabilità di variabile aleatoria gaussiana}
\end{figure}

La funzione di distribuzione di variabile aleatoria gaussiana non può essere calcolata in forma chiusa, ma può essere calcolata con metodi numerici la funzione di distribuzione di probabilità gaussiana standard
\begin{equation}
	\Phi_{X_N}=\intd{-\infty}{x}{\frac{1}{\sqrt{2\pi}}\;\e{-\tfrac{z^2}{2}}}{z}
\end{equation}
o la sua complementare $Q_{X_N}=1-\Phi_{X_N}$.

La funzione di distribuzione per la variabile aleatoria gaussiana si calcola applicando la trasformazione $X=\sigma_X X_N +\mu_X$ nella funzione di distribuzione di probabilità della variabile aleatoria normale standard
\begin{equation}
	F_X(x)=P(X\leq x)=P(\sigma_X\cdot X_N+\mu_X\leq x)=\f{P}{X_N\leq\frac{x-\mu_X}{\sigma_X}}=\f{\Phi_{X_N}}{\frac{x-\mu_X}{\sigma_X}}
\end{equation}
La probabilità che la v.a. gaussiana assuma valori in un intervallo $[a,b]$ si calcola
\[
	P(a<x\leq b)=F_X(b)-F_X(a)=\f{\Phi_{X_N}}{\frac{b-\mu_X}{\sigma_X}}-\f{\Phi_{X_N}}{\frac{a-\mu_X}{\sigma_X}}
\]
I valori di $\Phi_{X_N}$ sono tabulati o calcolati con le funzioni errore e errore complementare presenti nei calcolatori elettronici
\begin{equation}
	\erf{x}=\frac{2}{\sqrt{\pi}}\intd{0}{x}{\e{-z^2}}{z}
\end{equation}
\begin{equation}
	\erfc{x}=1-\erf{x}=\frac{2}{\sqrt{\pi}}\intd{x}{+\infty}{\e{-z^2}}{z}
\end{equation}
La funzione di distribuzione standard calcolata dalla funzione errore
\begin{equation}
	\Phi(x)=\frac{1}{2}\left(1+\erf{\frac{x}{\sqrt{2}}}\right)
\end{equation}
\begin{equation}
	Q(x)=\frac{1}{2}\erfc{\frac{x}{\sqrt{2}}}
\end{equation}

\begin{figure}[!ht]
	\centering
	\subfloat[][funzione di distribuzione di probabilità v.a. normale standard]{
		\begin{tikzpicture}
			\begin{axis}[axis lines=middle,no markers,enlargelimits,xtick={0},ytick={.5,1},xlabel=$x$,ylabel=$\Phi_{X_N}(x)$]
			\addplot[very thick,smooth] gnuplot[id=erfc]{.5*(1+erf(x/sqrt(2)))};
			\end{axis}
		\end{tikzpicture}}\quad
	\subfloat[][funzione errore e errore complementare]{
		\begin{tikzpicture}
			\begin{axis}[axis lines=middle,no markers,enlargelimits,xtick={0},ytick={-1,1,2},xlabel=$x$]
			\addplot[very thick,smooth] gnuplot[id=erf]{erf(x)} node [above,pos=1,pin={135:$\erf{x}$}] {};
			\addplot[dashed,smooth] gnuplot[id=erf]{erfc(x)} node [below,pos=.33,pin={-135:$\erfc{x}$}] {};
			\end{axis}
		\end{tikzpicture}}
\end{figure}

Si dimostra che l'integrale della funzione densità di probabilità
\[
	S=\intinf{f_X(x)}{x}=1
\]
\begin{proof}[Dim.]
Non avendo $\int\e{-z^2}\diff z$ soluzioni in forma chiusa, si considera che $S^2=1$ e si ha l'integrale doppio
\[
	S^2=\intinf{f_X(x)}{x}\intinf{f_Y(y)}{y}=\intinf{\intinf{\frac{1}{2\pi\sigma^2}\e{-\tfrac{(x-\mu)^2(y-\mu)^2}{2\sigma^2}}}{y}}{x}
\]
Per risolvere l'integrale si applica la trasformazione da variabili cartesiane a variabili polari
\[
	\begin{split}\frac{x-\mu}{\sigma}=\rho\cos{\theta}&\qquad\frac{y-\mu}{\sigma}=\rho\sen{\theta}\\
	\deriv{x}{\rho}=\sigma\cos{\theta}&\qquad\deriv{y}{\rho}=\sigma\sen{\theta}\\
	\deriv{x}{\theta}=-\sigma\rho\sen{\theta}&\qquad\deriv{y}{\theta}=\sigma\rho\cos{\theta}\\
	\diff x\diff y=\abs{J(\rho,\theta)}\diff\rho\diff\theta
\end{split}
\]
dove $J(\rho,\theta)$ è la matrice jacobiana della trasformazione che ha determinante
\[
	\abs{J(\rho,\theta)}=\abs{\begin{array}{cc} \sigma\cos{\theta}&-\sigma\rho\sen{\theta} \\
	\sigma\sen{\theta}&\sigma\rho\cos{\theta}\end{array}}=\sigma^2\rho\Cos^2{\theta}+\sigma^2\rho\Sen^2{\theta}=\sigma^2\rho
\]
\[
	S^2=\intd{0}{2\pi}{\intd{0}{+\infty}{\frac{1}{2\pi\sigma^2}\e{-\tfrac{\rho^2}{2}}\sigma^2\rho}{\rho}}{\theta}=\frac{1}{2\pi}\intd{0}{+\infty}{\rho\e{-\tfrac{\rho^2}{2}}}{\rho}\intd{0}{2\pi}{}{\theta}=\bound{0}{+\infty}{-\e{-\tfrac{\rho^2}{2}}}=1
\]
\end{proof}

Dimostriamo che la variabile aleatoria gaussiana ha media $\mu_X$ e varianza $\sigma^2_X$. La funzione generatrice dei momenti
\begin{proof}[Dim.]
\[
	\begin{split}G_X(t)=\E{\e{t X}}&=\intinf{\e{t x}\frac{1}{\sqrt{2\pi\sigma^2_X}}\;\e{-\tfrac{(x-\mu_X)^2}{2\sigma^2_X}}}{x}=\\
	&=\frac{\e{t\mu_X}}{\sqrt{2\pi\sigma^2_X}}\intinf{\e{t (x-\mu_X)}\;\e{-\tfrac{(x-\mu_X)^2}{2\sigma^2_X}}}{x}=\\
	&=\e{t\mu_X+\tfrac{\sigma^2_X t^2}{2}}\underbrace{\intinf{\frac{1}{\sqrt{2\pi\sigma^2_X}}\e{-\tfrac{(x-\mu_X-\sigma^2_X t)^2}{2\sigma^2_X}}}{x}}_{=1}=\\
	&=\e{t\mu_X+\tfrac{\sigma^2_X t^2}{2}}\end{split}
\]
dove l'esponente si è calcolato con i passaggi
\[
	t(x-\mu_X)-\frac{(x-\mu_X)^2}{2\sigma^2_X}=-\frac{(x-\mu_X)^2-2\sigma^2_X t(x-\mu_X)+\sigma^4_X t^2-\sigma^4_X t^2}{2\sigma^2_X}=\frac{\sigma^2_X t^2}{2}-\frac{(x-\mu_X-\sigma^2_X t)^2}{2\sigma^2_X}
\]

Il valor medio e la varianza della variabile aleatoria gaussiana calcolati dai momenti del primo e secondo ordine
\begin{equation}
\begin{split}
	\E{X}&=\restrict{\deriv{G_X(t)}{t}}{t=0}=\restrict{\e{t\mu_X+\tfrac{\sigma^2_X t^2}{2}}(\mu_X+\sigma^2_X t)}{t=0}=\mu_X
	\\
	\E{X^2}&=\restrict{\deriv[2]{G_X(t)}{t}}{t=0}=\restrict{\e{t\mu_X+\tfrac{\sigma^2_X t^2}{2}}(\mu_X+\sigma^2_X t)^2+\e{t\mu_X+\tfrac{\sigma^2_X t^2}{2}}\sigma^2_X}{t=0}=\mu^2_X+\sigma^2_X
	\\
	\E{(X-\mu_X)^2}&=\E{X^2}-\Esp^2[X]=\mu^2_X+\sigma^2_X-\mu^2_X=\sigma^2_X
\end{split}
\end{equation}
\end{proof}

% La VA paretiana è stata introdotta nell'A.A. 16/17
\section{Variabile aleatoria di Pareto (di tipo 1)}
La \textsc{variabile aleatoria di Pareto}\index{variabile aleatoria!di Pareto} (o \emph{paretiana}) ha funzione distribuzione di probabilità
\begin{equation}
	F_X(x) =
	\begin{cases}
		1-\left(\frac{x_m}{x}\right)^\alpha & x>x_m\\
		0 & \text{altrove}
	\end{cases}
\end{equation}
dove $x_m$ è il valore minimo che può assumere la funzione.

In teoria delle probabilità, la distribuzione paretiana è una distribuzione di probabilità continua, utilizzata in particolar modo per descrivere la distribuzione dei redditi.

La variabile aleatoria di Pareto è una è una variabile aleatoria a coda ‘‘pesante’’ (\textit{heavy tail}) ovvero la sua densità di probabilità è una funzione che tende a $0$ non in modo esponenziale ma come $\frac{1}{x^{\alpha}}$, infatti la sua densità di probabilità vale
\begin{equation}
	f_X(x) =
	\begin{cases}
		\alpha \frac{x_m^\alpha}{x^{\alpha + 1}} & x>x_m \\
		0 & \text{altrove}
	\end{cases}
\end{equation}
dove $0 < x_m \leq x < +\infty$ e $\alpha > 0$.

\begin{figure}[h]
	\centering
	\begin{tikzpicture}[scale=.8]
		\begin{scope}
			\begin{axis}[axis lines=middle,no markers,enlargelimits,xtick={1},xticklabels={$x_m$},xlabel=$x$,ytick={1},ylabel=$F_X(x)$]
				\addplot [very thick,samples=200,domain=1:5] {1-(1/x)^2};
				\addplot [dashed] coordinates {(1,0) (1,1)};
				\addplot [dashed,domain=0:5] {1};
			\end{axis}
		\end{scope}
		\begin{scope}[xshift=9cm]
			\begin{axis}[axis lines=middle,no markers,enlargelimits,xtick={1,3},xticklabels={$x_m$,$x_0$},xlabel=$x$,ytick={0},ylabel=$f_X(x)$]
				\addplot [very thick,samples=200,domain=1:5] {3*(1/x^3)};
				\addplot [dashed] coordinates {(1,0) (1,3)};
				\node[above] at (4,0) {\scriptsize Coda lunga};
			\end{axis}
		\end{scope}
	\end{tikzpicture}
	\caption{Funzione distribuzione e densità di probabilità di una variabile aleatoria paretiana.}
\end{figure}
\begin{flushleft}
Calcolo degli indici statistici:
\end{flushleft}
\begin{description}
\item[Valor medio]
\begin{equation}
	\mu_X=\E{X}=\intd{x_m}{+\infty}{\alpha\,x_m^\alpha\,x^{-(\alpha +1)} x}{x}=
	\bound{x_m}{+\infty}{\alpha\,x_m^\alpha\,\frac{x^{-\alpha +1}}{1-\alpha}}=
	\begin{cases}
		\frac{\alpha}{\alpha -1}\,x_m & \alpha>1\\
		+\infty & \alpha\leq 1
	\end{cases}
\end{equation}

\item[Potenza]
\begin{equation}
	\E{X^2}=\intd{x_m}{+\infty}{\alpha x_m^\alpha x^{-(\alpha +1)}x^2}{x}=
	\bound{x_m}{+\infty}{\alpha\, x_m^\alpha \frac{x^{-\alpha +2}}{2-\alpha}}=
	\begin{cases}
		\frac{\alpha}{\alpha -2}\,x_m^2 & \alpha>2 \\
		+\infty & \alpha\leq 2
	\end{cases}
\end{equation}

\item[Varianza]
\begin{equation}
\begin{split}
	\sigma_X^2=\E{X^2}-\Esp^2[X] &=
		\frac{\alpha}{\alpha -2} x_m^2-\left(\frac{\alpha}{\alpha-1} x_m \right)^2=
		\frac{\alpha}{\alpha -2} x_m^2-\frac{\alpha^2}{(\alpha-1)^2} x_m^2=\\
	&=\frac{(\alpha -1)^2-(\alpha -2)\alpha}{(\alpha -2)(\alpha -1)^2}(\alpha\,x_m)^2=
		\frac{\alpha^2 +1 -2\alpha -\alpha^2 +2\alpha}{(\alpha -2)(\alpha -1)^2} (\alpha\,x_m^2) = \\
	&=\frac{\alpha\,x_m^2}{(\alpha -2)(\alpha -1)^2}=\frac{\mu_X^2}{\alpha (\alpha -2)},\text{ per }\alpha>2
\end{split}
\end{equation}

\item[Deviazione standard]
\begin{equation}
	\sigma_X=\frac{x_m \sqrt{\alpha}}{(\alpha -1)\sqrt{\alpha -2}}= \frac{\mu_x}{\sqrt{\alpha(\alpha -2)}},\text{ per }\alpha>2
\end{equation}
\end{description}

% La VA Zeta è stata introdotta nell'A.A. 16/17
\section{Variabile aleatoria Zeta}
Una variabile aleatoria ha una distribuzione di probabilità \textsc{Zeta}\index{variabile aleatoria!Zeta} (talvolta detta di \emph{Zipf}) se la sua densità discreta è data da
\begin{equation}
	P(X = k) = \frac{C}{k^{\alpha+1}},
	\quad k=1,2,\dots
\end{equation}
per qualche valore $\alpha > 0$. Poiché la somma delle precedenti probabilità deve essere uguale a $1$, segue che
\begin{equation}
	C = \frac{1}{\sum\limits_{k=1}^{\infty} k^{-(\alpha+1)}}
\end{equation}

La distribuzione Zeta deve il suo nome al fatto che la funzione
\[
	\zeta(\alpha) = 1 + \left(\frac{1}{2}\right)^\alpha + \left(\frac{1}{3}\right)^\alpha + \dots + \left(\frac{1}{k}\right)^\alpha + \dots
\]
è nota in matematica come la \emph{funzione Zeta di Riemann}.

Se $X$ è una variabile aleatoria Zeta con parametro $\alpha$, allora la probabilità che $X$ assuma valore $k$ è data dalla \emph{funzione con massa di probabilità}
\begin{equation}
	f_\alpha(k) = \frac{k^{-\alpha}}{\zeta(\alpha)}
\end{equation}

Si calcolano gli indici statistici mediante la funzione generatrice dei momenti:
\begin{equation}
	G_X(t;\alpha) = \E{\e{tX}} = \frac{1}{\zeta(\alpha)} \sum_{k=1}^\infty \frac{\e{tk}}{k^{\alpha}}
\end{equation}

\paragraph{Valor medio (momento di primo ordine)}
\begin{equation}
	\mu_X=\restrict{\deriv{G_X(t;\alpha)}{t}}{t=0}=
	\frac{1}{\zeta(\alpha)}\sum_{k=1}^\infty\frac{1}{k^{\alpha-1}}=
	\frac{\zeta(\alpha-1)}{\zeta(\alpha)},
	\quad\text{ per }\alpha>2
\end{equation}

\paragraph{Potenza (momento di secondo ordine)}
\begin{equation}
	\E{X^2}=\restrict{\frac{\mathrm{d}^2 G_X(t;\alpha)}{\mathrm{d}t^2}}{t=0}=
	\frac{1}{\zeta(\alpha)}\sum_{k=1}^\infty\frac{1}{k^{\alpha -2}}=
	\frac{\zeta(\alpha -2)}{\zeta(\alpha)},
	\quad\text{ per }\alpha>3
\end{equation}

\paragraph{Varianza}
\begin{equation}
	\sigma_X^2=\E{X^2}-\Esp^2[X]=
	\frac{\zeta(\alpha -2)}{\zeta(\alpha)}-\left(\frac{\zeta(\alpha-1)}{\zeta(\alpha)}\right)^2=
	\frac{\zeta(\alpha-2)\zeta(\alpha)-\zeta(\alpha-1)^2}{\zeta(\alpha)^2},
	\quad\text{ per }\alpha>3
\end{equation}

\begin{flushleft}
Anche la variabile aleatoria Zeta è una variabile aleatoria a coda pesante, dunque nella progettazione di un sistema modellato da una variabile aleatoria di questo tipo bisogna sempre gestire il \textit{caso peggiore o worst case} poichè per le variabili aleatorie a coda pesante gli indici statistici possono assumere valori molto elevati.
\end{flushleft}

\section{Variabili aleatorie condizionate}
Data una variabile aleatoria X si ha la funzione distribuzione di probabilità, $F_X(x)=P(X\leq x)=P(A)$ che rappresenta la probabilità che si verifichi l'evento $A$ per cui la variabile aleatoria $X$ assuma valori minori o uguali a $x$.
Se l'evento $A$ può essere influenzato dal verificarsi o meno di un altro evento $B$, avente probabilità non nulla $P(B)$, si ha una \textsc{variabile aleatoria condizionata}\index{variabile aleatoria!condizionata}.
La sua funzione di distribuzione di probabilità condizionata dall'evento $B$ si definisce
\begin{equation}
	F_{X|B}(x)=\frac{P(A\cap B)}{P(B)}=\frac{P(\lbrace X\leq x\rbrace\cap B)}{P(B)}
\end{equation}
e funzione densità di probabilità
\[
	f_{X|B}(x)=\deriv{F_X(x)}{x}
\]
che godono di tutte le proprietà valide per funzioni e distribuzioni non condizionate.
Per brevità si scriverà di seguito $P(\lbrace X\leq x\rbrace\cap B)$ come $P(X\leq x, B)$

\section{Sistemi di variabili aleatorie}
Nello studio di un esperimento aleatorio in molti casi si eseguono misure di diverse grandezze fisiche a cui sono associate più variabili aleatorie. Spesso i risultati dell'esperimento forniscono informazioni significative solo considerando insieme le \textsc{variabili aleatorie congiunte}\index{variabili aleatorie congiunte}.

Data un coppia di v.a. congiunte $X,Y$ si definisce funzione distribuzione di probabilità congiunta
\begin{equation}
\label{eq:funz_dist_prob_congiunta}
	F_{XY}(x,y)=P(X\leq x,Y\leq y)
\end{equation}

In generale non è possibile derivare $F_{XY}(x,y)$ a partire dalle distribuzioni marginali $F_X(x)$ e $F_Y(y)$: il fenomeno va studiato come variabile aleatoria congiunta.
La \textsc{funzione di distribuzione di probabilità congiunta}\index{variabili aleatorie congiunte!funzione distribuzione di probabilità congiunta} ha proprietà simili alle funzioni di distribuzione di una sola variabile.\\

\textbf{Proprietà funzione distribuzione probabilità congiunta}
\begin{enumerate}
\item $0\leq F_{XY}(x,y)\leq 1$
\item Fissato un $y=y_0$ la $F_{XY}(x,y_0)$ è funzione di $x$, è monotona non decrescente, è continua a destra (per $\leq$).
Similmente si ha per $x=x_0$ fissato le stesse proprietà per $F_{XY}(x_0,y)$
\item Valori per eventi nulli
\[
	\begin{split}
		F_{XY}(-\infty,y)&=P(X\leq -\infty,Y\leq y)=0\\
		F_{XY}(-\infty,y)&=P(X\leq x,Y\leq -\infty)=0\\
		F_{XY}(-\infty,-\infty)&=P(X\leq -\infty,Y\leq -\infty)=0
	\end{split}
\]
\item Le funzioni di distribuzione marginale $F_X(x)$ e $F_Y(y)$, nota $F_{XY}(x,y)$, si ricavano come
\[
	\begin{split}
		F_X(x)=F_{XY}(x,+\infty)\\F_Y(y)=F_{XY}(+\infty,y)
	\end{split}
\]
\item $\lim\limits_{x,y\to\infty}F_{XY}(x,y)=1$
\item Dati degli intervalli di valori per le due v.a. la probabilità dell'evento descritto dall'area rettangolare $R=\{(x,y)|x_1<x\leq x_2,\, y_1<y\leq y_2\}$
\[
	P(x_1<X\leq x_2,\,y_1<Y\leq y_2)= F_{XY}(x_2,\,y_2)-F_{XY}(x_1,\,y_2)-F_{XY}(x_2,\,y_1)+F_{XY}(x_1,\,y_1)
\]
\end{enumerate}
L'ultima proprietà permette di definire il rapporto incrementale
\[
	\begin{split}
		P(&x<X\leq x+\Delta x,\,y<Y\leq y+\Delta y)=\\
		&=\left[F_{XY}(x+\Delta x,\,y+\Delta y)-F_{XY}(x,\,y+\Delta y)\right]-\left[F_{XY}(x+\Delta x,\,y_1)-F_{XY}(x,\,y)\right]\cong\\
		&\cong\frac{F_{XY}(x+\Delta x,\,y+\Delta y)-F_{XY}(x+\Delta x,\,y)}{\Delta y}\cdot\Delta y+\frac{F_{XY}(x,\,y+\Delta y)-F_{XY}(x,\,y)}{\Delta y}\cdot\Delta y=\\
		&\cong\pderiv{F_{XY}(x,\,y+\Delta y)}{y}\cdot\Delta y+\pderiv{F_{XY}(x,\,y)}{y}\cdot\Delta y
	\end{split}
\]
dove portando al limite $\Delta x\to 0$ e $\Delta y\to 0$ si ha la definizione di \textsc{funzione densità di probabilità congiunta}\index{variabili aleatorie congiunte!funzione densità di probabilità congiunta}
\begin{equation}
	f_{XY}(x,y)=\frac{\partial^2 F_{XY}(x,\,y)}{\partial x\partial y}
\end{equation}

\textbf{Proprietà funzione densità di probabilità congiunta}
\begin{enumerate}
\item La funzione densità è non negativa
\begin{equation}
	f_{XY}(x,y)\geq 0
\end{equation}
\item La funzione densità integra a 1 sul piano $\R^2$
\begin{equation}
	\intinf{\intinf{f_{XY}(x,y)}{x}}{y}=1
\end{equation}
\item Le densità marginali $f_X(x)$ e $f_Y(y)$ delle v.a. $X$ e $Y$ si ricavano
\begin{equation}
	f_X(x)=\intinf{f_{XY}(x,y)}{y}
\end{equation}
\begin{equation}
	f_Y(y)=\intinf{f_{XY}(x,y)}{x}
\end{equation}
\item La probabilità di un evento $A=\{(X,Y)\in D\}$ definito da un dominio $D$ nel piano cartesiano
\begin{equation}
\label{eq:funz_dens_prob_cong_dominio}
	P(A)=P(\{(X,Y)\in D\})=\iint\limits_{D}{f_{XY}(x,y)\diff x \diff y}
\end{equation}
\item La funzione distribuzione congiunta ricavata dalla funzione densità congiunta
\begin{equation}
	F_{XY}(x,y)=\int\limits_{\alpha=-\infty}^{x}\int\limits_{\beta=-\infty}^{y}{f_{XY}(\alpha,\beta)}{\,\diff \alpha\,\diff \beta}
\end{equation}
\end{enumerate}

% Helper per poter fare linee parallele con spazio variabile
\pgfdeclarepatternformonly[\LineSpace]{custom ne lines}{\pgfqpoint{-1pt}{-1pt}}{\pgfqpoint{\LineSpace}{\LineSpace}}{\pgfqpoint{\LineSpace}{\LineSpace}}%
{
    \pgfsetlinewidth{0.4pt}
    \pgfpathmoveto{\pgfqpoint{0pt}{0pt}}
    \pgfpathlineto{\pgfqpoint{\LineSpace + 0.1pt}{\LineSpace + 0.1pt}}
    \pgfusepath{stroke}
}
\makeatother %new code
\newdimen\LineSpace
\tikzset{
    line space/.code={\LineSpace=#1},
    line space=3pt
}
% Helper per fare un pattern di puntini con spazio intermedio variabile
\pgfdeclarepatternformonly{my crosshatch dots}{\pgfqpoint{-1pt}{-1pt}}{\pgfqpoint{5pt}{5pt}}{\pgfqpoint{6pt}{6pt}}%
{
    \pgfpathcircle{\pgfqpoint{0pt}{0pt}}{.5pt}
    \pgfpathcircle{\pgfqpoint{3pt}{3pt}}{.5pt}
    \pgfusepath{fill}
}

\begin{figure}[!ht]
	\centering
	\subfloat[\(F_{XY}(x_2,y_2)\)]
	{
		\begin{tikzpicture}[scale=.6]
			\begin{axis}[axis lines=middle,no markers,enlargelimits,xtick={1,4},xticklabels={\(x_1\),\(x_2\)},xlabel=\(x\),ytick={1,4},yticklabels={\(y_1\),\(y_2\)},ylabel=\(y\)]
				\addplot [dashed,domain=0:4] {1};
				\addplot [dashed,domain=0:4] {4};
				\addplot [dashed] coordinates {(1,0) (1,4)};
				\addplot [dashed] coordinates {(4,0) (4,4)};
				\addplot [draw=none,pattern=custom ne lines,line space=10pt,samples=200]  coordinates {(-2,-2) (-2,4) (4,4) (4,-2)};
			\end{axis}
		\end{tikzpicture}
	} \hspace{3em}
	\subfloat[\(F_{XY}(x_2,y_2) - F_{XY}(x_1,y_2)\)]
	{
		\begin{tikzpicture}[scale=.6]
			\begin{axis}[axis lines=middle,no markers,enlargelimits,xtick={1,4},xticklabels={\(x_1\),\(x_2\)},xlabel=\(x\),ytick={1,4},yticklabels={\(y_1\),\(y_2\)},ylabel=\(y\)]
				\addplot [dashed,domain=0:4] {1};
				\addplot [dashed,domain=0:4] {4};
				\addplot [dashed] coordinates {(1,0) (1,4)};
				\addplot [dashed] coordinates {(4,0) (4,4)};
				\addplot [draw=none,pattern=custom ne lines,line space=10pt,samples=200]  coordinates {(1,-2) (1,4) (4,4) (4,-2)};
			\end{axis}
		\end{tikzpicture}
	} \\
	\subfloat[\(F_{XY}(x_2,y_2) - F_{XY}(x_1,y_2) - F_{XY}(x_2,y_1)\)]
	{
		\begin{tikzpicture}[scale=.6]
			\begin{axis}[axis lines=middle,no markers,enlargelimits,xtick={1,4},xticklabels={\(x_1\),\(x_2\)},xlabel=\(x\),ytick={1,4},yticklabels={\(y_1\),\(y_2\)},ylabel=\(y\)]
				\addplot [dashed,domain=0:4] {1};
				\addplot [dashed,domain=0:4] {4};
				\addplot [dashed] coordinates {(1,0) (1,4)};
				\addplot [dashed] coordinates {(4,0) (4,4)};
				\addplot [draw=none,pattern=custom ne lines,line space=10pt,samples=200]  coordinates {(1,1) (1,4) (4,4) (4,1)};
				\addplot [draw=none,pattern=my crosshatch dots,samples=200]  coordinates {(-2,1) (1,1) (1,-2) (-2,-2)};
				\node[right,label={[align=center]\\\(\leadsto\)~area cancellata\\2 volte}] at (2.3,-2) {};
			\end{axis}
		\end{tikzpicture}
	} \hspace{3em}
	\subfloat[\(F_{XY}(x_2,y_2) - F_{XY}(x_1,y_2) - F_{XY}(x_2,y_1) + F_{XY}(x_1,y_1)\)]
	{
		\begin{tikzpicture}[scale=.6]
			\begin{axis}[axis lines=middle,no markers,enlargelimits,xtick={1,4},xticklabels={\(x_1\),\(x_2\)},xlabel=\(x\),ytick={1,4},yticklabels={\(y_1\),\(y_2\)},ylabel=\(y\)]
				\addplot [dashed,domain=0:4] {1};
				\addplot [dashed,domain=0:4] {4};
				\addplot [dashed] coordinates {(1,0) (1,4)};
				\addplot [dashed] coordinates {(4,0) (4,4)};
				\addplot [draw=none,pattern=custom ne lines,line space=10pt,samples=200]  coordinates {(1,1) (1,4) (4,4) (4,1)};
			\end{axis}
		\end{tikzpicture}
	}
	\caption{Dimostrazione grafica della proprietà 6 della funzione distribuzione di probabilità congiunta}
	% TODO: definire label e indicare la proprietà con riferimento
\end{figure}

Per una coppia di variabili aleatorie $(X,Y)$ con densità di probabilità congiunta $f_{XY}(x,y)$, si può osservare che una variabile, ad esempio $X$, assume un particolare valore $x$; la distribuzione marginale dell'altra variabile viene modificata da questo \emph{condizionamento}.

Si definisce pertanto la \textsc{funzione distribuzione condizionata}\index{variabili aleatorie congiunte!funzione distribuzione di probabilità condizionata} della variabile $Y$ rispetto all'evento $X=x$ calcolata dalla funzione densità congiunta $f_{XY}(x,y)$ e dalla densità marginale di $f_X(x)$:
\begin{equation}
	F_{Y|X}(y)=\frac{\intd{\beta=-\infty}{y}{f_{XY}(x,\beta)}{\beta}}{f_X(x)}
\end{equation}
La \textsc{funzione densità di probabilità condizionata}\index{variabili aleatorie congiunte!funzione densità di probabilità condizionata} della variabile aleatoria $Y$ rispetto all'evento $X=x$ si ricava derivando la funzione distribuzione di probabilità congiunta:
\begin{equation}
	f_{Y|X}(y)=\pderiv{F_{Y|X}(y)}{y}=\frac{f_{XY}(x,y)}{f_X(x)}
\end{equation}

Se la densità di probabilità marginale $f_Y(y)$ della variabile aleatoria $Y$ coincide con la densità di probabilità condizionata $f_{Y|X}(y)$ il comportamento statistico della variabile $Y$ non è influenzato dal valore assunto dalla variabile $X$.

In tal caso $f_Y(y)=f_{Y|X}(y)$, le variabili aleatorie X e Y si dicono indipendenti e la densità di probabilità congiunta $f_{XY}(x,y)$ può essere espressa come prodotto delle probabilità marginali:
\begin{equation}
\label{eq:funz_dens_prob_cong_var_indip}
	f_{XY}(x,y)=f_X(x)\cdot f_Y(y)
\end{equation}

\clearpage
\section{Funzioni di variabili aleatorie}
Si definisce una variabile aleatoria $Z$ come funzione di una coppia di variabili aleatorie $(X,Y)$, con funzione densità di probabilità congiunta $f_{XY}(x,y)$
\[
	Z=g(X,Y)
\]
con $g(\cdot,\cdot)$ funzione reale di due variabili reali. La funzione distribuzione di probabilità di $Z$
\[
	F_Z(z)=P(Z\leq z)=P(g(X,Y)\leq z)=\iint\limits_{R(z)}{f_{XY}(x,y)\diff x \diff y}
\]
dove la funzione $g(\cdot,\cdot)$ definisce nell'ultimo integrale, per l'eq.\ref{eq:funz_dens_prob_cong_dominio}, il dominio di integrazione nel piano cartesiano individuato dall'evento $R(z)=\{g(X,Y)\leq z\}$, ovvero l'insieme dei punti $(x,y)$ che verificano la diseguaglianza $g(x,y)\leq z$.

La funzione densità di probabilità $f_Z(z)$ si ricava per derivazione
\[
	f_Z(z)=\deriv{F_Z(z)}{z}
\].

\begin{esempio}
Si ha la variabile aleatoria $Z=X+Y$ con funzione distribuzione di probabilità
\[
	F_Z(z)=P(X+Y\leq z)=\iint\limits_{x+y\leq z}{f_{XY}(x,y)\diff x \diff y}=\int\limits_{y=-\infty}^{+\infty}\int\limits_{x=-\infty}^{z-y}{f_{XY}(x,y)}{\,\diff x\,\diff y}
\]
La funzione densità di probabilità calcolata come \[
	f_Z(z)=\deriv{F_Z(z)}{z}=\intinf{\deriv{}{z}\intd{x=-\infty}{z-y}{f_{XY}(x,y)}{x}}{y}=\intinf{f_{XY}(z-y,y)}{y}
\]
Se le due variabili sono indipendenti la densità di probabilità congiunta $f_{XY}(x,y)$ per l'eq.\ref{eq:funz_dens_prob_cong_var_indip} è fattorizzabile per cui
\[
	f_Z(z)=\intinf{f_X(z-y)f_Y(y)}{y}=f_X(x)\ast f_Y(y)
\]
ovvero la densità di probabilità di $Z=X+Y$ è pari alla convoluzione delle densità di probabilità marginali di $X$ e $Y$.
Il calcolo del valore medio $\mu_Z$ della variabile aleatoria $Z$ si ottiene generalizzando il teorema del valor medio (eq.\ref{eq:funz_dens_prob_cong_var_indip})
\[
	\mu_Z=\E{Z}=\E{g(X,Y)}=\intinf{\intinf{g(x,y)f_{XY}(x,y)}{x}}{y}
\]
\end{esempio}

\section{Correlazione e covarianza}
\`{E} possibile calcolare per coppie di variabili aleatorie indici caratteristici del comportamento statistico congiunto.

La \index{correlazione}\textsc{correlazione} $r_{XY}$ tra le variabili aleatorie $X$ e $Y$
\begin{equation}
\label{eq:v_a_correlazione}
	r_{XY}=\E{X Y}=\intinf{\intinf{x y f_{XY}(x,y)}{x}}{y}
\end{equation}

La \index{covarianza}\textsc{covarianza} $c_{XY}$ tra le variabili aleatorie $X$ e $Y$
\begin{equation}
\label{eq:v_a_covarianza}
	c_{XY}=\E{(X-\mu_X)(Y-\mu_Y)}=\intinf{\intinf{(x-\mu_X)(y-\mu_Y)f_{XY}(x,y)}{x}}{y}
\end{equation}
Essendo l'aspettazione un operatore lineare si ha che
\begin{equation}
\begin{split}
	c_{XY}&=\E{(X-\mu_X)(Y-\mu_Y)}=\E{x y-\mu_X\cdot y-\mu_Y\cdot x+\mu_X\mu_y}=\\
	&=\E{x y}-\mu_X \E{y}-\mu_Y \E{x}+\mu_X\mu_Y=\\
	&=r_{XY}-\mu_X\mu_Y
\end{split}
\end{equation}
La covarianza è un parametro statistico che da indicazioni su una dipendenza di tipo lineare tra variabili aleatorie, e misura la tendenza alla variazione congiunta, ovvero se le due v.a. tendono a discostarsi dal valor medio nella stessa direzione (covarianza positiva) o in direzioni opposte (covarianza negativa).

Il \textsc{coefficiente di correlazione} $\rho_{XY}$ da una misura di covarianza normalizzata e indica il grado di correlazione o dipendenza lineare tra due variabili aleatorie
\begin{equation}
	\rho_{XY}=\E{\frac{X-\mu_X}{\sigma_X}\cdot\frac{Y-\mu_Y}{\sigma_Y}}=\frac{c_{XY}}{\sigma_X\sigma_Y}=\frac{r_{XY}-\mu_X\mu_Y}{\sigma_X\sigma_Y}
\end{equation}
in cui le variabili aleatorie vengono trasformate per avere entrambe valor medio nullo e varianza unitaria. Il coefficiente di correlazione è limitato ai valori $\abs{\rho_{XY}}\leq 1$ e ha valore nullo per variabili aleatorie \emph{incorrelate}.

Se due variabili aleatorie sono indipendenti si ha
\[
	\begin{split}
		r_{XY}&=\E{X Y}=\intinf{\intinf{x y f_{XY}(x,y)}{x}}{y}=\intinf{\intinf{x f_{X}(x) y f_{Y}(y)}{x}}{y}=\\
		&=\intinf{x f_{X}(x)}{x}\intinf{y f_{Y}(y)}{y}=\mu_X\mu_Y\\
		c_{XY}&=r_{XY}-\mu_X\mu_Y=0\\
		\rho_{XY}&=\frac{c_{XY}}{\sigma_X\sigma_Y}=0
	\end{split}
\]
pertanto due variabili aleatorie indipendenti sono incorrelate. In generale non è vera l'implicazione inversa.

\begin{esempio}
Si vuole verificare la correlazione tra due variabili aleatorie linearmente dipendenti $Y=a X+b$
\[
	\begin{split}
		Y=a X+b&\implies\mu_Y=a\mu_X+b\implies Y-\mu_Y=a(X-\mu_X)\implies\\
		&\implies(Y-\mu_Y)^2=a^2(X-\mu_X)^2\implies\sigma^2_Y=a^2\sigma^2_X\implies\sigma_Y=\abs{a}\sigma_X
	\end{split}
\]
\[
	\rho_{XY}=\frac{1}{\sigma_X\sigma_Y}\,\E{a(X-\mu_X)^2}=\frac{1}{\sigma_X\sigma_Y}\cdot a\sigma^2_X=a\cdot\frac{\sigma_X}{\sigma_Y}=\frac{a}{\abs{a}}=\pm 1
\]
ovvero come ci si attendeva $Y$ è completamente correlata ad $X$ linearmente. Il segno di $a$ determina la correlazione diretta o inversa.
\end{esempio}

\begin{esempio}
Si suppone di avere una variabile aleatoria $Y=a X+b+Z$, un modello di dipendenza lineare dubbia per la presenza della v.a. $Z$. La soluzione lineare si ottiene imponendo che la media $\mu_Z=0$ e la varianza $\sigma^2_Z$ sia minima. Cerco $a,b$ tali da soddisfare le condizioni.
Ho che $\mu_Y=a\mu_X+b+\mu_Z$ quindi
\[
	\mu_Z=\mu_Y-a\mu_X-b=0\iff b=\mu_Y-a\mu_X
\]
e la varianza
\[
	\begin{split}
		\sigma^2_Z&=\E{Z^2}=\E{(Y-a X-b)^2}=\E{(Y-a X-\mu_X+a\mu_Y)^2}=\\
		&=\E{(Y-\mu_Y)^2}+a^2 \E{(X-\mu_X)^2}-2a \E{(Y-\mu_Y)(X-\mu_X)}=\sigma^2_Y+a^2\sigma^2_X-2a c_{XY}
	\end{split}
\]
il minimo si ha annullando la derivata rispetto al parametro $a$
\[
	\deriv{\sigma^2_Z}{a}=2a\sigma^2_X-2c_{XY}=0 \implies a=\frac{c_{XY}}{\sigma^2_X}
\]
per tale valore del parametro si ha
\[
	\sigma^2_{Z,\min}=\sigma^2_Y+\frac{c^2_{XY}}{\sigma^4_X}\sigma^2_X-2\frac{c^2_{XY}}{\sigma^2_X}=\sigma^2_Y-\frac{c^2_{XY}}{\sigma^2_X}=\sigma^2_Y(1-\rho^2)
\]
e come ci si attende $\abs{\rho}\leq 1$ essendo $\rho^2=\frac{\sigma^2_Y-\sigma^2_Z}{\sigma^2_Y}\leq 1$

Tanto più $\abs{\rho}\to 1$ tanto più le variabili saranno linearmente dipendenti. Per $\rho=0$ le due variabili sono incorrelate ($c_{XY}=0$).
\end{esempio}

\begin{esempio}
Si vuole calcolare la media e la varianza della somma di variabili aleatorie indipendenti, $Z=X+Y$:
\[
	\E{Z}=\E{X+Y}=\E{X}+\E{Y}=\mu_X+\mu_Y
\]
\[
	\E{Z^2}=\E{(X+Y)^2}=\E{X^2}+\E{Y^2}+2\E{X\cdot Y}
\]
Per $X$ e $Y$ variabili aleatorie indipendenti la correlazione $r_{XY}=\E{X\cdot Y}=\mu_X\mu_Y$
\[
	\begin{split}
		\sigma^2_Z&=\E{Z^2}-\E{Z}^2=\E{X^2}+\E{Y^2}+2r_{XY}-\mu^2_X-\mu^2_Y-2\mu_X\mu_Y=\\
		&=\E{X^2}+\E{Y^2}-\mu^2_X-\mu^2_Y=\sigma^2_X+\sigma^2_Y
	\end{split}
\]
\end{esempio}
\begin{nota}
	Si ha il risultato generale notevole per cui date $n$ variabili aleatorie indipendenti $X_1,\dots,X_n$, la variabile aleatoria $Z=\sum_{i=1}^{n}X_i$ ha valor medio $\mu_Z=\sum_{i=1}^{n}\mu_{X_i}$ e varianza $\sigma^2_Z=\sum_{i=1}^{n}\sigma^2_{X_i}$
\end{nota}

\begin{esempio}
Si hanno due variabili aleatorie indipendenti con distribuzione di probabilità gaussiana normale a media nulla e varianza $\sigma^2$: $X,Y\sim\mathcal{N}(0,\sigma^2)$
Si vuole trovare la media e la varianza della variabile aleatoria $Z=\sqrt{X^2+Y^2}$ che ha funzione densità di probabilità congiunta $f_{XY}(x,y)=f_X(x)\cdot f_Y(y)$ e funzione di distribuzione di probabilità $F_Z(z)=P(Z\leq z)=\iint\limits_{R(z)}{f_{XY}(x,y)\,\diff x\,\diff y}$
con $R(z)=\{(x,y):\sqrt{x^2+y^2}\leq z\}$ il cerchio di centro nell'origine e raggio $z$.

Si passa a coordinate polari per semplificare l'integrale doppio, cambiando le variabili $(x,y)$ in $\rho,\theta$, con la trasformazione
\[
	\begin{cases}
		x = \rho\cos{\theta}\\
		y = \rho\sen{\theta}
	\end{cases}
\]
lo jacobiano della trasformazione è il determinante della matrice
\[
	\abs{J}=\begin{array}{|cc|}
		\pderiv{x}{\rho}&\pderiv{x}{\theta}  \\
		\pderiv{y}{\rho}&\pderiv{y}{\theta}  \\
	\end{array}=\begin{array}{|cc|}
		\cos{\theta}&-\rho\sen{\theta}  \\
		\sen{\theta}&\rho\cos{\theta}  \\
	\end{array}=\rho\Cos^2\theta+\rho\Sen^2\theta=\rho
\]

La funzione distribuzione di probabilità in coordinate polari
\[
	F_Z(z)=\intd{0}{z}{\intd{0}{2\pi}{f_{\rho,\theta}(\rho,\theta) \rho}{\theta}}{\rho}
\]

La funzione densità di probabilità in coordinate polari
\[
	f_{XY}(x,y)=f_X(x) f_Y(y)=\frac{1}{2\pi\sigma^2}\e{-\frac{x^2+y^2}{2\sigma^2}}\quad\to\quad f_{\rho,\theta}(\rho,\theta)=\frac{1}{2\pi\sigma^2}\e{-\frac{\rho^2}{2\sigma^2}}
\]

\[
	F_Z(z)=\intd{0}{z}{\intd{0}{2\pi}{\frac{1}{2\pi\sigma^2}\e{-\frac{\rho^2}{2\sigma^2}}\rho}{\theta}}{\rho}=\frac{1}{\sigma^2}\intd{0}{z}{\rho\e{-\frac{\rho^2}{2\sigma^2}}}{\rho}
\]
una primitiva (a meno del segno) è proprio l'integranda $-\e{-\frac{\rho^2}{2\sigma^2}}=-\e{-\frac{\rho^2}{2\sigma^2}}(-\frac{2\rho}{2\sigma^2})=\frac{\rho}{\sigma^2}\e{-\frac{\rho^2}{2\sigma^2}}$ quindi invertendo gli estremi di integrazione si ha
\[
	F_Z(z)=\intd{0}{z}{\frac{1}{\sigma^2}\rho\e{-\frac{\rho^2}{2\sigma^2}}}{\rho}=\bound{z}{0}{\e{-\frac{\rho^2}{2\sigma^2}}}=1-\e{-\tfrac{z^2}{2\sigma^2}}
\]

La funzione densità di probabilità di $Z=\sqrt{X^2+Y^2}$ (\textsc{variabile aleatoria di Rayleigh})\index{variabile aleatoria!di Rayleigh}\label{eq:Rayleigh}
\begin{equation}
\label{eq:formula_rayleigh}
	f_Z(z)=\deriv{F_Z(z)}{z}=\frac{z}{\sigma^2}\e{-\tfrac{z^2}{2\sigma^2}}
\end{equation}

Il momento del primo ordine, valor medio
\[
	\begin{split}
	\mu_Z&=\intd{0}{+\infty}{z f_Z(z)}{z}=\intd{0}{+\infty}{\frac{z^2}{\sigma^2}\e{-\tfrac{z^2}{2\sigma^2}}}{z}=\\
\intertext{si integra per parti sul modulo $z$, ponendo $u=z, \diff u=\diff z, \diff v=\frac{z}{\sigma^2}\e{-\frac{z^2}{2\sigma^2}}, v=-\e{-\frac{z^2}{2\sigma^2}}$}
	&=\underbrace{\bound{0}{+\infty}{-z\e{-\frac{z^2}{2\sigma^2}}}}_{=0}+\intd{0}{+\infty}{\e{-\frac{z^2}{\sigma^2}}}{z}=\frac{\sqrt{2\pi}\sigma}{2}\intinf{\frac{1}{\sqrt{2\pi}\sigma}\e{-\frac{z^2}{2\sigma^2}}}{z}=\sigma\sqrt{\frac{\pi}{2}}
\end{split}
\]

Il momento del secondo ordine, potenza
\[	
\begin{split}
	\E{Z^2}&=\intd{0}{+\infty}{z^2 f_Z(z)}{z}=\intd{0}{+\infty}{\frac{z^3}{\sigma^2}\e{-\frac{z^2}{2\sigma^2}}}{z}=\\
\intertext{si integra per parti sul modulo $z$, ponendo $u=z^2, \diff u=2z\diff z, \diff v=\frac{z}{\sigma^2}\e{-\frac{z^2}{2\sigma^2}}, v=-\e{-\frac{z^2}{2\sigma^2}}$ in $\int v\diff v=u v-\int v\diff u$}
	&=\underbrace{\bound{0}{+\infty}{-z^2\e{-\frac{z^2}{2\sigma^2}}}}_{=0}+\sigma^2\intd{0}{+\infty}{\frac{2z}{\sigma^2}\e{-\frac{z^2}{2\sigma^2}}}{z}=2\sigma^2\bound{0}{+\infty}{-\e{-\frac{z^2}{2\sigma^2}}}=2\sigma^2
\end{split}
\]

La varianza
\[
	\sigma^2_Z=\E{Z^2}-\Esp^2[Z]=2\sigma^2-\frac{\pi}{2}\sigma^2=\frac{4-\pi}{2}\sigma^2
\]
\end{esempio}

\begin{esempio}
Data una variabile aleatoria di Rayleigh $Z$ ho la variabile aleatoria trasformata $W=Z^2$ con $Z\geq 0$ con densità di probabilità $f_W(w)$. La funzione $g(z)=z^2$ è invertibile per $z\geq 0$ pertanto
\begin{align*}
	f_W(w)&=\restrict{\frac{f_Z(z)}{\abs{g'(z)}}}{z=g^{-1}(w)} \\
	z&=g^{-1}(w)=\sqrt{w}\quad g'(z)=2z \implies \\
	f_W(w)&=\frac{f_Z(\sqrt{w})}{2\sqrt{w}}=\frac{1}{2\sqrt{w}}\frac{\sqrt{w}}{\sigma^2}\e{-\frac{w}{2\sigma^2}}=\frac{1}{2\sigma^2}\e{-\frac{w}{2\sigma^2}}=\lambda\e{-\lambda w},\,\lambda=\frac{1}{2\sigma^2}
\end{align*}
si ottiene una densità di probabilità di variabile aleatoria esponenziale.

Il valor medio e la varianza pari a $1/\lambda$ sono
\[
	\mu_W=2\sigma^2\qquad\sigma^2_W=2\sigma^2
\]
\end{esempio}

\begin{esempio}
Sia data la densità di probabilità
\[
	f_{(X,Y)} (x,y) = \begin{cases}
		2e^{-(x+y)}, & 0 \leq x \leq y,\, y \geq 0 \\
		0, & \text{altrove}
	\end{cases}
\]

\begin{figure}[!h]
	\centering
	\begin{tikzpicture}[scale=.7]
		\begin{axis}[axis lines=middle,no markers,enlargelimits,xscale=1.2,xtick={0},ytick={0},,xlabel={$x$},ylabel={$y$}]
			\addplot [very thick,samples=200, domain=-2:2] {x};
			\addplot [pattern=north west lines,draw=none] coordinates {(0,0) (0,2) (2,2)};
			\node[right] at (1.6,1.5) {$y=x$};
		\end{axis}
	\end{tikzpicture}
\end{figure}
dove l'area tratteggiata indica la regione del piano dove la densità non è nulla.

\paragraph{Calcolo la densità marginale in $x$ e $y$}
\begin{align*}
	f_X (x) &= \intd{0}{+\infty}{f_{XY} (x,y)}{y} =
		\intd{x}{+\infty}{2 \e{-x} \e{-y}}{y} =
		2\e{-x} \bound{x}{+\infty}{-\e{-y}} =
		2\e{-2x} u(x) \\
	f_Y (y) &= \intd{0}{+\infty}{f_{XY} (x,y)}{x} =
		\intd{0}{y}{2 \e{-y} \e{-x}}{x} =
		2\e{-y} \bound{0}{y}{-\e{-x}} =
		2\e{-y} \bigl(1 - \e{-y}\bigr) u(y)
\end{align*}
$f_{(X,Y)}(x,y) \neq f_X (x) f_Y (y) \implies X$ e $Y$ non sono indipendenti.

\paragraph{Calcolo il coefficiente di correlazione}
\[	
	\rho_{XY} = \frac{r_{XY} - \mu_X \mu_Y}{\sigma_X \sigma_Y}
\]
Al fine di calcolare il coefficiente di correlazione, calcolo la correlazione
$r_{XY}$, le medie $\mu_X$ e $\mu_Y$ e le deviazioni standard $\sigma_X$ e $\sigma_Y$
\begin{align*}
	r_{XY} &= \intinf{\intinf{xy f_{XY} (x,y)}{x}}{y} =
	\intd{0}{+\infty}{\intd{0}{y}{xy 2\e{-(x+y)}}{x}}{y} =
	\intd{0}{+\infty}{2y \e{-y} \intd{0}{y}{x \e{-y}}{x}}{y} = \\
	&= \intd{0}{+\infty}{2y\e{-y} \Biggl( \bound{0}{y}{-\e{-x}x} + \intd{0}{y}{\e{-x}}{x} \Biggr)}{y} =
	\intd{0}{+\infty}{2y\e{-y} \bigl(-\e{-y}y -\e{-y} +1\bigr)}{y} = \\
	&= 2\intd{0}{+\infty}{\underbrace{-y^2\e{-2y}-y\e{-2y}}_{- 1/2} + \underbrace{y\e{-y}}_1}{y} = 1
\end{align*}
Infatti, spezzando l'integrale in 2 integrali più semplici, risulta:
\[
	\intd{0}{+\infty}{y\e{-y}}{y} =
	\underbrace{\bound{0}{+\infty}{-y\e{-y}}}_0 + \intd{0}{+\infty}{\e{-y}}{y} =
	\bound{0}{+\infty}{-\e{-y}} = 1
\]
\begin{align*}
	\intd{0}{+\infty}{-y^2\e{-2y} -y\e{-2y}}{y} &= -\intd{0}{+\infty}{\underbrace{(y^2+y)}_{\text{fatt. diff.}} \underbrace{(\e{-2y})}_{\text{fatt. finito}}}{y} = \\
	&= \underbrace{\bound{0}{+\infty}{-\frac{1}{2}(y^2 +y)\e{-2y}}}_0 + \intd{0}{+\infty}{\frac{1}{2}\e{-2y}(2y+1)}{y} = \\
	&= -\frac{1}{2} \underbrace{\bound{0}{+\infty}{-\frac{1}{2}\e{-2y}(2y+1)}}_{1/2} + \intd{0}{+\infty}{\frac{1}{2}\e{-2y}(2)}{y} = \\
	&= -\frac{1}{2} \Biggl( \frac{1}{2} + \bound{0}{+\infty}{-\frac{1}{2}\e{-2y}} \Biggr) = -\frac{1}{2} \Bigl(\frac{1}{2} + \frac{1}{2}\Bigr) = -\frac{1}{2}
\end{align*}

\paragraph{Calcolo media e potenza sfruttando le funzioni generatrici dei momenti}
\begin{align*}
	G_X (t) &= \E{\e{tX}} = \intd{0}{+\infty}{\e{tx} 2 \e{-2x}}{x} = 2\intd{0}{+\infty}{\e{(t-2)x}}{x} = \frac{2}{t-2} \bound{0}{+\infty}{\e{(t-2)x}} = \frac{2}{2-t} \\
	G_Y (t) &= \E{\e{tY}} = \intd{0}{+\infty}{\e{ty} 2 \e{-y} \bigl(1 - \e{-y}\bigr)}{y} = 2\Biggl( \intd{0}{+\infty}{\e{y(t-1)}}{y} - \intd{0}{+\infty}{\e{y(t-2)}}{y} \Biggr) = \\
		&= -\frac{2}{t-1} + \frac{2}{t-2} = 2(t-2)^{-1} -2(t-1)^{-1} \\
	G'_X(t) &= -2\restrict{(2-t)^2}{t=0} = -\frac{1}{2} = \mu_X \\
	G''_X(t)&= 4\restrict{(2-t)^{-3}}{t=0} = \frac{1}{2} = m^2_X \\
	G'_Y(t) &= \restrict{-2(t-2)^{-2} - 2(t-2)^{-2}}{t=0} = \frac{3}{2} = \mu_Y \\
	G''_Y(t)&= -\frac{1}{2} + 4 = \frac{7}{2} = m^2_Y
\end{align*}

\paragraph{Calcolo varianza e deviazione standard}
\begin{align*}
	\sigma^2_X &= m^2_X - \mu^2_X = \frac{1}{2} - \frac{1}{4} = \frac{1}{4} \implies \sigma_X = \frac{1}{2} \\
	\sigma^2_Y &= m^2_Y - \mu^2_Y = \frac{7}{2} - \frac{9}{4} = \frac{5}{4} \implies \sigma_Y = \frac{\sqrt{5}}{2}
\end{align*}

Quindi il coefficiente di correlazione risulta essere
\[
	\rho_{XY}=\frac{r_{XY}-\mu_X \mu_Y}{\sigma_X \sigma_Y} =
	\frac{1-\frac{1}{2} \frac{3}{2}}{\frac{1}{2}\frac{\sqrt{5}}{2}} =\frac{\sqrt{5}}{5}
\]
\end{esempio}

\section{Vettori di variabili aleatorie}
\`{E} possibile generalizzare lo studio a sistemi di $n$ variabili aleatorie $X_1, X_2,\dots,X_n$ che saranno indicati con notazione vettoriale
\begin{equation}
	\vect{X}=\begin{bmatrix}X_1\\X_2\\\dots\\X_n\end{bmatrix}=\trasp{\begin{bmatrix}X_1&X_2&\dots&X_n\end{bmatrix}}
\end{equation}
Si definisce la funzione di distribuzione di probabilità congiunta
\begin{equation}
	F_\vect{X}(\bar{x})=F_{X_1,X_2,\dots,X_N}(x_1,x_2,\dots,x_n)=P(X_1\leq x_1,X_2\leq x_2,\dots,X_n\leq x_n)
\end{equation}
e la funzione densità di probabilità congiunta
\begin{equation}
	f_{X_1,X_2,\dots,X_N}(x_1,x_2,\dots,x_n)=\frac{\partial^n F_{X_1,X_2,\dots,X_N}(x_1,x_2,\dots,x_n)}{\partial x_1\partial x_2\dots\partial x_n}
\end{equation}
Dalla densità di probabilità congiunta è possibile ricavare la densità di probabilità marginale di ciascuna variabile integrando sulle restanti variabili. Le densità di probabilità congiunte di un sottoinsieme di variabili si ricava integrando rispetto alle variabili non presenti nell'insieme. Ad esempio
\begin{equation}
	f_{X_1,X_3,\dots,X_N}(x_1,x_3,\dots,x_n)=\intinf{f_{X_1,X_3,\dots,X_n}(x_1,x_3,\dots,x_n)}{x_2}
\end{equation}
Analogamente per funzioni densità di probabilità condizionate si divide la densità di probabilità congiunta delle variabili condizionate per la densità marginale ristretta alle variabili condizionanti
\begin{equation}
	f_{X_1,X_4,\dots,X_N|X_2,X_3}(x_1,x_4,\dots,x_n|x_2,x_3)=\frac{f_{X_1,X_4,\dots,X_N}(x_1,x_4,\dots,x_n)}{f_{X_2,X_3}(x_2,x_3)}
\end{equation}
Le variabili aleatorie sono indipendenti se la densità di un sottogruppo condizionata al sottogruppo complementare è pari alla relativa densità non condizionata.

Definite le funzioni distribuzione di probabilità congiunta $F_\vect{X}(\bar{x})$ e densità di probabilità congiunta $f_\vect{X}(\bar{x})$ possono essere indicati con la stessa notazione vettoriale gli indici statistici caratteristici.

Il valor medio $\mu_\vect{X}$ del vettore $\vect{X}$ è pari al vettore colonna dei valori attesi
\begin{equation}
	\mu_\vect{X}=\E{\vect{X}}=\trasp{\begin{bmatrix}\mu_{X_1}&\mu_{X_2}&\dots&\mu_{X_n}\end{bmatrix}}
\end{equation}

Dato il vettore $\vect{X}$ di variabili aleatorie si definisce la correlazione e la covarianza fra tutte le coppie di variabili aleatorie del vettore. Si possono esprimere in notazione matriciale come \textsc{matrice di correlazione} e \textsc{matrice di covarianza}
\begin{equation}
\label{eq:matrice_correlazione}
	R_\vect{X}=\E{\vect{X}\trasp{\vect{X}}}=\begin{bmatrix}r_{X_1,X_1}&r_{X_1,X_2}&\dots&r_{X_1,X_n}\\r_{X_2,X_1}&r_{X_2,X_2}&\dots&r_{X_2,X_n}\\\vdots&\vdots&\ddots&\vdots\\r_{X_n,X_1}&r_{X_n,X_2}&\dots&r_{X_n,X_n}\end{bmatrix}
\end{equation}
\begin{equation}
\label{eq:matrice_covarianza}
	C_\vect{X}=\E{(\vect{X}-\mu_\vect{X})\trasp{(\vect{X}-\mu_\vect{X})}}=\begin{bmatrix}c_{X_1,X_1}&c_{X_1,X_2}&\dots&c_{X_1,X_n}\\c_{X_2,X_1}&c_{X_2,X_2}&\dots&c_{X_2,X_n}\\\vdots&\vdots&\ddots&\vdots\\c_{X_n,X_1}&c_{X_n,X_2}&\dots&c_{X_n,X_n}\end{bmatrix}=R_\vect{X}-\mu_\vect{X}\trasp{\mu_\vect{X}}
\end{equation}

Le matrici di dimensione $n\times n$ sono simmetriche, essendo
\[
	r_{X_i,X_j}=r_{X_j,X_i}\qquad c_{X_i,X_j}=c_{X_j,X_i}
\]

I valori sulla diagonale di $R_\vect{X}$ sono i valori quadratici medii delle singole variabili aleatorie $X_i$ (eq.\ref{eq:v_a_valore_quad_medio}): $r_{X_i,X_i}=\E{X^2_i}=m^2_{X_i}$.

I valori sulla diagonale di $C_\vect{X}$ sono i valori delle varianze delle singole variabili aleatorie $X_i$ (eq.\ref{eq:v_a_varianza}): $c_{X_i,X_i}=\E{(X_i-\mu_{X_i})^2}=\sigma^2_{X_i}$.

Si ottiene un vettore aleatorio $\vect{Y}=g(\vect{X})$ applicando una funzione vettoriale $g(\cdot,\cdot,\dots,\cdot)$ che trasforma le $n$ variabili aleatorie di $\vect{X}$. Data la funzione densità di probabilità congiunta $f_\vect{X}$ si può calcolare con il teorema fondamentale generalizzato la
\begin{equation}
	f_\vect{Y}(\bar{y})=\sum_{i}\restrict{\frac{f_\vect{X}(x)}{\abs{\det J(\bar{x})}}}{x=\bar{x}_i}
\end{equation}
dove $\bar{x}_i$ sono le soluzioni per un dato $\vect{y}$ del sistema di equazioni $g(\vect{x})=\vect{y}$ e dove il determinante della matrice Jacobiana della trasformazione
\[
	J(\vect{x})=\begin{bmatrix}
		\pderiv{g_1}{x_1}&\cdots&\pderiv{g_1}{x_n}\\
		\vdots&\ddots&\vdots\\
		\pderiv{g_n}{x_1}&\cdots&\pderiv{g_n}{x_n}\\
	\end{bmatrix}
\]
Un vettore aleatorio può essere trasformato in un vettore di dimensioni differenti. Nel caso è necessario determinare prima la funzione di distribuzione congiunta $F_Z(z)=\int\lim\limits_{R(z)}f_Z(\vect{x})\diff\vect{x}$ nel dominio che soddisfa la disuguaglianza data dalla trasformazione $R(z)=\{\vect{x}\in g(\vect{x})\leq \vect{Z}\}$. La funzione densità $f_Z(z)$ si ricava per derivazione.

Caso di particolare interesse è il vettore aleatorio trasformato in un'unica variabile aleatoria $Z=g(\vect{X})=g(X_1,X_2,\dots,X_n)$.
\begin{esempio}
Si ha la variabile aleatoria $Z=\sum_{i=1}^n X_i=\trasp{1}\vect{X}$, dove si è indicato con $\trasp{1}$ il vettore di 1.
Il valor medio e la varianza del vettore aleatorio trasformato in $Z$:
\[
	\mu_Z=\E{Z}=\E{\trasp{1}X}=\trasp{1}\E{X}=\trasp{1}\mu_X=\sum_i\mu_{X_i}
\]
\[
	\begin{split}\sigma^2_Z&=\E{(Z-\mu_Z)^2}=\E{\trasp{(Z-\mu_Z)}(Z-\mu_Z)}=\\
	&=\E{\trasp{(\trasp{1}\vect{X}-\trasp{1}\mu_\vect{X})}(\trasp{1}\vect{X}-\trasp{1}\mu_\vect{X})}=\\
	&=\E{(\trasp{\vect{X}}1-\trasp{\mu_\vect{X}}1)(\trasp{1}\vect{X}-\trasp{1}\mu_\vect{X})}=\\
	&=\E{(\underbrace{\trasp{\vect{X}}-\trasp{\mu_X}}_{1\times N})\underbrace{1\trasp{1}}_{N\times N}(\underbrace{\vect{X}-\mu_\vect{X}}_{N\times 1})}=\\
	&=\sum_i\sum_j c_{X_i,X_j}=C_\vect{X}\end{split}
\]

Nel caso di variabili aleatorie a due a due incorrelate (o indipendenti) si ha $c_{X_i,X_j}=0$ per $i\neq j$, si ha pertanto
\[
	\sigma^2_Z=\sum_{i=1}^{n}\sigma^2_{X_i}
\]
\end{esempio}

\begin{esempio}\label{es:v_a_congiuntamente_gaussiane}
Nel caso di variabili aleatorie congiuntamente gaussiane che compongono un vettore aleatorio $\vect{X}=\trasp{[X_1,X_2,\dots,X_n]}$ con le $X_i\sim\mathcal{N}(\mu_{X_i},\sigma^2_{X_i})$.

Se le variabili aleatorie sono indipendenti tra loro la funzione densità di probabilità congiunta
\begin{equation}
	f_\vect{X}(\bar{x})=\prod_{i=1}^{n}f_{X_i}(x_i)
\end{equation}
dove le densità di probabilità delle gaussiane
$f_{X_i}(x)=\frac{1}{\sqrt{2\pi\sigma^2_{X_i}}}\;\e{-\tfrac{(x-\mu_{X_i})^2}{2\sigma^2_{X_i}}}$ quindi
\[
	f_\vect{X}(\bar{x})=\frac{1}{\sqrt{(2\pi)^n\prod_{i=1}^{n}\sigma^2_{X_i}}}\;\e{-\frac{1}{2}\sum_{i=1}^{n}\tfrac{(x-\mu_{X_i})^2}{2\sigma^2_{X_i}}}
\]
che può essere riscritta tenendo conto del vettore dei valori medii $\mu_\vect{X}$ e che per l'indipendenza delle variabili il determinante della matrice di covarianza $\det C_\vect{X}=\prod_{i=1}^{n}c_{X_i}$
\[
	f_\vect{X}(\bar{x})=\frac{1}{\sqrt{(2\pi)^n\det\abs{ C_\vect{X}}}}\;\e{-\frac{1}{2}\trasp{(\vect{x}-\mu_\vect{X})}C^{-1}_\vect{X}(\vect{x}-\mu_\vect{X})}
\]

Tale espressione della densità di probabilità congiunta di $n$ variabili gaussiane resta valida anche in caso di variabili non indipendenti, dove la matrice di covarianza non sarà più diagonale.

Il vettore aleatorio gaussiano ha pertanto le seguenti proprietà
\begin{enumerate}
\item il comportamento statistico dato da $f_\vect{X}(\bar{x})$ è determinato dal vettore $\mu_\vect{X}$ e dalla matrice di covarianza $C_\vect{X}$
\item se le v.a. gaussiane $X_i$ sono incorrelate (elementi fuori diagonale $C_{X_i,X_j}=0$) si ha $f_\vect{X}(\bar{x})=\prod_{i=1}^{n}f_{X_i}(x_i)$ ovvero per le v.a. gaussiane la incorrelazione implica l'indipendenza.
\end{enumerate}
\end{esempio}

\section{Disuguaglianza di Chebyshev}
Anche senza conoscere la funzione di distribuzione di variabile aleatoria $X$ si può, grazie al seguente teorema, maggiorare la probabilità di discostarsi dal valor medio data la varianza.

Data la v.a. $X$ e $\forall\eta>0$ si ha
\begin{equation}
	P(\abs{X-\mu_X}>\eta)\leq\frac{\sigma^2}{\eta^2}
\end{equation}
\begin{proof}[Dim.]
Sia $Y$ la variabile aleatoria così definita
\[
	Y=\begin{cases}\eta^2&\abs{X-\mu_X}>\eta\\0&\abs{X-\mu_X}\leq\eta\end{cases}\quad\implies\abs{X-\mu_X}\geq Y
\]
che per come è definita implica $\abs{X-\mu_X}\geq Y$, infatti ho due casi: se $\abs{X-\mu_X}>\eta$ si ha $(\abs{X-\mu_X})^2>\eta^2=Y$, mentre se $\abs{X-\mu_X}\leq\eta$ la $Y=0$ e $\abs{X-\mu_X}\geq 0$.

Calcolando l'aspettazione di ambo i membri si ha
\[
	\sigma^2_X=\E{(X-\mu_X)^2}\geq \E{Y}=\eta^2 P(\abs{X-\mu_X}>\eta)
\]
\end{proof}

La disuguaglianza di Chebyshev è in grado di dare una misura quantitativa della dispersione, in termini di probabilità, di una variabile aleatoria attorno al suo valor medio attraverso la varianza; infatti questo teorema rende rigorosa l'interpretazione intuitiva di varianza come misura di dispersione attorno alla media. Un piccolo valore di varianza vuol dire piccola probabilità che X prenda valori lontani dalla media.
Tale disuguaglianza è fondamentale per dimostrare la seguente legge dei grandi numeri.

\section{Legge dei Grandi Numeri}
Date $n$ variabili aleatorie $X_i$ indipendenti, con la stessa legge, media $\mu$ e varianza $\sigma^2$.

Si definisce la variabile aleatoria che rappresenta la media di $n$ misure
\[
	\hat{X}_n=\frac{1}{n}\sum_{i=1}^{n}X_i
\]

Si ha che, considerando il limite per $n\to\infty$, per la disuguaglianza di Chebyshev,
\begin{equation}
\label{eq:legge_grandi_numeri}
	\lim\limits_{n\to\infty} P\left(\abs{\hat{X}_n-\mu}\geq\eta\right)=0\quad\forall\eta
\end{equation}

\begin{proof}[Dim.]
La variabile aleatoria $\hat{X}_n$ ha anch'essa media $\mu$:
\[
	\E{\hat{X}_n}=\E{\frac{1}{n}\sum_{i=1}^{n}X_i}=\frac{1}{n}\sum_{i=1}^{n}\E{X_i}=\frac{1}{n}n\mu=\mu
\]
e varianza pari a
\[
	\sigma^2_{\hat{X}_n}=\frac{1}{n^2}\sum_{i=1}^{n}\sigma^2_{X_i}=\frac{1}{n^2}n\sigma^2=\frac{\sigma^2}{n}\xrightarrow{n\to\infty}0
\]

Si ha quindi che la disuguaglianza di Chebyshev è maggiorata da una quantità che tende a zero per $n\to\infty$, il che dimostra la tesi
\[
	0\leq P\left(\abs{\hat{X}_n-\mu}>\eta\right)\leq \frac{\sigma^2_{\hat{X}_n}}{\eta^2}=\frac{\sigma^2}{n\eta^2} \xrightarrow{n\to\infty}0
\]
\end{proof}

\section{Teorema del Limite Centrale}
Si considerino $n$ variabili aleatorie indipendenti $X_i$ tutte con la stessa funzione densità di probabilità $f_{X_i}(x)=f_X(x)$ e quindi con medesima media $\mu$ e varianza $\sigma^2$. Sappiamo che la somma delle v.a. $S_n=\sum_{i=1}^{n}X_i$ avrà media pari alla somma delle medie e varianza pari alla somma delle varianze, ovvero $\mu_n=n\mu$ e $\sigma^2_n=n\sigma^2$, inoltre esse sono quantità che tendono a divergere al crescere di $n$.
Si consideri quindi la variabile aleatoria normalizzata
\[
	Z_n=\frac{S_n-\mu_n}{\sigma_n}=\frac{S_n-n\mu}{\sqrt{n}\sigma}
\]
che, indipendentemente da $n$, ha valor medio nullo e varianza unitaria, e quind è una gaussiana standard.

Il \textsc{teorema del limite centrale}\index{Teorema!del limite centrale} (di Lyapunov), ci dice quindi che la somma normalizzata di $n$ v.a. indipendenti con la stessa densità di probabilità, per $n$ che tende ad infinito, tende ad una gaussiana standard (a prescindere dalla distribuzione di ciascuna di esse)
\begin{equation}
	\lim\limits_{n\to\infty}f_{Z_n}(x)=f_\mathcal{N}(x)=\frac{1}{\sqrt{2\pi}}\e{-\frac{x^2}{2}}
\end{equation}
\begin{nota}
	Il teorema del limite è centrale, ovvero fondamentale, perché modella un fenomeno fisico importante come il rumore termico che è composto dalla sovrapposizione di moltissimi contributi elementari indipendenti che complessivamente sono descritti da una statistica gaussiana standard.
\end{nota}

\chapter{Processi stocastici}
Si considera un fenomeno aleatorio di cui si effettua una misura per un esperimento. L'evoluzione del segnale registrato non è predicibile a priori come in un segnale deterministico. Il segnale che viene effettivamente osservato si dice una realizzazione di un \keyword[processo stocastico]{processo stocastico}.

\begin{figure}[h!]
\begin{tikzpicture}
\draw[-latex] (0,0) -- (8,0) node[right]{$t$};
\draw[-latex] (0,0) -- (0,5);
\draw [decorate, decoration={random,segment length=5pt,amplitude=20pt}] (0,1) -- (8,1) node [right] {$x_3(t)$} ;
\draw [decorate, decoration={random,segment length=5pt,amplitude=20pt}] (0,2.5) -- (8,2.5) node [right] {$x_2(t)$};
\draw [decorate, decoration={random,segment length=5pt,amplitude=20pt}] (0,4) -- (8,4) node [right] {$x_1(t)$};
\draw (13,2.5) circle(2);
\node at (14.8,4) {$\Omega$};
\node (x1) at(13,3.8){};
\node (x2) at(12,2.2){};
\node (x3) at(13.5,1.2){};
\fill (x1) circle(2pt) node[below]{$\omega_1$};
\fill (x2) circle(2pt) node[below]{$\omega_2$};
\fill (x3) circle(2pt) node[below]{$\omega_3$};
\draw[-latex] (9.2,4) to[bend left](x1);
\draw[-latex] (9.2,2.5) to[bend left](x2);
\draw[-latex] (9.2,1) to[bend right](x3);
\draw[dashed] (4,5)--(4,0)node[below]{$t_0$};
\end{tikzpicture}
\caption{Processo aleatorio}
\end{figure}

In altre parole un processo aleatorio mette in corrispondenza un particolare campione nello spazio di probabilità, $\omega_i$ in $\Omega=\{\omega_i\}$, con una sua particolare realizzazione $x_i(t)$ che ha la sua particolare evoluzione temporale. Tale corrispondenza è riassunta nella scrittura
\begin{equation}X(t;\omega_i)=x_i(t)\end{equation}
In forma abbreviata, omettendo $\omega_i$ si indica con $X(t)$ il \textsc{processo aleatorio} per la generica realizzazione $x_i(t)$.
Una realizzazione del processo stocastico non è più aleatoria, a posteriori dell'osservazione essa è funzione deterministica del tempo.

Viceversa è possibile fissare un istante di tempo $t=t_0$ ed osservare il valore di tutte le realizzazioni del processo $X(t_0)$: i valori assunti dalle varie realizzazioni non sono predicibili a priori quindi rappresentano i risultati di una variabile aleatoria.

Un \textsc{processo aleatorio parametrico} è un processo in cui il segnale nel tempo ha uno o più parametri descritti da una variabile aleatoria. Ad esempio
\[X(t;\omega)=\e{-A(\omega)t}\step(t)\]
con $A(\omega)$ variabile aleatoria uniformemente distribuita nell'intervallo $[0,T]$.
Si può omettere l'indicazione di $\omega$
\[X(t)=\e{-A t}\step(t)\quad A\sim\text{unif.dist.} [0,T]\]
Ulteriore esempio, oscillatore reale a fase incerta
\[X(t)=A\sen{2\pi f_0 t+\phi}\quad \phi\sim\text{unif.dist.} [0,2\pi[\]

\section{Caratterizzazione statistica di processi aleatori}
Per un fissato istante $t_0$ , i valori che assumono tutte le realizzazioni di un processo $X(t)$ rappresentano una variabile aleatoria. Di questa variabile aleatoria si può definire una \textsc{funzione di distribuzione di probabilità del primo ordine} che dipende da $t_0$:
\begin{equation}
F_X(x;t_0)=P(X(t_0)\leq x)
\end{equation}

Per meglio caratterizzare un processo si può definire una \textsc{funzione di distribuzione di probabilità del secondo ordine o congiunta} per una coppia di variabili aleatorie in due istanti di tempo differenti $t_1$ e $t_2$:
\begin{equation}
F_X(x_1,x_2;t_1,t_2)=P(X(t_1)\leq x_1,X(t_2)\leq x_2)
\end{equation}

Per estensione una caratterizzazione completa di un processo stocastico richiede di determinare la \textsc{funzione di distribuzione di probabilità di ordine $n$}, fissati $n$ istanti di tempo e $n$ variabili aleatorie che si hanno estraendo i valori delle realizzazioni negli istanti $t_1,t_2,\dots,t_n$:
\begin{equation}
F_X(x_1,\dots,x_n;t_1,\dots,t_n)=P(X(t_1)\leq x_1,\dots,X(t_n)\leq x_n)
\end{equation}

Dalle funzioni di distribuzioni di probabilità si ricava la \textsc{funzione densità di probabilità di ordine $n$}:
\begin{equation}
f_X(x_1,\dots,x_n;t_1,\dots,t_n)=\frac{\partial^n F_X(x_1,\dots,x_n;t_1,\dots,t_n)}{\partial x_1,\dots,\partial x_n}
\end{equation}

La conoscenza delle funzioni $f_X$ per ogni $n$ e per ogni $n$-pla di istanti di tempo caratterizza completamente il processo aleatorio. Una conoscenza completa è un'impresa praticamente impossibile. Nella maggioranza dei casi si cerca di determinare le distribuzioni e densità di probabilità del primo e secondo ordine e alcuni parametri statistici.

\section{Parametri statistici di primo ordine}
I parametri statistici valor medio, potenza e varianza permettono di determinare le caratteristiche principali di un processo aleatorio.

\subsection{Funzione valor medio}
La \textsc{funzione valor medio} è definita come il valor medio della variabile aleatoria che si ottiene estraendo i valori delle realizzazioni nell'istante assegnato:
\begin{equation}\label{eq:proc_valor_medio}
\mu_X(t)=\E{X(t)}=\intinf{x f_X(x;t)}{x}
\end{equation}
La funzione valor medio è una statistica del primo ordine di $X(t)$ poiché il suo calcolo richiede la funzione densità di probabilità del primo ordine (funzione di una variabile aleatoria estratta dal processo e un istante di tempo).

\begin{esempio}
Si ha ad esempio il processo aleatorio parametrico
\[X(t)=a\cos{2\pi f_0 t+\Theta}\quad\Theta\sim\text{v.a.unif.} [0,\pi[\]
Le funzioni campione del processo sono segnali cosinusoidali aventi medesima ampiezza e frequenza ma fase iniziale incerta. Per un fissato valore di $t$ si ha una variabile aleatoria $a\cos{2\pi f_0 t+\Theta}$ ottenuta come trasformazione della variabile aleatoria $\theta$.
Per il teorema del valor medio della variabile aleatoria
\begin{equation}\label{eq:esempio_processo_non_stazionario_valor_medio}
\begin{split}\mu_X(t)&=\E{X(t)}=\E{a\cos{2\pi f_0 t+\Theta}}=\intinf{ a\cos{2\pi f_0 t+\theta}f_\Theta(\theta)}{\theta}=\\
&=\frac{a}{\pi}\intd{0}{\pi}{\cos{2\pi f_0 t+\theta}}{\theta}=\frac{a}{\pi}\bound{0}{\pi}{\sen{2\pi f_0 t+\theta}}=\\
&=\frac{a}{\pi}\bound{0}{\pi}{\sen{2\pi f_0 t}\Cos\theta+\cos{2\pi f_0 t}\Sen\theta}=-\frac{2a}{\pi}\sen{2\pi f_0 t}\end{split}\end{equation}

Il valor medio è una funzione del tempo $t$.
\end{esempio}

\begin{esercizio}Dato il processo aleatorio parametrico
\[X(t)=a\cos{2\pi f_0 t+\Theta}\]
con $\Theta$ variabile uniforme in $[0,2\pi[$ mostrare che il valor medio $\mu_X(t)=0$.
\end{esercizio}

\subsection{Potenza media statistica istantanea}
La \textsc{potenza media} è una grandezza statistica del primo ordine che caratterizza un processo $X(t)$
\begin{equation}
P_X(t)=\E{X^2(t)}=\intinf{x^2 f_X(x;t)}{x}
\end{equation}

\subsection{Funzione varianza}
La \textsc{funzione varianza} del processo aleatorio $X(t)$ è l'indice statistico del primo ordine
\begin{equation}
\sigma^2_X(t)=\E{(X(t)-\mu_X(t))^2}=\intinf{(x-\mu_X)^2 f_X(x;t)}{x}
\end{equation}
La relazione di dipendenza tra varianza, funzione valor medio e potenza istantanea
\begin{equation}
\sigma^2_X(t)=P_X(t)-\mu^2_X(t)
\end{equation}

\section{Parametri statistici di secondo ordine}\index{processo stocastico!parametri statistici di secondo ordine}
Due parametri statistici del secondo ordine, fondamentali per lo studio dei processi stocastici, sono la \textsc{funzione autocorrelazione} e la \textsc{funzione autocovarianza}.

Dati due istanti di tempo arbitrari $t_1$ e $t_2$ è possibile estrarre dal processo due variabili aleatorie $X_1=X(t_1)$ e $X_2=X(t_2)$.

\subsection{Funzione di auto-correlazione}
La \keyword[processo stocastico!parametri statistici di secondo ordine!funzione di autocorrelazione]{funzione di autocorrelazione} misura la correlazione tra due variabili aleatorie estratte da uno stesso processo in due istanti di tempo $t_1$ e $t_2$:
\begin{equation}\label{eq:proc_autocorrelazione}
R_X(t_1,t_2)=\E{X(t_1)X(t_2)}=\intinf{\intinf{x_1 x_2 f_X(x_1,x_2;t_1,t_2)}{x_1}}{x_2}
\end{equation}

\subsection{Funzione di auto-covarianza}
Analogamente, la \keyword[processo stocastico!parametri statistici di secondo ordine!funzione di autocovarianza]{funzione di auto-covarianza} misura la covarianza di due variabili aleatorie estratte da uno stesso processo in due istanti di tempo $t_1$ e $t_2$:
\begin{equation}\label{eq:proc_autocovarianza}\begin{split}
C_X(t_1,t_2)&=\E{(X(t_1)-\mu_X(t_1)(X(t_2)-\mu_X(t_2)}=\\
&=\intinf{\intinf{(x_1-\mu_X(t_1))(x_2-\mu_X(t_2)) f_X(x_1,x_2;t_1,t_2)}{x_1}}{x_2}=\\
&=R_X(t_1,t_2)-\mu_X(t_1)\mu_X(t_2)
\end{split}\end{equation}

\begin{esempio}\label{es:oscillatore_non_stazionario}
Si calcoli l'autocorrelazione del processo aleatorio\[X(t)=a\cos{2\pi f_0 t+\theta}\text{ con }\theta\text{ v.a. unif.dist. in }[0,\pi[\]
Estraendo dal processo negli istanti $t_1$ e $t_2$ le variabili aleatorie
\[X(t_1)=a\cos{2\pi f_0 t_1+\theta}\quad X(t_2)=a\cos{2\pi f_0 t_2+\theta}\]
e ricordando che $\Cos\alpha\Cos\beta=\frac{\cos{\alpha+\beta}+\cos{\alpha-\beta}}{2}$

\begin{equation}\label{eq:autocorrelazione_oscillatore_non_stazionario}
\begin{split}R_X(t_1,t_2)&=\E{X(t_1)X(t_2)}=\E{a\cos{2\pi f_0 t_1+\theta}a\cos{2\pi f_0 t_2+\theta}}=\\
&=\frac{a^2}{2}\E{\cos{2\pi f_0(t_1+t_2)+2\theta}+\underbrace{\cos{2\pi f_0(t_1-t_2)}}_{\text{non dipende da $\theta$}}}=\\
&=\frac{a^2}{2}\frac{1}{\pi}\intd{0}{\pi}{\cos{2\pi f_0(t_1+t_2)+2\theta}}{\theta}+\frac{a^2}{2}\cos{2\pi f_0(t_1-t_2)}=\\
&=\frac{a^2}{2\pi}\bound{0}{\pi}{\frac{\sen{2\pi f_0(t_1+t_2)+2\theta}}{2}}+\frac{a^2}{2}\cos{2\pi f_0(t_1-t_2)}=\\
&=\frac{a^2}{\pi}\bound{0}{\pi}{\frac{1}{2}\frac{\sen{2\pi f_0(t_1+t_2)}\cos{2\theta}}{2}+\frac{1}{2}\frac{\cos{2\pi f_0(t_1+t_2)}\sen{2\theta}}{2}}+\\
&\quad+\frac{a^2}{2}\cos{2\pi f_0(t_1-t_2)}=\\
&=\frac{a^2}{\pi}\underbrace{\left[\frac{1}{2}\frac{\sen{2\pi f_0(t_1+t_2)}}{2}-\frac{1}{2}\frac{\sen{2\pi f_0(t_1+t_2)}}{2}\right]}_{=0}+\frac{a^2}{2}\cos{2\pi f_0(t_1-t_2)}=\\
&=\frac{a^2}{2}\cos{2\pi f_0(t_1-t_2)}
\end{split}\end{equation}
con $\theta\sim U([0,\pi[)$ si ha $R_X(t_1,t_2)$ che è funzione della differenza $t_1-t_2$, quindi di una sola variabile.
\end{esempio}
\begin{esempio}\label{es:oscillatore_stazionario}
Si supponga di avere lo stesso processo aleatorio dell'esempio precedente ma con variabili aleatoria distribuita uniformemente in $[0,2\pi[$. Si voglia calcolare la funzione valor medio, la funzione di autocorrelazione e autocovarianza.
\[\mu_X(t)=\E{X(t)}=\intd{0}{2\pi}{\frac{1}{2\pi}a\cos{2\pi f_0 t+\theta}}{\theta}=0\]
Essendo la funzione valor medio nulla allora
\begin{equation}\label{eq:autocorrelazione_oscillatore_stazionario}
\begin{split}C_X(t_1,t_2)&=R_X(t_1,t_2)=\E{X(t_1)X(t_2)}=\\&=\intd{0}{2\pi}{\frac{1}{2\pi}a\cos{2\pi f_0 t_1+\theta}a\cos{2\pi f_0 t_2+\theta}}{\theta}=\frac{a^2}{2}\cos{2\pi f_0(t_1-t_2)}\end{split}\end{equation}
che è il medesimo risultato dell'esempio precedente.
\end{esempio}

\begin{esempio}
Si ha il processo aleatorio oscillatorio con ampiezza variabile uniformemente distribuita in $[0,1]$:
\[X(t)=A\cos{2\pi f_0 t}\text{ con }A\sim U([0,1]) \]
La funzione valor medio, come parametro del primo ordine funzione del tempo $t$:
\[\mu_X(t)=\E{X(t)}=\E{A\cos{2\pi f_0 t}}=\E{A}\cos{2\pi f_0 t}=\frac{1}{2}\cdot\cos{2\pi f_0 t}\]
Il parametro di secondo ordine, la funzione di autocorrelazione, funzione di $t_1$ e $t_2$:
\[\begin{split}R_X(t_1,t_2)&=\E{X(t_1)X(t_2)}=\E{A\cos{2\pi f_0 t_1}A\cos{2\pi f_0 t_2}}=\\&=\E{A^2}\cos{2\pi f_0 t_1}\cos{2\pi f_0 t_2}=\frac{1}{3}\cos{2\pi f_0 t_1}\cos{2\pi f_0 t_2}\end{split}\]
essendo $\E{A^2}=\intd{0}{1}{x^2}{x}=\frac{1}{3}$. In questo caso la funzione autocorrelazione è funzione delle due variabili $t_1$ e $t_2$, e non della differenza $t_1-t_2$.

La funzione autocovarianza, dai parametri precedenti, si valuta
\[\begin{split}C_X(t_1,t_2)&=R_X(t_1,t_2)-\mu_X(t_1)\mu_X(t_2)=\\
&=\frac{1}{3}\cos{2\pi f_0 t_1}\cos{2\pi f_0 t_2}-\frac{1}{2}\cos{2\pi f_0 t}\cdot\frac{1}{2}\cos{2\pi f_0 t}=\\&=\frac{1}{12}\cos{2\pi f_0 t_1}\cos{2\pi f_0 t_2}\end{split}\]
\end{esempio}

\section{Processo di Bernoulli e suoi derivati}\index{processo aleatorio!di Bernoulli}
Si consideri il processo aleatorio tempo discreto di Bernoulli, le cui realizzazioni indipendenti assumono valori $I_n=\{0,1\}$ in istanti di tempo discreti indicati con $n\in\N$. I valori assunti si presentano con probabilità
\begin{equation}I_n=\begin{cases}0&p\\1&1-p\end{cases}\end{equation}
La funzione valor medio vale
\begin{equation}\mu_I(n)=\E{I_n}=p\cdot 0+(1-p)\cdot 1=1-p\end{equation}
La varianza, calcolando il valore quadratico medio,
\[\E{I^2_n}=p\cdot 0^2+(1-p)\cdot 1^2=1-p\]
\begin{equation}\sigma^2_I=\E{I^2_n}-\Esp^2[I_n]=(1-p)-(1-p)^2=p(1-p)\end{equation}
La funzione di autocorrelazione, essendo le realizzazioni indipendenti:
\begin{equation}R_I(n,m)=\E{I_n I_m}=\E{I_n}\E{I_m}=(1-p)^2\end{equation}
Si possono definire processi che sono trasformazioni del processo di Bernoulli, ad esempio
\begin{equation}D_n=2I_n-1=\begin{cases}-1&p\\1&1-p\end{cases}\end{equation}
La funzione valor medio vale
\begin{equation}\mu_D(n)=\E{D_n}=p\cdot -1+(1-p)\cdot 1=1-2p\end{equation}
La varianza, calcolando il valore quadratico medio,
\[\E{D^2_n}=p\cdot -1^2+(1-p)\cdot 1^2=p+1-p=1\]
\begin{equation}\sigma^2_D=\E{D^2_n}-\Esp^2[D_n]=1-(1-2p)^2=1-1+4p-4p^2=4p(1-p)\end{equation}
La funzione di autocorrelazione, essendo le realizzazioni indipendenti:
\begin{equation}R_D(n,m)=\E{D_n D_m}=\E{D_n}\E{D_m}=(1-2p)^2\end{equation}

\subsection{Random walk unidimensionale}
Dal processo precedente si deriva il processo \textsc{random walk unidimensionale}
\begin{equation}S_n=D_1+D_2+\dots+D_n\end{equation}
La funzione valor medio vale
\begin{equation}\mu_S(n)=\E{S_n}=n\E{D_n}=n(1-2p)\end{equation}
che dipende da $n$. Essendo i processi indipendenti la varianza è somma delle varianze
\begin{equation}\sigma^2_S(n)=\sum_{i=1}^{n}\sigma^2_{D_i}(n)=n\,4p(1-p)\end{equation}
La funzione di autocorrelazione vale
\begin{equation}R_S(n,m)=\E{S_n S_m}=\E{\sum D_n \sum D_m}=\sum_{j=1}^{n}\sum_{i=1}^{m}\E{D_j D_i}=n m(1-2p)^2\end{equation}
Nella doppia sommatoria precedente si ha $n\times m$ prodotti $\E{D_j}\E{D_i}$ di cui $n\cdot m-n$ $\E{D_j}\E{D_i}$ e $n$ $\E{D^2_i}=1$. [TODO:controllare appunti diversi]

Il range di valori che può assumere il processo è variabile con $n$. $S_n$ con $n$ fissato può assumere valori compresi tra $[-n,n]$. La probabilità che k valori siano pari a $+1$ e $n-k$ valori siano pari a $-1$ è dato dalla formula di Bernoulli per la probabilità
\[P(S_n=k-(n-k)=2k-n)=\binom{n}{k}(1-p)^k p^{n-k}\]

\subsection{Segnale telegrafico casuale}
Una variante del processo di Bernoulli è il \textsc{segnale telegrafico casuale}. Le realizzazioni del processo sono funzioni continue che assumono solo valori discreti pari a $-1$ e $1$ e cambiano stato secondo il modello degli eventi aleatori o arrivi di un processo di Poisson con intensità $\alpha$. Inoltre si suppone che lo stato iniziale sia equiprobabile: $P(X(0)=1)=P(X(0)=-1)=\frac{1}{2}$

\begin{figure}[h!]
\centering\begin{tikzpicture}
\begin{axis}[axis lines=middle,no markers,enlargelimits,xscale=1.5,xtick={0},xlabel={$t$},ytick={-1,1}]
\addplot [very thick]coordinates {(0,1)(1,1)(1,-1)(2,-1)(4,-1)(4,1)(5,1)(5,-1)(6,-1)(6,1)(8,1)(8,-1)(10,-1)(10,1)(11,1)};
\end{axis}\end{tikzpicture}
\end{figure}

Si calcola la probabilità che ad un dato istante $t$ la singola realizzazione abbia uno dei due valori:
\begin{equation}\begin{split}P(X(t)=1)=&P(X(t)=1|X(0)=1)P(X(0)=1)+\\&+P(X(t)=1|X(0)=-1)P(X(0)=-1)\end{split}\end{equation}
in cui la prima probabilità condizionata si ha per un numero di eventi pari di Poisson, mentre la seconda si ha per un numero di eventi dispari, ovvero
\begin{equation}\begin{split}P(X(t)=1|X(0)=1)&=\sum_{k=0}^{+\infty}{\frac{(\alpha t)^{2k}}{(2k)!}\e{-\alpha t}}=\\
&=\e{-\alpha t}\sum_{k=0}^{+\infty}{\frac{1}{2}\left[\frac{(\alpha t)^k}{k!}+\frac{(-\alpha t)^k}{k!}\right]}=\\
&=\frac{\e{-\alpha t}}{2}(\e{\alpha t}+\e{-\alpha t})=\frac{1}{2}(1+\e{-2\alpha t})
\end{split}\end{equation}
\begin{equation}\begin{split}P(X(t)=1|X(0)=-1)&=\sum_{k=0}^{+\infty}{\frac{(\alpha t)^{2k+1}}{(2k+1)!}\e{-\alpha t}}=\\
&=\e{-\alpha t}\sum_{k=0}^{+\infty}{\frac{1}{2}\left[\frac{(\alpha t)^k}{k!}-\frac{(-\alpha t)^k}{k!}\right]}=\\
&=\frac{\e{-\alpha t}}{2}(\e{\alpha t}-\e{-\alpha t})=\frac{1}{2}(1-\e{-2\alpha t})
\end{split}\end{equation}
da cui
\begin{equation}P(X(t)=1)=\frac{1}{2}\left[\frac{1}{2}(1+\e{-2\alpha t})+\frac{1}{2}(1-\e{-2\alpha t})\right]=\frac{1}{2}\end{equation}
Analogamente si ha $P(X(t)=-1)=\frac{1}{2} \implies$ \text{processo senza memoria}.

Si calcola la funzione valor medio e la funzione varianza del processo:
\begin{equation}\mu_X(t)=\E{X(t)}=\frac{1}{2}\cdot(-1)+\frac{1}{2}\cdot(+1)=0\end{equation}
\begin{equation}\sigma^2(t)=\E{X^2(t)}-0=\frac{1}{2}(-1)^2+\frac{1}{2}(+1)^2=1\end{equation}

La funzione di autocorrelazione e la funzione di autocovarianza, data la funzione valor medio, sono \[C_X(t_1,t_2)=R_X(t_1,t_2)=\E{X(t_1)X(t_2)}\]
il prodotto $X(t_1)X(t_2)$ può assumere solo due valori $+1$ o $-1$ a seconda del numero pari o dispari di arrivi nell'intervallo $[t_1,t_2[$, come si è visto con probabilità
\[\begin{split}P(X(t_1)X(t_2)=1)&=\frac{1}{2}(1+\e{-2\alpha(t_2-t_1)})\\
P(X(t_1)X(t_2)=-1)&=\frac{1}{2}(1-\e{-2\alpha(t_2-t_1)})\end{split}\]
pertanto si ha che
\begin{equation}\begin{split}\E{X(t_1)X(t_2)}=\frac{1}{2}\left[(+1)(1+\e{-2\alpha(t_2-t_1)})+(-1)(1+\e{-2\alpha(t_2-t_1)})\right]=\e{-2\alpha(t_2-t_1)}\end{split}\end{equation}
ovvero funzione autocorrelazione e autovarianza dipendono dalla differenza dei due istanti di tempo generici e non dagli istanti stessi.

\section{Processi stazionari}
I \keyword[processo stazionario]{processi stazionari} hanno la notevole proprietà di mantenere costanti i parametri statistici determinati in $X(t)$ e in $X(t+\Delta t)$.

In generale si è visto che le funzioni densità di probabilità congiunta di ordine $n$ e i momenti di ordine $n$ dipendono dalla $n$-pla degli istanti di tempo $t_1,t_2,\dots,t_n$.

\subsection{Processo stazionario in senso stretto (SSS)}
Un processo aleatorio $X(t)$ è \keyword[processo stazionario!in senso stretto]{stazionario in senso stretto (SSS)} se le funzioni densità di probabilità congiunta di ogni ordine siano invarianti ad una traslazione rigida degli istanti di tempo, ovvero che per ogni ordine $n$
\begin{equation}
\forall\Delta t\quad f_X(x_1,\dots,x_n;t_1,\dots,t_n)=f_X(x_1,\dots,x_n;t_1+\Delta t,\dots,t_n+\Delta t)
\end{equation}
Come corollario si ha che la stazionarietà di ordine $n$ implica la stazionarietà di ogni ordine inferiore.
Un processo stazionario in senso stretto ha quindi indici statistici che non distinguono il processo $X(t)$ dal processo $X(t+\Delta t)$.

Questo vuol dire ad esempio che la funzione densità di probabilità del primo ordine di un processo SSS è invariante rispetto al tempo $t$:
\[f_X(x;t)=f_X(x;t+\Delta t)\quad\forall\Delta t\]
Di conseguenza tutte gli indici statistici del primo ordine non dipendono dal tempo $t$:
\[\mu_X(t)=\mu_X\quad \sigma^2_X(t)=\sigma^2_X\]

Estraendo da un processo aleatorio stazionario in senso stretto di ordine 2 negli istanti di tempo $t_1$ e $t_2$ le variabili aleatorie $X(t_1)$ e $X(t_2)$ si ha la densità di probabilità congiunta
\[f_X(x_1,x_2;t_1,t_2)=f_X(x_1,x_2;t_1+\Delta t,t_2+\Delta t)\quad\forall\Delta t\]
ovvero la funzione densità di probabilità congiunta non dipende dagli istanti di tempo separatamente ma dalla differenza dei due
\begin{equation}
f_X(x_1,x_2;t_1,t_2)=f_X(x_1,x_2;t_1-t_2)
\end{equation}
Di conseguenza tutti gli indici statistici del secondo ordine, funzione autocorrelazione e autocovarianza, dipendono non dagli istanti di tempo ma dalla differenza $t_1-t_2$:
\[R_X(t_1,t_2)=R_X(t_1-t_2)\quad C_X(t_1,t_2)=C_X(t_1-t_2)\]

In generale la densità di probabilità congiunta e tutte le statistiche di ordine $n$ di un processo stazionario in senso stretto dipenderanno solo dalle differenze $t_1-t_2, t_2-t_3, \dots, t_{n-1}-t_n$, che restano invariate in una traslazione rigida dei tempi.

\begin{nota}Per gli scopi ingegneristici la richiesta di stazionarietà in senso stretto è di difficile applicazione perché raramente le funzioni densità di probabilità di ogni ordine sono date in forma chiusa (notevole eccezione sono i processi gaussiani). Inoltre raramente si è interessati ad indici statistici di ordine superiore a due.
\end{nota}

\subsection{Processo stazionario in senso lato (SSL)}
Una definizione di stazionarietà meno restrittiva e più semplice da verificare:

Un processo aleatorio $X(t)$ è \keyword[processo stazionario!in senso lato]{stazionario in senso lato} se
\begin{equation}\label{eq:processo_stazionario_senso_lato}
\begin{cases}
\mu_X(t)=\mu_X\\
R_X(t_1,t_2)=R_X(t_1-t_2)
\end{cases}
\end{equation}
La funzione di autocorrelazione si può scrivere, ponendo $t_1=t$ e $t_2=t-\tau$, come \begin{equation}R_X(t_1,t_2)=R_X(t,t-\tau)=\E{X(t)X(t-\tau)}=R_X(\tau)\end{equation}
In tal caso anche la funzione autocovarianza dipende da $\tau=t_1-t_2$
\begin{equation}C_X(t_1,t_2)=R_X(t_1-t_2)-\mu^2_X=R_X(\tau)-\mu^2_X\end{equation}

Non è richiesta alcuna proprietà di invarianza delle densità di probabilità che possono anche non essere conosciute in forma chiusa.
La stazionarietà di ordine 2 è condizione sufficiente per avere un processo SSL.

La stazionarietà in senso lato non implica la stazionarietà in senso stretto.

\begin{esempio}
Si è visto un esempio di processo aleatorio non stazionario definito dall'oscillatore
\[X(t)=a\cos{2\pi f_0 t+\theta}\]
con fase $\theta$ variabile aleatoria a distribuzione uniforme in $[0,\pi[$
di cui si è calcolata la funzione valor medio funzione del tempo $t$ (eq.\ref{eq:esempio_processo_non_stazionario_valor_medio})
\[\mu_X(t)=-\frac{2a}{\pi}\sen{2\pi f_0 t}\]
Il processo è non stazionario nonostante l'autocorrelazione (eq.\ref{eq:autocorrelazione_oscillatore_non_stazionario}) sia funzione di $t_1-t_2$.
\end{esempio}

\begin{esempio}\label{es:oscillatore_stazionario_senso_lato}
Esempio di processo aleatorio stazionario in senso lato è l'oscillatore
\[X(t)=a\cos{2\pi f_0 t+\theta}\]
con fase $\theta$ variabile aleatoria a distribuzione uniforme in $[0,2\pi[$
che ha funzione valor medio funzione $\mu_X(t)=0$, quindi non dipende da $t$ e funzione di autocorrelazione (eq.\ref{eq:autocorrelazione_oscillatore_stazionario}) che dipende solo da $t_1-t_2$:
\[R_X(t_1,t_2)=\frac{a^2}{2}\cos{2\pi f_0(t_1-t_2)}\]
\end{esempio}

\section{Processo telegrafico casuale: segnale dati}
Si supponga di avere un processo stocastico che modelli un segnale dati binario con frequenza di clock $\frac{1}{T}$ modello della trasmissione di bit tra due sistemi.

Le realizzazioni sono funzioni $V(t)$ che possano assumere solo due valori discreti $+1$ e $-1$ e i cambi di stato avvengono in istanti di tempo multipli interi di un periodo $T$.
\begin{equation}V_n=\begin{cases}
1&p=\frac{1}{2}\\-1&1-p=\frac{1}{2}
\end{cases}\end{equation}
I valori $V_n$ sono assunti in modo indipendente l'uno dall'altro e sono equiprobabili. La funzione si dice di \emph{sample and hold}: il segnale cambia di stato in istanti prefissati e mantiene costante il valore nell'intervallo di tempo tra transizioni successive $V(t)=V_n$ per $n T\leq t<(n+1)T$.
La funzione $V(t)$ è un treno di impulsi rettangolari scalati e traslati
\begin{equation}\label{eq:segnale_dati}
V(t)=\sum_{n=-\infty}^{+\infty}{V_n\rect{\frac{t-n T-\frac{T}{2}}{T}}}
\end{equation}

Dall'osservazione dei valori assunti e dall'equiprobabilità si desume la funzione densità di probabilità del primo ordine:
\begin{equation}f_V(v;t)=\frac{1}{2}\impulse(v-1)+\frac{1}{2}\impulse(v+1)\end{equation}
La funzione densità di probabilità del primo ordine non dipende dal tempo quindi il processo è stazionario in senso stretto per il primo ordine. Infatti la funzione valor medio è costante
\begin{equation}\label{eq:segnale_binario_media}\mu_V(t)=\intinf{v f_V(v;t)}{v}=\intinf{\frac{1}{2}v\impulse(v-1)+\frac{1}{2}v\impulse(v+1)}{v}=\frac{1}{2}-\frac{1}{2}=0 \end{equation}

Il calcolo della funzione di autocorrelazione , indice statistico del secondo ordine, avviene per due istanti di tempo generici $t_1$ e $t_2$. Come si può vedere in figura nel grafico di una realizzazione i due istanti di tempo possono assumere gli stessi valori $V(t_1)=V(t_2)=V_n$ all'interno di uno stesso intervallo di tempo, per cui
\begin{equation}R_V(t_1,t_2)=\E{V(t_1)V(t_2)}=\E{V(t_1)}\E{V(t_2)}=\E{V_n^2}=1^2\cdot\frac{1}{2}+(-1)^2\cdot\frac{1}{2}=1\end{equation}
In un'altra coppia di istanti di tempo è possibile trovarsi nella condizione $V(t'_1)\neq V(t'_2)$, ad esempio a cavallo di due intervalli in cui è avvenuto un cambio di stato, per i quali
\begin{equation}R_V(t'_1,t'_2)=\E{V(t'_1)V(t'_2)}=\E{V(t'_1)}\E{V(t'_2)}=\left(1\cdot\frac{1}{2}-1\cdot\frac{1}{2}\right)^2=0\end{equation}
\begin{figure}[h!]
	\centering\begin{tikzpicture}
	\begin{axis}[axis lines=middle,no markers,enlargelimits,xscale=1.5,xtick={0},xlabel={$t$},ytick={-1,1},clip=false]
	\addplot [very thick]coordinates {(0,1)(1,1)(1,-1)(2,-1)(4,-1)(4,1)(5,1)(5,-1)(6,-1)(6,1)(8,1)(8,-1)(10,-1)(10,1)(11,1)};
	\draw [dashed] (axis cs:2.5,-1.1) node[below]{$t_1$} -- (axis cs:2.5,1.2)
	(axis cs:3.5,-1.1) node[below]{$t_2$} -- (axis cs:3.5,1.2);
	\draw [dashed] (axis cs:7.5,-1.1) node[below]{$t'_1$} -- (axis cs:7.5,1.2)
	(axis cs:8.5,-1.1) node[below]{$t'_2$} -- (axis cs:8.5,1.2);
	\end{axis}\end{tikzpicture}
\caption{Realizzazione di un processo dati binario}\label{fig:processo_dati_binario}
\end{figure}

Pertanto l'autocorrelazione dipende dai due istanti di tempo $t_1$ e $t_2$, questo implica che il processo non è stazionario in senso lato pur essendo stazionario in senso stretto per il primo ordine.

\section[Processo aleatorio con riferimento temporale aleatorio]{Processo aleatorio con riferimento temporale aleatorio: segnale eco radar}
Si modella il processo stocastico di un segnale di cui si conosce l'andamento ma il cui riferimento temporale è una variabile aleatoria $\theta$:
\begin{equation}X(t)=p(t-\theta)\end{equation}
Ad esempio il segnale può essere un segnale periodico di periodo $T$: $p(t)=p(t+T)$. La variabile aleatoria uniformemente distribuita in $[0,T]$.

Si determinano le proprietà del processo. La funzione valor medio, calcolata sulla trasformata della variabile aleatoria $\theta$ (che ha funzione densità $f_\Theta(\theta)=\frac{1}{T}\rect{\frac{t-T/2}{T}}$):
\[\mu_X(t)=\E{p(t-\theta)}=\intd{0}{T}{p(t-\theta)\frac{1}{T}}{\theta}=\frac{1}{T}\intd{t}{t-T}{-p(\alpha)}{\alpha}=\frac{1}{T}\intd{t-T}{t}{p(\alpha)}{\alpha}\]
dove si è posto $\alpha=t-\theta$, $\diff\alpha=-\diff\theta$, gli estremi di integrazione $\theta=0\to\alpha=t$, $\theta=T\to\alpha=t-T$.
L'integrale sulla funzione periodica su un intervallo di ampiezza $T$ non dipende dal valore $t$. Il valor medio statistico è pari al valor medio temporale della funzione $p(\alpha)$.

La funzione autocorrelazione
\[R_X(t_1,t_2)=\E{p(t_1-\theta)p(t_2-\theta)}=\frac{1}{T}\intd{0}{T}{p(t_1-\theta)p(t_2-\theta)}{\theta}=\]
cambio di variabile pongo $\alpha=t_1-\alpha$, $\diff\alpha=-\diff\theta$
\[=\frac{1}{T}\intd{t_1}{t_1-T}{-p(\alpha)p(t_2-t_1+\alpha)}{\alpha}=\frac{1}{T}\intd{t_1-T}{t_1}{p(\alpha)p(t_2-t_1+\alpha)}{\alpha}=R_X(t_1-t_2)\]
l'integranda è il prodotto di due termini periodici di periodo $T$, che è ancora di periodo $T$, quindi il suo integrale non dipende dal particolare istante iniziale di integrazione sul periodo. Pertanto la funzione autocorrelazione del processo $X(t)$ non dipende separatamente da $t_1$ e $t_2$ ma è funzione di $t_1-t_2$.

Essendo la funzione valor medio indipendente dal valor $t$ e la funzione autocorrelazione dipendente dal valore di $t_1-t_2$ il processo risulta stazionario in senso lato.

\clearpage
\section{Proprietà funzione autocorrelazione processo SSL}
Proprietà della funzione di autocorrelazione di un processo stazionario in senso lato:
\begin{enumerate}
\item La funzione di autocorrelazione è pari
\[R_X(\tau)=R_X(-\tau)\]
\begin{proof}[Dim.]
\[R_X(\tau)=\E{X(t)X(t-\tau)}=\E{X(t+\tau)X(t)}=R_X(-\tau)\]
\end{proof}

\item Il valore assunto da $R_X(\tau)$ nell'origine è pari alla potenza statistica del processo
\[R_X(0)=P_X=\E{X^2(t)}\]
\begin{nota}I processi aleatori SSL sono sempre segnali di potenza perché le varie realizzazioni che si estraggono da un processo non possono essere tutte infinitesime.\end{nota}

\item La funzione autocorrelazione è massima in modulo nell'origine
\[R_X(0)\geq\abs{R_X(\tau)}\]

\begin{proof}[Dim.]
\[\E{(X(t)\pm X(t-\tau))^2}\geq 0\]
disuguaglianza sempre vera perché aspettazione di una quantità sempre positiva, sviluppando la relazione
\begin{gather*}\E{X^2(t)}+\E{X^2(t-\tau)}\pm 2 \E{X(t)X(t-\tau)}\geq 0\\2R_X(0)\pm 2R_X(\tau)\geq 0\\R_X(0)\geq\pm R_X(\tau)\\ R_X(0)\geq\abs{R_X(\tau)}\end{gather*}
\end{proof}

\item Se $R_X(\tau)$ non è periodica allora \begin{equation}\lim\limits_{\tau\to\infty}R_X(\tau)=\mu^2_X\end{equation}
\begin{proof}[Dim.]
Per funzione di autocorrelazione che dipende solo dalla differenza dei tempi si ha che la funzione autocovarianza $C_X(\tau)=R_X(\tau)-\mu^2_X\xrightarrow{\tau\to\infty}0$ ovvero a crescere della distanza $\tau$ si hanno valori delle variabili aleatorie sempre meno correlati. La funzione di autocorrelazione pertanto $R_X(\tau)\xrightarrow{\tau\to\infty}\mu^2_X$.
\end{proof}
\end{enumerate}

\section[Processo aleatorio con riferimento temporale aleatorio]{Processo aleatorio: segnale dati binari}
Si considera nuovamente il segnale dati (eq.\ref{eq:segnale_dati}) con riferimento temporale non noto. Tale fenomeno si verifica ad esempio quando il ricevente riceve un segnale con un ritardo che dipende dalla distanza dal trasmittente. Un modello di questo processo con $\theta$ variabile aleatoria uniformemente distribuita in $[0,T]$, e valori $V_n$ indipendenti tra loro ed equiprobabili,
\[X(t)=V(t-\theta)=\sum_{n=-\infty}^{+\infty}V_n\rect{\frac{t-\theta-n T-T/2}{T}}\]
Come abbiamo visto il segnale telegrafico casuale ha funzione valor medio costante $\mu_X(t)=\mu_X=0$ (eq.\ref{eq:segnale_binario_media}).
Inoltre la funzione di autocorrelazione
\[\begin{split}R_X(t_1,t_2)&=\E{X(t_1)X(t_2)}=\\
&=\E{\sum_{n=-\infty}^{+\infty}\sum_{m=-\infty}^{+\infty}V_n V_m\rect{\frac{t_1-\theta-n T-\frac{T}{2}}{T}}\rect{\frac{t_2-\theta-m T-\frac{T}{2}}{T}}}=\\
\intertext{essendo $\E{V_nV_m}\neq 0$ solo se $n=m$ la doppia sommatoria si riduce ad un elemento}
&=\sum_{n=-\infty}^{+\infty}\frac{1}{T}\intd{0}{T}{\rect{\frac{t_1-\theta-n T-T/2}{T}}\rect{\frac{t_2-\theta-m T-T/2}{T}}}{\theta}=\\
\intertext{cambio di variabili $t_1=t, t_2=t-\tau, \alpha=t-\theta-n T,\diff\alpha=-\diff\theta, \theta=0\to\alpha=t-nT, \theta=T\to t-(n+1)T$, inverto gli estremi per il segno di $-\diff\theta$}
&=\frac{1}{T}\sum_{n=-\infty}^{+\infty}\intd{t-n T-T}{t-n T}{\rect{\frac{\alpha-T/2}{T}}\rect{\frac{\alpha-\tau-T/2}{T}}}{\alpha}=\\
\intertext{l'integranda non dipende da $n$, la sommatoria degli integrali su intervalli disgiunti $[n T-T,n T]$ può essere sostituita da un unico integrale}
&=\frac{1}{T}\intd{-\infty}{+\infty}{\rect{\frac{\alpha-T/2}{T}}\rect{\frac{\alpha-\tau-T/2}{T}}}{\alpha}=\\
&=\frac{1}{T}\rect{\frac{\tau-T/2}{T}}\ast\rect{\frac{-\tau-T/2}{T}}
\end{split}\]
La funzione autocorrelazione dipende dalla sola variabile $\tau$ e rappresenta una correlazione deterministica convoluzione di due funzioni rettangolo che è pari ad una funzione triangolare.
Il segnale dati binario con ritardo casuale, pertanto, è un processo stazionario in senso lato.

\section{Significato funzione autocorrelazione processo SSL}
Presi due processi stocastici stazionari in senso lato $X(t)$ e $Y(t)$, dotati degli stessi parametri statistici del primo ordine (funzione valor medio, funzione potenza, funzione varianza), ad esempio con valor medio $\mu_X(t)=\mu_Y(t)=\text{costante}$. Per poter distinguere statisticamente i segnali si analizzano i parametri statistici del secondo ordine (funzione autocorrelazione, funzione autocovarianza).
Supponendo che $R_X(\tau)\neq R_Y(\tau)$ ho modo di distinguere i processi.
In particolare quello che si osserva per istanti di tempo distanti $\tau$ è una diversa velocità di variazione: infatti uno dei due segnali assomiglierà di più a se stesso (autocorrelazione maggiore), ovvero le variazioni delle realizzazioni sono più lente e quindi più somiglianti tra loro.

\section{Filtraggio processo aleatorio}\index{processo aleatorio!filtraggio}
La teoria dei processi stocastici modella fenomeni reali descrivibili da grandezze fisiche che variano nel tempo in modo non predicibile a priori. Essendo le grandezze fisiche manipolabili ha senso chiedersi cosa significa filtrare un processo aleatorio attraverso un sistema.

In particolare per sistemi lineari tempo-invarianti si avrà che un segnale $x(t)$ in ingresso al sistema sarà convoluto con la risposta all'impulso del sistema per avere in uscita un segnale $y(t)=x(t)\ast h(t)$.

Nei processi aleatori avremo usualmente un segnale deterministico noto a cui è sovrapposto un segnale di disturbo o rumore che è modellato dal processo aleatorio a valor medio nullo. Compito dell'ingegnere è progettare filtri che eliminino, almeno in parte, la componente rumorosa dalle realizzazioni del processo.

\begin{figure}[h!]\centering
	\begin{tikzpicture}[scale=.6]
	\begin{axis}[axis lines=middle,no markers,enlargelimits,xscale=3.33,line join=bevel]
	\addplot [thick,domain=0:6.28,samples=500] {sin(pi*x+0.5)+.5*rand};
	\end{axis}\end{tikzpicture}
	\caption{Esempio di segnale deterministico affetto da rumore}
\end{figure}

Per ogni realizzazione $x(t;\omega)$ del processo $X(t)$ dello spazio campione $\Omega$ si otterrà in uscita dal sistema filtro una funzione $y(t)=x(t;\omega)\ast h(t)$. L'insieme dei segnali di uscita costituisce un nuovo processo, $Y(t)$ che indichiamo con
\begin{equation}
Y(t)=X(t)\ast h(t)
\end{equation}

\begin{figure}[!h]\centering
	\begin{tikzpicture}[node distance=3cm]
	\node [block] (system) at (0,0) {$h(t)$};
	\node [left of=system](input) {};
	\node [right of=system] (output) {};
	\draw [-latex] (input) -- (system) node[pos=.5,above]{$X(t)$};
	\draw [-latex] (system) -- (output) node[pos=.5,above]{$Y(t)$};
	\end{tikzpicture}
	\caption{Filtraggio del processo $X(t)$}
\end{figure}

Generalmente il problema di determinare la funzione densità di probabilità congiunta di qualunque ordine per il processo d'uscita, ammesso che sia nota quella del processo di ingresso, è insolubile.

Si determina allora la relazione dei parametri statistici di primo e secondo ordine in ingresso, almeno la funzione valor medio e la funzione autocorrelazione di $X(t)$, e i corrispondenti dell'uscita.

\begin{equation}
\begin{split}
\mu_Y(t)&=\E{Y(t)}=\E{X(t)\ast h(t)}=\E{\intinf{h(\tau)X(t-\tau)}{\tau}}=\\
&=\intinf{\E{h(\tau)X(t-\tau)}}{\tau}=\intinf{h(\tau)\E{X(t-\tau)}}{\tau}=\\
&=\intinf{h(\tau)\mu_X(t-\tau)}{\tau}=\\
&=\mu_X(t)\ast h(t)
\end{split}
\end{equation}
Il risultato $\mu_Y(t)=\mu_Y(t)\ast h(t)$ è notevole: la funzione valor medio in uscita si ottiene come convoluzione della funzione valor medio in ingresso con la risposta all'impulso del sistema. Pertanto un sistema con in ingresso un segnale deterministico a cui si somma un processo stocastico a valor medio nullo, per la linearità del sistema, darà in uscita un processo somma di una componente deterministica e di una componente statistica che avrà ancora valor medio nullo. Il filtraggio ottimo elimina la componente statistica e preserva la componente deterministica.
\begin{equation}
\underset{\text{proc.aleat.}}{X(t)} = \underset{\text{proc.aleat. a media nulla}}{X_0(t)} + \underset{\text{segn.determ.}}{\mu_x(t)}
\end{equation}

La funzione autocorrelazione in uscita
\begin{equation}
\begin{split}
R_Y(t_1,t_2)&=\E{Y(t_1)Y(t_2)}=\E{(X(t_1)\ast h(t_1))\cdot(X(t_2)\ast h(t_2))}=\\
&=\E{\intinf{\intinf{X(\alpha)h(t_1-\alpha)X(\beta)h(t_2-\beta)}{\alpha}}{\beta}}=\\
&=\intinf{\intinf{h(t_1-\alpha)h(t_2-\beta)\E{X(\alpha)X(\beta)}}{\alpha}}{\beta}=\\
&=\intinf{\intinf{h(t_1-\alpha)h(t_2-\beta)R_X(\alpha,\beta)}{\alpha}}{\beta}=\\
&=R_X(t_1,t_2)\ast h(t_1)\ast h(t_2)
\end{split}
\end{equation}

\section{Filtraggio processo stazionario in senso lato}\label{sec:filtraggio_processo_SSL}
Si ha in ingresso al filtro un processo stazionario in senso lato, che ha funzione valor medio costante e funzione di autocorrelazione dipendente da $\tau=t_1-t_2$ (eq.\ref{eq:processo_stazionario_senso_lato}).

La funzione valor medio in uscita
\begin{equation}
\mu_Y(t)=\mu_X(t)\ast h(t)=\intinf{h(\tau)\mu_X(t-\tau)}{\tau}=\mu_X\intinf{h(\tau)}{\tau}=\mu_X\cdot \restrict{H(f)}{f=0}=\text{cost}
\end{equation}
dove $H(0)$ è il valore della trasformata di Fourier della risposta all'impulso nell'origine, e la funzione valor medio in uscita assume valore costante.

La funzione di autocorrelazione in uscita
\begin{equation}\begin{split}
R_Y(t,t-\tau)&=\E{Y(t)Y(t-\tau)}=\E{(X(t)\ast h(t))(X(t-\tau)\ast h(t-\tau))}=\\&=\E{\intinf{\intinf{h(\alpha)X(t-\alpha)h(\beta)X(t-\tau-\beta)}{\alpha}}{\beta}}=\\&=\intinf{\intinf{h(\alpha)h(\beta)\E{X(t-\alpha)X(t-\tau-\beta)}}{\alpha}}{\beta}=\\&=\intinf{\intinf{h(\alpha)h(\beta)R_X(\tau+\beta-\alpha)}{\alpha}}{\beta}=\\
\intertext{si può osservare che la funzione di autocorrelazione non dipende da $t$ ma solo da $\tau$ pertanto filtrando un processo stazionario si ottiene un processo stazionario,}
&=\intinf{h(\beta)\underbrace{\left[\intinf{R_X(\tau+\beta-\alpha)h(\alpha)}{\alpha}\right]}_{R_X(\tau+\beta)\ast h(\tau+\beta)=g(\tau+\beta)}}{\beta}=\\
&=\intinf{h(\beta)g(\tau+\beta)}{\beta}=\intinf{h(-z)g(\tau-z)}{z}=g(\tau)\ast h(-\tau)
\end{split}
\end{equation}
Si ottiene quindi
\begin{equation}
R_Y(\tau)=R_X(\tau)\ast h(\tau)\ast h(-\tau)=R_X(\tau)\ast R_h(\tau)
\end{equation}
dove la convoluzione del segnale $h(\tau)\ast h(-\tau)$ è l'autocorrelazione del segnale deterministico risposta all'impulso.

Un processo stazionario in senso lato in ingresso ad un sistema lineare tempo invariante viene filtrato e da in uscita un processo stazionario in senso lato.

Il filtraggio di un processo aleatorio $X(t)$ ha quindi due casi per cui la funzione valor medio e di autocorrelazione del processo in uscita sono legate a quelle del processo in ingresso

\begin{table}[!h]
\centering
\begin{tabular}{c|c}
\toprule
\multicolumn{2}{c}{$Y(y)=X(t)\ast h(t)$} \\
\midrule
caso $X(t)$ generico & caso $X(t)$ SSL  \\
$\begin{cases}
\mu_Y(t)=\mu_X(t)\ast h(t)\\R_Y(t_1,t_2)=R_X(t_1,t_2)\ast h(t_1)\ast h(t_2)
\end{cases}$ & $\begin{cases}
\mu_Y(t)=\mu_X\cdot H(0)\\R_Y(\tau)=R_X(\tau)\ast R_h(\tau)
\end{cases}$ \\
\bottomrule
\end{tabular}
\end{table}


\chapter{Analisi spettrale dei processi aleatori}
\label{ch:teoriasegnali-capitolo7}
\index{processo!stocastico!analisi spettrale}
Si è data una descrizione del problema del filtraggio di un processo aleatorio, e il calcolo delle statistiche di primo e secondo ordine, per un sistema lineare stazionario in ambito temporale.

Ha senso analizzare in frequenza la risposta di un sistema lineare tempo invariante ad un processo aleatorio $X(t,\omega)$ e ad una analisi di Fourier delle realizzazioni, i segnali $x(t)$ estratti dal processo.

Ci si limiterà qui a studiare le proprietà in frequenza per soli processi aleatori stazionari in senso lato. L'analisi di Fourier di un processo richiederebbe lo studio in frequenza di ampiezza e fase dello spettro di ogni realizzazione del processo, e delle relazioni tra gli indici statistici nel tempo e in frequenza.
\`{E} comune limitarsi alla descrizione degli spettri di potenza del segnale aleatorio.

Le funzioni campione di un processo stazionario non possono essere segnali a energia finita, perché andando asintoticamente a zero la funzione valor medio tenderebbe a zero per tutte le funzioni campione, mentre in generale i processi SSL hanno media costante (non necessariamente nulla).
In generale le funzioni campione di un processo stazionario sono segnali a potenza finita, perciò il processo aleatorio ammette spettro di potenza.

La funzione densità spettrale di potenza di un processo aleatorio è la media delle funzioni densità spettrale di potenza ottenute per le realizzazioni
\begin{equation}
	S_X(f,\omega)=\E{S_X(f;\omega)}
\end{equation}

Lo spettro di potenza del processo come valor medio dello spettro di potenza delle funzioni campione
\begin{equation}
\label{eq:densita_spettrale_processo_aleatorio}
	S_X(f)=\E{\lim\limits_{t\to+\infty}\frac{X_T(f)^2}{T}}=\E{\lim\limits_{t\to+\infty}\frac{\abs{\fourier{x_T(t;\omega)}}^2}{T}}
\end{equation}
dove la trasformata di Fourier si applica al segnale $x(t)$ troncato all'intervallo $[-\frac{T}{2},\frac{T}{2}]$.
Questa definizione, valida anche per processi non stazionari, è molto difficile da utilizzare.

\section{Teorema di Wiener-Kintchine}
\index{Teorema!di Wiener-Kintchine}
La densità spettrale di potenza dei processi stazionari in senso lato è calcolabile come trasformata di Fourier della funzione di autocorrelazione
\begin{equation}
	S_X(f)=\fourier{R_X(\tau)}=\intinf{R_X(\tau)\e{-\imath 2\pi f\tau}}{\tau}
\end{equation}
Proprietà:
\begin{enumerate}
	\item La densità spettrale è non negativa per definizione eq.\ref{eq:densita_spettrale_processo_aleatorio}: $S_X(f)\geq 0$
	\item La densità spettrale di potenza di un processo aleatorio stazionario in senso lato è una funzione reale e pari.
\begin{proof}[Dim.]
Si ricorda che $R_X(\tau)=R_X(-\tau)$ è pari pertanto posso sommare i contributi nell'integrale
\[
	\intinf{R_X(\tau)\e{-\imath 2\pi f\tau}}{\tau}=\intd{0}{\infty}{R_X(\tau)\e{-\imath 2\pi f\tau}+R_X(-\tau)\e{+\imath 2\pi f\tau}}{\tau}=\intd{0}{\infty}{R_X(\tau)2\cos{2\pi f\tau}}{\tau}
\]
\end{proof}
\item La potenza media statistica del processo, costante per processo stazionario, è pari all'integrale della densità spettrale di frequenza
\begin{equation}
	P_X=R_X(0)=\E{X^2(t)}=\intinf{S_X(f)}{f}
\end{equation}
\end{enumerate}

\section{Filtraggio processo stazionario}\index{processo stazionario!filtraggio}
Si può caratterizzare la densità spettrale del processo in uscita ad un sistema lineare tempo invariante nota la densità spettrale del processo in ingresso.
\`E noto che se il processo in ingresso è stazionario in senso lato lo è anche il processo in uscita.

La densità spettrale del processo in uscita
\begin{equation}
	S_Y(f)=\fourier{R_X(\tau)\ast h(\tau)\ast h(-\tau)}=S_X(f)\cdot H(f)\cdot H(-f)
\end{equation}

Per sistemi reali la risposta all'impulso è una funzione reale, $H(-f)=\conj{H}(f)$
\begin{equation}
	S_Y(f)=S_X(f)\cdot\abs{H(f)}^2
\end{equation}
che è la stessa relazione che vale per gli spettri di potenza dei segnali deterministici.
La risposta in fase del sistema non influenza la densità spettrale del processo in uscita.

Nella densità spettrale di potenza sono contenute tutte le informazioni spettrali del processo, ovvero come è distribuita la potenza alla varie frequenze. Il significato di densità spettrale di potenza è lo stesso per segnali deterministici e per processi aleatori.

\begin{esempio}
Si calcoli la densità spettrale di potenza del processo aleatorio stazionario 
\[
	X(t)=a\sen{2\pi f_0 t+\theta}
\]
con $\theta$ v.a. uniformemente distribuita in $[0,2\pi[$. Si è calcolata la funzione di autocorrelazione
\[
	R_X(\tau)=\frac{a^2}{2}\cos{2\pi f_0\tau}
\]
Secondo la definizione di Wiener-Kintchine si può calcolare la densità spettrale può essere calcolata come
\[
	S_X(f)=\fourier{R_X(\tau)}=\frac{a^2}{4}[\impulse(f-f_0)+\impulse(f+f_0)]
\]
\begin{figure}[!ht]
	\centering
	\begin{tikzpicture}
		\draw [-latex] (-3,0)--(3,0) node[below] {$f$};
		\draw [-latex] (0,0)--(0,1.5) node[right] {$S_X(f)$};;
		\draw [very thick,-latex] (-2,0)--(-2,1);
		\draw [very thick,-latex] (2,0)--(2,1);
		\draw (-2,0) -- (-2,-1mm) node [below] {$-f_0$};
		\draw (0,0) -- (0,-1mm) node [below] {$0$};
		\draw (2,0) -- (2,-1mm) node [below] {$f_0$};
	\end{tikzpicture}
\end{figure}
\end{esempio}

\begin{nota}
	Poiché la densità spettrale di potenza è la trasformata di Fourier della funzione di autocorrelazione per processi stazionari si ha che tanto più rapidamente variano le singole realizzazioni di un processo, tanto più larga è la banda passante della densità spettrale di potenza. In altre parole a variazioni rapide corrispondono termini spettrali a potenza non nulla a frequenze più alte.
\end{nota}

\section{Processo stocastico gaussiano}
\index{processo!stocastico!gaussiano}
Un processo aleatorio $X(t)$ è gaussiano se, presi $n$ istanti di tempo distinti, le variabili aleatorie $X_1(t),X_2(t),\dots,X_n(t)$ risultano congiuntamente gaussiane.
$X(t)$ è gaussiano se la densità di probabilità congiunta del vettore delle variabili aleatorie ha la forma
\begin{equation}
	f_\vect{X}(x_1,\dots,x_n;t_1,\dots,t_n)=\frac{1}{\sqrt{(2\pi)^n\det\abs{ C_\vect{X}}}}\;\e{-\frac{1}{2}\trasp{(\vect{x}-\mu_\vect{X})}C^{-1}_\vect{X}(\vect{x}-\mu_\vect{X})}
\end{equation}
Per la conoscenza completa della funzione di densità di probabilità congiunta, e quindi dell'intero processo, è necessario conoscere la funzione valor medio del vettore $\vect{X}$, $\mu_\vect{X}(t)$, e la funzione di autocovarianza, la matrice $C_\vect{X}$ (si veda es.\ref{es:v_a_congiuntamente_gaussiane}) per ogni $n$-pla di istanti di tempo $(t_1,t_2,\dots,t_n)$. Si ricorda che la funzione valor medio del vettore delle variabili aleatorie è
\[
	\mu_\vect{X}(t)=\trasp{[\mu_X(t_1),\mu_X(t_2),\dots,\mu_X(t_n)]}
\]
e gli elementi della matrice di covarianza si calcolano come
\[
	c_{ij}=\E{(X(t_i)-\mu_X(t_i))\cdot (X(t_j)-\mu_X(t_j))}=C_X(t_i,t_j)=R_X(t_i,t_j)-\mu_X(t_i)\mu_X(t_j)
\]

Per i processi gaussiani si ha la notevole proprietà per cui la stazionarietà in senso lato implica la stazionarietà in senso stretto. Questo perché la funzione densità di probabilità congiunta del processo dipende dalla funzione valor medio, che è costante, e dalla funzione di autocorrelazione, che dipende solo dalla differenza tra gli istanti di tempo ($\mu_X(t)=\mu_X$ e $R_X(t_1,t_2)=R_X(\tau)$), e quindi la stessa funzione di autocovarianza.

I processi gaussiani in ingresso a sistemi lineari tempo invarianti vengono filtrati e risultano ancora processi gaussiani in uscita.
\begin{proof}[Dim.]
\[
	Y(t)=X(t)\ast h(t)=\intinf{x(\alpha)h(t-\alpha)}{\alpha}
\]
L'operazione di integrale si può pensare come somma di infiniti termini $X(k\Delta\alpha)h(t-k\Delta\alpha)\Delta\alpha$. Il processo in uscita risulta una combinazione lineare di tanti processi in ingresso tutti gaussiani pertanto è esso stesso gaussiano.
\end{proof}

\section{Processo aleatorio bianco}
\index{processo!stocastico!bianco}
Si supponga un modello teorico di un processo con densità spettrale di potenza la cui banda tende a crescere illimitatamente, mantenendo il valore presente in $f=0$. La densità spettrale di potenza del processo $X(t)$ tende a diventare costante a tutte le frequenze
\[
	S_X(f)=\xi
\]
Il tempo di correlazione tende a ridursi fino ad arrivare al limite ad una funzione di autocorrelazione impulsiva
\[
	R_X(\tau)=\xi\impulse(\tau)
\]
Il valore medio
\[
	\mu_X(t)=\lim\limits_{\tau\to +\infty}R_X(\tau)=\mu^2_X=0
\]

\begin{nota}
	Il processo aleatorio bianco è solo una astrazione matematica: uno spettro reale non potrà mai avere potenza a tutte le frequenze, avrebbe potenza infinita.
\end{nota}

\subsection{Esempio resistore con rumore termico}
\begin{esempio}
L'agitazione termica degli elettroni in un resistore causa una tensione di rumore a vuoto proporzionale alla temperatura del componente. Rispetto al modello ideale si ha quindi un resistore con in serie un generatore di tensione pari a quella prodotta dal rumore termico. Quest'ultimo può essere modellato come un processo aleatorio stazionario.

\begin{figure*}[h]
	\centering
	\begin{circuitikz}[american voltages]
		\draw (0,0)	to[V,v^>=${N(t)}$] (0,3)
		to[R,l=${R}$] (2,3)
		to[open,*-*] (2,0) -- (0,0);
	\end{circuitikz}
\end{figure*}

La descrizione del fenomeno del rumore termico coinvolge considerazioni di meccanica quantistica che portano a determinare l'espressione della densità spettrale di potenza della tensione di rumore (formula di Nyquist)

\begin{equation}
	S_N(f)=2 R \frac{\frac{\abs{f}}{f_0}}{\e{\frac{\abs{f}}{f_0}-1}}
	=2 R \frac{\hbar\abs{f}}{\e{\frac{\hbar\abs{f}}{k_B T_R}-1}} \quad[\si{\volt\squared\per\hertz}]
\end{equation}

dove $f_0=\frac{k_B T_R}{\hbar}$, con la costante di Boltzmann $k_B=\SI{1.38e-23}{\joule\per\kelvin}$, la costante di Planck $\hbar=\SI{6.63e-34}{\joule\second}$, la temperatura ambiente $T_R=\SI{293}{\kelvin}$.
A temperatura ambiente la frequenza $f_0\cong\SI{6}{\tera\hertz}$

A frequenze $f\ll f_0$ posso approssimare
\[
	\e{\frac{\hbar\abs{f}}{k_B T_R}}-1\cong\frac{\hbar\abs{f}}{k_B T_R}
\]
essendo $\e{x}=1+x+\frac{x^2}{2}+\dots$ la serie per $x\to 0$ si approssima $\e{x}\cong 1+x\implies\e{x}-1\cong x$.
\[
	S_N(f)= 2 R k_B T_R
\]

La tensione quadratica media misurata con un voltmetro di banda $B$ è pari alla densita spettrale di potenza monolatera
\[
	S_N^m(f)=4 R K_B T_R
\]

Il modello del rumore bianco pertanto è utile: per calcolare gli effetti del rumore termico sull'uscita di un sistema filtrante si può sostituire la densità di potenza del rumore termico $S_N(f)$ con il suo modello semplificato di valore costante.
\end{esempio}

\begin{nota}
	Nel dimensionamento dei sistemi di telecomunicazione è importante la potenza trasmessa alla quale si somma la potenza del rumore. Si progettano i sistemi in modo da massimizzare la potenza trasmessa utile.
\end{nota}

\begin{figure*}[h]
	\centering
	\begin{circuitikz}[american voltages]
		\draw (0,0)	to[V,v^>=${v_n(t)}$] (0,3)
		to[R,l=${R}$,-*] (3,3)--(4,3)
		to[R,l=${R_L}$,v=${v_L(t)}$] (4,0) to[short,-*] (3,0)--(0,0);
		\draw node at(7,1.5) {$v_L(t)=\frac{R_L}{R+R_L}v_N(t)$};
	\end{circuitikz}
\end{figure*}
La massima potenza in uscita $P(t)$ misurata in $\si{\watt}$ si ha quando il carico (\emph{Load}) è adattato, ovvero per $R_L=R$
\[
	P(t)=v_L(t)\cdot i(t)=v_L(t)\frac{v_N(t)}{R+R_L}=\frac{R_L}{R+R_L}v_N(t)\frac{v_N(t)}{R+R_L}\overset{R_L=R}{=}\frac{1}{2}\frac{v^2_N(t)}{2 R}
\]
La densità spettrale monolatera di rumore disponibile in uscita in una banda $B$
\[
	\E{\frac{v^2_N(t)}{4 R}}=\frac{1}{4 R}\E{v^2_N(t)}=\frac{1}{4 R}4 R k_B T B=k_B T B \si{\watt\per\hertz}
\]

\section{Esempio filtro passa basso con rumore termico}
\begin{esempio}
Esempio filtro passa basso con rumore termico.

\begin{figure*}[h]
	\centering
	\begin{circuitikz}[american voltages]
		\draw (0,0)	to[V,v^>=${V_S(t)}$] (0,3)
		to[R,l=${R}$,-*] (3,3)--(4,3)
		to[C,l=${C}$,v=${v_c(t)}$] (4,0) to[short,-*] (3,0)--(0,0);
	\end{circuitikz}
\end{figure*}

Trasformata della risposta all'impulso del sistema passa basso:
\[
	H(f)=\frac{V_c(f)}{V_n(f)}=\frac{1}{1+\imath 2\pi R C f}
\]
La frequenza di taglio \[
	f_0=\frac{1}{2\pi R C}
\]
Potenza media di rumore sulla resistenza
\[
	P_n(f)=2 R k_B T [\si{\volt\squared\per\hertz}]
\]
Potenza media di rumore sul condensatore
\[
	P_{V_c}(f)=2 R k_B T \frac{1}{1+4\pi^2 R^2 C^2 f^2} [\si{\volt\squared\per\hertz}]
\]

Calcolo la potenza media bilatera
\[
	\begin{split}
		\E{V_c^2}&=\intd{0}{+\infty}{\frac{4 R k_B T}{1+4\pi^2 R^2 C^2 f^2}}{f}=\\
\intertext{cambio variabile $\alpha=2\pi R C f$, $\diff\alpha=2\pi R C\diff f$}
		&=\intd{0}{+\infty}{\frac{4 R k_B T}{2\pi R C}\frac{1}{1+\alpha^2}}{\alpha}=\bound{0}{+\infty}{\frac{4 k_B T}{2\pi C}\arctg\alpha}=\frac{4 k_B T}{2\pi C}\frac{\pi}{2}=\frac{k_B T}{C}
	\end{split}
\]
\end{esempio}

\begin{esempio}
Si hanno due resistenze in serie a diverse temperature.

\begin{figure*}[!h]
	\centering
	\begin{circuitikz}[american voltages]
		\draw (0,6)	to[R,l={$R_1$},v=${v_{n_1}(t)}$] (0,3)
		to[R,l={$R_2$},v=${v_{n_2}(t)}$] (0,0)
		(0,6) to[short,-*] (3,6)
		to[open,-*,v=${v_n(t)}$] (3,0)--(0,0);
	\end{circuitikz}
\end{figure*}

La tensione di uscita è pari alle tensioni di rumore
\[
	v_n(t)=v_{n_1}(t)+v_{n_2}(t)
\]

Sono tensioni a media nulla quindi posso sommare media e varianza.
\[
	\E{V^2_n}=\E{V^2_{n_1}}+\E{V^2_{n_2}}=\intinf{P_{n_1}(f)}{f}+\intinf{P_{n_2}(f)}{f}=\intinf{P_n(f)}{f}
\]

La densità di potenza monolatera
\[
	P_n(f)=2 R_1 k_B T_1 + 2 R_2 k_B T_2
\]

La densità di potenza bilatera
\[
	P_n^m(f)=4 R_1 k_B T_1 + 4 R_2 k_B T_2 = 4 R_\text{eq} k_B T_\text{eq}
\]
dove la resistenza equivalente alla serie è $R_\text{eq}=R_1+R_2$ e la temperatura equivalente è data da
\[
	T_\text{eq}=\frac{T_1 R_1+T_2 R_2}{R_1+R_2}
\]
\end{esempio}

\section{Rumore passa banda o a banda stretta}
Si suppone un processo aleatorio stazionario $X(t)=\mu_X(t)+N(t)$ somma di un segnale deterministico più un rumore a valor medio nullo che attraversa un filtro passa banda ideale, per modulazione del segnale.

Il rumore ha le seguenti proprietà:
\begin{enumerate}
\item Il rumore gaussiano bianco può essere rappresentato nelle sue componenti
\[
	N(t)=N_I(t)\cos{2\pi f_0 t}-N_Q(t)\sen{2\pi f_0 t}
\]
dove $N_I(t)$ è la componente in fase del rumore e $N_Q(t)$ è la componente in quadratura.

\item $N_I(t)$ e $N_Q(t)$ sono processi aleatori passa basso $\abs{f}\leq B$

\item In uscita la potenza media a frequenza zero è nulla pertanto non ho potenza in continua.

\item Le componenti $N_I$ e $N_Q$ hanno valor medio nullo.

\item Se $N(t)$ è gaussiano anche le componenti $N_I$ e $N_Q$ sono gaussiane.

\item Se $N(t)$ è stazionario anche le componenti $N_I$ e $N_Q$ sono stazionarie.

\item La densità spettrale di potenza delle componenti in uscita è
\[
	S_{N_I}(f)=S_{N_Q}(f)=
	\begin{cases}
		S(f-f_0)+S(f+f_0)&\abs{f}<B\\
		0&\text{altrimenti}
	\end{cases}
\]

\item La varianza $\sigma^2_{N_I}=\sigma^2_{N_Q}=\sigma^2_N$
\end{enumerate}

\section{Sistema di filtraggio di rumore bianco su portante aleatoria}
\begin{figure}[h]
	\centering
	\begin{tikzpicture}[node distance=2cm,>=latex']
		\node [block] (integrator) at (0,0) {$g(t)$};
		\node [left of=integrator](input) {$N(t)$};
		\node [sum,cross,right of=integrator,node distance=3cm] (mult) {};
		\node [below of=mult](modulator) {$p(t)$};
		\node [block,right = of mult] (filter) {$H(f)$};
		\node [right of=filter] (output) {$Z(t)$};
		\draw [->] (input) -- (integrator);
		\draw [->] (integrator) -- (mult) node[pos=.5,above]{$X(t)$};
		\draw [->] (modulator) -- (mult);
		\draw [->] (mult) -- (filter) node[pos=.5,above]{$Y(t)$};
		\draw [->] (filter) -- (output);
	\end{tikzpicture}
\end{figure}

Nel sistema illustrato si ha
\begin{itemize}
\item il processo aleatorio $N(t)$ stazionario in senso lato con densità spettrale costante di rumore bianco $S_n(f)=n$
\item filtro integratore a finestra mobile
\[
	g(t)=\frac{1}{T}\rect{\frac{t-\frac{T}{2}}{T}}
\]
\item è un segnale portante
\[
	p(t)=2\cos{2\pi f_0+\theta}
\]
con fase $\theta$ variabile aleatoria uniformemente distribuita in $[0,2\pi[$
\item filtro passa banda ideale $H(f)$
\end{itemize}

Il processo $X(t)$ è stazionario in senso lato perché risulta il filtraggio di un sistema LTI di un processo rumore bianco, quindi SSL.

La funzione valor medio $\mu_X(t)$ ha valore nullo perché la funzione valor medio $\mu_N(t)=0$ è identicamente nulla (rumore bianco).

La funzione di autocorrelazione
\[
	R_X(\tau)=R_N(\tau)\ast g(\tau)\ast g(-\tau)=n\impulse(\tau)\ast R_g(\tau)= \frac{n}{T}\left(1-\frac{\abs{\tau}}{T}\right)\rect{\frac{\tau}{2T}}
\]
che è l'espressione di un triangolo di altezza $\frac{1}{T}$ base $2T$ e pendenza $\frac{\abs{\tau}}{T}$

La densità spettrale di potenza
\[
	S_X(f)=n \abs{\fourier{g(t)}}^2=n\abs{G(f)}^2=n\Sinc^2(T f)
\]

La portante è un processo aleatorio stazionario in senso lato (oscillatore es.\ref{es:oscillatore_stazionario_senso_lato}) con valor medio $\mu_p(t)=0$ e funzione di autocorrelazione
\[
	R_X(\tau)=2\cos{2\pi f_0(\tau)}
\]

Il processo prodotto $Y(t)$ avrà ancora valor medio nullo
\[
	\mu_Y(t)=\E{Y(t)}=\E{X(t)P(t)}=2 \E{X(t)\cos(2\pi f_0 t+\theta)}
\]
la variabile aleatoria $\theta$ è indipendente dal processo $N(t)$, e quindi da $X(t)$, pertanto
\[
	\mu_Y(t)=\E{X(t)}\E{P(t)}=0
\]
La funzione autocorrelazione di $Y(t)$
\[
	\begin{split}
		R_Y(\tau)&=\E{Y(t)Y(t-\tau)}=\\
		&=\E{X(t)2\cos{2\pi f_0 t+\theta} X(t-\tau)2\cos{2\pi f_0 (t-\tau)+\theta}}=\\
		&=2 R_X(\tau)\cos{2\pi f_0\tau}
	\end{split}
\]

Risulta quindi che il processo $Y(t)$ è stazionario in senso lato avendo valor medio nullo e funzione di autocorrelazione che non dipende da $t$ ma da $\tau$.

La densità spettrale di potenza di $Y(t)$
\[
	\begin{split}
		S_Y(f)&=\fourier{R_Y(\tau)}=S_X(f-f_0)+S_X(f+f_0)=\\
		&=n\left\lbrace\Sinc^2[(f-f_0)T]+\Sinc^2[(f+f_0)T]\right\rbrace
	\end{split}
\]
Se $f_0\gg\frac{1}{T}$ le due repliche di $S_X(f)$ centrate in $\pm f_0$ si possono considerare non sovrapposte, ovvero le code dei $\Sinc^2(\cdot)$ essersi smorzate completamente.

Infine si ha il filtro passa banda $H(f)$ che mi da garanzia del filtraggio delle frequenze. Per il teorema fondamentale del filtraggio si ha che la densità spettrale di potenza
\[
	S_Z(f)\cong S_Y(f)\abs{H(f)}^2=n\left\lbrace\Sinc^2[(f-f_0)T]\rect{\frac{f-f_0}{2/T}}+\Sinc^2[(f+f_0)T]\rect{\frac{f+f_0}{2/T}}\right\rbrace
\]

\begin{figure}[!ht]
	\centering
	\begin{tikzpicture}
		\begin{axis}[axis lines=middle,no markers,enlargelimits,xscale=2,xtick={-10,-5,0,5,10},ytick={1}]
		\addplot [thick,domain=-10:10,samples=200] {(sin(x-5)/(x-5))^2+(sin(x+5)/(x+5))^2};
		\addplot [very thick,dashed,samples=100,domain=-10:0] {abs((x+5)/5)<.5?1:0};
		\addplot [very thick,dashed,samples=100,domain=0:10] {abs((x-5)/5)<.5?1:0};
		\end{axis}
	\end{tikzpicture}
\end{figure}

\section{Processi ergodici}
\index{processo!stocastico!ergodico}
I parametri statistici di un processo aleatorio sono misure effettuate sull'insieme delle funzioni campione o realizzazioni del processo. La funzione valor medio, ad esempio, determina per un dato istante $t$, la media della funzione densità di probabilità di primo ordine calcolata nell'istante $t$. Questa operazione teorica richiede di saper scrivere in forma chiusa ogni possibile realizzazione con la funzione densità di probabilità di primo ordine (o superiore per le altre statistiche).

Se la funzione densità di probabilità non è nota, è possibile fare ipotesi sul comportamento statistico di un processo dalle misure effettuate da una singola realizzazione?
In generale nulla si può dire tranne che per processi ergodici stazionari in media o in autocorrelazione.

\begin{definizione}
Un processo aleatorio stazionario in media ($\mu_X(t)=\mu_X \;\text{cost}$) si dice \textsc{ergodico in media} se
\begin{equation}
	P\Bigg( \E{X(t)}=\lim\limits_{T\to\infty}\intd{-\frac{T}{2}}{\frac{T}{2}}{x(t)}{t}\Bigg)=1
\end{equation}
ovvero se con probabilità che tende a 1 la media d'insieme coincide con la media temporale su una sola realizzazione.
\end{definizione}

In generale la misura della media temporale è una variabile aleatoria: può essere differente da realizzazione a realizzazione oppure anche se uguale per tutte le realizzazioni essere differente dalla media d'insieme del processo.

Un processo ergodico in media è quindi un processo in cui la singola realizzazione si comporta statisticamente come l'intero processo.
Affinché un processo sia ergodico è innanzitutto necessario che sia stazionario: la media temporale è necessariamente un valor singolo quindi il valor medio del processo (media di insieme) non può essere funzione del tempo.
Inoltre l'uguaglianza tra variabili aleatorie può essere espressa sono in termini probabilistici, affermando cioè che il valor medio temporale coincida con la media d'insieme e la varianza sia nulla.

Non potendo osservare il processo per un tempo illimitato si considera la funzione ristretta all'intervallo limitato $[-T/2,T/2]$, la media temporale è il suo limite per $T\to\infty$
\[
	X_T=\frac{1}{T}\intd{-\frac{T}{2}}{\frac{T}{2}}{x(t)}{t}\qquad X_m=\lim\limits_{T\to +\infty}X_T
\]
si deve dimostrare che la media e la varianza della media temporale
\[
	\mu_{X_m}=\lim\limits_{T\to\infty}\mu_{X_T}\qquad\sigma^2_{X_m}=\lim\limits_{T\to\infty}\sigma^2_{X_T}=0
\]

Dato il sistema in figura \ref{fig:sistema_processo_ergodico} si considera il caso di filtraggio di un processo stazionario in senso lato (v.\ref{sec:filtraggio_processo_SSL}). Il filtro è un integratore a finestra mobile su un intervallo di ampiezza $T$ la cui risposta all'impulso $h(t)=\frac{1}{T}\rect{\frac{t-T/2}{T}}$
\[
	\begin{cases}
		\mu_Y(t)&=\mu_X\cdot H(0)\\
		R_Y(\tau)&=R_X(\tau)\ast h(\tau)\ast h(-\tau)
	\end{cases}
\]

\begin{figure}[h]
	\centering
	\begin{tikzpicture}
		\node [block] (integrator) at (0,0) {
			\begin{tikzpicture}[scale=.8]
				\draw [-latex] (-1,-1.1)--(-1,.7) node[left]{$\frac{1}{T}$} --(-1,1) node[right]{$h(t)$};
				\draw [-latex] (-1.1,-1)--(1,-1)node[below]{$T$}--(1.4,-1);
				\draw (-1,-1) rectangle (1,.7);
			\end{tikzpicture}};
		\node [left = of integrator](input) {$X(t)$};
		\node [right = of integrator](switch) {};
		\node [right = of switch] (output) {};
		\draw [-latex] (input) -- (integrator);
		\draw [-latex] (integrator) -- node[pos=.5,above]{$Y(t)$} (switch);
		\draw [-latex] (switch) to[cspst] node[pos=0,above]{$t=\frac{T}{2}$} (output) node[right]{$X_T$};
	\end{tikzpicture}
	\caption{Variabile aleatoria valor medio temporale delle funzioni campione di un processo}
	\label{fig:sistema_processo_ergodico}
\end{figure}

\begin{proof}[Dim. Media $\mu_{X_m}$]
Il processo SSL in uscita ha media pari al valore
\[
	\mu_{X_T}=\mu_Y\left(\frac{T}{2}\right)=\mu_Y=\mu_X \cdot H(0)
\]

\[
	H(f)=\sinc{f T}\implies H(0)=1\implies \mu_{X_T}=\mu_X \implies \mu_{X_m}=\mu_X 
\]
Si ha un processo stazionario in media con funzione valor medio in uscita uguale al valor medio in ingresso costante.
\end{proof}

\begin{proof}[Dim. Varianza $\sigma^2_{X_m}$]
Devo dimostrare che $\sigma^2_{X_m}=0$
\[
	\sigma^2_{X_T}=\sigma^2_{Y}=C_Y(0)
\]
ipotizzando un processo $X(t)$ stazionario in senso lato
\[
	R_Y(\tau)=R_X(\tau)\ast h(\tau)\ast h(-\tau)
\]
\[
	C_Y(\tau)=R_Y(\tau)-\mu^2=C_X(\tau)\ast h(\tau)\ast h(-\tau) 
\]
\[
	C_Y(\tau)=\E{(Y(t)-\mu)(Y(t-\tau)-\mu)}
\]
per $\tau=0$ si ha l'espressione della varianza $\sigma^2_Y=C_Y(0)$.
La convoluzione dei due $h(t)=\frac{1}{T}\rect{\frac{t-T/2}{T}}$ da il triangolo
\[
	\begin{split}
		C_Y(\tau)&=C_X(\tau)\ast \frac{1}{T}\left(1-\frac{\abs{\tau}}{T}\right)\rect{\frac{\tau}{2 T}}=\\
		&=\intinf{C_X(\alpha)\;\frac{1}{T}\left(1-\frac{\abs{\tau-\alpha}}{T}\right)\rect{\frac{\tau-\alpha}{2 T}}}{\alpha}
	\end{split}
\]

Per $\tau=0$, considerando la parità del rect e del valore assoluto,
\[
	\sigma^2_{X_T}=\sigma^2_{Y}=C_Y(0)=\frac{1}{T}\intinf{C_X(\alpha)\;\left(1-\frac{\abs{\alpha}}{T}\right)\rect{\frac{\alpha}{2 T}}}{\alpha}=\frac{1}{T}\intd{-T}{T}{C_X(\alpha)\; \left(1-\frac{\abs{\alpha}}{T}\right)}{\alpha}
\]

Per avere ergodicità si deve avere
\[
	\sigma^2_{X_m}=\lim\limits_{T\to\infty}\sigma^2_{X_T}=\lim\limits_{T\to\infty}\frac{1}{T}\intd{-T}{T}{C_X(\alpha)\left(1-\frac{\abs{\alpha}}{T}\right)}{\alpha}=0
\]

L'ergodicità del valor medio (statistica del primo ordine) è subordinata alla autocovarianza (statistica del secondo ordine).
\end{proof}

L'operatore media temporale può essere utilizzato per definire l'autocorrelazione di un segnale deterministico a potenza finita
\begin{equation}
	\langle x(t)x(t-\tau)\rangle=\lim\limits_{T\to\infty}\frac{1}{T}\intd{-\frac{T}{2}}{\frac{T}{2}}{x(t) x(t-\tau)}{\tau}
\end{equation}

\begin{definizione}
Un processo aleatorio stazionario in senso lato è \textsc{ergodico in autocorrelazione}\index{processo!stocastico!ergodico!in autocorrelazione} se con probabilità pari ad uno risulta vera
\begin{equation}
	R_X(\tau)=\E{X(t)X(t-\tau)}=\langle x(t)x(t-\tau)\rangle=\lim\limits_{T\to\infty}\intd{-\frac{T}{2}}{\frac{T}{2}}{x(t)x(t-\tau)}{t}
\end{equation}
\end{definizione}
L'ipotesi di stazionarietà è importante per avere una funzione di autocorrelazione d'insieme dipendente da una sola variabile, come l'autocorrelazione temporale.

L'ergodicità in autocorrelazione è importante perché consente di determinare la funzione di autocorrelazione mediante l'osservazione di una singola realizzazione. Da cui è possibile calcolare la densità spettrale di potenza del processo ergodico.

\begin{definizione}
Un processo ergodico in valor medio e in autocorrelazione si dice \textsc{ergodico in senso lato}\index{processo!stocastico!ergodico!in senso lato}.
\end{definizione}

\begin{definizione}
Un processo ergodico si dice \textsc{ergodico in senso stretto}\index{processo!stocastico!ergodico!in senso stretto} se la proprietà di ergodicità vale per qualunque grandezza statistica di qualunque ordine venga estratta dal processo.
\[
	\E{g(X(t),X(t-\tau_1),\dots,X(t-\tau_{n-1}))}=\langle g(X(t),X(t-\tau_1),\dots,X(t-\tau_{n-1}))\rangle
\]
\end{definizione}

\begin{esempio}
Abbiamo visto nell'esempio \ref{es:oscillatore_stazionario_senso_lato} che il processo aleatorio
\[
	X(t)=a\cos{2\pi f_0 t+\theta}
\]
con frequenza e ampiezza noti e fase $\theta$ variabile aleatoria a distribuzione uniforme in $[0,2\pi[$ è un processo stazionario in senso lato, ovvero ha funzione valor medio costante, $\mu_X(t)=0$, e funzione di autocorrelazione $R_X(\tau)=\frac{a^2}{2}\cos{2\pi f_0 \tau}$ dipendente da $\tau=t_1-t_2$.

Si dimostra che tale oscillatore è anche un processo ergodico in senso lato.
\begin{proof}[Dim. ergodicità in media]
Si ha che la media temporale del segnale periodico risulti nulla indipendentemente dalla v.a. $\theta$
\[
	\langle x(t)\rangle=\lim\limits_{T\to\infty}\frac{1}{T}\intd{-\frac{T}{2}}{\frac{T}{2}}{a\cos{2\pi f_0 t+\theta}}{t}=0
\]
\end{proof}

\begin{proof}[Dim. ergodicità in autocorrelazione]
\[
	\begin{split}\langle x(t)x(t-\tau)\rangle&=R_X(\tau)=\lim\limits_{T\to\infty}\frac{1}{T}\intd{-\frac{T}{2}}{\frac{T}{2}}{a\cos{2\pi f_0 t+\theta}a\cos{2\pi f_0 (t-\tau)+\theta}}{t}=\\
	&=\lim\limits_{T\to\infty}\frac{1}{T}\intd{-\frac{T}{2}}{\frac{T}{2}}{a\cos{2\pi f_0 t+\theta}a\cos{2\pi f_0 (t-\tau)+\theta}}{t}=\\
	\intertext{applicando la formula di Werner $\cos\alpha\cos\beta=\frac{\cos{\alpha+\beta}+\cos{\alpha-\beta}}{2}$}
	&=\lim\limits_{T\to\infty}\frac{1}{T}\frac{a^2}{2}\intd{-\frac{T}{2}}{\frac{T}{2}}{[\cos{4\pi f_0 t+2\theta-2\pi f_0\tau}+\cos{2\pi f_0\tau}]}{t}=\\
	&=\lim\limits_{T\to\infty}\frac{a^2}{2 T}\Bigg[\underbrace{\intd{-\frac{T}{2}}{\frac{T}{2}}{\cos{4\pi f_0 t+2\theta-2\pi f_0\tau}}{t}}_{=0}+\intd{-\frac{T}{2}}{\frac{T}{2}}{\cos{2\pi f_0\tau}}{t}\Bigg]=\\
	&=\lim\limits_{T\to\infty}\frac{a^2}{2 T}\intd{-\frac{T}{2}}{\frac{T}{2}}{\cos{2\pi f_0\tau}}{t}=\lim\limits_{T\to\infty}\frac{a^2}{2 T}T\cos{2\pi f_0\tau}=R_X(\tau)
\end{split}
\]
pertanto la funzione autocorrelazione è funzione solo di $\tau$.
\end{proof}
Si è dimostrato quindi che il processo oscillatorio stazionario in senso lato è ergodico in valor medio e in autocorrelazione quindi lo è in senso lato.
\end{esempio}

\chapter[Trasmissione dei segnali]{Principi di base sulla trasmissione dei segnali: teorema del campionamento e quantizzazione}
\section{Trasmissione dei segnali}
La trasmissione di un segnale è un problema comune presente ogni qual volta è necessario trasportare l'informazione associata ad una grandezza fisica da un punto dello spazio ad un altro. Uno schema generico di un sistema di trasmissione prevede sempre i seguenti elementi base:
\begin{itemize}
	\item un \textsc{trasmettitore}, l'apparato in prossimità della sorgente del segnale, che ha il compito di fornire la potenza necessaria al segnale per attraversare il mezzo trasmissivo e giungere riconoscibile al ricevente;
	\item un \textsc{mezzo trasmissivo}, che rappresenta il mezzo fisico attraverso cui si propaga il segnale trasmesso, le cui caratteristiche possono alterare il messaggio. Esistono due grandi categorie: mezzi ad onde convogliate (non dispersivi), mezzi ad onde irradiate (dispersivi);
	\item un \textsc{ricevitore}, l'apparato in prossimità della destinazione del segnale, che ha il compito di estrarre dal mezzo trasmissivo il segnale utile, ovvero quello che contiene l'informazione che trasporta il messaggio originale.
\end{itemize}
\begin{figure}[!ht]
	\begin{center}\begin{tikzpicture}[node distance=2cm]
		\node [block,minimum width=4cm, minimum height=.5cm] (mt) {$MT$};
		\node [block, left =of mt] (tx) {$T_X$};
		\node [block, right =of mt] (rx) {$R_X$};
		\draw [-latex] (tx) -- (mt);
		\draw [-latex] (mt) -- (rx);
		\end{tikzpicture}
	\end{center}
\end{figure}

I mezzi trasmissivi ad onde irradiate sono lo spazio vuoto e l'atmosfera, attraverso cui si propagano onde elettromagnetiche. Il trasmettitore e il ricevitore sono due antenne che irradiano e ricevono potenza del campo elettromagnetico. Le più semplici sono antenne isotrope in cui la potenza del segnale si distribuisce allo stesso modo in ogni direzione dello spazio, propagando il fronte di una superficie sferica di raggio via via crescente alla velocità delle onde elettromagnetiche nel vuoto $c$.

Ad una distanza $R$ dall'antenna trasmittente la potenza ricevuta è attenuata rispetto alla potenza trasmessa secondo una legge
\begin{equation}
P_R=P_T R^{-\alpha}
\end{equation}
con $\alpha$ indice di attenuazione, compreso tra $\alpha=2$ in spazio aperto e $\alpha=4.5$ in ambienti indoor.

I mezzi trasmissivi ad onde convogliate trasmettono la potenza del segnale di tensione o di corrente attraverso sistemi a cavo, come doppino in rame, cavo coassiale, fibra ottica. Per le loro dimensioni non possono essere studiati come circuiti a parametri concentrati e bisogna tener conto degli effetti dissipativi e di ritardo di un mezzo a costanti distribuite.
La potenza trasmessa risulta attenuata, secondo le leggi fisiche del mezzo, in modo lineare con la distanza in unità logaritmiche
\begin{equation}
P_R=P_T\cdot{10}^{-\alpha_\text{TOT}}
\end{equation}
Esprimendo la potenza in decibel
\[P_{R\text{[dB]}}=10\Log P_R=10\Log P_T-\alpha_\text{TOT}=P_{T\text{[dB]}}-\alpha_S\cdot\ell \]
dove si è indicato con $\alpha_S$ l'attenuazione specifica in decibel per unità di lunghezza. Per conduttori in metallo l'attenuazione è funzione della frequenza $\left(\alpha_S=\alpha_\text{r}\sqrt{\frac{f}{f_r}}\right)$.

Un mezzo trasmissivo ideale si suppone lineare e tempo-invariante almeno nella banda di interesse. Il segnale trasmesso $s_T(t)$ giunge in uscita attenuato e ritardato, a causa della velocità di propagazione finita, come $s_R(t)=k\cdot s_T(t-t_0)$, ovvero con una funzione di trasferimento in frequenza \[H(f)=k\e{-\imath 2\pi f t_0}\]

Un mezzo trasmissivo reale è solo approssimativamente lineare, e può avere caratteristiche che variano lentamente nel tempo, e una funzione di trasferimento che attenui diversamente alle varie frequenze, \[H(f)=H_T(f)\e{-\imath 2\pi f t_0}\]

Tale comportamento richiede un filtro di equalizzazione in ricezione che compensi l'effetto del mezzo trasmissivo \[H_\text{eq}(f)=\frac{k}{H(f)}\]

Un mezzo trasmissivo reale introduce sempre una qualche forma di disturbo del segnale trasmesso. In ricezione dunque oltre il segnale distorto dal mezzo sarà presente un segnale indesiderato sovrapposto, genericamente indicato come \textsc{rumore}.

\section{Trasmissione analogica e numerica}

I sistemi di trasmissione distinguono tra \textsc{trasmissione analogica} e \textsc{trasmissione numerica}.
Nella trasmissione analogica l'informazione trasmessa è il segnale stesso, come questo è disponibile al trasmettitore.
Nella trasmissione numerica il segnale viene codificato in \textsc{simboli} a cui corrispondono forme d'onda analogiche.

La trasmissione numerica consente di non dover modificare il sistema a seconda del segnale originale da trasmettere. Si riesce a controllare con precisione l'entità dei disturbi in rapporto al segnale codificato. Si risparmia potenza a parità di informazione trasmessa o, equivalentemente, si può trasmettere più informazione a parità di potenza in trasmissione.

Lo schema di trasmissione/ricezione è più complesso. Per rendere un segnale analogico un segnale numerico è necessario eseguire  operazioni di filtraggio, \textsc{campionamento} e \textsc{quantizzazione}.  Operazioni che hanno la caratteristica di essere invertibili per poter tornare al segnale originario dal lato del ricevitore.

\begin{figure}[!h]
	\begin{center}\begin{tikzpicture}
		\node [block,node distance=1cm] (f) {filtro};
		\node [left of= f,node distance=2cm](input) {$s(t)$};
		\node [block, right of= f,node distance=3cm] (c) {campionamento};
		\node [block, right of= c,node distance=4cm] (q) {quantizzazione};
		\node [block,right of= q,minimum width=3cm, minimum height=.5cm,node distance=4cm] (mt) {$MT$};
		\draw [-latex] (input) -- (f);
		\draw [-latex] (f) -- (c);
		\draw [-latex] (c) -- (q);
		\draw [-latex] (q) -- (mt);
		\end{tikzpicture}
	\end{center}
\end{figure}

\section{Campionamento}\label{sec:campionamento}
Dato un segnale analogico l'operazione di campionamento estrae i valori del segnale in istanti discreti del tempo. Si ottiene un segnale numerico, una serie di numeri reali che rappresentano i campioni del segnale. Il numero di campioni deve essere sufficiente a poter ricostruire il segnale originale.

\begin{figure}[!ht]
	\centering
	\begin{tikzpicture}
	\begin{axis}[axis lines=middle,no markers,enlargelimits,xscale=2,xtick={},ytick={},xlabel=$x$,
	]
	\addplot+[quiver={u=0,v=-y},latex-,black,thick,samples=25,domain=-4:4] {sin(2*x)-cos(pi*x)};
	\addplot[very thick,samples=200,smooth,domain=-4:4] {sin(2*x)-cos(pi*x)};
	\end{axis}\end{tikzpicture}
	\caption{Campionamento}
\end{figure}

Bisogna dimensionare il passo di campionamento $T$ al fine di avere un numero gestibile di campioni che consenta di ricostruire il segnale senza perdere informazione.

\subsection{Teorema del campionamento}
Per ricostruire fedelmente un segnale con spettro a banda limitata $[-B,B]$ su cui si è operato un campionamento a frequenza $f_S=\frac{1}{T}$ si deve avere $f_S\geq B$.
\begin{proof}[Dim.]
Dato il segnale $s(t)$ con spettro $S(f)$ limitato alle frequenze $f\in[-B,B]$ (in banda base), e data la proprietà dell'impulso di estrarre un campione del segnale, si definisce il segnale campionato $s_C(t)$ come il prodotto tra il segnale $s(t)$ e un treno di impulsi di ampiezza unitaria ad intervalli di tempo regolari di periodo $T$
\begin{equation}
s_C(t)=s(t)\cdot\sum_{n=-\infty}^{+\infty}{\impulse(t-n T)}
\end{equation}

Lo spettro del segnale campionato risulta essere dalla trasformata di Fourier la somma di tutte le repliche dello spettro del segnale di partenza $S(f)$ traslate a frequenze multiple di quella di campionamento
\begin{equation}
S_C(f)=S(f)\ast\fourier{\sum_{n=-\infty}^{+\infty}{\impulse(t-n T)}}=S(f)\ast\frac{1}{T}\sum_{n=-\infty}^{+\infty}{\f{\impulse}{f-\frac{n}{T}}}=\frac{1}{T}\sum_{n=-\infty}^{+\infty}{S(f-\frac{n}{T})}
\end{equation}

\begin{figure}[!ht]
	\centering
	\subfloat[][$S(f)$]
	{\begin{tikzpicture}[scale=.6]
		\begin{axis}[axis lines=middle,no markers,enlargelimits,xtick={-1,1},ytick={0},xticklabels={$-B$,$B$},xlabel={$f$}]
		\addplot [very thick]coordinates {(-2,0)(-1,0)(0,1)(1,0)(2,0)};
		\end{axis}\end{tikzpicture}}\qquad\subfloat[][$S_C(f)$] {
		\begin{tikzpicture}[scale=.6]
		\begin{axis}[axis lines=middle,no markers,enlargelimits,xscale=1.8,xtick={-3,-1,1,2,3,4},ytick={1},xticklabels={$-f_S$,$-B$,$B$,$f_S\!-\!B$,$f_S$,$f_S\!+\!B$},yticklabels={$\frac{1}{T}$},xlabel={$f$}]
		\addplot [very thick]coordinates {(-5,0)(-4,0)(-3,1)(-2,0)(-1,0)(0,1)(1,0)(2,0)(3,1)(4,0)(5,0)};
		\addplot [dashed] coordinates{(-1,0)(-1,1)(1,1)(1,0)};
		\end{axis}\end{tikzpicture}	
	}
	\caption{Esempio campionamento con $f_S\geq 2 B$}
\end{figure}\label{fig:campionamento}

Per poter ricostruire il segnale è necessario che le repliche spettrali del segnale campionato non si sovrappongano al segnale in banda base, fenomeno detto \textsc{aliasing}. Deve essere $f_S-B\geq B$ ovvero la frequenza di campionamento deve essere almeno il doppio della banda unilatera del segnale
\begin{equation}
f_S\geq 2 B
\end{equation}
o, equivalentemente, definita \textsc{frequenza di Nyquist} $f_N=f_S / 2$, il segnale di partenza può essere ricostruito se la sua banda unilatera è $B<f_N$.

\begin{figure}[!h]
\centering\begin{tikzpicture}[scale=.6]
\begin{axis}[axis lines=middle,no markers,enlargelimits,xscale=1.8,xtick={-1,.5,1,1.5,2.5},ytick={1},xticklabels={$-B$,$f_S\!-\!B$,$B$,$f_S$,$f_S\!+\!B$},yticklabels={$\frac{1}{T}$},xlabel={$f$}]
\addplot [very thick]coordinates {(-3,0)(-1,0)(0,1)(1,0)(3,0)};
\addplot [very thick,dashed]coordinates {(-3.5,0)(-2.5,0)(-1.5,1)(-.5,0)(.5,0)(1.5,1)(2.5,0)(3.5,0)};
\end{axis}\end{tikzpicture}	
\caption{Esempio aliasing, campionamento con $f_S<2 B$}
\end{figure}\label{fig:aliasing}
\end{proof}
\begin{esempio}
Un segnale audio con una banda $B=\SI{20}{\kilo\hertz}$ nella \textsc{Pulse Code Modulation (PCM)} viene campionato alle frequenze di $\SI{40}{\kilo\hertz}$ o $\SI{48}{\kilo\hertz}$.
\end{esempio}
\begin{nota}
Il segnale viene filtrato prima di essere campionato perché al segnale è sempre sovrapposto il rumore termico a tutte le frequenze.
\end{nota}
\begin{nota}
Se non è possibile campionare alla frequenza $f_S$ è necessario applicare un filtro di \emph{antialiasing} in una banda più stretta di $[-B,B]$, che fa perdere parte dell'informazione del segnale ma consente la ricostruzione seppur meno fedele del segnale originale.
\end{nota}

\subsection{Ricostruzione}

I campioni del segnale numerico giunti al ricevitore attraversano il \textsc{filtro di ricostruzione} per riottenere il segnale analogico di partenza. L'operazione di ricostruzione è necessaria per eliminare le repliche spettrali che non fanno parte dello spettro del segnale di partenza e che sono il risultato dell'operazione di campionamento.

Si utilizza un filtro passa basso ideale con banda pari alla frequenza di Nyquist, che faccia passare le frequenze comprese nell'intervallo $[-\frac{f_S}{2},\frac{f_S}{2}]$, e che compensi l'attenuazione di $\frac{1}{T}$ di $S_C(f)$, per l'eq.\ref{eq:filtro_passa_basso_ideale}
\begin{equation}
H(f)=T \rect{\frac{f}{f_S}}\qquad h(t)=\sinc{\frac{t}{T}}
\end{equation}

Il segnale ricostruito è la convoluzione del segnale campionato con il filtro
\begin{equation}\begin{split}
s_R(t)&=s_C(t)\ast h(t)=\intinf{s_C(\tau)h(t-\tau)}{\tau}=\\
&=\intinf{[s(\tau)\sum_{n=-\infty}^{+\infty}\impulse(\tau-n T)]\sinc{\frac{t-\tau}{T}}}{\tau}=\\
&=\intinf{\left[\sum_{n=-\infty}^{+\infty}s(n T)\impulse(\tau-n T)\right]\sinc{\frac{t-\tau}{T}}}{\tau}=\\
&=\intinf{\sum_{n=-\infty}^{+\infty}s(n T)\sinc{\frac{t-n T}{T}}\impulse(\tau-n T)}{\tau}=\\
\intertext{per la proprietà di setaccio dell'impulso l'integrale si riduce al valore in $\tau=n T$ si ha}
&=\sum_{n=-\infty}^{+\infty}s(n T)\sinc{\frac{t-n T}{T}}=\sum_{n=-\infty}^{+\infty}s(n T)\sinc{\frac{n T-t}{T}}
\end{split}
\end{equation}

Il segnale ricostruito nell'istante $t$ si ottiene come somma dei prodotti tra i campioni del segnale e il valore della funzione seno cardinale calcolata nei corrispondenti istanti di campionamento.

\begin{nota}
Se il segnale è passa banda, il teorema del campionamento continua a valere ma può essere impraticabile campionare al doppio della frequenza massima. In tal caso si sfruttano le repliche in banda base del segnale. Si seleziona una banda $B$ tale che la frequenza massima sia un multiplo intero della banda base, $\frac{f_M}{B}=m\in\N$, e si porrà $f_S\geq B$.
\end{nota}
\begin{nota}
Un campionamento perfetto dovrebbe estrarre l'informazione del segnale istantaneamente, mentre in realtà è necessario periodo di osservazione seppur breve. Un campionatore reale effettua la convoluzione del segnale con un rettangolo molto stretto, $s(t)\ast\rect{\frac{t}{\tau}}$ con un duty cycle molto piccolo, la cui trasformata risulta essere un $\Sinc$ molto ampio ($\frac{1}{\tau}\gg B$). Se in ricezione è richiesta alta fedeltà al segnale originale è sempre possibile equalizzare per compensare l'effetto sagomatore del filtro.
\end{nota}

\section{Quantizzazione}
L'operazione di \textsc{quantizzazione} trasforma i valori continui dei campioni del segnale appartenenti ad un intervallo $[-a,a]$ in valori discreti  dell'intervallo diviso in $Q$ livelli distinti di ampiezza $\Delta=\frac{2a}{Q}$.

Nei sistemi di elaborazione a logica binaria si assegna ad ogni campione del segnale un numero finito fisso di bit $N$ che consente di individuare $2^N$ livelli del segnale.

\begin{figure}[!h]
\centering
\begin{tikzpicture}
\begin{axis}[axis x line=middle,axis y line=left,enlargelimits,domain=-4:4,xlabel={$s(t)$},ylabel={$s_q(t)$},grid=major,xtick={-4,-3,-2,-1,0,1,2,3,4},ytick={-4,-3,-2,-1,0,1,2,3,4},extra y ticks={-3.5,-2.5,-1.5,-.5,0.5,1.5,2.5,3.5},extra y tick labels={$000$,$001$,$010$,$011$,$100$,$101$,$110$,$111$},extra y tick style={ticklabel pos=right}]
\addplot [const plot,very thick] coordinates {(-4,-3.5)(-3,-2.5)(-2,-1.5)(-1,-.5)(0,.5)(1,1.5)(2,2.5)(3,3.5)(4,3.5)};
\addplot [dashed] coordinates {(-4,-4)(4,4)};
\end{axis}
\end{tikzpicture}
\caption{Quantizzatore a 3 bit di segnale $s(t)\in[-4,4]$}
\end{figure}

L'operazione comporta una perdita di informazione irreversibile.

Dato un processo aleatorio stazionario il campionamento di una sua realizzazione $x(t)$ da luogo ad una variabile aleatoria, con la sua funzione densità di probabilità $f(x)$. Si supponga che la dinamica della v.a. assuma valori nell'intervallo $[-a,a]$, e che la quantizzazione in $Q$ livelli sia uniforme con intervalli di quantizzazione di ampiezza $\Delta=\frac{2a}{Q}$. I bordi degli intervalli si trovano in $x_i=-a+i\cdot\Delta$, con $i=0,\dots,Q$.  Per minimizzare l'errore di quantizzazione tra due intervalli successivi il livello assume valore intermedio
\begin{equation}
x_{i-1}\leq x<x_i,\,i=1,\dots,Q,\quad
x_q=\frac{x_i+x_i-1}{2}=-a+i\cdot\Delta-\frac{\Delta}{2}
\end{equation}
da cui risulta un errore di quantizzazione assoluto massimo di $\frac{\Delta}{2}$ in corrispondenza dei bordi degli intervalli.

L'operazione introduce quindi un rumore di quantizzazione che si può valutare con l'errore quadratico medio come
\[N_q=\E{(x-x_q)^2}=\intd{-a}{a}{(x-x_q)^2 f(x)}{x}=\sum_{i=1}^{Q}\intd{x_{i-1}}{x_i}{(x-x_q)^2 f(x)}{x}\]
Per un segnale con distribuzione uniforme si ha $f(x)=\frac{1}{2a},\,x\in[-a,a]$ e 0 altrove, da cui
\[\begin{split}N_q&=\sum_{i=1}^{Q}\intd{-a+(i-1)\Delta}{-a+i\Delta}{\left(x+a-i\Delta+\frac{\Delta}{2}\right)^2 \frac{1}{2a}}{x}=\\\intertext{ponendo $y=x+a-i\Delta+\frac{\Delta}{2}$}&=\frac{1}{2a}\sum_{i=1}^{Q}\intd{-\frac{\Delta}{2}}{\frac{\Delta}{2}}{y^2}{y}=\frac{Q}{2a}\bound{-\frac{\Delta}{2}}{\frac{\Delta}{2}}{\frac{y^3}{3}}=\frac{1}{\Delta}\frac{\Delta^3}{12}=\frac{\Delta^2}{12}
\end{split}\]
Tale quantità costituisce un disturbo che va confrontato con la potenza del segnale dato dallo spettro di potenza
\[S_x=\intd{-a}{a}{x^2\frac{1}{2a}}{x}=\frac{1}{2a}\bound{-a}{a}{\frac{x^3}{3}}=\frac{a^2}{3}=\frac{Q^2\Delta^2}{12}\]
Si ha il \textsc{rapporto segnale rumore di quantizzazione}
\begin{equation}
\frac{S_x}{N_q}=Q^2
\end{equation}
che migliora aumentando il numero di livelli di quantizzazione.
Per un numero di intervalli potenza di due si ha $\frac{S_x}{N_q}=2^{2n}$ che espresso in decibel \[\restrict{\frac{S_x}{N_q}}{\text{dB}}=10\Log 2^{2n}\cong n\cdot\SI{6}{\decibel}\]
ovvero si quadruplica il rapporto segnale rumore per ogni bit di quantizzazione in più.

Tale rapporto segnale rumore è stato calcolato per un processo con densità di probabilità uniforme, ma in generale la quantizzazione può essere non lineare, ad esempio per un processo gaussiano le cui realizzazioni si concentrano attorno al valor medio nullo e con varianza molto piccola. Per ottimizzare il rapporto segnale rumore di quantizzazione è necessaria una suddivisione più fine in livelli dove il segnale è più probabile, descrivendo i campioni con maggiore precisione.


\chapter[Teoria dell’informazione]{Principi di teoria dell’informazione e teorema di Shannon}
\label{ch:teoriasegnali-capitolo9}
In un canale di comunicazione basato su un sistema di trasmissione numerico l'informazione viene trasformata nei blocchi di filtraggio, campionamento e quantizzazione in una sequenza di bit corrispondenti ad un alfabeto di $N$ simboli. In ricezione giunge in generale una sequenza di bit a cui sono associati $M$ simboli. In un canale di comunicazione ideale l'informazione viene trasmessa senza errori di trasmissione pertanto non è possibile scambiare un simbolo per un altro ($N=M$) per cui
\[
\begin{cases}
	p(a_{i_R}|a_{i_T})=1\\
	p(a_{j_R}|a_{i_T})=0 \quad i\neq j
\end{cases}
\]
Un canale binario può trasmettere e ricevere solo due simboli, ad esempio 0 e 1, emessi con probabilità $P(0_T)=P_0$ e $P(1_T)=P_1$. Ogni simbolo trasmesso può essere ricevuto correttamente con probabilità o con errore
\begin{figure}[!h]
	\centering
	\begin{tikzpicture}
		\node (t0) {0};
		\node [below= of t0](t1) {1};
		\node [right of = t0,node distance=4cm](r0) {0};
		\node [below= of r0](r1) {1};
		\draw [-latex] (t0)--(r0) node[above,pos=.3] {$q_0$};
		\draw [-latex] (t0)--(r1) node[above,pos=.3] {$p_0$};
		\draw [-latex] (t1)--(r0) node[below,pos=.8] {$p_1$};
		\draw [-latex] (t1)--(r1) node[below,pos=.8] {$q_1$};
	\end{tikzpicture}
\end{figure}

\[
\begin{cases}
	q_0=P(0_R|0_T)& \text{prob. trasmissione corretta simbolo 0}\\
	p_0=P(1_R|0_T)&\text{prob. trasmissione errata simbolo 0}\\
	q_0+p_0=1\\
	q_1=P(1_R|1_T)& \text{prob. trasmissione corretta simbolo 1}\\
	p_1=P(0_R|1_T)& \text{prob. trasmissione errata simbolo 1}\\
	q_1+p_1=1
\end{cases}
\]

Si ha un \textsc{canale binario simmetrico} quando la probabilità di errore o \textsc{Bit Error Rate} non distingue i due simboli  $p_0=p_1=\text{BER}$

La probabilità di errore per trasmissione su un canale binario
\[
	P(E)=P(E\cap 0_T)+P(E\cap 1_T)=P(1_R|0_T)P(0_T)+P(0_R|1_T)P(1_T)=p_0 P_0+p_1 P_1
\]

Se il canale è binario simmetrico
\[
	P(E)=\text{BER}\cdot(P_0+P_1)=\text{BER}
\]

Le probabilità di ricevere i due simboli
\[
	P(0_R)=P(0_R|0_T)P(0_T)+P(0_R|1_T)P(1_T)=q_0 P_0+p_1 P_1
\]
\[
	P(1_R)=P(1_R|0_T)P(0_T)+P(1_R|1_T)P(1_T)=p_0 P_0+q_1 P_1
\]

\begin{figure}[!h]
	\begin{center}\begin{tikzpicture}[node distance=.5cm]
		\node [draw,circle] (S) {S};
		\node [block, right= of S,text width=2cm] (cs) {codifica sorgente};
		\node [block, right=of cs,text width=2cm] (cc) {codifica canale};
		\node [block,right=of cc,text width=2cm, minimum height=.5cm,node distance=2cm] (c) {canale};
		\node [block, right=of c,text width=2cm] (dc) {decodifica canale};
		\node [draw, dashed, fit=(cc)(c)(dc), rounded corners, inner sep=2mm] {};
		\node [block, right=of dc,text width=2cm] (ds) {decodifica sorgente};
		\node [draw,circle,right=of ds] (R) {R};
		\draw [-latex] (S) -- (cs);
		\draw [-latex] (cs) -- (cc);
		\draw [-latex] (cc) -- (c);
		\draw [-latex] (c) -- (dc);
		\draw [-latex] (dc) -- (ds);
		\draw [-latex] (ds) -- (R);
		\end{tikzpicture}
	\end{center}
\caption{Schema a blocchi di trasmissione numerica}
\end{figure}

\section{Codifica di sorgente}
L'errore sul canale binario può essere ridotto ricorrendo alla \textsc{codifica di sorgente}\index{codifica di sorgente} che trasforma i simboli generati dalla sorgente in una sequenza di bit che possa risolvere o ridurre alcuni problemi causati dalla trasmissione su canale rumoroso.

La codifica di canale è la parte del sistema che trasforma la sequenza di bit in forme d'onda adattate al mezzo trasmissivo.

In ricezione si eseguono le operazioni contrarie per riottenere l'informazione trasmessa.

\subsection{Codice a ripetizione}
Una codifica di sorgente è il \textsc{codice a ripetizione}\index{codifica di sorgente!codice a ripetizione} per un canale binario simmetrico. Ogni bit generato dalla sorgente viene ripetuto $2n+1$ volte, il che riduce la velocità di trasmissione (\emph{bit rate}) dello stesso fattore. Il ricevitore interpreta la sequenza di $2n+1$ bit come il bit che si è presentato in maggioranza. Questo consente in ricezione di rilevare e correggere fino a $n$ errori. La sequenza è interpretata erroneamente se sono errati la maggioranza dei bit, ovvero almeno $n+1$.

La sequenza di bit in ricezione può essere assimilata ad un processo di Bernoulli, con due simboli che arrivano indipendentemente l'uno dall'altro. La probabilità di errore è data da tutte le combinazioni che hanno tra $n+1$ e $2n+1$ bit errati, ovvero
\[p(E)=\sum_{k=n+1}^{2n+1}\binom{2n+1}{k}p^k(1-p)^{2n+1-k}\]

\begin{esempio}
Dato un canale di trasmissione binario con probabilità di errore sul bit $p=10^{-3}$, si vuole dimensionare un codice a ripetizione tale da avere una probabilità di errore $p(E)=10^{-9}$.
Con la formula di Bernoulli si può approssimare con una stima per difetto la probabilità di errore con
\[
	\binom{2n+1}{n+1}p^{n+1}(1-p)^n=\frac{(2n+1)!}{(n+1)!n!}\,p^{n+1}(1-p)^n
\]
con $n=1$ si ha
\[
	p(E)=\frac{3!}{2!1!}(10^{-3})^2(1-10^{-3})\cong 10^{-6}
\]
con $n=2$ si ha 
\[
	p(E)=\frac{5!}{3!2!}(10^{-3})^3(1-10^{-3})^2\cong 10^{-8}
\]
\end{esempio}

\subsection{Codice a controllo di parità}
Nella codifica di sorgente \textsc{a controllo di parità}\index{codifica di sorgente!codice a controllo di parità} si aggiunge alla stringa di bit da trasmettere un bit di parità. In una codifica a parità pari ci si assicura che la stringa di $n$ bit contenga un numero pari di bit uguali ad 1.

Se il canale di trasmissione introduce un errore su un bit con probabilità $p$ il ricevitore rivela la parità errata. In generale il ricevitore può rivelare un numero dispari di errori.
La probabilità di avere un numero dispari di errori è
\[
	p(R)=\sum_{k=1}^{n/2}\binom{n}{2k-1}\,p^{2k-1}(1-p)^{n-2k+1}
\]

Il ricevitore non può rivelare l'errore in caso di un numero pari di errori, il che si verifica con probabilità
\[
	p(E_{NR})=\sum_{k=1}^{n/2}\binom{n}{2k}\,p^{2k}(1-p)^{n-2k}
\]

La probabilità che la trasmissione della stringa di $n$ bit sia corretta è
\[
	p(C)=(1-p)^n
\]

Date le tre alternative si ha
\[
	p(R)+p(E_{NR})+p(C)=1
\]

Il ricevitore può chiedere la ritrasmissione della stringa di $n$ bit quando rivela la parità errata.
Se il ricevitore continua a chiedere la ritrasmissione sino a che non rivela un errore si ha una probabilità di errore totale
\[
	p(E)=p(E_{NR})+p(R)p(E_{NR})+p(R)^2p(E_{NR})+\dots=p(E_{NR})\cdot\sum_{k=0}^{\infty}p(R)^k=\frac{p(E_{NR})}{1-p(R)}
\]

Il numero di ritrasmissioni è una variabile casuale.
La ritrasmissione non è necessaria se la trasmissione è corretta o si verificano uno o più errori non rivelabili, con probabilità
\[
	p(n_R=0)=p(E_{NR})+p(C)=1-p(R)
\]

\`E necessaria $k$ volte con probabilità
\[
	p(n_R=k)=[1-p(R)]p(R)^k
\]

Il numero medio di ritrasmissioni
\[
	\E{n_R}=\sum_{k=0}^{\infty}k\cdot p(n_R=k)=\sum_{k=0}^{\infty}k\cdot[1-p(R)]p(R)^k=\frac{p(R)}{1-p(R)}
\]

Il numero medio di trasmissioni compressa la prima è $n_T=1+n_R$:
\[
	\E{n_T}=1+\E{n_R}=\frac{1}{1-p(R)}
\]
La probabilità di ricezione corretta complessiva considerato il caso con ritrasmissioni
\[
	p(C_T)=p(C)+p(R)p(C)+p(R)^2p(C)+\dots=p(C)\cdot\sum_{k=0}^{\infty}p(R)^k=\frac{p(C)}{1-p(R)}
\]

\begin{esempio}
Dato un canale di trasmissione binario con probabilità di errore sul bit $p=10^{-3}$, si vuole dimensionare un codice a ripetizione tale da avere una probabilità di errore $p(E)=10^{-9}$.

Si da una stima approssimata di errore non rilevato per il caso più probabile di doppio errore di trasmissione
\[
	p(E_{NR})\cong\binom{n}{2}p^2(1-p)^2=\frac{n!}{2!(n-2)!}\cdot 10^{-6}
\]
Non è possibile ridurre il \emph{bit error rate} al di sotto di tale stima ed avere un $\text{BER}\leq 10^{-9}$
\end{esempio}

\begin{esercizio}
Dato un canale di banda $B = \SI{40}{\kHz}$ e un rapporto segnale rumore di quantizzazione $\frac{S}{N_q} = \SI{40}{\decibel}$ calcolare il \emph{bitrate}.

Sapendo che il rapporto segnale rumore è espesso in decibel, vale la relazione $6n \geq \SI{40}{\decibel} \implies n \geq 7$, essendo $n$ il numero di bit, quindi necessariamente intero. Dovendo essere la \emph{frequenza di campionamento} almeno il doppio della banda, risulta $f_s \geq \SI{80}{\kHz}$. Questo implica che il \emph{bitrate} deve essere almeno
\[
	R \geq f_s \cdot N = \SI{560}{\kilo\bit\per\second}
\]

Si vuole adesso calcolare la memoria minima per registrare $t_R = \SI{15}{\minute}$ di questo segnale. La memoria deve essere 
\[
	M \geq R \cdot t_R = \SI{560}{\kilo\bit\per\second} \cdot \SI{900}{\second} = \SI{504}{\mega\bit} = \SI{63}{\mega\byte}
\]

Si osserva che, se si riduce il canale di un fattore $10$, anche la memoria minima necessaria si riduce di un fattore analogo: 
\[
	B=\SI{4}{\kHz} \qquad M\geq\SI{6,3}{\mega\byte}
\]

Si vuole adesso ripetere lo stesso esercizio considerando una \emph{quantizzazione uniforme su distribuzione triangolare}. Da cui risulta che il rapporto segnale rumore $\frac{S}{N_q} = 6n-\SI{3}{\decibel}$. Risolvendo l'esercizio in maniera analoga a quello precedente perveniamo a 
\[
	R \geq \SI{640}{\kilo\bit\per\second} \qquad M \geq \SI{72}{\mega\byte}
\]
\end{esercizio}

\begin{esercizio}
Si consideri un processo aleatorio $W = N+\mu$ con $N$ processo aleatorio SSL con $\mu_N = 0$. Si calcoli la potenza del segnale $W$.

A tal fine notiamo che $\mu_W = \mu_N + \mu = \mu$ e che l'autocorrelazione vale:
\begin{align*}
	R_W(t,t-\tau) &= \E{W(t)W(t-\tau)} = \E{(N(t)+\mu)(N(t-\tau)+\mu)} = \\
	&= \E{N(t)N(t-\tau)} +\mu\E{N(t)} +\mu\E{N(t-\tau)} +\E{\mu^2} = \\
	&= R_N(\tau) +\mu^2
\end{align*}

Quindi la potenza del segnale è $$S_W(f) = S_N(f)+\mu^2\delta(f)$$
Inoltre se $N$ è bianco: $S_W(f) = N_0 +\mu^2\delta(f)$.

\begin{figure}[!h]
	\begin{tikzpicture}[scale=.7]
		\begin{axis}[axis equal, axis lines=middle,no markers,xtick={0},xlabel=$f$,ytick={0,.5,1},yticklabels={0,$N_0$,$\mu^2\delta(f)$},ylabel=$S_W (f)$]
		\addplot [domain=-1:1] {.5};
		\draw [very thick, ->] (axis cs:0,0) -- (axis cs:0,1);
		\end{axis}
\end{tikzpicture}
\end{figure}
\end{esercizio}

\section{Teoria dell'informazione}
Dato un canale di comunicazione basato su un sistema di trasmissione è importante quantificare il limite teorico di informazione trasmissibile (\textsc{Shannon}, 1948). \`E fondamentale poter confrontare i sistemi reali affetti da rumore gaussiano rispetto ad un sistema teorico, che astrae i dettagli implementativi e fissa un limite teorico di efficienza del sistema di trasmissione.

Nel modello l'informazione viene generata da una sorgente che emette, a velocità costante e incorrelati tra loro, $m$ simboli $x_1,x_2,\dots,x_m$ scelti da un alfabeto. Un legge di codifica associa ad ogni simbolo una sequenza di bit. Per rappresentare $m$ simboli di un alfabeto sono necessari $\log_2 m$ bit/simbolo.

Ogni simbolo $x_i$ può occorrere con una certa probabilità $p_i=p(x_i)$ ed essere rappresentato con una stringa di $n_i$ bit.
Se i simboli sono equiprobabili è ragionevole una codifica a lunghezza fissa. Se i simboli non sono equiprobabili è ragionevole attribuire stringhe di bit più corte ai simboli più probabili.

La quantità di informazione media che transita sul canale sarà quindi una media pesata delle lunghezze in bit dei simboli per la probabilità dei simboli:
\[
	\sum_{i}p_i\cdot n_i
\]

Nel caso di una sorgente ergodica è possibile derivare le statistiche del processo dall'osservazione di una singola realizzazione. In un messaggio costituito da una sequenza di $N$ simboli, con $N$ grande, ogni simbolo $x_i$ si presenterà mediamente $n_i=p_i N$ volte. Con questi $N$ simboli si possono costruire numerosi messaggi che rispettino tali statistiche di sorgente e che si differenziano tra loro per la posizione dei simboli all'interno del messaggio. La probabilità di generare un messaggio lecito è
\begin{equation}
	p_\text{mess}=p_1^{p_1 N}\cdot p_2^{p_2 N}\cdot\dots\cdot p_m^{p_m N}
\end{equation}

Per una sorgente ergodica ogni messaggio lecito di lunghezza $N$ è equiprobabile, essendo i simboli emessi in modo indipendente il particolare ordine di emissione non è importante, quindi tutti i possibili messaggi leciti con $N$ simboli sono proprio $1/p_\text{mess}$.
Per rappresentare quindi un tal numero di messaggi leciti sono necessari un numero di bit di informazione pari a
\[
	n=\log_2\frac{1}{p_\text{mess}}=-\log_2 p_\text{mess}
\]
e quindi il numero medio di bit per simbolo generato dalla sorgente S, anche detta \textsc{entropia della sorgente}\index{entropia della sorgente}
\begin{equation}
	\begin{split}
		H(S)=\frac{n}{N}&=\frac{1}{N}\log_2\frac{1}{p_\text{mess}}=-\frac{1}{N}\log_2 p_\text{mess}=\\
		&=-\frac{1}{N}\log_2\prod_{i=1}^{m}p_i^{N p_i}=-\frac{1}{N}\sum_{i=1}^{m}\log_2 p_i^{N p_i}=\\
		&=-\sum_{i=1}^{m}p_i\log_2 p_i
	\end{split}
\end{equation}
La quantità $I(x_i)=-\log_2 p(x_i)$ rappresenta il minimo numero di bit necessari per descrivere il simbolo $x_i$ e quindi una misura dell'informazione associata al simbolo. Quanto più probabile è l'emissione di un simbolo tanto meno informazione esso trasporta. Si possono riassumere le seguenti proprietà per l'informazione associata alla sorgente:
\begin{enumerate}
\item L'emissione di un simbolo come evento certo non contiene alcuna informazione
\[
	p(x_i)=1 \iff I(x_i)=0
\]
\item Vi è maggiore informazione associata ad un simbolo meno probabile
\[
	I(x_i)>I(x_j)\iff p(x_i)<p(x_j)
\]
\item Per coppie di simboli emessi in modo indipendente si ha
\[
	P(x_i,x_j)=p(x_i)\cdot p(x_j)\implies I(x_i,x_j)=I(x_i)+I(x_j)
\]
infatti $I(x_i,x_j)=\log_2\frac{1}{p(x_i,x_j)}=\log_2\frac{1}{p(x_1)p(x_2)}=\log_2\frac{1}{p(x_i)}+\log_2\frac{1}{p(x_j)}=I(x_i)+I(x_j)$
\item Dato un alfabeto di $M$ simboli, l'entropia della sorgente verifica la disuguaglianza
\begin{equation}
	H(S)\leq\log_2 M
\end{equation}
\begin{proof}[Dim.]
\[
	H(S)-\log_2 M\leq 0
\]
\[
	\sum_{i=1}^{M}p_i\log_2\frac{1}{p_i}-\log_2 M\leq 0
\]
sviluppando il primo membro, ed essendo $\sum_i p_i=1$
\[
	\sum_{i=1}^{M}p_i\log_2\frac{1}{p_i}-\sum_{i=1}^{M}p_i\log_2 M=\sum_{i=1}^{M}p_i\left[\log_2\frac{1}{p_i}-\log_2 M\right]=\sum_{i=1}^{M}p_i\log_2\frac{1}{p_i M}
\]
che è possibile maggiorare essendo $\log y\leq y-1$
\[
	\leq \sum_{i=1}^{M}p_i\left(\frac{1}{p_i M}-1\right)\log_2\e{}= \sum_{i=1}^{M}\left(\frac{1}{M}-p_i\right)\log_2\e{}=0
\]
\end{proof}

\begin{esempio}
Nel caso di sorgente con $M$ simboli equiprobabili si ha
\[
	p_i=p=\frac{1}{M}\qquad I(x)=\log_2\frac{1}{M}
\]
L'entropia della sorgente
\[
	H(S)=\sum_{i=1}^{M}p_i I(x_i)=-\sum_{i=1}^{M}\frac{1}{M}\log_2\frac{1}{M}=\log_2 M
\]
assume il valore massimo teorico con l'uguaglianza $H(S)=\log_2 M$.
\end{esempio}

\begin{esempio}
Nel caso di sorgente binaria con 2 simboli con probabilità $p_0=p$ e $p_1=1-p$ si ha l'entropia
\[
	H(S)=-p\log_2 p-(1-p)\log_2(1-p)
\]

\begin{figure}[!h]
	\centering
	\begin{tikzpicture}[scale=.6]
		\begin{axis}[axis lines=middle,no markers,enlargelimits,xtick={0.5,1},xticklabels={$\frac{1}{2}$,$1$},ytick={1},xlabel={$p$},ylabel={$H(S)$},xscale=2]
		\addplot [thick,domain=0:1] { -x*ln(x)/ln(2)-(1-x)*ln(1-x)/ln(2) };
		\end{axis}
	\end{tikzpicture}
	\caption{Entropia di una sorgente binaria}
\end{figure}
\end{esempio}
\end{enumerate}

\section{Sorgente con memoria}
Una sorgente può emettere simboli che non siano statisticamente indipendenti tra loro. La definizione di entropia della sorgente deve tener conto dell'informazione associata alla dipendenza statistica fra simboli.

L'informazione legata all'emissione del simbolo $x_1$ condizionata all'emissione di un simbolo precedente $x_0$:
\[
	I(x_1|x_0)=-\log_2(x_1|x_0)
\]

L'entropia della sorgente, ovvero l'informazione media per simbolo generato da sorgente con memoria, condizionata all'emissione di un simbolo $x_0$ risulta
\[
	H(S|x_0)=\sum_i p(x_i|x_0) I(x_i|x_0)=-\sum_i p(x_i|x_0)\log_2 p(x_i|x_0)\]

\`E possibile definire l'informazione media o entropia del primo ordine considerando la media pesata di tutte le possibili emissioni di simboli precedenti. Per estensione si può supporre una sorgente con memoria più estesa e dipendenza tra più simboli in sequenza.

\section{Teorema di Shannon}
\index{Teorema!di Shannon}
Si definisce la \textsc{capacità del canale}\index{canale trasmissivo!capacità} trasmissivo la misura dell'informazione che il canale è in grado di far transitare, ovvero alla quantità di bit nell'unità di tempo che riescono a transitare correttamente sul canale.

Su un canale reale i simboli in uscita dal mezzo trasmissivo sono affetti da errori, l'entropia del canale contiene anche informazione errata a causa degli errori di trasmissione.

Dato un canale affetto da errore di tipo gaussiano, a media nulla e data varianza, si ha che la capacità del canale, calcolata in bit/s,
\begin{equation}
	C=B \log_2 \left(1+\frac{S}{N}\right)
\end{equation}

Tale risultato, noto come \textsc{teorema di Shannon}, permette di stabilire un limite superiore alla capacità di trasmettere bit sul canale, dato il rapporto tra potenza statistica del segnale sorgente $S$ e potenza statistica del rumore di canale $N$.

\begin{nota}
	\`E possibile aumentare il rapporto segnale/rumore incrementando la potenza del segnale trasmesso o riducendo il rumore sul canale (ad esempio avvicinando Tx e Rx o modificando la tecnologia del mezzo trasmissivo).
\end{nota}

%\chapter*{Appendice}

\begin{sidewaystable}
\centering
\caption{Variabili aleatorie}
\begin{tabular}{cccccc}
	\toprule
	\rule[-2ex]{0pt}{5.5ex} v.a.
		& \(f_X(x)\) \quad
			\tablefootnote{funzione probabilità di massa per v.a. discrete
			\(f_X(x) = \sum_k p_k \impulse(x-k)\)}
		& \(\mu_X\)
		& \(\E{X^2}\)
		& \(\sigma^2_X\)
		& \(G_X(t)\) \quad
			\tablefootnote{funzione generatrice dei momenti} \\
	\midrule
	\rule[-2ex]{0pt}{5.5ex} uniforme
		& \(\frac{1}{b-a}\)
		& \(\frac{b+a}{2}\)
		& \(\frac{b^2+ab+a^2}{3}\)
		& \(\frac{(b-a)^2}{12}\) \\
	\midrule
	\rule[-2ex]{0pt}{5.5ex} esponenziale
		& \(\lambda\e{-\lambda x}\step(x)\)
		& \(\frac{1}{\lambda}\)
		& \(\frac{2}{\lambda^2}\)
		& \(\frac{1}{\lambda^2}\)\\
	\midrule
	\rule[-2ex]{0pt}{5.5ex} disc. Poisson
		& \(\frac{(\Lambda T)^k}{k!}\e{-\Lambda T}\)
		& \(\Lambda T\)
		& \((\Lambda T)^2+\Lambda T\)
		&
		& \(\e{-\Lambda T}\e{\Lambda T\e{t}}\) \\
	\midrule
	\rule[-2ex]{0pt}{5.5ex} disc. Bernoulli
		& \(\binom{n}{k}p^k(1-p)^{n-k}\)
		& \(n p\)
		& \(n p [p(n-1)+1]\)
		& \(n p(1-p)\)
		& \((p\e{t}+1-p)^n\) \\
	\midrule
	\rule[-2ex]{0pt}{5.5ex} disc. geometrica
		& \(p^k(1-p)\)
		& \(\frac{p}{1-p}\)
		& \(\frac{p(1-p)}{(1-p)^2}\)
		& \(\frac{p}{(1-p)^2}\)
		& \((1-p)(1-p\e{t})^{-1}\) \\
	\midrule
	\rule[-2ex]{0pt}{5.5ex} gaussiana
		& \(\frac{1}{\sqrt{2\pi\sigma^2_X}}\;\e{-\tfrac{(x-\mu_X)^2}{2\sigma^2_X}}\)
		& \(\mu_X\)
		& \(\mu^2_X+\sigma^2_X\)
		& \(\sigma^2_X\)
		& \(\e{t\mu_X+\frac{\sigma^2_X t^2}{2}}\) \\
	\midrule
	\rule[-2ex]{0pt}{5.5ex} paretiana
		& \(\begin{cases} \alpha \frac{x_m^\alpha}{x^{\alpha + 1}}
		& x > x_m \\ 0
		& \text{altrove} \end{cases}\)
		& \(\begin{cases} \frac{\alpha}{\alpha -1}x_m
		& \alpha > 1 \\ +\infty
		& \alpha \leq 1 \end{cases}\)
		& \(\begin{cases} \frac{\alpha}{\alpha -2}x_m^2
		& \alpha > 2 \\ +\infty
		& \alpha \leq 2 \end{cases}\)
		& \(\frac{\mu_X^2}{\alpha (\alpha -2)}\) \\
	\midrule
	\rule[-2ex]{0pt}{5.5ex} disc. Zeta
		& \(\frac{C}{k^{\alpha+1}}\)
		& \(\frac{\zeta(\alpha-1)}{\zeta(\alpha)}\)
		& \(\frac{\zeta(\alpha -2)}{\zeta(\alpha)}\)
		& \(\frac{\zeta(\alpha-2)\zeta(\alpha) - \zeta(\alpha-1)^2}{\zeta(\alpha)^2}\)
		& \(\frac{1}{\zeta(\alpha)} \sum_{k=1}^\infty \frac{\e{tk}}{k^{\alpha}}\) \\
	\bottomrule
\end{tabular}
\end{sidewaystable}

\part{Fondamenti di Telecomunicazioni}
\chapter{Sistemi di telecomunicazione}\label{sec:sistemi-di-telecomunicazione}\index{sistema di telecomunicazione}

\section{Obiettivi del corso}

Applicando la teoria dei segnali si vogliono acquisire la capacità di progettare sistemi di telecomunicazione per la trasmissione dell'informazione da una sorgente ad un ricevitore su un canale trasmissivo garantendo un sufficiente livello di accuratezza in presenza di fenomeni di disturbo.

Un sistema di telecomunicazione tra una \keyword[sistema di telecomunicazione!sorgente]{sorgente} dell'informazione e un \keyword[sistema di telecomunicazione!destinatario]{destinatario} è modellato per componenti: un \keyword[sistema di telecomunicazione!trasmettitore]{trasmettitore}, un \keyword[sistema di telecomunicazione!canale di trasmissione]{canale di trasmissione}, un \keyword[sistema di telecomunicazione!ricevitore]{ricevitore}. Il messaggio contenente l'informazione generata alla sorgente $m(t)$ viene trasmesso sul canale di trasmissione e ricevuto come $\tilde{m}(t)$ al destinatario. Ogni passaggio può alterare il contenuto del messaggio che viene elaborato, filtrato, affetto da disturbi, non linearità, attenuazione e distorsione. Per ricostruire il messaggio originale è necessario che tali modifiche siano reversibili.

Per sistemi analogici l'informazione è generata, elaborata e trasmessa come onda elettromagnetica. Nei sistemi digitali l'informazione è rappresentata da stringhe di bit alla sorgente che il trasmettitore trasforma in onde elettromagnetiche per la trasmissione.

\begin{figure}[!ht]
\centering
\resizebox{\textwidth}{!}{
\begin{tikzpicture}[node distance=3.5cm,minimum width=2.5cm,text width=2.5cm,align=center, >=latex'];
\node [block](t1) {Elaborazione del segnale};
\node [block,right of=t1](t2) {Modulatore}edge[<-](t1);
\node [block,right of=t2,node distance=5cm,minimum height=1em](c) {Mezzo Tx};
\node [fitted,fit=(t1) (t2),label=above left:Trasmettitore] {};
\node [fitted,fit=(c),label=above:Canale]{};
\node [above of=c,node distance=1.5cm,pin=above:Rumore](n) {$n(t)$}edge[->](c);
\node [block,right of=c,node distance=5cm](r1) {Demodulatore};
\node [block,right of=r1](r2) {Elaborazione del segnale}edge[<-](r1);
\node [fitted,fit=(r1) (r2),label=above left:Ricevitore] {};
\node [left of=t1,node distance=3cm,minimum width=1.5cm,text width=.5cm,pin=below:Informazione in ingresso](m1) {$m(t)$}edge[->](t1);
\node [right of=r2,node distance=3cm,minimum width=1cm,text width=.5cm,pin=below:Dati per l'utente](m2) {$\hat{m}(t)$} edge[<-](r2);
\draw [->] (t2)--node[above]{$s(t)$}(c);
\draw [->] (c)--node[above]{$r(t)$}(r1);
\end{tikzpicture}
}
\caption{Schema di principio di un sistema di TLC}\label{fig:schema_sistema_telecomunicazioni}
\end{figure}

L'informazione prodotta dalla sorgente in forma analogica o digitale viene elaborata dal trasmettitore per ottenere un segnale adattato al mezzo di trasmissione: segnali con spettri in \keyword[spettro!in banda base]{banda base} concentrati attorno alla frequenza nulla possono essere trasmessi direttamente su canali passa basso, che lasciano passare non attenuate le frequenze nella banda base. Se il canale risulta già utilizzato a tali frequenze o se per caratteristiche del canale tali frequenze sono troppo attenuate o distorte è possibile modulare ovvero traslare lo spettro del segnale nella \keyword[spettro!in banda passante]{banda passante} del canale trasmissivo. \footnote{Se la potenza a frequenza zero è nulla è assente il segnale in componente continua}

In base alle frequenze utilizzabili nel canale e alla tipologia di trasmissione per mezzo di conduttori o via radio i mezzi trasmissivi si distinguono per tecnologia di canale ad onde elettromagnetiche convogliate o ad onde irradiate e per la risposta in frequenza nei canali passa basso o passa banda.

Ogni componente del sistema di telecomunicazione, a causa di vari fenomeni fisici, è affetto da \keyword[sistema di telecomunicazione!rumore]{rumore} elettronico che altera il segnale in modo indesiderato. Tale segnale di disturbo aleatorio è sovrapposto al segnale contenente l'informazione, ne altera il contributo in potenza a varie frequenze, e risulta indistinto dal segnale utile e causa un \keyword[sistema di telecomunicazione!tasso di errore di trasmissione]{tasso di errore di trasmissione} caratteristico del sistema di telecomunicazione, o \ac{BER} nei sistemi di trasmissione numerica. 

In ogni punto del sistema di telecomunicazione sarà possibile misurare il rapporto segnale rumore (\ac{SNR}). La progettazione di un sistema di telecomunicazione ha lo scopo di dimensionare ogni componente del sistema al fine di minimizzare la potenza necessaria a trasmettere l'informazione stabilendo un rapporto segnale rumore (\ac{SNR}) tale che sia soddisfatto il tasso di errore per sistemi analogici o il \ac{BER} per sistemi numerici digitali.

\begin{table}[!ht]\centering
	\begin{tabular}{c|p{0.3\textwidth}|p{0.3\textwidth}}
		\hline \rule[-2ex]{0pt}{5.5ex}  & Onde Convogliate & Onde Irradiate \\ 
		\hline \rule[-2ex]{0pt}{5.5ex} \textsc{LP} & \parbox[c]{5cm}{Doppino telefonico\\ Cavo coassiale} &  \\ 
		\hline \rule[-2ex]{0pt}{5.5ex} \textsc{BP} & Fibra Ottica & Mezzo Radio \\ 
		\hline 
	\end{tabular} 
	\caption{Tipologie di mezzi trasmissivi}\label{tab:mezzi-trasmissivi}
\end{table}

\section{Rumore}
Nel corso di \textsc{Fondamenti delle Telecomunicazioni} si adotterà l'ipotesi semplificativa di avere sempre un disturbo o rumore descritto da un processo aleatorio \emph{gaussiano bianco}. Tale modello dato dalla sovrapposizione di un gran numero di eventi elementari è valido per frequenze $f\ll\SI{10}{\giga\hertz}$. Si considereranno due tipologie di rumore.

\begin{definizione}
Il \textbf{rumore termico} gaussiano bianco è originato dall'agitazione termica degli elettroni. Per frequenze sufficientemente piccole si hanno i seguenti spettri di potenza:
\begin{equation}\begin{split}
N_W(f)=k_B T \,[\si{\watt\per\hertz}] &\quad \text{spettro potenza monolatero}\footnotemark\\
N_V(f)=4 k_B T R \,[\si{\volt\squared\per\hertz}] &\quad \text{spettro potenza a vuoto}
\end{split}\end{equation}
\footnotetext{ costante di Boltzmann $k_B=\SI{1.3806488e-23}{\joule\per\kelvin}$}
\end{definizione}
\begin{definizione}
\textbf{Rumore granulare} generato dal movimento di cariche elettriche attraverso una barriera di potenziale (es. giunzione $np$):
\begin{equation}\begin{split} N_I(f)= 2 q I [\si{\ampere\squared\per\hertz}]\quad\text{spettro potenza monolatero} \end{split}\end{equation}
\end{definizione}

\section{Rumore in catene di amplificazione}
Si ipotizza una catena di amplificatori nel ricevitore, ognuno dei quali modifica il segnale amplificando di un fattore l'ampiezza del segnale.

Si assume che i dispositivi siano a carichi adattati, ovvero abbiano la medesima impedenza alla porta (l'equivalente di Thevenin o di Norton del circuito). Essendo ogni resistore affetto da rumore termico si ha all'uscita la somma del rumore presente in ingresso amplificato da ogni stadio della catena di amplificatori. Il rumore introdotto da un amplificatore o catena di amplificatori può essere riportato in modo equivalente all'ingresso.

Possiamo usare due modelli equivalenti di rumore in catena di amplificazione.

\begin{figure}[t]
	\centering
	\subfloat[Modello additivo]{
	\begin{circuitikz}[scale=.6,american currents]
		\draw[dashed] (3,4)--(3,-1)node[below]{$k T_g$} (7,4)--(7,-1)node[below]{$k T_s$};
		\draw (2,0)--(0,0)	to[I=${T_g}$] (0,3) to[short,-o] (3,3) --(5,3) (6,3) to[short,-o] (7,3)--(8,3) to[R, l=${R_L}$] (8,0)--(7,0) (5,1.5) node[block,minimum height=3cm]{\parbox{1.5cm}{\centering$A$\\ $T_a$}}
		(2,3) to[R, l=${R}$] (2,0) to[short,-o] (3,0)--(5,0) (6,0) to[short,-o] (7,0);
	\end{circuitikz}\label{fig:modelli_eq_rumore_additivo}
	}\quad\subfloat[Modello fattore di rumore]{\begin{circuitikz}[scale=.6,american currents]
		\draw[dashed] (3,4)--(3,-1)node[below]{$k T_g$} (7,4)--(7,-1)node[below]{$k T_s$};
		\draw (2,0)--(0,0)	to[I=${T_0}$] (0,3) to[short,-o] (3,3) --(5,3) (6,3) to[short,-o] (7,3)--(8,3) to[R, l=${R_L}$] (8,0)--(7,0) (5,1.5) node[block,minimum height=3cm]{\parbox{1.5cm}{\centering$A$\\ $F$}}
		(2,3) to[R, l=${R}$] (2,0) to[short,-o] (3,0)--(5,0) (6,0) to[short,-o] (7,0);
	\end{circuitikz}
	\label{fig:modelli_eq_rumore_moltiplicativo}}
	\caption{Modelli equivalenti di rumore in catena di amplificazione}
\end{figure}

\subsubsection{Modello equivalente di rumore additivo in catena di amplificazione}

\begin{itemize}
	\item Amplificatore con guadagno di potenza $A$
	\item Temperatura equivalente di rumore dell'amplificatore $T_a$ \footnote{ $T_a$ è funzione della frequenza ma per ipotesi semplificativa si considera costante}
	\item Temperatura equivalente di rumore del sistema $T_s=T_g+T_a$ 
	\item Densità spettrale di rumore in uscita all'amplificatore 
	\begin{equation} h_n=k(T_g+T_a)=k T_s \,[\si{\watt\per\hertz}] \end{equation}
\end{itemize}

\subsubsection{Modello equivalente di rumore moltiplicativo in catena di amplificazione}
\begin{itemize}
	\item Fattore $F$ di rumore dell'amplificatore
	\item $F$ è definito ad una data temperatura $T_0$ dell'impedenza (es. $T_0=\SI{293}{\kelvin}$)
	\item Densità spettrale di rumore in uscita
	\begin{equation} h_n=F k T_0 \end{equation}
\end{itemize}

Se $T_g=T_0 \implies F k T_0 = k(T_0+T_a) \implies T_a=(F-1) T_0 \quad F=1+\frac{T_a}{T_0}$

\subsubsection{Catene di amplificazione}
\begin{figure}[!ht]
	\centering
	\begin{tikzpicture}[start chain=going right,node distance=8mm,>=latex',every node/.style={on chain},every join/.style={->},	block/.style={draw,align=center,minimum width=3em}]
	\node[join]{$k T_g$};
	\foreach\i in{1,2,3} {
		\node[sum,join]{$+$};
		\begin{scope}[start branch=above,every join/.style={<-,thick,shorten <=1pt}]
			\node[on chain=going above,join]{$k T_{A_\i}$};
		\end{scope}
		\node[block,join]{$A_\i$};
	}
	\coordinate[join](end);
	\end{tikzpicture}
	\caption{Catena di 3 amplificatori di potenza con modello additivo di temp. equiv. di rumore}
	\label{fig:catena_amplificazione}
\end{figure}

La temperatura equivalente di rumore della catena di amplificazione di potenza in fig.\ref{fig:catena_amplificazione} si ottiene riportando all'ingresso del I stadio tutti i contributi.

\begin{equation}\label{eq:temp_eq_rumore}
T_A=T_{A_1}+\frac{T_{A_2}}{A_1}+\frac{T_{A_3}}{A_1 A_2}
\end{equation}

\begin{itemize}
\item Il contributo principale al rumore è dato dal I stadio
\item Il contributo del II e III stadio sono attenuati
\end{itemize}
\begin{figure}[!ht]
	\centering
	\begin{tikzpicture}[start chain=going right,node distance=8mm,>=latex',every node/.style={on chain},every join/.style={->},	block/.style={draw,align=center,minimum width=3em}]
	\node[join]{$k T_0$};
	\foreach\i in{1,2,3} {
		\node[sum,join]{$+$};
		\begin{scope}[start branch=above,every join/.style={<-,thick,shorten <=1pt}]
		\node[on chain=going above,join]{$(F_\i-1)k T_0$};
		\end{scope}
		\node[block,join]{$A_\i$};
	}
	\coordinate[join](end);
	\end{tikzpicture}
	\caption{Catena di 3 amplificatori di potenza con modello moltiplicativo di temp. equiv. di rumore}
	\label{fig:catena_amplificazione_fattore_rumore}
\end{figure}

Si ricava facilmente il fattore di rumore sostituendo $T_A$ nell'eq.\ref{eq:temp_eq_rumore} in $F=1+\frac{T_A}{T_0}$.
\begin{equation}
F=F_1+\frac{F_2-1}{A_1}+\frac{F_3-1}{A_1 A_2}
\end{equation}
Il primo termine $F_1$ incide maggiormente essendo il secondo e terzo termine scalati di valori di amplificazione a monte $A_1$ e $A_1 A_2$.

\section{Attenuatore passivo}
\`{E} importante calcolare il peggioramento di prestazioni introdotto da un cavo o una guida d'onda frapposta tra due componenti. Ad esempio una \keyword[sistema di telecomunicazione!antenna]{Antenna} collegata con un cavo al \keyword[sistema di telecomunicazione!amplificazione]{Primo Stadio di Amplificazione} di un sistema di ricezione:
\begin{figure}[!ht]
	\centering
	\begin{circuitikz}[scale=.8]
		\draw (0,0) node[antenna,label=left:Antenna]{} to[short] (4,0) node[draw,block,label=below:Amplificatore]{$A$};
		\draw [decorate,decoration={brace,amplitude=10pt,mirror},yshift=-12pt] (0,0)-- node [black,pos=.5,yshift=-25pt] {Attenuatore passivo} (3,0);
	\end{circuitikz}
\end{figure}
\begin{itemize}
	\item L'attenuatore attenua la potenza del segnale in ingresso $P_i$ di un fattore $\frac{1}{\alpha}$ con $\alpha>1$
	\item Il generatore di segnale adattato e l'attenuatore sono equivalenti ad un elemento resistivo di $T_0$ pertanto il rumore in uscita all'attenuatore ed in ingresso all'amplificatore è lo stesso di una rete passiva a temperatura $T_0$:
	\[h_n=k T_0\]
	\item A maggiore attenuazione corrisponde maggiore rumore
\end{itemize}

\begin{equation}
k T_0 = \underbrace{\frac{k T_0}{\alpha}}_{\parbox{2.5cm}{contributo del\\generatore}}+\underbrace{\frac{k T_\text{att}}{\alpha}}_{\parbox{2cm}{\centering contributo\\dell'attenuatore}}
\end{equation}

Temperatura equivalente rumore attenuatore passivo:
\begin{equation} T_\text{att}=(\alpha-1) T_0\end{equation} 
con fattore di rumore di attenuazione (di quanto moltiplicare il rumore del generatore per considerare l'effetto dell'attenuatore passivo): $ F_\text{att}=\alpha $

\begin{esempio}
	Si consideri un sistema con temperatura di rumore al generatore $T_g=\SI{100}{\kelvin}$ e un amplificatore con temperatura equivalente di rumore $T_a=\SI{150}{\kelvin}$. Cosa accade se il collegamento tra antenna e ricevitore è realizzato con una linea che attenua di $\SI{1}{\decibel}$ e $\SI{2}{\decibel}$?
	
	\begin{itemize}
		\item In assenza di perdite, ipotizzando l'antenna connessa direttamente all'amplificatore, la temperatura equivalente di rumore complessiva del sistema è
		\[T_S=T_g+T_a=\SI{250}{\kelvin}\]
		\item Inserendo un cavo con effetto di attenuatore passivo si ha una attenuazione in decibel di 
		\[\alpha_\text{dB}=10\Log\alpha=1 \iff \alpha=10^{0.1}\cong\num{1.26}\]
		ovvero un fattore di rumore sulla linea
		\[ F= \alpha= \num{1.26}\]
		sapendo che il contributo della linea non dipende dall'ingresso si ha
		\[ T_\text{att}=(\alpha-1) T_0 \]
		da cui si calcola la temperatura della catena di amplificazione
		\[ T_S=T_g+T_\text{att}+\frac{ T_a }{\frac{1}{\alpha}}= T_g+(\alpha-1)T_0+\alpha T_a=\SI{100}{\kelvin}+0.26\cdot\SI{293}{\kelvin}+1.26\cdot\SI{150}{\kelvin}=\SI{365}{\kelvin}\]
		
		\item Con una perdita di $\SI{2}{\decibel}$ si ha una attenuazione
		\[10\Log\alpha=2 \iff \alpha=10^{0.2}\cong\num{1.58}\]
		\[ T_S=T_g+T_\text{att}+\alpha T_a= T_g+(\alpha-1)T_0+\alpha T_a=\SI{100}{\kelvin}+0.58\cdot\SI{293}{\kelvin}+1.58\cdot\SI{150}{\kelvin}=\SI{507}{\kelvin}\]
	\end{itemize}
\end{esempio}

\begin{esempio}Si ha un sistema composto da due filtri $H_1(f)$ passa basso e $H_2(f)$ passa banda in parallelo con in ingresso un segnale di rumore con densità spettrale di potenza\index{densità spettrale di potenza} $h_n(f)$ costante. 
	
\begin{figure}[!ht]\centering
	\begin{tikzpicture}[>=latex']
	\draw node (hn) {$h_n$};
	\coordinate [right=1cm of hn] (n1);
	\draw node[above right=1cm of n1,passabasso,label=below:=Filtro passa basso](h1){};
	\draw node[below right=1cm of n1,passabanda,label=below=:Filtro passa banda](h2){};
	\draw node[sum,right=2.75cm of n1] (s1){$+$};
	\draw node[right=1cm of s1] (hnu) {$h_{n_u}$}edge[<-](s1);
	\draw (hn)--(n1);
	\draw [->] (n1)|-(h1);
	\draw [->] (n1)|-(h2);
	\draw [->] (h1)-|(s1);
	\draw [->] (h2)-|(s1);
	\end{tikzpicture}
	\caption{Sistema composto da filtri in parallelo passa basso e passa banda}
\end{figure}

La funzione di trasferimento del sistema $H_1(f)+H_2(f)$ trasforma il segnale in ingresso nella somma dei due segnali filtrati che hanno componenti spettrali che si sommano:
\[h_{n_u}(f)=\abs{H_1(f)+H_2(f)}^2 \cdot h_n(f)\]

In funzione delle frequenze di taglio $f_1$ e $f_2$ dei filtri si hanno due casi:
\begin{description}
	\item[I caso] $f_1<f_2$ , le uscite dei due filtri non hanno componenti alla stessa frequenza. La densità spettrale $h_{n_u}$ si ottiene sommando le densità in uscita da ciascun filtro. Infatti complessivamente il sistema è un filtro la cui funzione di trasferimento è $H_1(f)+H_2(f)$ e $H_1(f)$ e $H_2(f)$ non sono mai $\neq 0$ alla stessa frequenza. La funzione di trasferimento di potenza è $\abs{H_1(f)+H_2(f)}^2$
	\item[II caso] $f_2<f_1$, le uscite dei due filtri si sovrappongono in frequenza, si hanno tre sotto casi:
	\begin{description}
		\item[$0<f_1<f_2$] l'unico contributo all'uscita è dato dal I filtro
		\item[$f_2<f<f_1$] l'uscita dei due filtri è uguale e si sommano i quadrati delle ampiezze pertanto \[ h_{n_u} = 4 h_n \]
		\item[$f_1<f<f_3$] l'unico contributo all'uscita è dato dal I filtro
	\end{description}
\end{description}

\begin{figure}[!ht]\centering
	\subfloat[$f_1 < f_2$]{
		\begin{tikzpicture}[scale=.6]
		\begin{scope}
		\begin{axis}[xtick={1,2,3},xticklabels={$f_2$,$f_1$,$f_3$},ytick={1,2,3,4},xlabel=$f$,ylabel=$H_1(f)+H_2(f)$,yscale=.5]
		\addplot[black] coordinates {(0,1)(1,1)(1,0)(2,0)(2,1)(3,1)(3,0)};
		\end{axis}
		\end{scope}
		\begin{scope}
		\begin{axis}[xshift=8cm,xtick={1,2,3},xticklabels={$f_2$,$f_1$,$f_3$},ytick={1},yticklabel={$h_n$},xlabel=$f$,ylabel=$h_{nu}$,yscale=.5]
		\addplot[black] coordinates {(0,1)(1,1)(1,0)(2,0)(2,1)(3,1)(3,0)};
		\end{axis}
		\end{scope}
		\end{tikzpicture}}
	
	\subfloat[$f_2 < f_1$]{
		\begin{tikzpicture}[scale=.6]
		\begin{scope}
		\begin{axis}[xtick={1,2,3},xticklabels={$f_2$,$f_1$,$f_3$},ytick={1,2},xlabel=$f$,ylabel=$H_1(f)+H_2(f)$,yscale=.5]
		\addplot[black] coordinates {(0,1)(1,1)(1,2)(2,2)(2,1)(3,1)(3,0)};
		\end{axis}
		\end{scope}
		\begin{scope}
		\begin{axis}[xshift=8cm,xtick={1,2,3},xticklabels={$f_2$,$f_1$,$f_3$},ytick={1,4},ytick={1,4},yticklabels={$h_n$,$4h_n$},xlabel=$f$,ylabel=$h_{n_u}$,yscale=.5]
		\addplot[black] coordinates {(0,1)(1,1)(1,4)(2,4)(2,1)(3,1)(3,0)};
		\end{axis}
		\end{scope}
		\end{tikzpicture}}
\end{figure}
\end{esempio}

\begin{esempio}
Si abbia un apparato TV con fattore di rumore $F_\text{TV}=\SI{10}{\decibel}$ (ovvero una attenuazione $\alpha=10 \quad\alpha_\text{dB}=10\Log\alpha=10$). Un cavo collega l'antenna alla TV ha una attenuazione $\alpha_\text{dB}=\SI{3}{\decibel}$.

Si calcoli il fattore di rumore del sistema ricevente. Come si può abbassarlo a $\SI{6}{\decibel}$?
\begin{figure}[!ht]
	\centering
	\begin{circuitikz}[scale=.8]
		\draw (0,0) node[antenna,label=left:Antenna]{} to[short,label=$F_1$] (4,0) node[draw,block,label=above:$F_\text{TV}$]{$TV$};
		\draw [decorate,decoration={brace,amplitude=10pt,mirror},yshift=-12pt] (0,0)-- node [black,pos=.5,yshift=-20pt] {ampl. $\frac{1}{\alpha}$} (3,0);
	\end{circuitikz}
\end{figure}

L'attenuatore passivo che attenua di $\alpha_\text{dB}=\SI{3}{\decibel}$ dimezza la potenza $\alpha=2$ (amplifica di $1/2$).

Il fattore di rumore del sistema in catena di amplificazione:
\[F_S=F_1+F_2=\alpha+\frac{F_\text{TV}-1}{1/\alpha}=2+\frac{10-1}{1/2}=20\]
Espresso in decibel $10\Log 20=10\Log 2\cdot 10=10\Log 2+10\Log 10=(3+10)\si{\decibel}=\SI{13}{\decibel}$.

Per ridurre il fattore di rumore da $\SI{13}{\decibel}$ a $\SI{6}{\decibel}$ è necessario inserire un amplificatore a monte del cavo che collega l'antenna al TV, dove il segnale arriva debole e prima dell'attenuazione del cavo.
\begin{figure}[!ht]
	\centering
	\begin{circuitikz}[scale=.8]
		\draw (0,0) node[antenna,label=left:Antenna]{} to[short] (1,0) node[draw,block,label=above:$F_A$]{$A, T_a$} to[short,label=$F_2$] (5,0) node[draw,block,label=above:$F_\text{TV}$]{$TV$};
		\draw [decorate,decoration={brace,amplitude=10pt,mirror},yshift=-12pt] (2,0)-- node [black,pos=.5,yshift=-20pt] {ampl. $\frac{1}{\alpha}$} (4,0);
	\end{circuitikz}
\end{figure}
Si ottiene una catena di amplificazione a tre stadi con fattore di rumore:
\[F_S=F_A+F_2+F_\text{TV}=F_A+\frac{\alpha-1}{A}+\frac{F_\text{TV}-1}{A/\alpha}\leq 4 \quad(F_S\leq\SI{6}{\decibel})\]
\[F_S=F_A+\frac{2-1}{A}+\frac{10-1}{A/2}=F_A+\frac{1}{A}+\frac{18}{A}=F_A+\frac{19}{A}\leq 4\]
Si deve scegliere l'amplificatore con il guadagno $A$ minore che soddisfi la disequazione e eviti la saturazione e conseguente distorsione del segnale.
\end{esempio}

\chapter{Sistemi di telecomunicazione su canale passa basso}
\section{Introduzione}
Obiettivo del sistema di trasmissione è di far giungere l'informazione codificata in un segnale al destinatario. Le apparecchiature di trasmissione $T_X$ e ricezione $R_X$ sono necessarie per adattare e recuperare il segnale trasmesso tenendo conto delle caratteristiche del canale: si ha attenuazione $\alpha(f)$ funzione delle frequenza e una temperatura equivalente di rumore $T_0$ che quantifica l'entità del segnale di disturbo di varia natura che si sovrappone inevitabilmente al segnale trasmesso.

\`{E} necessario quindi determinare la potenza del segnale trasmesso $P_T$ necessaria per ottenere al ricevitore un sufficiente rapporto tra la potenza attenuata del segnale ricevuto $P_R$ e la potenza media del rumore usualmente di tipo termico gaussiano bianco.
\begin{figure}[h!]
	\centering
	\begin{tikzpicture}[node distance=2cm,>=latex',thick];
	\node [block](tx) {$T_X$};
	\node [block,right of=tx,minimum height=1em,node distance=3cm](c) {$\alpha, T_0$} edge[<-](tx);
	\node [sum,right of=c](s) {$+$} edge[<-](c);
	\node [above of=s,node distance=1.5cm](n) {$h_n$} edge[->](s);
	\node [block,right of=s](rx) {$R_X$} edge[<-](s);
	\node [left of=tx](s1) {$s(t)$} edge[->](tx);
	\node [right of=rx](s2) {$\hat{s}(t)$}edge[<-](rx);
	\draw [section={$P_R$}] (s)--(rx);
	\draw [dot=$O$] (s)--(rx);
	\draw [section={$P_T$}] (tx)--(c);	
	\end{tikzpicture}
	\caption{Schema sistema di telecomunicazione su canale passa basso ad onde convogliate}
	\label{fig:sistema_trasmissione_passa_basso}
\end{figure}

La potenza del segnale in ricezione:
\begin{equation}
P_R=\frac{P_T}{\alpha} \qquad \restrict{P_R}{\text{dBm}}=\restrict{P_T}{\text{dBm}}-\restrict{\alpha}{\text{dB}}
\end{equation}

La potenza del rumore in ricezione:
\begin{equation}
P_N=k T_0 B \quad\footnotemark
\end{equation}
\footnotetext{$B$ banda monolatera del segnale}

Rapporto segnale rumore in ingresso al ricevitore:
\begin{equation}
\restrict{\frac{S}{N}}{o}=\frac{P_R}{P_N}=\frac{P_T}{\alpha k T_0 B}
\end{equation}

Rapporto segnale rumore alla uscita dal ricevitore, corretto dal fattore di rumore $F$ del ricevitore:
\begin{equation}
\restrict{\frac{S}{N}}{u}=\frac{P_R}{P_N}=\frac{P_T / \alpha}{F k T_0 B}
\end{equation}

Si dimostra che trasmettendo su due canali di trasmissione in parallelo non si può ottenere una migliore efficienza:
\begin{figure}[h!]
	\centering
	\resizebox{\textwidth}{!}{
		\begin{tikzpicture}[node distance=2cm,>=latex',thick];
		% canale superiore
		\node [block](tx) {$T_X$};
		\node [block,right of=tx,minimum height=1em,node distance=3cm](c) {$\alpha, T_0$} edge[<-](tx);
		\node [sum,right of=c](s1) {$+$} edge[<-](c);
		\node [above of=s,node distance=1.5cm](n) {$h_n$} edge[->](s1);
		\node [block,right=1.3cm of s1](rx) {$R_X$} edge[<-](s1);
		\draw [dot={$o$}] (s1)--(rx);
		\draw [section={$P_R$}] (s1)--(rx);
		\draw [section={$P_T$}] (tx)--(c);
		% canale inferiore
		\coordinate[below=1cm of tx](c0);
		\node [block,below=1cm of c0](tx2) {$T_X$};
		\coordinate[left=1.5cm of c0](c1);
		\node [left of=c1](i) {$s(t)$};
		\draw [->] (i)--(c1)|-(tx);
		\draw [->] (i)--(c1)|-(tx2);		
		\node [block,right of=tx2,minimum height=1em,node distance=3cm](c2) {$\alpha, T_0$} edge[<-](tx2);
		\node [sum,right of=c2](s21) {$+$} edge[<-](c2);
		\node [above of=s21,node distance=1.5cm](n2) {$h_n$} edge[->](s21);
		\coordinate[below=1cm of rx](c4);
		\node[sum,right=1.5cm of c4](s3){$+$};
		\node [block,right=1.3cm of s21](rx2) {$R_X$} edge[<-](s21);
		\node [right of=s3](o) {$s_T(t)$};
		\draw [->] (rx)-|node[above]{$\hat{s}(t)$}(s3);
		\draw [->] (rx2)-|node[below]{$\hat{s}(t)$}(s3);
		\draw [->](s3)--(o);		
		\draw [dot={$o$}] (s21)--(rx2);
		\draw [section={$P_R$}] (s21)--(rx2);
		\draw [section={$P_T$}] (tx2)--(c2);
		\draw [dot={$u$}] (s3)--(o);	
		\end{tikzpicture}
	}
	\caption{Schema sistema di telecomunicazione su doppio canale passa basso ad onde convogliate}
	\label{fig:sistema_trasmissione_passa_basso_doppio_canale}
\end{figure}

In uscita si ha il segnale $s_T(t)=\hat{s}_1(t)+\hat{s}_2(t)$ che ha una potenza $P_{s_T}=4 P_R$ (raddoppiando il segnale quadrupla la potenza).

In uscita si ha la somma dei rumori dei due canali, due realizzazioni di processi aleatori gaussiani (non bianchi perché filtrati)
$n_T(t)=n_1(t)+n_2(t)$, contributi incorrelati con la medesima potenza $P_N$, che risultano in una potenza del rumore in uscita
$P_{n_T}=2 P_N$

Si ha pertanto che il rapporto segnale rumore in uscita è
\[\restrict{\frac{S}{N}}{T}=2\restrict{\frac{S}{N}}{S}\]
il doppio del singono canale, ma al costo di un raddoppio della potenza trasmessa.
Raddoppiando i costi dell'intero sistema di telecomunicazione si raddoppia il rapporto segnale rumore.

\section{Sistema di trasmissione su cavo coassiale}
Il mezzo trasmissivo del sistema di telecomunicazione con trasmissione analogica in fig.\ref{fig:sistema_trasmissione_passa_basso} ipotizza che la densità spettrale del segnale $h_s$ sia costante al variare delle frequenze nella banda monolatera di trasmissione $B$ del canale, $f\in[0,B]$.

L'ipotesi di un mezzo trasmissivo che attenua in modo uguale tutte le frequenze è ideale. I mezzi di trasmissione in materiale conduttore attenuano il segnale con legge esponenziale con la distanza e in modo crescente all'aumentare della frequenza a causa dell'\keyword[effetto pelle]{effetto pelle}
\begin{equation}
\f{\alpha}{f}=\alpha_\text{sp}\sqrt{\frac{f}{f_\text{sp}}}\;[\si{\decibel\per\kilo\meter}]
\label{eq:attenuazione_esponenziale}
\end{equation}
dove la misura dell'attenuazione $\f{\alpha}{f}$ è espressa in unità logaritmiche per kilometro in riferimento all'attenuazione $\alpha_\text{sp}$\footnote{Qualche dB per km o decine dB per km su doppino} ad una specifica frequenza $f_\text{sp}$.

\begin{esempio}
	Si consideri un cavo di lunghezza $\SI{1}{\kilo\meter}$ con attenuazione $\alpha_\text{sp}= \SI{2}{\decibel\per\kilo\meter}$ alla frequenza di $\SI{1}{\mega\hertz}$. Un sistema di trasmissione su cavo coassiale costituisce un filtro passa basso affetto da attenuazione. Supponendo uno spettro di potenza del segnale $s(t)$ costante nella banda del canale, si avrà in ricezione per l'attenuazione del canale un effetto filtro con funzione di trasferimento:
	\[ H(f)=\e{-2\f{\alpha}{f}}\]
	
\begin{figure}[!ht]
\def\ampiezza{10}
\def\alphasp{2}%db/km
\def\freqsp{440}%freq taglio
\subfloat[Densità spettrale segnale trasmesso]{
\begin{tikzpicture}% coordinates
\begin{semilogxaxis}[yscale=.7,xlabel=$f$,ytick={\ampiezza},yticklabels={$h_s$},extra x ticks={\freqsp},extra x tick labels={$B$},extra x tick style={grid=major},domain=1:1e4]
\addplot coordinates {(1,\ampiezza)(\freqsp,\ampiezza)(\freqsp,0)(10000,0)};
\end{semilogxaxis}
\end{tikzpicture}
}\quad\subfloat[Densità spettrale segnale ricevuto ($\alpha_\text{sp}=\SI{2}{\decibel/\kilo\meter}$)]{
\begin{tikzpicture}% coordinates
\begin{semilogxaxis}[yscale=.7,xlabel=$f$,ylabel={$H(f)=\e{-2\f{\alpha}{f}}$},ytick={1},yticklabels={$h_s$},domain=1:1e4]
\addplot {exp(-2*\alphasp*sqrt(x/\freqsp))};
\end{semilogxaxis}
\end{tikzpicture}
}
\end{figure}
\end{esempio}

\section{Equalizzazione attiva e passiva}
Per compensare l'attenuazione del mezzo trasmissivo alle alte frequenze si può \keyword[equalizzazione]{equalizzare} in ricezione applicando un filtro con funzione di trasferimento inversa rispetto al canale:
\[\f{A}{f}=k\e{2\f{\alpha}{f}}\]
Tale equalizzazione amplifica anche il rumore, che non sarà più bianco ma avrà le componenti in alta frequenza esaltate. 

In alternativa è possibile realizzare un equalizzatore passivo che invece di amplificare lo spettro alle alte frequenze per ottenere uno spettro costante, attenui le basse frequenze, senza la necessità di una alimentazione.

L'equalizzazione viene utilizzata per segnali a banda stretta, mentre risulta problematica per segnali multiplati in frequenza.

In alternativa all'equalizzazione non lineare in ricezione si può \keyword[pre-enfasi]{pre-enfatizzare} $s(t)$ in trasmissione in modo da ottenere un segnale piatto in banda di ricezione, avendo cura di non distorcere e saturare il segnale trasmesso.

\section{Sistema di trasmissione multi-tratta}\index{sistema di trasmissione!multi-tratta}
Per compensare il fenomeno dell'attenuazione su collegamenti via cavo di grande lunghezza si progettano sistemi multi-tratta, in cui sono posti in cascata più sistemi elementari in cui si amplifica il segnale prima di trasmetterlo sulla tratta successiva.
Il sistema multi-tratta è equivalente ad un sistema con una unica tratta con densità spettrale di rumore pari alla somma delle densità spettrali di rumore incorrelate; per $m$ tratte omogenee si ha un rumore moltiplicato $m$ volte e un rapporto segnale rumore
\[\frac{S}{N}=\frac{h_s B}{\intd{0}{B}{m\cdot h_n A(f)}{f}}\]
con $B$ la banda monolatera del segnale, $h_n$ la densità spettrale di rumore in banda, $A(f)$ la curva di amplificazione di equalizzazione.
\begin{figure}[h!]
	\centering
	\subfloat[Schema sistema di telecomunicazione multi-tratta]{
		\begin{tikzpicture}[>=latex',thick,start chain=going right,node distance=5mm,every node/.style={on chain},every join/.style={->},	block/.style={draw,align=center}]
		\coordinate[join](start);
		\foreach \i in {1,...,3} {
			\node[block,join,minimum width=3em]{$\alpha$};
			\node[sum,join]{$+$};
			\begin{scope}[start branch=above, every join/.style={<-,thick,shorten <=1pt},]
				\node[on chain=going above,join]{$h_{n_\i}$};
			\end{scope}
			\node[block,join,minimum width=2em,minimum height=2em]{$A$};
			%\coordinate[right=1.3cm of s\i](m\i);
		}
		\coordinate[join](end);
		\end{tikzpicture}
	}
	
	\subfloat[Schema sistema equivalente al multi-tratta]{
		\begin{tikzpicture}[>=latex',thick,start chain=going right,node distance=5mm,every node/.style={on chain},every join/.style={->},	block/.style={draw,align=center}]
		\coordinate[join](start);
		\node[block,join,minimum width=3em]{$\alpha$};
		\node[sum,join]{$+$};
		\begin{scope}[start branch=above, every join/.style={<-,thick,shorten <=1pt}]
			\node[on chain=going above,join]{$m\cdot h_n$};
		\end{scope}
		\node[block,join,minimum width=2em,minimum height=2em]{$A$};
		\coordinate[join](end);
		\end{tikzpicture}
	}
	\caption{Schema sistema di telecomunicazione multi-tratta}
	\label{fig:sistema_trasmissione_multi_tratta}
\end{figure}

\begin{figure}[ht!]\centering
\begin{tikzpicture}[scale=.3]
\def\ampiezza{1}
\def\alphasp{1}%db/km
\def\freqsp{100}%freq taglio
\foreach \y/\s in {0/$h_s$,1/$h_{n_1}$,2/$h_{n_2}$,3/$h_{n_3}$} {
	\foreach \x in {1,...,7} {
		\ifnum \x=1\relax
		\node [left] at(-1,-\y*7+9)(n\y\x) {\s};
		\fi
		\pgfmathparse{int(\y+\y-1)}
	\def\z{\pgfmathresult}
	\ifnum\numexpr\x>\z\relax
		\begin{semilogxaxis}[black,xshift=(\x-1)*7cm,yshift=-(\y-1)*7cm,xlabel=$f$,ylabel=$\abs{S(f)}^2$,ytick={\ampiezza},yticklabels={$h_s$},extra x ticks={\freqsp},extra x tick labels={$B$},domain=1:1e3]
		\ifnum \y > 0 \relax
			\ifodd\x\relax
				\addplot {exp(.5*\alphasp*sqrt(x/\freqsp))};
			\else
				\addplot coordinates {(1,\ampiezza)(\freqsp,\ampiezza)(\freqsp,0)(\freqsp*10,0)};
			\fi
		\else
			\ifodd\x\relax
				\addplot coordinates {(1,\ampiezza)(\freqsp,\ampiezza)(\freqsp,0)(\freqsp*10,0)};
			\else
				\addplot {exp(-.5*\alphasp*sqrt(x/\freqsp))};
			\fi
		\fi
		\end{semilogxaxis}
	\fi
}}
\end{tikzpicture}
\caption{Densità spettrali di segnale e di rumore in un sistema multitratta con tratte identiche}
\end{figure}

\chapter{Sistemi di telecomunicazione su canale passa banda}\label{cap:sis_telcom_band_pass}
\section{Introduzione}
Nei sistemi di trasmissione su canale radio è necessario tener conto delle caratteristiche di risposta in frequenza del mezzo trasmissivo: lo spettro in una banda di frequenze che si può allocare, in generale, è diverso da quello del segnale che si vuole trasmettere. Per abilitare la comunicazione è pertanto necessario traslare la banda del segnale originario per adattarsi alla banda di frequenze permesse dal mezzo trasmissivo, che ha funzione di trasferimento tipica di un filtro passa banda.

Questo tipo di canali si comportano in modo soddisfacente quando la frequenza relativa $f_r=\frac{f_0}{B}$ è sufficientemente piccola. Tale requisito è necessario ad alte frequenze per limitare la risposta tempovariante che si riduce operando in banda stretta.

I segnali che possono essere trasmessi su questo tipo di canali sono sinusoidi o somma di sinusoidi. L'informazione è contenuta nell'ampiezza del segnale e nella sua fase.

\clearpage
\section{Modulazione di ampiezza}
\[s_t(t)=s(t)\cdot\cos{\omega_0 t}\]
\section{In doppia banda laterale e portante soppressa (DSB-SC)}
\section{Demodulazione coerente (DSB-SC)}
\section{Modulazione d'ampiezza con portanti in quadratura}
\section{Effetto di un filtro frapposto tra modulatore e demodulatore}
\section{Rapporto Segnale Rumore (DSB-SC)}
\section{Modulazione d'ampiezza in banda laterale unica (SSB)}
\section{Modulatore SSB con filtraggio}
\section{Demodulazione SSB}
\section{Modulazione d'ampiezza con portante trasmessa}

\clearpage
\section{Modulazione angolare}
Nella modulazione d'ampiezza sono necessari amplificatori lineari: lo stadio di trasmissione è poco efficiente. Per questo motivo si introduce la \keyword[modulazione!angolare]{modulazione angolare} in cui l'informazione del segnale modulante è codificata nella fase (eq.\ref{eq:modulazione_fase}) o nella frequenza (eq.\ref{eq:modulazione_frequenza}).

Il \textsc{segnale trasmesso} ha formulazione non lineare \[s_T(t)=A\cos{\Phi(t)}\] e si può avere \keyword[modulazione!di fase]{modulazione di fase}
\begin{equation}s_T(t)=A_p \cos{\omega_0 t + \phi(t)}\label{eq:modulazione_fase}\end{equation}
e \keyword[modulazione!di frequenza]{modulazione di frequenza}
\begin{equation}s_T(t)=A_p \cos{\omega(t) t + \phi}\label{eq:modulazione_frequenza}\end{equation}

Il segnale trasmesso passa banda può essere rappresentato nelle sue componenti in fase e quadratura come nella modulazione di ampiezza:
\begin{equation}
s_T(t)=s_T^\text{I}(t)\cos{\omega_0 t}-s_T^\text{Q}(t)\sen{\omega_0 t}
\end{equation}
Nella modulazione angolare con portante a frequenza $f_0=\omega_0/2\pi$:
\begin{equation}\begin{split}
s_T(t)&=A_T(t)\cos{\omega_0 t+\phi_T(t)}=\\&=A_T(t)\cos{\omega_0 t}\cos{\phi_T(t)}-A_T(t)\sen{\omega_0 t}\sen{\phi_T(t)}
\end{split}\end{equation}
dove
\begin{equation}
\begin{cases}
s_T^\text{I}(t)=A_T(t)\cos{\phi_T(t)}\\s_T^\text{Q}(t)=A_T(t)\sen{\phi_T(t)}
\end{cases}\iff\begin{cases}
A_T(t)=\sqrt{{s_T^\text{\small I}}^2(t)+{s_T^\text{Q}}^2(t)}\\\phi_T(t)=\arctan\frac{s_T^\text{Q}(t)}{s_T^\text{I}(t)}
\end{cases}
\end{equation}

\begin{definizione}Si definiscono la \keyword[pulsazione!istantanea]{pulsazione istantanea}\begin{equation}\omega_i(t)=\deriv{\Phi(t)}{t}= \omega_0+\deriv{\phi(t)}{t}\end{equation} e la \keyword[pulsazione!deviazione di]{deviazione di pulsazione}\begin{equation}\Delta\omega(t)=\omega_i(t)-\omega_0
\end{equation}\end{definizione}
\begin{definizione}Si definiscono la \keyword[frequenza!istantanea]{frequenza istantanea}\begin{equation}f_i(t)=\frac{1}{2\pi}\omega_i(t)=f_0+\frac{1}{2\pi}\deriv{\phi(t)}{t}=f_0+\Delta f(t)\end{equation} e \keyword[frequenza!deviazione di]{deviazione di frequenza}\begin{equation}\Delta f(t)=\frac{\dot{\phi}(t)}{2\pi}\end{equation}\end{definizione}

\begin{nota}
La fase e la frequenza si possono determinare contando gli attraversamenti dell'asse dei tempi, rilevando i cambi di segno del segnale. La tecnica di modulazione angolare è robusta alle distorsioni sull'ampiezza introdotte da amplificatori non lineari.
\end{nota}

Il segnale ricevuto su canale passa banda può sempre essere espresso nelle sue componenti in fase e quadratura:
\begin{equation}\begin{split}
s_R(t)&=k s_T^\text{I}(t-T_g)\cos{\omega_0(t-T_f)}+\\
&+k s_T^\text{Q}(t-T_g)\cos{\omega_0(t-T_f)}
\end{split}\end{equation}
dove il fattore $k$ tiene conto dell'attenuazione del canale, $T_g$ è il ritardo di gruppo, $T_f$ il ritardo di fase.

In particolare per la modulazione angolare il segnale ricevuto con le sue portanti in fase e quadratura
\begin{equation}\begin{split}
s_R(t)&=S_T^\text{I}(t)\cos{\omega_0 t}-S_T^\text{Q}(t)\sen{\omega_0 t}=\\&=A_R(t)\cos{\omega_0 t+\phi_R(t)}
\end{split}\end{equation}

Al segnale ricevuto si somma il rumore, variabile aleatoria con distribuzione uniforme, si può esprimere nelle componenti in fase e quadratura:
\begin{equation}
N_T(t)=n^\text{I}(t)\cos{\omega_0 t}-n^\text{Q}(t)\sen{\omega_0 t}=\rho(t)\cos{\omega_0 t+\phi_n(t)}
\end{equation}

Nella modulazione angolare si riceve pertanto la somma dei segnali esprimibili in notazione fasoriale
\begin{equation}\begin{split}
&A_R(t)\cos{\omega_0 t+\phi_R(t)}+\rho(t)\cos{\omega_0 t+\phi_n(t)}=\\
&=\Re{A_R\e{\jmath\phi_R(t)}\e{\jmath\omega_0 t}+\rho(t)\e{\jmath\omega_0 t}\e{\phi_n(t)}}=\\
&=\Re{\rho(t)\e{\jmath(\phi_R(t)+\beta(t))}}
\end{split}\end{equation}

\section{Spettro di una sinusoide modulata angolarmente}
La non linearità della modulazione angolare non consente il calcolo dello spettro della portante modulata per un segnale qualsiasi. La banda del segnale modulato angolarmente in frequenza si può determinare nei due casi:
\begin{description}
\item[I caso] \[\abs{\phi(t)}\ll 1\]
Il segnale modulante $A\cos{\omega_0 t+\phi(t)}$ modulato angolarmente per piccolo indice di modulazione tende ad un segnale trasmesso modulato in ampiezza con portante trasmessa
\[s_T(t)=\frac{A}{2}\cos{\omega_0 t}\cos{\phi(t)}-\frac{A}{2}\sen{\omega_0 t}\sen{\phi(t)}\cong \underbrace{\frac{A}{2}\cos{\omega_0 t}}_{\parbox[c]{2cm}{portante\\trasmessa}}-\underbrace{\phi(t)\frac{A}{2}\sen{\omega_0 t}}_{\parbox[c]{3cm}{segnale modulato\\in ampiezza}}\]
La banda minima occupata da una sinusoide modulata angolarmente non può essere inferiore alla modulazione di ampiezza in doppia banda laterale \ac{DSB-SC}.

La potenza media del segnale portante $A\cos{\omega_0 t}$ è $\frac{A^2}{2}$ essendo \[\frac{1}{T}\intd{0}{T}{[A\cos{\omega_0 t}]^2}{t}=\frac{A^2}{2}\bound{0}{T}{\sen[2]{\omega_0 t}}\]

\item[II caso] \[\abs{\phi(t)}\nless 1\]
Non è calcolabile lo spettro per un qualunque segnale modulante non potendosi semplificare $\sen{\phi(t)}\cong\phi(t)$ e $\cos{\phi(t)}\cong 1$

Nel caso di segnale modulante di natura sinusoidale con pulsazione $\omega_s$ con battimenti con la portante di pulsazione $\omega_0$:
\[\phi_s(t)=B\sen{\omega_s t}\]
\end{description}

Il segnale trasmesso modulato angolarmente può essere espresso come serie di Fourier o in forma fasoriale con gli esponenziali complessi
\begin{equation}\begin{split}
s_T(t)&=A\cos{\omega_0 t+B\sen{\omega_s t}}=\\
&=\Re{A\e{\jmath\omega_0 t}\e{\jmath B\sen{\omega_s t}}}=\\
&=\Re{\tilde{s}_T(t)\e{\jmath\omega_0 t}}
\end{split}\end{equation}
Da tale relazione è possibile ricavare l'ampiezza delle righe spettrali nelle bande laterali che formano un segnale inviluppo complesso $\tilde{s}_T(t)=A\e{\jmath B\sen{\omega_s t}}$ di natura periodica, di periodo $T_s=\frac{2\pi}{\omega_s}$, pertanto esprimibile in serie di Fourier come
\begin{equation}
\tilde{s}_T(t)=\sum_{n=-\infty}^{+\infty}{c_n\e{\jmath \frac{2\pi}{T_s}n t}}
\end{equation}

Tale inviluppo in fig.\ref{fig:inviluppo_modulazione_angolare} presenta uno spettro a righe a frequenze multiple della fondamentale, $n\cdot\frac{1}{T_S}$ modulate per il termine $\e{\jmath\omega_0 t}$ alle frequenze $-f_0$ e $f_0$:
\begin{figure}[h!]
\centering\subfloat[Spettro segnale inviluppo complesso $\tilde{s}_T(t)$]{\begin{tikzpicture}
\begin{axis}[width=8cm,height=4cm,black,xlabel=$f$,ylabel=$\tilde{s}_T(f)$,ytick={1},xtick={-1,-.334,.334,1},xticklabels={$-\frac{2}{T_S}$,$-\frac{1}{T_S}$,$\frac{1}{T_S}$,$\frac{2}{T_S}$}]
\addplot+ [quiver={u=0,v=exp(-x*x)},-stealth,samples=16,domain=-2.5:2.5] {0};
\end{axis}
\end{tikzpicture}
\label{fig:inviluppo_modulazione_angolare}}\quad\subfloat[Spettro segnale modulato]{\begin{tikzpicture}
\begin{axis}[width=8cm,height=4cm,black,xlabel=$f$,ylabel=$\tilde{s}_T(f)$,ytick={1},xtick={-2,2},xticklabels={$-f_0$,$f_0$}]
\addplot+ [quiver={u=0,v=exp(-(x-2)^2)+exp(-(x+2)^2)},-stealth,samples=32,domain=-5:5] {0};
\end{axis}
\end{tikzpicture}}
\end{figure}

Lo spettro a righe del segnale modulato ha infiniti contributi ma dal punto di vista ingegneristico sono rilevanti i contributi alla potenza di un insieme finito di righe.
La banda di frequenze di interesse è definita secondo la regola o \keyword[principio!di Carson]{principio di Carson} in funzione della frequenza del tono modulante $f_s$ e la deviazione di frequenza di picco:
\begin{equation}B_T\cong 2f_s+2\Delta f_p\label{eq:Carson}\end{equation}
Nel caso non si utilizzi un tono sinusoidale ma un generico segnale portante con banda $B$ la regola diventa
\begin{equation}B_T\cong 2B+2\Delta f_p\label{eq:Carson_generico}\end{equation}
Nel dimensionamento si cercherà la minore deviazione di frequenza di picco $\Delta f_p$ per non eccedere nella banda $B_T$. Se $\frac{\Delta f_p}{B}\ll 1\implies B_T\cong 2B$.

\section{Calcolo del rapporto \ac{SNR}}
Nel canale di trasmissione passa banda al segnale modulato angolarmente si somma un segnale di disturbo descritto da un rumore bianco gaussiano a banda stretta, per cui all'ingresso del ricevitore al demodulatore si ha il segnale ricevuto:
\begin{equation}
s_R(t)=A\cos{\omega_0 t+\phi(t)}+N(t)
\end{equation}
Il rumore $N(t)$ sovrapposto alla portante in ricezione può essere scomposto nelle sue componenti in fase e quadratura rispetto alla portante $\cos{\omega_0 t}$:
\begin{equation}
s_R(t)=A\cos{\omega_0 t+\phi(t)}+n_I(t)\cos{\omega_0 t}-n_Q(t)\sen{\omega_0 t}
\end{equation}
dove il rumore espresso con forma fasoriale:
\begin{equation}
N(t)=\rho(t)\cos{\omega_0 t+\phi_N(t)}\quad\begin{cases}
\rho(t)=\sqrt{n_I^2(t)+n_Q^2(t)}\\\phi_N(t)=\arctan\frac{n_I(t)}{n_Q(t)}\end{cases}
\end{equation}

Essendo le componenti del rumore $n_I$ e $n_Q$ due gaussiane ho che l'ampiezza del rumore $\rho(t)$ è descritto da una variabile aleatoria di Rayleigh.
Si ha inoltre che il rumore sulla fase $\phi_N(t)$ è una variabile aleatoria con distribuzione continua uniforme costante in $[0,2\pi]$.

Il segnale ricevuto si può esprimere pertanto come somma dei contributi
\begin{equation}
s_R(t)=\Re{A\e{\jmath\omega_0 t}\e{\jmath\phi (t)}+\rho(t)\e{\jmath\phi_N(t)}\e{\jmath\omega_0 t}}=\Re{R\e{\jmath(\omega_0 t+\phi (t)+\beta(t))}}
\end{equation}
\begin{figure}[!ht]\centering
\begin{tikzpicture}[>=latex',black,scale=.9]
\draw[<->](0,4)node[above]{$\Imaginarypart$}--(0,0)--(6,0)node[right]{$\Realpart$};
\coordinate (A) at ($(30:5)$);
\coordinate (B) at ($(A)+(90:1)$);
\draw[thick,->](0,0)--node[below]{$A$}(A);
\draw[thick,->](0,0)--node[above]{$R$}(B);
\draw[gray](A)--+(120:.86) (A)--+(30:.5);
\draw[thick,->](A)--node[pos=1,above right]{$\rho(t)$}(B);
\node[draw,cloud,cloud puffs=15,minimum width=2cm,minimum height=2cm]at(A){};
\draw[double](1,0) arc(0:30:1)node[pos=.5,right]{\footnotesize$\omega_0 t$};
\draw($(A)+(0,.3)$) arc(90:120:.3)node[pos=.5,above]{\footnotesize$\beta$};
\draw(30:1) arc(30:40:1)node[pos=.75,right]{\footnotesize$\beta(t)$};
\draw[double]($(A)+(30:.3)$) arc(30:90:.3)node[pos=.5,right]{\footnotesize{$\phi_N(t)-\phi(t)$}};
\end{tikzpicture}
\caption{Segnale modulato angolarmente affetto da rumore gaussiano con $\rho\ll 1$}\label{fig:segnale_modulato_angolarmente_affetto_da_rumore}
\end{figure}

La deviazione di fase $\beta(t)$ è data dalla componente del rumore in quadratura $n_Q(t)$ e il modulo del segnale modulato angolarmente sarà approssimativamente sempre di ampiezza $A$ se il segnale rumore ha modulo $\rho(t)\ll 1$ essendo:
\[\sen{\beta(t)}\cong\beta(t)\cong\frac{\rho\sen{\phi_N(t)-\phi(t)}}{A}=\frac{\rho\sen{\phi_N(t)}}{A}=\frac{n_Q(t)}{A}\]
Tale deviazione assume le caratteristiche statistiche del rumore in quadratura ovvero una distribuzione uniforme in $[0,2\pi]$.

In definitiva si è trasmesso un segnale modulato angolarmente con fase $\phi(t)$ e si ha un segnale ricevuto con fase $\phi(t)+\beta(t)=\phi(t)+\tfrac{n_Q(t)}{A}$ quando il sistema lavora “\emph{sopra soglia}” ovvero con una potenza del segnale portante $A^2/2$ sufficientemente grande in rapporto alla potenza del segnale rumore.
\begin{equation}\begin{split}
P_S&=\lim\limits_{n\to\infty}{\frac{1}{2 n T}\intd{-n T}{n T}{A^2(t)\cos[2]{\omega_0 t}}{t}}=\\
&=\lim\limits_{n\to\infty}{\frac{1}{2 n T}\intd{-n T}{n T}{\frac{A^2(t)}{2}\left[1+\cos{2\omega_0 t}\right]}{t}}=\frac{P_A}{2}
\end{split}\end{equation}

Il rapporto segnale rumore dopo il demodulatore espresso come rapporto tra la potenza picco-picco del segnale e la potenza media del segnale rumore valutata in uscita dal demodulatore calcolata sulla densità spettrale bilatera del rumore in uscita dal canale di trasmissione $h_n=\frac{N_0}{2}$ moltiplicata per la banda di interesse $B_T$ ottenuta con il principio di Carson:
\begin{equation}
\restrict{\frac{S}{N}}{o}=\frac{P_\text{pp}}{\frac{N_0}{2}\frac{2}{A^2}\cdot 2B_T}
\end{equation}

\section{Demodulazione FM}
Per la modulazione di frequenza in ricezione si valuta con un derivatore in cascata al demodulatore di fase la deviazione di frequenza del segnale trasmesso essendo $\Delta f(t)=\frac{1}{2\pi}\deriv{\phi(t)}{t}$. Tale deviazione di fase è affetta dal rumore sovrapposto al segnale per cui si ha
\[\Delta f(t)=\frac{1}{2\pi}\deriv{(\phi(t)+\beta(t))}{t}=\frac{1}{2\pi}\dot{\phi}(t)+\frac{1}{2\pi}\dot{\beta}(t)\]

La potenza del segnale rumore si ottiene integrando nella banda del segnale la densità spettrale di potenza di rumore all'uscita del demodulatore angolare.
Essendo la derivata del rumore di fase $h_{nu}=\frac{h_n}{P_R}f^2$, dove $P_R=A^2/2$ è la potenza del segnale ricevuto, si calcola la potenza media del rumore all'uscita
\begin{equation}
P_{N_u}=\intd{-B}{B}{\frac{h_n}{P_R}f^2}{f}=2\intd{0}{B}{\frac{N_0}{A^2}f^2}{f}=\frac{2}{3}\frac{N_0}{A^2}B^3
\end{equation}

Per il dimensionamento del sistema si può calcolare dato un richiesto rapporto \ac{SNR} imponendo che sia
\begin{equation}
\frac{A^2}{2}\gg\frac{N_0}{2}\cdot 2B_T
\end{equation}
\begin{equation}
\restrict{\frac{S}{N}}{\text{dopo demod}}=\frac{3 P(\Delta f)}{2\frac{N_0}{A^2}B^3}=\frac{\Delta f_p^2}{\frac{N_0}{\frac{A^2}{2}}\frac{B^3}{3}}=3\left(\frac{\Delta f_p}{B}\right)^2\frac{A^2/2}{N_0 B}
\end{equation}

\begin{esempio}
Se ad esempio dato un rapporto segnale rumore richiesto tale che $\frac{S}{N}\gg\num{1e4} [\SI{40}{\decibel}]$ devo risolvere il sistema per il minimo $\Delta f_P$
\[\begin{cases}
3\left(\frac{\Delta f_p}{B}\right)^2\frac{P_R}{N_0 B}\gg\num{1e4}\\\frac{P_R}{N_0 B_T}\gg 10 \iff \frac{P_R}{N_0(2B+2\Delta f)}\gg 10
\end{cases}\]

Se ad esempio $\frac{S}{N}=\SI{40}{\decibel}$, $B=\SI{15}{\kilo\hertz}$, $F=\SI{10}{\decibel}$, $\alpha=\SI{100}{\decibel}$ con la modulazione di frequenza si ottiene dividendo membro a membro:
\[\frac{3\left(\frac{\Delta f_p}{B}\right)^2\frac{P_R}{N_0 B}}{\frac{P_R}{N_0(2B+2\Delta f)}}=\frac{10^4}{10}\]
da cui si ottiene $\Delta f_p=\SI{77}{\kilo\hertz}$ e una potenza ricevuta $P_R=\SI{-101}{\decibel}$.

Rispetto alla modulazione di ampiezza \ac{DSB-SC} per cui si è calcolata la potenza ricevuta $P_R=\SI{-79}{\decibel}$ con la modulazione di frequenza si ha un risparmio di potenza a spese di un maggiore uso di banda.
\end{esempio}
\clearpage
\section{Trasmissione su canale radio}\label{cap:canale_radio_attenuazione_spazio_libero}
In alternativa ai mezzi trasmissivi basati sulle guida d'onda con effetti dissipativi è possibile realizzare sistemi di telecomunicazione basati sulla propagazione di onde irradiate nello spazio libero. Il segnale trasmesso subisce una attenuazione dovuta all'aumentare della superficie del fronte d'onda. La densità di potenza per unità di superficie diminuisce con il quadrato della distanza.

Nel caso di antenna omnidirezionale l'\keyword[attenuazione!di spazio libero]{attenuazione di spazio libero} calcolata ipotizzando l'assenza di ostacoli tra il trasmettitore $T_X$ e ricevitore $R_X$, è
\begin{equation}
\alpha_\text{SL}=\frac{4\pi R^2}{G_T A_R}
\label{eq:radio_attenuazione_spazio_libero}
\end{equation}
dove $G_T$ è il guadagno d'antenna trasmittente, $A_R$ l'area efficace d'antenna ricevente. L'area efficace di una antenna è legata al suo guadagno dalla relazione
\begin{equation}
\frac{G}{A}=\frac{4\pi}{\lambda^2}
\label{eq:radio_legame_area_efficace_guadagno_antenna}
\end{equation}

\begin{figure}[ht]\centering
	\begin{tikzpicture}
	\draw (0,3) pic{antenna} --++(0,-.5)--++(-.5,-1.5)++(.5,1.5)--++(.5,-1.5)++(-.5,0) node[below]{Tx};
	\draw (9,3) pic[rotate=180]{antennarx} --++(0,-.5)--++(-.5,-1.5)++(.5,1.5)--++(.5,-1.5)++(-.5,0)node[below]{Rx};
	\draw [dotted] (3,3)--node[above]{$R$}(6,3);
	\end{tikzpicture}
\end{figure}

L'attenuazione di spazio libero è un fenomeno deterministico che vale per frequenze $f\ll\SI{3}{\giga\hertz}$, per le quali è sufficiente considerare solo questa attenuazione causata dalla divergenza sferica; per $f>\SI{3}{\giga\hertz}$ è necessario considerare le perdite dissipative aleatorie dovute all'atmosfera.

Tale fenomeno causa un ulteriore effetto di attenuazione, esponenziale con la distanza, dovuto alle molteplici stratificazioni di gas ionizzati, paralleli alla superficie terrestre, che causano percorsi di diffrazione. Tali fenomeni consentono la comunicazione anche non in linea d'aria tra stazioni radio situate in luoghi remoti della Terra.

\begin{figure}[ht]\centering
	\begin{tikzpicture}[scale=.66]
	\draw (4,0)arc(0:180:4);
	\foreach\x in{8,7.8,7.5,6.75}
		\draw [dashed,gray](\x,0)arc(0:180:\x);
	\draw [rotate=60](0,5.5) pic[rotate=60,scale=.33]{antenna}--++(0,-.5)--++(-.5,-1)++(.5,1)--++(.5,-1)++(-.5,0)node[below]{Tx};
	\draw [rotate=-60](0,5.5) pic[rotate=120,scale=.33]{antennarx} --++(0,-.5)--++(-.5,-1)++(.5,1)--++(.5,-1)++(-.5,0)node[below]{Rx};
	\draw[rotate=90,decorate,decoration={random,amplitude=2}](60:5.5)--(30:7)--(-30:7)--(-60:5.5);
\end{tikzpicture}
\caption{Effetto degli strati di gas ionizzati in atmosfera terrestre}
\label{fig:percorsi_diffrazione_atmosfera}
\end{figure}

\clearpage
\section{Effetti dovuti a cammini multipli}\label{cap:canale_radio_effetti_cammini_multipli}
A causa delle disomogeneità dell'atmosfera le onde trasmesse si propagano dal trasmettitore al ricevitore seguendo cammini multipli. Ciascun fronte d'onda fornisce un contributo diverso in ampiezza e fase. Le ampiezze a grande distanza avranno attenuazioni simili quindi contributi comparabili mentre le fasi si presenteranno con differenze casuali che vanno descritte statisticamente.
\begin{figure}[ht]\centering
	\begin{tikzpicture}
	\draw (0,3) pic{antenna} node[below]{Tx};
	\draw (9,3) pic[rotate=180]{antennarx} node[below]{Rx};
	\foreach\i in{-15mm,-10mm,-5mm,0,5mm,10mm,15mm}
		\draw [dotted,decorate,decoration={bent,amplitude=\i}] (1,3)--(8,3);
	\end{tikzpicture}
	\caption{Rappresentazione di cammini multipli}
	\label{fig:cammini_multipli}
\end{figure}

Il segnale ricevuto $s_R(t)$ sarà somma di tanti contributi indipendenti le cui componenti in fase e quadratura hanno distribuzione gaussiana normale, $\begin{array}{c}X\\Y\end{array}\sim\mathcal{N}(0,\sigma^2)$:

\begin{equation}
s_R(t)=x(t)\cos{\omega_0 t}-y(t)\sen{\omega_0 t}=R\cos{\omega_0 t+\phi(t)}=\Re{R\e{\imath\omega_0 t}\e{\imath\phi(t)}}
\end{equation}

L'ampiezza del segnale risultante dato dalla combinazione delle due variabili gaussiane incorrelate assume le caratteristiche statistiche di una variabile aleatoria di Rayleigh $R=\sqrt{X^2+Y^2}$ con funzione densità di probabilità \[f_R(r)=\frac{r}{\sigma^2}\e{-\frac{r^2}{2\sigma^2}}\]
\begin{figure}[ht!]\centering
	\begin{tikzpicture}
	\begin{axis}[yscale=0.6,xlabel=$r$,ylabel=$f_R(r)$,xtick=\empty,ytick=\empty]
	\foreach\var in{.25,1.0,5.0} {
		\addplot[domain=0:10,samples=200]{x/\var*exp(-x^2/(2*\var))};
%		\node[pin=45:{$\sigma=\var$}] at(0,0){}; %at(1,\eval{exp(-1/(2*\var))/\var})
	}
	\end{axis}
	\end{tikzpicture}
	\caption{Funzione densità di probabilità di Rayleigh ($\sigma^2=\{1/4,1,5\}$)}
	\label{fig:funz_dens_prob_rayleigh}
\end{figure}

La probabilità che la sovrapposizione delle onde dovuta a cammini multipli causi un \keyword[probabilità!di fuori servizio]{fuori servizio} coincide con la probabilità che l'ampiezza del segnale risultante sia inferiore ad una soglia ovvero che la potenza picco-picco del segnale ricevuto sia inferiore alla \keyword[potenza!di fuori servizio]{potenza di fuori servizio} $P_\text{FS}$:
\begin{equation}
\P{P_R^\text{pp}<P_\text{FS}}=1-\e{-\frac{h_s}{2\sigma^2}}
\label{eq:probabilita_fuori_servizio}
\end{equation}

La potenza di picco del segnale $s_R(t)$ è una variabile aleatoria esponenziale con \[R^2=\E{R^2}=\E{X^2+Y^2}=\sigma^2+\sigma^2=2\sigma^2\]

\clearpage
\section{Modello ad echi}\label{cap:canale_radio_modello_ad_echi}
Il fenomeno fisico dei cammini multipli può essere schematizzato come un sistema che somma un segnale ricevuto da un percorso diretto più lo stesso segnale ritardato. Il sistema può essere generalizzato alla somma di più echi.
\begin{figure}[h!]
\centering
	\begin{tikzpicture}[>=latex',thick,start chain=going right,node distance=1cm,every node/.style={on chain},every join/.style={->},block/.style={draw,align=center}]
	\node[join](sT) {$s_T(t)$};
	\coordinate[on chain,join](n);
	\draw[join,fill](n) circle(2pt);
	\begin{scope}[start branch=right]
	\coordinate[on chain=going below](n2);
	\node[block,join,minimum width=15mm](r){$a\e{\imath\omega\tau}$};
	\end{scope}
	\node[sum,join,right=3cm of n](s) {$+$};
	\draw[->](r)-|(s);
	\draw(n)--(n2);
	\node[join,right=of s](sR) {$s_R(t)$};
	\end{tikzpicture}
\caption{Modello sistema ad eco}
\label{fig:sistema_con_ritardo}
\end{figure}

La funzione di trasferimento del sistema per la sovrapposizione di un segnale con il suo eco:
\begin{equation}\begin{split}
H(\omega)&=1+a\e{-\jmath\omega\tau}\\
\abs{H(\omega)}&=\sqrt{(1+a\cos{\omega\tau})^2+a^2\sen[2]{\omega\tau}}=\\&=\sqrt{1+a^2+2a\cos{\omega\tau}}
\end{split}\end{equation}
Tale funzione presenta i massimi per le frequenze che si sommano costruttivamente al nodo sommatore ovvero per le quali il ritardo $\tau$ corrisponde ad uno sfasamento di multipli di $2\pi$. Nel caso di più ritardi i massimi e minimi della funzione di trasferimento possono presentarsi in modo periodico se i ritardi sono in rapporto armonico.
\begin{figure}[!ht]\centering
\begin{tikzpicture}\def\a{3.0}
\begin{axis}[xlabel=$\omega\tau$,yscale=0.5,ylabel=$\abs{H(\omega)}$,xtick={3.1415,6.2832},ytick={4,16},yticklabels={$1-a$,$1+a$},xticklabels={$\pi$,$2\pi$},extra x ticks={1},extra x tick labels={$B$},extra x tick style={grid=major}]
\addplot[domain=0:6.28,smooth] {sqrt{1+\a^2+2*\a*cos((x))}};
\end{axis}
\end{tikzpicture}
\caption{Diagramma funzione di trasferimento per modello ad echi con attenuazione $a$ e ritardo $\tau$}
\label{fig:ritardo_echi}
\end{figure}

Si può approssimativamente considerare $\abs{H(\omega)}$ costante per $B\ll\frac{1}{\tau}$. Nel caso di echi multipli si deve avere $B\ll\frac{1}{\tau_\text{max}}$. Se la banda non è piccola rispetto all'inverso del massimo ritardo il modello non è valido.


\section{Diversità spaziale e di frequenza}
Per ridurre la probabilità di fuori servizio dovuto ai cammini multipli e al modello ad echi è possibile utilizzare il concetto di \keyword[cammini multipli!diversità spaziale]{diversità spaziale} in ricezione per aumentare l'affidabilità di un ponte radio senza aumentare la potenza trasmessa. La soluzione ha il costo di porre una seconda antenna in ricezione posta ad una certa distanza dalla prima antenna ricevente e di selezionare in ricezione il segnale captato che ha subito minore attenuazione per sfasamento. La probabilità che si verifichi una attenuazione tale da causare il fuori servizio su entrambe le antenne è necessariamente minore: $P_\text{FS}^\text{Tot}=P_\text{FS}\cdot P_\text{FS}$.

Una alternativa alla trasmissione con un cammino radioelettrico differente tra antenne è possibile ottenere il risultato trasmettendo lo stesso segnale su due portanti a frequenza differente, che avranno attenuazione massima diversa per la \keyword[cammini multipli!diversità in frequenza]{diversità in frequenza} e selezionando in ricezione il segnale più potente. Tale tecnica richiede più banda ma si ha il risparmio di non dover installare più antenne.

\chapter{Sistemi \ac{RADAR}}
Diversamente dal sistema di trasmissione analogica utilizzato per trasmettere informazione in un segnale da far giungere da una sorgente ad un destinatario quanto più fedelmente possibile, un sistema \acf{RADAR} ha lo scopo individuare un bersaglio inviando un segnale e ricevendo un eco che ne consenta l'individuazione, calcolandone la distanza, la direzione e lo spostamento.

\begin{figure}[ht]\centering
	\begin{tikzpicture}
	\draw (0,3) pic[rotate=10]{antenna} --++(0,-.5)--++(-.5,-1)++(.5,1)--++(.5,-1)++(-.5,0) node[below]{RADAR};
	\draw (6,4) circle(3pt)node[right,outer sep=3pt]{bersaglio};
	\draw [decorate,decoration={expanding waves,angle=3}] (6,4)--(3,3.5);
	\end{tikzpicture}
	\caption{Sistema \ac{RADAR} con una antenna ricetrasmittente}
	\label{fig:radar_tx_rx}
\end{figure}

\section{Sistema RADAR}\index{RADAR}
L'antenna emette un segnale con una potenza di trasmissione $P_T$. La densità di potenza per unità di superficie che incide su un bersaglio a distanza $R$ è
\begin{equation}
\frac{P_T G_T}{4\pi R^2}
\end{equation}
Parte del segnale si riflette sul bersaglio in proporzione alla sua \keyword[RADAR!sezione RADAR]{sezione RADAR} rilevabile $\sigma [\si{\square\meter}]$ che ingloba anche la direttività dell'“antenna” bersaglio. Della potenza riflessa giunge al ricevutore la quota parte intercettabile dall'area efficace $A_R$ dell'antenna ricevente, attenuata per la divergenza sferica pari al fattore $1/4\pi R^2$. Il sistema ricevitore deve riconoscere il segnale inviato che si è notevolmente attenuato in proporzione inversa alla quarta potenza della distanza $1/R^4$ che è ricevuto con una potenza
\begin{equation}
P_R = \frac{P_T G_T}{4\pi R^2}\,\sigma\,\frac{A_R}{4\pi R^2}
\end{equation}
Elaborando il segnale con un demodulatore d'ampiezza si determina se la potenza ricevuta supera una soglia: dal tempo  $\tau$ intercorso tra l'invio del segnale e il suo ritorno è possibile stimare la distanza del bersaglio $R=c\,\tau/2$, considerando una velocità di propagazione del fronte d'onda pari alla velocità della luce $c$. Il \ac{RADAR} attende un tempo massimo $t_p$ prima di emettere un nuovo impulso, per cui resta definito il massimo raggio di visibilità dei bersagli $R_\text{max}=c\frac{t_p}{2}$.

\begin{nota}In un \ac{RADAR} usualmente si usa la stessa antenna per trasmettere e ricevere potendo disaccoppiare il segnale trasmesso da quello ricevuto.\end{nota}

L'apparecchiatura ricevente è costituita dall'antenna affetta dal rumore del canale radio, da un blocco filtro passa basso, da un demodulatore d'ampiezza a inviluppo che misura l'ampiezza del segnale, da un blocco decisore impostato su una soglia.
\begin{figure}[ht]
\centering
\begin{tikzpicture}[>=latex',
start chain=going right,
node distance=5mm,
every node/.style={on chain,rounded corners=1mm},
every join/.style={->},
block/.style={draw,align=center,text width=2.5cm}]
\pic[rotate=180,on chain,join]{antennarx};
\coordinate[join,on chain](start);
\node[sum,on chain,join]{$+$};
\begin{scope}[start branch=above, every join/.style={<-},]
\node[on chain=going above,join]{$n(t)$};
\end{scope}
\node[block,join]{Filtro Passa Banda $H(f)$};
\node[block,join]{Demodulatore AM a inviluppo};
\node[block,join]{Decisore a soglia};
\coordinate[join](end);
\end{tikzpicture}
\caption{Schema a blocchi di ricevitore \ac{RADAR}}
\label{fig:ricevitore_radar}
\end{figure}

\section{Risoluzione radiometrica}
L'obiettivo del dimensionamento di un sistema \ac{RADAR} è di ottimizzare la potenza trasmissibile e la struttura dell'antenna per ottenere la massima capacità di distinguere il segnale dell'eco riflesso dal rumore.

Data la massima distanza $R_\text{max}$ a cui si vuole rilevare un bersaglio con sezione \ac{RADAR} $\sigma_\text{min}$ si calcola il rapporto segnale rumore che garantisca una minima probabilità di mancato bersaglio e di falso allarme. Non è possibile realizzare \ac{RADAR} in grado di rilevare un qualsiasi bersaglio a qualsiasi distanza.

Per migliorare il rapporto segnale rumore si può aumentare la potenza trasmessa e l'area efficace dell'antenna ma è possibile massimizzare tale rapporto grazie alla teoria dei segnali.

In assenza di bersaglio non vi è il segnale riflesso, in ingresso al ricevitore si ha esclusivamente rumore gaussiano bianco $n(t)$.
In presenza del bersaglio si otterrà una replica del segnale trasmesso attenuata e ritardata nel tempo in proporzione alla distanza, quindi un segnale $s_R(t)=k\cdot s_T(t-\tau)$ a cui si somma il rumore $n(t)$.

Il demodulatore non può essere coerente perché non può conoscere la fase del segnale ricevuto data la distanza incognita. Si utilizza pertanto un demodulatore ad inviluppo che fornisce l'ampiezza del segnale estraendo le componenti in fase e in quadratura, del segnale e del rumore:
\begin{equation}
s_R(t)=s_R^\text{I}(t)\cos{\omega_0 t}-s_R^\text{Q}(t)\cos{\omega_0 t}=A(t)\cos{\omega_0 t+\phi(t)}\quad A(t)=\sqrt{s_R^\text{I}(t)+s_R^\text{Q}(t)}
\end{equation}
\begin{equation}
n(t)=n^\text{I}(t)\cos{\omega_0 t}-n^\text{Q}(t)\cos{\omega_0 t}=r(t)\cos{\omega_0 t+\phi(t)}\quad r(t)=\sqrt{n^\text{I}(t)+n^\text{Q}(t)}
\end{equation}

Essendo il rumore la composizione di due variabili aleatorie gaussiane si ottiene una densità di probabilità di variabile aleatoria di Rayleigh (v.\ref{eq:Rayleigh}). 

In assenza di bersaglio la probabilità di rilevare una tensione all'uscita del demodulatore ad inviluppo superiore alla soglia è la \keyword[probabilità!di falso allarme]{probabilità di falso allarme} $P_\text{FA}$ (v.\ref{fig:radar_prob_falso_allarme}).

In presenza di bersaglio ricevo il segnale riflesso cui si somma il rumore per cui il demodulatore ha in ingresso il segnale
\begin{equation}
A(t)\cos{\omega_0 t+\phi(t)}+r(t)\cos{\omega_0 t+\phi(t)}=D(t)\cos{\omega_0 t+\phi(t)}
\end{equation}
di cui il demodulatore estrarrà il modulo $D(t)$, pari al modulo della somma dei vettori rotanti del segnale e del rumore. 

Nel funzionamento nominale del \ac{RADAR} il termine del segnale è dominante sul rumore (v.fig.\ref{fig:segnale_modulato_angolarmente_affetto_da_rumore}), si può approssimare il modulo con la somma delle componenti in fase, considerando l'errore di fase uniformente distribuito in $[0,2\pi]$:
\[D(t)\cong A(t)+r(t)\cos{\phi_N(t)-\phi(t)}\cong A(t)+r(t)\cos{\phi_N(t)}\]

Il modulo del segnale ricevuto avrà approssimativamente una statistica gaussiana con il valore medio dipendente dalla potenza dell'eco che ritorna al ricevitore: maggiore è la potenza ritornata maggiore sarà la tensione all'uscita del demodulatore. Un valore di tensione al demodulatore sotto la soglia si verifica con una certa \keyword[probabilità!di bersaglio mancato]{probabilità di bersaglio mancato}.

La scelta del valore di soglia è un compromesso tra una bassa probabilità di bersaglio mancato e una bassa probabilità di falso allarme. Il valore di soglia ottimo, che determina la probabilità di falso allarme e bersaglio mancato, è influenzato dalla varianza del rumore gaussiano e assieme ai parametri di progetto, la sezione \ac{RADAR} minima $\sigma_\text{min}$ e la distanza massima di rilevazione $R_\text{max}$, si determina la potenza da trasmettere tale che il rapporto segnale rumore \ac{SNR} garantisca che la potenza dell'eco ricevuto sia sufficientemente grande rispetto alla potenza del rumore.
Il dimensionamento del sistema con i parametri $\sigma_\text{min},R_\text{max},P_\text{FA},P_\text{BM}$ segue il criterio di \emph{Neymann-Pearson}.

\begin{figure}[ht!]\centering\def\soglia{5}
	\subfloat[Probabilità di falso allarme]{\begin{tikzpicture}
		\begin{axis}[xlabel=$X$,ylabel=$f_R(X)$,ytick=\empty,xtick={5.0},xticklabels={soglia},yticklabels={},yscale=.8,domain=0:10]
		\addplot [name path=rayleigh5,smooth]{x/5.*exp(-(x^2)/10.)};
		\path[name path=axis] (axis cs:0,0)--node[above right,pin=45:$P_\text{FA}$]{}(axis cs:10,0);
		\addplot[pattern=north west lines] fill between[of=rayleigh5 and axis,soft clip={domain=\soglia:10}];
		\end{axis}
		\label{fig:radar_prob_falso_allarme}
		\end{tikzpicture}
	}\quad\subfloat[Probabilità di bersaglio mancato]{
		\begin{tikzpicture}
		\begin{axis}[xlabel=$X$,ylabel=$f_D(X)$,xtick={10},ytick=\empty,xticklabels={A},yticklabels={},yscale=.8,extra x ticks={5.0},extra x tick labels={soglia},extra x tick style={grid=major},domain=0:20]
		\addplot [name path=gauss5,smooth] {100*gauss(x,10,5)};
		\path[name path=axis] (axis cs:0,0.75)--node[above right,pos=.33,pin=135:$P_\text{BM}$]{}(axis cs:\soglia,.75);
		\addplot[pattern=north east lines] fill between[of=gauss5 and axis,soft clip={domain=0:\soglia}];
		\end{axis}
		\label{fig:radar_prob_bersaglio_mancato}
		\end{tikzpicture}
	}\quad\subfloat[Valore di soglia]{
		\begin{tikzpicture}
		\begin{axis}[xscale=1.3,xlabel=$X$,xtick=\empty,ytick=\empty,xticklabels={},extra x ticks={5.0},extra x tick labels={soglia},extra x tick style={grid=major},yscale=.8,domain=0:20]
		\addplot [name path=gauss5,smooth] {gauss(x,10,5)};
		\addplot [name path=rayleigh5,smooth]{x/5.*exp(-(x^2)/10.)};
		\path[name path=axis] (axis cs:0,0)--node[above right,pos=.25,pin=45:$P_\text{FA}$]{}node[above left,pos=.25,pin=135:$P_\text{BM}$]{}(axis cs:20,0);
		\addplot[pattern=north west lines] fill between[of=rayleigh5 and axis,soft clip={domain=\soglia:20}];
		\addplot[pattern=north east lines] fill between[of=gauss5 and axis,soft clip={domain=0:\soglia}];
		\end{axis}
		\end{tikzpicture}
	}
\label{fig:radar_soglia}
\end{figure}

Per ottimizzare la potenza di picco del segnale ricevuto bisogna agire sulla funzione di trasferimento $H(f)$ del filtro passa banda del ricevitore. Per massimizzare il valore di uscita del demodulatore in presenza di bersaglio bisogna esaltare la potenza di picco del segnale, nell'istante $t_m$ del suo massimo, più della potenza del rumore, agendo sulla funzione di trasferimento del filtro che viene adattata al segnale trasmesso. 

Data la trasformata di Fourier del segnale ricevuto $k s(t-\tau) \stackrel{\Fourier}{\longrightarrow} k S(f) \e{-\jmath\omega_0\tau}$ si calcola la potenza di picco del segnale nell'istante $t_m$:
\begin{equation}
P_S^\text{PP}=\abs{\intinf{k S(f)\e{-\jmath\omega_0 \tau} H(f)\e{\jmath\omega_0 t_m}}{f}}^2
\end{equation}
e la potenza media del rumore (a banda stretta)
\begin{equation}
P_N=\intinf{h_n\abs{H(f)}^2}{f}
\end{equation}

Il rapporto segnale rumore, definito come rapporto della potenza picco-picco del segnale e la potenza media del rumore, risulta
\begin{equation}\begin{split}
\restrict{\frac{S}{N}}{\text{post dem}}&=\frac{P_S^\text{PP}}{P_N}=\frac{\abs{\intinf{k S(f) H(f)\e{\jmath\omega_0 (t_m-\tau)}}{f}}^2}{\intinf{h_n\abs{H(f)}^2}{f}}=
\frac{\intinf{k^2\abs{S(f)}^2\abs{H(f)}^2}{f}}{\intinf{h_n\abs{H(f)}^2}{f}}=\\
&<\frac{\intinf{k^2\abs{S(f)}^2}{f}\cdot\intinf{\abs{H(f)}^2}{f}}{h_n\intinf{\abs{H(f)}^2}{f}}=\frac{\intinf{k^2\abs{S(f)}^2}{f}}{h_n}=\frac{E_R}{h_n}
\end{split}\end{equation} 

Per il teorema di Parseval (v.eq.\ref{eq:parseval}) il termine a numeratore $E_R=\intinf{\abs{k S(f)}^2}{f}$ è l'energia del segnale ricevuto.

Il rapporto segnale rumore è maggiorato dall'energia del segnale ricevuto rapportata alla densità spettrale di rumore gaussiano bianco.

L'uguaglianza si ha per \keyword[filtro adattato]{filtro adattato} ovvero con spettro\footnote{Il filtro adattato opera il rifasamento di tutte le componenti spettrali.} \begin{equation}H(f)\propto\conj{S}(f)\label{eq:radar_filtro_adattato_rumore_bianco}\end{equation}

Il filtro adattato massimizza il rapporto segnale rumore ottenendo la massima autocorrelazione tra segnale ricevuto e filtro nell'istante $\tau=t_m$
\begin{equation}\begin{split}
\restrict{\frac{S}{N}}{\text{post dem}}&=\dfrac{\abs{\intinf{k\alpha\abs{S(f)}^2\e{\jmath\omega_0(t_m-\tau)}}{f}}^2}{h_n\intinf{\alpha^2\abs{S(f)}^2}{f}}=\dfrac{\alpha^2 k^2\abs{\intinf{\abs{S(f)}^2\e{\jmath\omega_0(t_m-\tau)}}{f}}^2}{\alpha^2 h_n\intinf{\abs{S(f)}^2}{f}}=\\
&=\frac{k^2\intinf{\abs{S(f)}^2}{f}}{h_n}=\frac{E_R}{h_n}
\end{split}\end{equation}

In caso di rumore gaussiano non bianco è necessario pesare in misura maggiore le componenti che hanno un rapporto più grande possibile tra ampiezza dello spettro del segnale e ampiezza dello spettro del rumore.
\begin{equation}
H(f)\propto\frac{\conj{S}(f)}{h'_n}
\label{eq:radar_filtro_adattato_rumore_colorato}
\end{equation}

\section{Risoluzione geometrica azimutale}
Un sistema \ac{RADAR} con una antenna omnidirezionale può rilevare la presenza di un bersaglio e calcolare la sua distanza che è proporzionale al ritardo col quale l'eco torna all'antenna
\begin{equation}
R=c\,\frac{\tau}{2}
\end{equation}

Per poter determinare la direzione da cui proviene il segnale riflesso una antenna direzionale viene fatta ruotare sul proprio asse per spazzare lo spazio aereo con un fascio di radiazioni ristretto in un piano azimutale in cui è concentrata la potenza trasmessa dal \ac{RADAR}.

\newcommand{\radar}{
\pgfmathparse{\targetlong-\waveazimutalfov/2}\let\longmin=\pgfmathresult
\pgfmathparse{\targetlong+\waveazimutalfov/2}\let\longmax=\pgfmathresult
\pgfmathparse{\longmin+\longstep}\let\longfor=\pgfmathresult
\pgfmathparse{\longmax-\longstep}\let\longlast=\pgfmathresult
\pgfmathparse{\latmin+\latstep}\let\latfor=\pgfmathresult
\pgfmathparse{\latmax-\latstep}\let\latlast=\pgfmathresult
\begin{tikzpicture}[x={(1cm,0)},y={(-.5cm,-.5cm)},z={(0cm,1cm)}]% sistema di riferimento 3D

% piano di base
\draw [help lines](-4,-3,0)--(-4,3,0)--(4,3,0)--(4,-3,0)--cycle;

% draw base circle in plane xy and axis
\pgfsetstrokeopacity{0.5}
\pgfpathmoveto{\pgfpointorigin}
\pgfpathlineto{\pgfpointspherical{0}{90}{\waveradius}}
\pgfpathmoveto{\pgfpointspherical{0}{0}{\waveradius}}
\foreach\azimuth in {0,6,...,360} {
	\pgfpathlineto{\pgfpointspherical{\azimuth}{0}{\waveradius}}
}
\pgfusepath{stroke}

% draw tangetial plane intersection
\pgfsetdash{{1mm}{1mm}}{0pt}
\pgfpathmoveto{\pgfpointspherical{\targetlong-90}{0}{3}}
\pgfpathlineto{\pgfpointspherical{\targetlong+90}{0}{3}}
\pgfusepath{stroke}

% draw front wave cone
\pgfpathmoveto{\pgfpointspherical{\longmin}{0}{\waveradius}}
\pgfpathlineto{\pgfpointspherical{\longmin}{\latmin}{\waveradius}}
\pgfpathlineto{\pgfpointorigin}
\pgfpathlineto{\pgfpointspherical{\longmax}{\latmin}{\waveradius}}
\pgfpathlineto{\pgfpointspherical{\longmax}{0}{\waveradius}}
\pgfpathmoveto{\pgfpointspherical{\longmin}{\latmax}{\waveradius}}
\pgfpathlineto{\pgfpointorigin}
\pgfpathlineto{\pgfpointspherical{\longmax}{\latmax}{\waveradius}}
\pgfpathlineto{\pgfpointspherical{\longmax}{90}{\waveradius}}
\pgfpathlineto{\pgfpointspherical{\longmin}{\latmax}{\waveradius}}
\pgfusepath{stroke}

% draw target projection
\pgfsetdash{{2mm}{2mm}{1pt}{2mm}}{0pt}
\pgfpathmoveto{\pgfpointorigin}
\pgfpathlineto{\pgfpointspherical{\targetlong}{0}{\targetdistance*cos(\targetlat)}}
\pgfpathlineto{\pgfpointspherical{\targetlong}{\targetlat}{\targetdistance}}
\pgfusepath{stroke}
\pgfsetdash{}{0pt}

% draw the antenna directional wave
\pgfpathmoveto{\pgfpointspherical{0}{0}{0}}
\pgfpathlineto{\pgfpointspherical{\targetlong}{\targetlat}{\waveradius}}
\pgfusepath{stroke}
\pgfsetfillcolor{lightgray}
\pgfsetfillopacity{0.5}
\foreach \latitude in {\latmin,\latfor,...,\latlast}
{
	\foreach \longitude in {\longmin,\longfor,...,\longlast}
	{
		\pgfpathmoveto{\pgfpointspherical{\longitude}{\latitude}{\waveradius}}
		\pgfpathlineto{\pgfpointspherical{\longitude+\longstep}{\latitude}{\waveradius}}
		\pgfpathlineto{\pgfpointspherical{\longitude+\longstep}{\latitude+\latstep}{\waveradius}}
		\pgfpathlineto{\pgfpointspherical{\longitude}{\latitude+\latstep}{\waveradius}}
		\pgfpathclose
	}
	\pgfusepath{fill,stroke}
}

%\pgfdeclarefading{fading}{\tikz\shade[left color=pgftransparent!0,right color=pgftransparent!100] (0,0) rectangle (1,1);}
% draw azimutal plane
\pgfsetfillopacity{0.25}
\pgfpathmoveto{\pgfpointorigin}
\pgfpathlineto{\pgfpointspherical{\targetlong}{0}{\maxdistance*cos(\targetlat)}}
\pgfpathlineto{\pgfpointspherical{\targetlong}{\targetlat}{\maxdistance}}
\pgfpathlineto{\pgfpointspherical{\targetlong}{90}{\maxdistance*sin(\targetlat)}}
\pgfpathclose
%\pgfsetfadingforcurrentpath{fading}{\pgftransformshift{\pgfpoint{-1cm}{-1cm}}}
\pgfusepath{fill}

% draw target vector
\pgfsetstrokeopacity{1}
\pgfsetfillcolor{white}
\pgfsetfillopacity{1}
\pgfpathmoveto{\pgfpointspherical{\targetlong}{\targetlat}{\waveradius}}
\pgfpathlineto{\pgfpointspherical{\targetlong}{\targetlat}{\targetdistance}}
\pgfusepath{stroke}
\pgfpathcircle{\pgfpointspherical{\targetlong}{\targetlat}{\targetdistance}}{2pt}
\pgfusepath{fill,stroke}
\end{tikzpicture}
}
\begin{figure}[ht!]\centering
\subfloat{
	% dati bersaglio e fronte d'onda
	\def\targetlong{60}
	\def\targetlat{30}
	\def\targetdistance{4}
	\def\maxdistance{4}
	\def\waveradius{2}
	\def\waveazimutalfov{30}
	\def\longstep{5}
	\def\latstep{6}
	\def\latmin{0}
	\def\latmax{60}
	\radar
}\quad\subfloat{
	% dati bersaglio e fronte d'onda
	\def\targetlong{-30}
	\def\targetlat{45}
	\def\targetdistance{3}
	\def\maxdistance{4}
	\def\waveradius{2}
	\def\waveazimutalfov{30}
	\def\longstep{5}
	\def\latstep{6}
	\def\latmin{30}
	\def\latmax{60}
	\radar
}
\caption{Risoluzione geometrica azimutale}
\end{figure}

Il sistema \ac{RADAR} può rilevare e distinguere l'eco di due bersagli a distanza radiale $R$ dall'antenna e distanza trasversale $d$ fra loro se il fascio azimutale ha una risoluzione angolare inferiore a 
\begin{equation}
\theta=\arctan\frac{d}{R}
\end{equation}

\begin{figure}[ht!]\centering
\def\maxdistance{4}
\begin{tikzpicture}[x={(1cm,0)},y={(-.5cm,-.5cm)},z={(0cm,1cm)}]% sistema di riferimento 3D
% piano di base
\draw [help lines](-4,-3,0)--(-4,3,0)--(4,3,0)--(4,-3,0)--cycle;

% draw base circle in plane xy and axis
\begin{pgfscope}
\pgfpathmoveto{\pgfpointorigin}
\pgfpathlineto{\pgfpointspherical{0}{90}{\maxdistance/2}}
\pgfpathmoveto{\pgfpointspherical{0}{0}{\maxdistance/2}}
\foreach\azimuth in {0,6,...,360} {
	\pgfpathlineto{\pgfpointspherical{\azimuth}{0}{\maxdistance/2}}
}
\pgfsetstrokeopacity{0.5}
\pgfusepath{stroke}
\end{pgfscope}

% draw for each target
\foreach \long/\lat/\distance in {60/60/2.,90/30/2} {
	\begin{pgfscope}
	% draw targets projection
	\pgfpathmoveto{\pgfpointorigin}
	\pgfpathlineto{\pgfpointspherical{\long}{0}{\distance*cos(\lat)}}
	\pgfpathlineto{\pgfpointspherical{\long}{\lat}{\distance}}
	\pgfsetstrokeopacity{0.5}
	\pgfsetdash{{1mm}{1mm}}{0pt}
	\pgfusepath{stroke}
	
	% draw azimutal plane
	\pgfpathmoveto{\pgfpointorigin}
	\pgfpathlineto{\pgfpointspherical{\long}{0}{\maxdistance*cos(\lat)}}
	\pgfpathlineto{\pgfpointspherical{\long}{\lat}{\maxdistance}}
	\pgfpathlineto{\pgfpointspherical{\long}{90}{\maxdistance*sin(\lat)}}
	\pgfpathclose
	\pgfsetfillcolor{lightgray}
	\pgfsetfillopacity{0.25}
	\pgfusepath{fill}
	\end{pgfscope}
	
	% draw target vector
	\begin{pgfscope}
	\pgfpathmoveto{\pgfpointorigin}
	\pgfpathlineto{\pgfpointspherical{\long}{\lat}{\distance}}
	\pgfsetstrokeopacity{1}
	\pgfusepath{stroke}
	\pgfpathcircle{\pgfpointspherical{\long}{\lat}{\distance}}{2pt}
	\pgfsetdash{}{0pt}
	\pgfsetfillcolor{white}
	\pgfsetfillopacity{1}
	\pgfusepath{fill,stroke}
	\end{pgfscope}
}
\pgfsetfillcolor{black}

% {60/60/2.,90/30/2}
\begin{pgfscope}
\pgfsetstrokeopacity{0.5}
\pgfsetdash{{1mm}{1mm}}{0pt}
\pgfpathmoveto{\pgfpointspherical{60}{0}{2*cos(60)}}
\pgfpathlineto{\pgfpointspherical{90}{0}{2*cos(30)}}
\pgfusepath{stroke}
\pgftext[right,at={\pgfpointspherical{75}{0}{1.5}}]{$d$}
\pgftext[right,at={\pgfpointspherical{75}{0}{.5}}]{$\theta$}
\end{pgfscope}
\end{tikzpicture}
\end{figure}

Per rilevare anche la quota del bersaglio è necessario utilizzare un altro \ac{RADAR} il cui fascio di radiazioni sia concentrato in un piano che ruoti trasversalmente al piano azimutale.

\section{Risoluzione geometrica radiale}
In presenza di più bersagli a distanze diverse per ogni impulso trasmesso il sistema \ac{RADAR} riceverà più segnali eco riflessi con attenuazioni e ritardi differenti. Le forme d'onda sono risolvibili l'una dall'altra fino a quando sono distinte all'uscita del filtro in ricezione, pertanto la capacità risolutiva è tanto maggiore quanto più stretta è la forma d'onda. L'uscita del filtro per un segnale sinusoidale sarà del tipo
\begin{equation}
s(t)=\intinf{S_R(f)\conj{S}(f)\e{\jmath\omega t_m}}{f}
\end{equation}
L'operazione del filtro è quella di calcolare l'autocorrelazione tra il segnale ricevuto e il modello del segnale inviato in modo tale da riconoscerlo. L'autocorrelazione di due segnali si può distinguere fino a quando è possibile distinguere gli spettri dei due segnali in uscita dal filtro adattato. Per un segnale rettangolare di periodo $T$ (o a inviluppo rettangolare) si ha per convoluzione un segnale triangolare di periodo $2T$: sarà possibile discriminare i due echi fino ad un ritardo temporale minimo pari a $\Delta T$, che corrisponde ad una capacità discriminatoria nello spazio pari a $\Delta R=c \Delta T/2$.

Per lo stesso motivo un sistema \ac{RADAR} ha una zona d'ombra entro la quale l'eco riflesso da un bersaglio torna prima che si sia terminato di trasmettere tutto il segnale di periodo $T$.

\begin{figure}[ht]\centering
\subfloat[Echi risolvibili]{
\begin{tikzpicture}
\begin{axis}[yscale=0.6,xlabel=$t$,ylabel=$s_T(t)$,xtick={1,2.5,4},xticklabels={$T$,$\tau_1$,$\tau_2$},ytick={-1,1},black,smooth,samples=300]
\def\omegazero{2*pi*8}
\addplot[domain=0:1] {sin((\omegazero*x))};
\addplot[domain=2.5:3.5] {.3*sin((\omegazero*x))};
\addplot[domain=4:5] {.25*sin((\omegazero*x))};
\end{axis}
\end{tikzpicture}
}\quad\subfloat[Echi sovrapposti]{
\begin{tikzpicture}
\begin{axis}[yscale=0.6,xlabel=$t$,ylabel=$s_T(t)$,xtick={1,2.5,3},xticklabels={$T$,$\tau_1$,$\tau_2$},ytick={-1,1},black,smooth,samples=300]
\def\omegazero{2*pi*8}\def\ampa{.3}\def \ampb{.25}
\addplot[domain=0:1] {sin((\omegazero*x))};
\addplot[domain=2.5:3] {\ampa*sin((\omegazero*x))};
\addplot[domain=3:4] {\ampa*sin((\omegazero*x))+\ampb*sin((\omegazero*x))};
\addplot[domain=4:4.5] {\ampa*sin((\omegazero*x))};
\end{axis}
\end{tikzpicture}
}
\caption{Schema segnale \ac{RADAR} trasmesso ed echi ricevuti}
\end{figure}

Per poter migliorare la risoluzione geometrica radiale è possibile ridurre la durata $T$ del segnale trasmesso, consentendo di discriminare echi con ritardi più vicini tra loro. Riducendo il periodo di trasmissione del segnale però si riduce l'energia trasmessa per cui è necessario amplificare il più possibile il segnale trasmesso.

In alternativa si può ottimizzare la capacità di riconoscere un segnale di durata molto breve ampliandone la banda trasmessa per ottenere una autocorrelazione quanto più stretta possibile: un segnale sinusoidale a frequenza crescente nel periodo $[0,T]$, detto \keyword[segnale!chirp]{chirp}. L'autocorrelazione di due segnali chirp sovrapposti ha carattere impulsivo di periodo $T_E=\frac{1}{B}$. La risoluzione radiale diventa non più funzione del periodo ma della banda del segnale trasmesso $\Delta R=\frac{c}{2B}$.

\begin{figure}[ht]\centering
	\begin{tikzpicture}
	\begin{axis}[yscale=.6,xlabel=$t$,ylabel=$s_T(t)$,xtick={1},xticklabels={$T$},ytick={-1,1},black,smooth,samples=500]
	\addplot[domain=0:1] {sin(((7*pi+32*pi*x)*x))};
	\end{axis}
	\end{tikzpicture}
	\caption{Segnale \keyword{Chirp}}
\end{figure}

\section{Bersagli fissi e mobili}
La conformazione del territorio attorno al \ac{RADAR} costituisce un bersaglio fisso il cui eco riflesso disturba la rilevazione di bersagli di interesse. Ad esempio un bersaglio aereo interposto tra l'antenna e una montagna (fig.~\ref{fig:radar_bersaglio_fisso}) sarà indistinguibile per il demodulatore ad inviluppo che rileverà un oggetto di sezione radar ordini di grandezza maggiore del bersaglio.

\begin{figure}[ht]\centering
	\begin{tikzpicture}[scale=.1]
	\draw (0,3) pic[rotate=35,scale=.5]{antenna} --++(-1,-3)--++(2,0)--++(-1,3);
	\draw [fill] (35:50) circle(10pt)node[left,outer sep=10pt]{bersaglio};
	\draw [decorate,decoration={expanding waves,angle=3}] (6,4)--(3,3.5);
	\draw [decorate,decoration={random,amplitude=10}](0,0)--(30,0)--++(60:50)--++(30:3)--++(-30:3)--++(-60:50)--++(0:5);
	\end{tikzpicture}
	\caption{Sistema \ac{RADAR} con bersaglio fisso}
	\label{fig:radar_bersaglio_fisso}
\end{figure}

Dato il segnale trasmesso $A\cos{\omega_0 t+\phi(t)}$ si riceverà con ritardi $\tau_1$ e $\tau_2$ la sovrapposizione degli echi della montagna e del bersaglio aereo:
\[\text{montagna}\colon\quad A_1\cos{\omega_0(t-\tau_1)+\phi(t-\tau_1)}=\Re{A_1\e{\jmath 2\pi\Phi_1}}\]
\[\text{bersaglio}\colon\quad A_2\cos{\omega_0(t-\tau_2)+\phi(t-\tau_2)}=\Re{A_2\e{\jmath 2\pi\Phi_2}}\]
\[A_R\cos{\omega_0 t-\phi_R(t)}\]

Per poter distinguere l'eco del bersaglio in movimento con $A_2\ll A_1$ si confronta non l'ampiezza dell'eco tra due scansioni successive ma la fase. L'ampiezza e la fase dell'eco della montagna risultano sostanzialmente costanti tra due scansioni successive dell'antenna in rotazione, mentre varierà la fase dell'eco del bersaglio mobile. Effettuando la differenza tra segnali ottenuti da due scansioni successive è possibile cancellare l'eco dei bersagli fissi e riuscire ad isolare l'eco del bersaglio in movimento (\ac{MTI}).

Per estrapolare la fase del segnale riflesso si utilizza un \keyword[demodulatore!I-Q]{demodulatore I-Q} con due portanti in quadratura.
\begin{figure}[h!]
	\centering
		\begin{tikzpicture}[>=latex',thick];
		\coordinate(c1);
		\coordinate[below=1cm of c1](c0);
		\coordinate[below=1cm of c0](c2);
		\node [left=1cm of c0](n0){$s_R$};
		\draw (n0)--(c0)|-(c1)(c0)|-(c2);
		\node [mult,right=1cm of c1](m1){} edge[<-](c1);
		\node [above=5mm of m1]{$2\cos{\omega_0 t}$} edge[->](m1);
		\node [passabasso,right=1cm of m1](pb1){} edge[<-](m1);
		\node [right=1cm of pb1]{$s_R^\text{I}$} edge[<-](pb1);
		\node [mult,right=1cm of c2](m2){} edge[<-](c2);
		\node [above=5mm of m2]{$2\sen{\omega_0 t}$} edge[->](m2);
		\node [passabasso,right=1cm of m2](pb2){} edge[<-](m2);
		\node [right=1cm of pb2]{$s_R^\text{Q}$} edge[<-](pb2);
		\end{tikzpicture}
	\caption{Schema di demodulatore I-Q in radar coerente}
	\label{fig:radar_demodulatore_IQ}
\end{figure}

Dato l'impulso \ac{RADAR} trasmesso $s_T(t)=\cos{\omega_0 t+\phi_T(t)}$ si suppone di ricevere il segnale riflesso sovrapposizione degli echi (ad es. della montagna e del bersaglio mobile) \[s_R(t)=a_R\cos{\omega_0 t+\phi_R(t)}=a_1\cos{\omega_0 (t-\tau_1)+\phi_R(t-\tau_1)}+a_2\cos{\omega_0 (t-\tau_2)+\phi_R(t-\tau_2)}\]
che in ingresso al demodulatore I-Q in modulazione con le portanti in quadratura danno
\[s_R^\text{I}(t)=a_R\cos{\omega_0 t+\phi_R(t)}\cdot 2\cos{\omega_0 t}=a_R\cos{\phi_R(t)}\]
\[s_R^\text{Q}(t)=a_R\cos{\omega_0 t+\phi_R(t)}\cdot 2\sen{\omega_0 t}=a_R\sen{\phi_R(t)}\]

Dai segnali in fase e quadratura si può ricavare l'ampiezza e la fase del segnare ricevuto:
\[a_R(t)=\sqrt{s_R^\text{I}(t)^2+s_R^\text{Q}(t)^2}\quad\phi_R(t)=\arctan\frac{s_R^\text{Q}(t)}{s_R^\text{I}(t)}\]

Tra due scansioni successive posso avere inviluppi di ampiezza simili ma è possibile apprezzare i cambiamenti di fase essendo
\[\Phi_R(t)=\Phi_1(t)+\arctan\frac{a_2\sen{\Phi_2-\Phi_1}}{a_R}\]

\[			\begin{split}\Phi_R''-\Phi_R'=\phi_R''-\phi_R'&=-\arctan\frac{a_2'\sen{\Phi_2'-\Phi_1}}{a_R'}+\arctan\frac{a_2''\sen{\Phi_2''-\Phi_1}}{a_R''}=\\
&\cong\frac{a_2''\sen{\Phi_2''-\Phi_1}-a_2'\sen{\Phi_2'-\Phi_1}}{a_1}\quad(a_1\gg a_2\implies a_R=a_1)\end{split}\]

\begin{figure}[!ht]\centering
	\begin{tikzpicture}[>=latex',black,scale=.9]
	\draw[<->](0,4)node[above]{$\Imaginarypart$}--(0,0)--(6,0)node[right]{$\Realpart$};
	\coordinate (O) at (0,0);
	\coordinate (A) at (30:5);
	\coordinate (B) at ($(A)+(60:1.5)$);
	\coordinate (C) at ($(A)+(30:1.5)$);
	\coordinate (D) at ($(A)+(120:1.5)$);
	\coordinate (E) at (3,0);
	\coordinate (F) at (5,0);
	\draw[thick,->](O)--node[below]{$a_1$}(A);
	\draw[thick,->](A)--node[pos=0.99,right]{$a_2$}(B);	
	\draw[thick,->](O)--node[above]{$a_R$}(B);
	\draw[thin] (A)--(C);
	\draw[thin] (E)--(B);
	\draw[thin] (1,0) arc(0:30:1) node[pos=.5,right]{$\Phi_1$};
	\draw[thin] (4,0) arc(0:60:1) node[pos=.5,right]{$\Phi_2$};
	\draw[thin] ($(A)+(30:1)$) arc(30:60:1) node[pos=.66,right]{$\Phi_R$};
-	\end{tikzpicture}
	\caption{Fasori segnale modulato in quadratura}\label{fig:segnale_modulato_in_quadratura}
\end{figure}

In un breve intervallo di tempo tra due scansioni il bersaglio si sarà spostato di poco pertanto è possibile approssimare il seno dell'angolo con l'angolo
\[\Phi_R(t)\cong\Phi_1(t)+\frac{a_2}{a_1}(\Phi_2-\Phi_1)\]
\[\begin{split}\Delta\Phi_R&=\frac{a_2}{a_1}\Delta\Phi=\frac{a_2}{a_1}\left[\underbrace{(\omega_0 t-\omega_0\tau_2''+\phi_T(t-\tau_2''))}_{\Phi_2''}-\underbrace{(\omega_0 t-\omega_0\tau_2'+\phi_T(t-\tau_2'))}_{\Phi_2'}\right]=\\
&=\frac{a_2}{a_1}[\omega_0 (\tau_2''-\tau_2'')+\phi_T(t-\tau_2'')-\phi_T(t-\tau_2')]\end{split}\]


\clearpage
\section{Shift Doppler}
Il sistema \ac{RADAR} può rilevare se il bersaglio in movimento è in avvicinamento o allontanamento sulla congiungente antenna-bersaglio. Si determina la distanza radiale $v_R$ del bersaglio in movimento, considerando costante la velocità radiale nell'intervallo tra due impulsi radar:
\begin{equation}
R(t)=R_0+v_R t
\end{equation}

Il tempo necessario al segnale di propagarsi, riflettersi sul bersaglio e tornare indietro diventa una funzione del tempo
\begin{equation}
\tau(t)=\frac{2 R(t)}{C}=2\frac{R_0+v_R t}{c}=\tau_0+k t\quad k=\frac{v_R}{c}
\end{equation}

Il segnale riflesso subisce un ritardo tempovariante $s_R(t)=s_T(t-\tau_0-k t)$ ha spettro per la trasformata di Fourier
\[S_R(f)=\intinf{s(t-\tau_0-k t)\e{-\jmath 2\pi f t}}{f}\]
effettuando il cambio di variabile $\alpha=t-\tau_0-k t$, $t=\frac{\alpha+\tau_0}{1-k}$, $\diff t=\frac{\diff\alpha}{1-k}$,
\[\begin{split}S_R(f)&=\intinf{\frac{s(\alpha)}{1-k}\e{-\jmath 2\pi f\left(\frac{\alpha}{1-k}+\frac{\tau_0}{1-k}\right)}}{\alpha}=\\
&=\frac{\e{-\jmath 2\pi f\frac{\tau_0}{1-k}}}{1-k}\intinf{s(\alpha)\e{-\jmath 2\pi f\frac{\alpha}{1-k}}}{\alpha}=\\
&=\frac{\e{-\jmath 2\pi f\frac{\tau_0}{1-k}}}{1-k} S\left(\frac{f}{1-k}\right)\end{split}\]

Si può notare che la singola componente sinusoidale dello spettro moltiplicata per l'operatore lineare ritardo, che ha funzione di trasferimento $\e{-\jmath\omega\tau}=\e{-\jmath 2\omega\frac{R_0}{c}}\e{-\jmath 2\omega\frac{v_R}{c}t}$, risulta variata nell'ampiezza e ruotata nella fase: alla frequenza $f_A$ corrisponde la $\frac{1}{1-k}f_A$, con una variazione $\Delta f=f_A-\frac{f_A}{1-k}=\left(1-\frac{1}{1-k}\right)f_A=\frac{-k}{1-k}f_A\cong-k f_A=-\frac{v_R}{c}f_A$

Un segnale passa banda stretta avrà il centro banda $f_0$ traslato in $f_0-\frac{v_R}{c}f_0$, e l'estremo $f_\text{min}$ in $f_\text{min}-\frac{v_R}{c}f_\text{min}$
\begin{figure}[ht!]\centering
	\def\fmin{500.}\def\fmax{3000.}
	\def\fmina{450.}\def\fmaxa{2700.}
	\begin{tikzpicture}
	\begin{axis}[hide y axis,yscale=.1,xlabel=$f$,extra x ticks={\fmina,\fmaxa,\fmin,\fmax},xtick={100,10000},extra x tick labels={$f_\text{min}'$,$f_\text{max}'$,$f_\text{min}$,$f_\text{max}$},extra x tick labels/.style={inner sep=10pt},domain=10:10000,every x tick scale label/.style={at={(xticklabel* cs:.9,5pt)},/pgfplots/near ticklabel align=outside,anchor=near xticklabel opposite,inner sep=0pt},]
	\addplot coordinates {(\fmin,0)(\fmin,1)(\fmax,1)(\fmax,0)};
	\addplot coordinates {(\fmina,0)(\fmina,.5)(\fmaxa,.5)(\fmaxa,0)};
	\end{axis}
	\end{tikzpicture}
\end{figure}

\chapter{Sistemi di trasmissione numerica}
Un \keyword[sistema di trasmissione!numerico]{sistema di trasmissione numerico} può essere utilizzato per trasmettere qualunque tipo di informazione codificata in una stringa di simboli numerici binari $1$ e $0$. Tali simboli generati da una sorgente con una certa velocità o \keyword[bit rate]{bitrate} in $\si{\bit\per\second}$, vengono trasformati a livello fisico in forme d'onda analogiche, e, affette da disturbi e rumore del canale di trasmissione, giungono al ricevitore. 

Al ricevitore non interessa ricostruire fedelmente le forme d'onda trasmesse ma è necessario elaborare e campionare il segnale per decidere in ogni intervallo di tempo quale simbolo si è ricevuto e poter ricostruire, con una certa probabilità di errore, la sequenza numerica dei simboli trasmessi.

\begin{figure}[!ht]
	\centering
	\resizebox{\textwidth}{!}{
		\begin{tikzpicture}[>=latex']
		\coordinate(c0);
		\node[block,right=1.5cm of c0](b0){GFO};
		\node[block,right=1.5cm of b0](b1){MT};
		\node[sum,right=1cm of b1](s0){$+$};
		\coordinate[above=1cm of s0](n);
		\node[block,right=1.5cm of s0](b2){FR};
		\node[campionatore,right=1cm of b2](q0){};
		\node[block,right=1cm of q0](b3){DEC};
		\coordinate[right=1.5cm of b3](c1);
		\draw[->](c0)--node[above,near start]{bit}(b0);
		\draw[dot=O](b0)--(b1);
		\draw[->](b1)--(s0);
		\draw[->](n)node[above]{$n(t)$}--(s0);
		\draw[dot=A](s0)--(b2);
		\draw[dot=B](b2)--(q0);
		\draw[dot=C](q0)--(b3);
		\draw[->](b3)--node[above,near end]{bit}(c1);
		\node[fitted, fit=(b0),label=above:Trasmettitore]{};
		\node[fitted, fit=(b1)(s0),label=above:Canale]{};
		\node[fitted, fit=(b2)(q0)(b3),label=above:Ricevitore]{};
		\end{tikzpicture}
	}
	\caption{Schema sistema di trasmissione numerico in banda base}
	\label{fig:schema_sistema_trasmissione_numerico_banda_base}
\end{figure}

Se il generatore di forme d'onda genera una tensione maggiore di zero per $T$ secondi per codificare “1” e una tensione nulla per $T$ secondi per codificare “0” si è in una condizione analoga a quella del sistema \ac{RADAR}: l'assenza di bersaglio corrisponde in ricezione al solo rumore tradotto nel simbolo “0”, la presenza di bersaglio corrisponde alla ricezione di segnale rumoroso sovrapposto alla forma d'onda ritardata e attenuata da interpretare come simbolo “1”.

In generale si può supporre di trasmettere una forma d'onda $s_0(t)$ per codificare il simbolo “$0$” e una forma d'onda diversa $s_1(t)$ per l'“$1$”. In tal caso al ricevitore si avranno due filtri che applicati ai due segnali forniranno due diverse uscite che vengono confrontate per decidere quale segnale è stato trasmesso.

Un filtro del ricevitore può essere adattato massimizzando l'uscita di tensione in corrispondenza del segnale che codifica per il simbolo da riconoscere. Per ricostruire la sequenza di bit in uscita si campiona l'uscita del filtro ad intervalli regolari con lo stesso ritmo della sorgente, e si determina il simbolo statisticamente più probabile con il decisore a soglia. La soglia è posta in modo tale da rendere equiprobabile l'errore di decisione.

\begin{figure}[ht!]\centering
\foreach\avg/\var in{1/0.707,1/0.5,2/.707} {
\subfloat[$\rho=\avg, \sigma^2=\var$]{
\begin{tikzpicture}
\begin{axis}[scale=.66,clip=false,xlabel=$V$,hide y axis,axis x line=middle,xtick={-\avg,\avg},ytick=\empty,xticklabels={$v_0$,$v_1$},extra x ticks={0.},extra x tick labels={soglia},extra x tick style={grid=major},samples=300,domain=-\avg-.5:\avg+.5]
\addplot [name path=gauss0] {gauss(x,-\avg,\var)};
\addplot [name path=gauss1] {gauss(x,\avg,\var)};
\path[name path=axis] (axis cs:-2.5,0)--node[above right,pos=0.5,pin=60:\footnotesize $P(1_R|0_T)$]{}node[above left,pos=0.5,pin=120:\footnotesize $P(0_R|1_T)$]{}(axis cs:2.5,0);
\addplot[pattern=north west lines] fill between[of=gauss0 and axis,soft clip={domain=0:2}];
\addplot[pattern=north east lines] fill between[of=gauss1 and axis,soft clip={domain=-2:0}];
\end{axis}
\end{tikzpicture}
}}
\caption{Modello statistico della decisione}
\label{fig:trasmissione_numerica_modello_statistico_decisione}
\end{figure}

Nei sistemi a trasmissione numerica la probabilità di commettere un errore su un simbolo è funzione del \ac{SNR} in ricezione:
\begin{equation}
p(\epsilon) = p(0_T)p(1_R|0_T)+p(1_T)p(0_R|1_T)
\end{equation}

Per ipotesi l'informazione contenuta nel segnale trasmesso avrà simboli 0 e 1 equiprobabili. Le probabilità condizionate corrispondono alle aree sottese alle code delle gaussiane suddivise dal valore di soglia.

Per ridurre la probabilità di errore si aumenta la distanza tra i picchi nella densità di probabilità in uscita ai filtri o si riduce la varianza dell'errore, utilizzando filtri a banda stretta.

Un filtro in ricezione a banda stretta limita l'energia passante del rumore ma anche la massima velocità di variazione del segnale, quindi un periodo minimo di tempo per ogni simbolo, ovvero il \keyword{bitrate}.

\section{Codifica ortogonale e antipodale}
Si possono scegliere le due forme d'onda $s_0(t)$ e $s_1(t)$ e i filtri adattati in modo tale che quando l'uscita di un filtro è massima l'altra sia nulla (\keyword{codifica ortogonale}). Ad esempio si può usare una forma d'onda per un simbolo e il valore costante nullo per l'altro.

In alternativa si può ottimizzare la potenza trasmessa facendo in modo che quando un'uscita è massima e positiva l'altra sia massima e negativa (\keyword{codifica antipodale}): ad esempio si può usare la stessa forma d'onda cambiata di segno per i due simboli. A parità di potenza di picco sarà immediato per il decisore determinare e confrontare, con la minore probabilità d'errore, il segno della tensione d'uscita ai filtri.

\begin{figure}[h!]
	\centering
	\begin{tikzpicture}[>=latex',thick];
	\coordinate(c0);
	\coordinate[below=1cm of c0](c2);
	\coordinate[below=1cm of c2](c1);
	\node [left=1cm of c2](n0){$s_R$};
	\draw (n0)--(c2)|-(c0)(c2)|-(c1);
	\node [block,right=1cm of c0](h0){$H_0(f)$} edge[<-](c0);
	\node [block,right=1cm of c1](h1){$H_1(f)$} edge[<-](c1);
	\coordinate[right=2cm of h0](c3);
	\coordinate[right=2cm of h1](c4);
	\node [block,minimum height=3cm,fit=(c3)(c4)](dec){$\lessgtr$};
	\coordinate[right=1cm of dec](u);
	\draw [->](h0)--(dec.126);
	\draw [->](h1)--(dec.235);
	\draw [->](dec)--node[above]{0}node[below]{1}(u);
	\end{tikzpicture}
	\caption{Schema di filtri adattati e decisore in ricezione per sistema di trasmissione numerica}
	\label{fig:trasmissione_numerica_filtri_adattati}
\end{figure}


\section{Pulse Amplitude Modulation}
Le forme d'onda che rappresentano i simboli binari di una trasmissione numerica possono essere segnali impulsivi modulati in ampiezza, o \ac{PAM}, scelti con codifica ortogonale o antipodale 
\[\begin{cases}0 \to n(t) \\ 1 \to s_1(t)+n(t)\end{cases} \quad \begin{cases}0 \to s_0(t)+n(t) \\ 1 \to s_1(t)+n(t)\end{cases} \]

Gli impulsi rettangolari sono la forma d'onda di base più semplice, con ampiezza costante di valore $s_0(t)=v_0$ o nullo per codificare uno “0” e valore $s_1(t)=v_1$ per codificare “1”.

Il segnale ricevuto affetto da rumore viene campionato ad intervalli regolari. Il ricevitore opera il campionamento utilizzando un segnale di sincronismo che indichi gli istanti in cui il segnale all'uscita del filtro adattato ha valore massimo. 

\`{E} possibile utilizzare un circuito integratore con azzeramento al campionatore per determinare il valore del campione alla fine di ogni periodo (tratteggiato in fig.\ref{fig:modulazione_numerica_PAM_ortogonale}c e \ref{fig:modulazione_numerica_PAM_antipodale}c).

\begin{figure}[ht!]\centering
\def\delay{.5}
\subfloat[Segnale trasmesso]{
	\begin{tikzpicture}[scale=.66]
	\begin{axis}[yscale=.5,xtick={},ytick={1},domain=0:7,ylabel=$s_T(t)$]
	\addplot[samples=125] {rect(x,1,2)+rect(x,3,5)};
	\end{axis}\end{tikzpicture}}\quad\subfloat[Segnale ricevuto]{
	\begin{tikzpicture}[scale=.66]
	\begin{axis}[yscale=.5,xtick={},ytick={1},domain=0:7,ylabel=$s_R^\text{A}(t)$]
	\addplot[samples=250] {noise(0,.1)+rect(x-\delay,1,2)+rect(x-\delay,3,5)};
	\end{axis}\end{tikzpicture}}%
\quad\subfloat[Segnale campionato]{
	\begin{tikzpicture}[scale=.66]
	\begin{axis}[yscale=.5,xtick={},ytick={1},domain=0:7,ylabel=$s_R^\text{C}(t)$]
	\addplot[samples=250] {noise(0,.1)+ramp(x-\delay,1,2)-ramp(x-\delay,2,3)+ramp(x-\delay,3,4)-ramp(x-\delay,5,6)};
	\addplot[only marks,mark=x,red] coordinates {(\delay,0)(\delay+1,0)(\delay+2,1)(\delay+3,0)(\delay+4,1)(\delay+5,1)(\delay+6,0)};
	\addplot[samples=125,dashed] {ramp(x-\delay,1,2)-grad(x-\delay,2)+ramp(x-\delay,3,4)-grad(x-\delay,4)+ramp(x-\delay,4,5)-grad(x-\delay,5)};
	\end{axis}\end{tikzpicture}}
\caption{Esempio di \acl{PAM} con codifica ortogonale}
\label{fig:modulazione_numerica_PAM_ortogonale}
\end{figure}

\begin{figure}[ht!]\centering
	\def\delay{.5}
	\subfloat[Segnale trasmesso]{
		\begin{tikzpicture}[scale=.66]
		\begin{axis}[yscale=.5,xtick={},ytick={1},domain=0:7,ylabel=$s_T(t)$]
		\addplot[samples=125] {-rect(x,0,1)+rect(x,1,2)-rect(x,2,3)+rect(x,3,5)};
		\end{axis}\end{tikzpicture}}\quad\subfloat[Segnale ricevuto]{
		\begin{tikzpicture}[scale=.66]
		\begin{axis}[yscale=.5,xtick={},ytick={1},domain=0:7,ylabel=$s_R^\text{A}(t)$]
		\addplot[samples=250] {noise(0,.1)-rect(x-\delay,0,1)+rect(x-\delay,1,2)-rect(x-\delay,2,3)+rect(x-\delay,3,5)-rect(x-\delay,5,6)};
		\end{axis}\end{tikzpicture}}\quad\subfloat[Segnale campionato]{
		\begin{tikzpicture}[scale=.66]
		\begin{axis}[yscale=.5,xtick={},ytick={1},domain=0:7,ylabel=$s_R^\text{C}(t)$]
		\addplot[samples=250] {noise(0,.1)-ramp(x-\delay,0,1)+grad(x-\delay,1)+ramp(x-\delay,1,2)-grad(x-\delay,2)-ramp(x-\delay,2,3)+grad(x-\delay,3)+ramp(x-\delay,3,4)-grad(x-\delay,4)+ramp(x-\delay,4,5)-grad(x-\delay,5)-ramp(x-\delay,5,6)+grad(x-\delay,6)};
		\addplot[only marks,mark=x,red] coordinates {(\delay,0)(\delay+1,-1)(\delay+2,1)(\delay+3,-1)(\delay+4,1)(\delay+5,1)(\delay+6,-1)};
		\addplot[samples=125,dashed] {-ramp(x-\delay,0,1)+grad(x-\delay,1)+ramp(x-\delay,1,2)-grad(x-\delay,2)+ramp(x-\delay,3,4)-grad(x-\delay,4)+ramp(x-\delay,4,5)-grad(x-\delay,5)};
		\end{axis}\end{tikzpicture}}
	\caption{Esempio di \acl{PAM} con codifica antipodale}
	\label{fig:modulazione_numerica_PAM_antipodale}
\end{figure}

Le sequenze di simboli numerici “0” e “1” possono essere associati a treni di impulsi rettangolari. 

Ogni simbolo binario viene codificato al trasmettitore con un impulso rettangolare $p_\text{O}(t)$ di durata $T$, che determina la frequenza di cifra $f_S=1/T$.

La sequenza di bit viene tradotta in un segnale dato dalla sovrapposizione di impulsi ritardati con l'informazione codificata nell'ampiezza, nei coefficienti $a_k$ pari ai livelli di tensione associati ai simboli:
\begin{equation}\begin{split} s_\text{O}(t)&=\sum_{k=-\infty}^{+\infty}a_k\,p_\text{O}(t-k T) \\
S_\text{O}(f)&=\sum_{k=-\infty}^{+\infty}{a_k\,P_\text{O}(f)\e{-\jmath 2\pi f k T}}
\end{split}\end{equation}

Al ricevitore si ha, sovrapposto al rumore gaussiano bianco, il segnale trasmesso modificato dalla funzione di trasferimento del canale e del filtro adattato. Nel punto B dello schema \ref{fig:schema_sistema_trasmissione_numerico_banda_base} a valle del filtro adattato si ha la convoluzione del segnale nel tempo, il prodotto in frequenza, con la risposta all'impulso del canale e del filtro in ricezione
\begin{equation}
\begin{split}
s_\text{B}(t)&=s_\text{O}(t)\ast q(t)\ast h(t) \\
S_\text{B}(f)&=S_\text{O}(f)\cdot Q(f)\cdot H(f)
\end{split}
\end{equation}

\begin{equation}
S_\text{B}(f)=\sum_{k=-\infty}^{+\infty}{a_k\, p_\text{O}(f)Q(f)H(f)\e{-\jmath 2\pi k T f}}=\sum_{k=-\infty}^{+\infty}{a_k\,p_\text{B}(f)\e{-\jmath 2\pi k T f}}
\end{equation}

Al campionatore con tempo di campionamento $T$ si ha che l'ennesimo campione del segnale ricevuto ha espressione
\begin{equation}
s_\text{B}(i T)=\sum_{k=-\infty}^{+\infty}{a_k\,p_\text{B}[(i-k)T]}=a_i\,p_\text{B}(i T)+\sum_{k\neq i}{a_k\,p_\text{B}[(i-k)T]}
\label{eq:interferenza_intersimbolica}
\end{equation}
in cui al valore del segnale in un istante $t=i T$ concorrono le forme d'onda relative a simboli successivi e precedenti, causando il fenomeno di \keyword{interferenza intersimbolica}.

\section{Interferenza intersimbolica}
Un sistema di trasmissione numerica deve utilizzare forme d'onda che non siano affette da \keyword{interferenza intersimbolica}\index{interferenza intersimbolica}, ovvero avere contributi indesiderati dalla trasmissione di simboli precedenti. 

Essendo impredicibile la combinazione di simboli trasmessi è necessario scegliere forme d'onda che all'uscita del filtro di ricezione assumano il massimo dell'energia trasmessa per il simbolo trasmesso e che si annullino negli istanti di campionamento relativi agli altri simboli, ovvero forme d'onda a zeri equidistanti. 

Una funzione a zeri equidistanti a distanza $T$ ha lo spettro che periodicizzato a passo $1/T$ risulta in una costante. Un esempio di tale funzione è il $\sinc{t/T}$ avente spettro rettangolare. Nel tempo è una forma d'onda di durata infinita, che decade a zero lentamente come $1/t$.

\begin{figure}[ht!]\centering
\subfloat[Funzione impulsiva $\sinc{t/T}$]{%
	\begin{tikzpicture}          
	\begin{axis}[xlabel=$t$,ylabel=$p_\text{B}(t)$,xtick={1,2,3,4},xticklabels={$T$,$2T$,$3T$,$4T$},samples=300]
	\addplot[black] {sinc(x,1)+sinc(x-1,1)-sinc(x-2,1)+sinc(x-3,1)};
	\addplot[gray,dashed] {sinc(x,1)};
	\addplot[gray,dashed] {sinc(x-1,1)};
	\addplot[gray,dashed] {-sinc(x-2,1)};
	\addplot[gray,dashed] {sinc(x-3,1)};	\end{axis}                   
	\end{tikzpicture}}%
\quad\subfloat[Errori di campionamento funzione $\sinc{\frac{i t+\Delta}{T}}$]{%
	\begin{tikzpicture}          
	\begin{axis}[xlabel=$t$,ylabel=$p_\text{B}(t)$,xtick={1,2,3,4},xticklabels={$T$,$2T$,$3T$,$4T$},samples=150]
	\addplot[black] {sinc(x,1)+sinc(x-1,1)-sinc(x-2,1)+sinc(x-3,1)};
	\addplot[blue,only marks,mark=*,samples at={0.0001,1.0001,2.0001,3.0001,4.0001}] {sinc(x,1)+sinc(x-1,1)-sinc(x-2,1)+sinc(x-3,1)};
	\addplot[red,only marks,mark=*,samples at={0.1,1.1,2.1,3.1,4.1}] {sinc(x,1)+sinc(x-1,1)-sinc(x-2,1)+sinc(x-3,1)};
	\end{axis}
	\label{fig:errori_campionamento_sinc}
	\end{tikzpicture}}%
\end{figure}

A causa di errori nel passo di campionamento, ad esempio un ritardato sincronismo (in rosso in fig.~\ref{fig:errori_campionamento_sinc}), è possibile dimostrare che un segnale impulsivo a zeri equidistanti $p_\text{B}(t)=\sinc{t/T}=\frac{\sen{\frac{\pi t}{T}}}{\frac{\pi t}{T}}$ sia affetto da problemi di interferenza intersimbolica, infatti
\[\begin{split}s_\text{B}(i t+\Delta)&=\sum_{k}{a_k\,p_\text{B}[(i-k)T+\Delta]}=\\
&=a_k\,p_\text{B}(\Delta)+\sum_{k\neq i}{a_k\,p_\text{B}[(i-k)T+\Delta]}=\\
&=a_k\,p_\text{B}(\Delta)+\sum_{k\neq i}{a_k\,\frac{\sen{\frac{\pi[(i-k)T+\Delta]}{T}}}{\frac{\pi[(i-k)T+\Delta]}{T}}}=\\
&=a_k\,p_\text{B}(\Delta)+\sum_{k\neq i}{a_k\,\frac{\sen{\pi[(i-k)+\Delta\cdot T]}}{\pi[(i-k)+\Delta\cdot T]}}\end{split}\]

Per ottenere la convergenza della serie nell'eq.~\ref{eq:interferenza_intersimbolica} è necessaria una funzione che decada nel tempo come $1/t^2$.

\section{Forme d'onda di Nyquist} 
Le \keyword{forme d'onda di Nyquist}\index{forme d'onda!di Nyquist} sono una famiglia di curve a zeri equidistanti con inviluppo che decade almeno come $1/t^2$. L'ampiezza nello spettro di frequenze varia gradualmente dal valore unitario al valore nullo, nella banda monolatera che si estende in proporzione al coefficiente di roll-off $\delta$ secondo la legge: 
\begin{equation}
B=\frac{1+\delta}{2T}=(1+\delta)\frac{f_s}{2}
\label{eq:banda_nyquist}
\end{equation}

\begin{figure}[ht!]\centering
\begin{tikzpicture}
\def\fs{10}
\pgfmathparse{\fs/2}\let\fzero=\pgfmathresult
\begin{axis}[xtick={\fzero,\fs},xticklabels={$\frac{f_S}{2}$,$f_S$},ytick={1},yscale=.75,samples=200]
\foreach\delt in{0.001,0.5,1.0} {
	\pgfmathparse{(1+\delt)*\fzero}\let\band=\pgfmathresult
	\pgfmathparse{\band-\fzero}\let\fdelta=\pgfmathresult
	\pgfmathparse{\fzero-\fdelta}\let\funo=\pgfmathresult
	\addplot[domain=0:\fs]{ (abs(x)<\funo)?1.0:((abs(x)>\band)?0.0:(.5*(1+cos( (pi*(abs(x)-\funo)/(2*\fdelta)))))) };
}
\node at(axis cs:.5*\fs,0.15)[pin=135:{$\delta=0$}]{};
\node at(axis cs:.4*\fs,0.68)[pin=45:{$\delta=0.5$}]{};	
\node at(axis cs:.7*\fs,0.15)[pin=45:{$\delta=1$}]{};
\end{axis}
\end{tikzpicture}
\caption{Spettro monolatero forme d'onda di Nyquist}
\label{fig:spettro_forme_donda_Nyquist}
\end{figure}

All'aumentare del roll-off aumenta la banda e aumenta la velocità di decadimento a zero della funzione impulsiva.
Per $\delta=0$ si ha lo spettro monolatero rettangolare con banda $B=f_s/2=1/2T$, per $\delta=1$ si ha lo \keyword[spettro!a coseno rialzato]{spettro a coseno rialzato} con banda $B=f_s=\frac{1}{T}$.

Conoscendo le funzioni di trasferimento del canale di trasmissione $Q(f)$ e del filtro adattato $H(f)$ è possibile scegliere il coefficiente $\delta$ per il migliore spettro di Nyquist a coseno rialzato, $p_\text{O}(f)$.

Dato lo spettro della forma d'onda in uscita al mezzo trasmissivo $p_\text{A}(f)$ si sceglie il filtro adattato con la funzione di trasferimento $H(f)=\conj{p}_\text{A}(f)$ tale che lo spettro della forma d'onda in uscita del filtro di ricezione adattato sia $p_\text{B}(f)=\abs{p_\text{A}}^2$. Tale forma d'onda ha modulo ad intersimbolo nullo, ma è ancora possibile determinare la fase del segnale in ricezione. Ad esempio si può scegliere la fase in modo che la forma d'onda relativa abbia potenza di picco minore per evitare fenomeni di distorsione per saturazione degli amplificatori. La potenza media del segnale, che è proporzionale al modulo del segnale, non cambia.

\section{Dimensionamento}
Una volta stabilite le prestazioni desiderate da un sistema di trasmissione numerica, la quantità di informazione trasmessa per unità di tempo, il \keyword{bitrate}, e la probabilità di errore $P(\epsilon)$ massima accettabile, è necessario determinare la potenza del segnale da trasmettere.

Nel dimensionamento del sistema di trasmissione analogica si considerano fissate la banda passante del segnale da trasmettere e il rapporto segnale rumore in ricezione. Nel sistema numerico in analogia alla banda si ha la frequenza di cifra ovvero la quantità di simboli binari trasferiti al secondo, mentre la probabilità di ricevere un simbolo diverso da quello trasmesso è imputabile unicamente al disturbo sovrapposto al segnale a causa del rumore termico degli apparati di trasmissione.

Aumentare il bitrate implica aumentare la banda passante del filtro in ricezione aumentando la rumorosità del sistema causano una maggiore probabilità di errare al decisore a soglia.

Bisogna valutare la potenza da trasmettere $P_T$ mantenendo la probabilità di errore $P(\epsilon)$ inferiore al massimo tollerabile.

La probabilità di commettere un errore su un simbolo, con simboli equiprobabili, è
\begin{equation}
p(\epsilon) = p(0_T)p(1_R|0_T)+p(1_T)p(0_R|1_T) = \frac{1}{2}[p(\epsilon|0_T)+p(\epsilon|1_T)]
\end{equation}
dove le probabilità condizionate corrispondono alle aree sottese alle code delle gaussiane suddivise dal valore di soglia in fig.~\ref{fig:trasmissione_numerica_modello_statistico_decisione}
\begin{equation}
\begin{split}
p(\epsilon|0_T)&=\intd{S}{+\infty}{f_{v_C|0_T}(v)}{v}=\intd{S}{+\infty}{\frac{1}{\sqrt{2\pi}\sigma_n}\,\e{-\frac{(v-v_0)^2}{2\sigma_n^2}}}{v} \\
p(\epsilon|1_T)&=\intd{-\infty}{S}{f_{v_C|1_T}(v)}{v}=\intd{-\infty}{S}{\frac{1}{\sqrt{2\pi}\sigma_n}\,\e{-\frac{(v-v_1)^2}{2\sigma_n^2}}}{v}
\end{split}
\end{equation}
con il cambio di variabili
\[y=\frac{v-v_0}{\sigma_n}\quad \diff y=\frac{\diff v}{\sigma_n}\qquad z=\frac{v-v_1}{\sigma_n}\quad \diff z=\frac{\diff v}{\sigma_n}\]
\begin{equation}
\begin{split}
p(\epsilon|0_T)&=\intd{\frac{S-v_0}{\sigma_n}}{+\infty}{\frac{1}{\sqrt{2\pi}}\,\e{-\frac{y^2}{2}}}{y}=\f{Q}{\frac{s-v_0}{\sigma_n}} \\
p(\epsilon|1_T)&=\intd{-\infty}{\frac{S-v_1}{\sigma_n}}{\frac{1}{\sqrt{2\pi}}\,\e{-\frac{z^2}{2}}}{z}=\intd{\frac{v_1-S}{\sigma_n}}{+\infty}{\frac{1}{\sqrt{2\pi}}\,\e{-\frac{z^2}{2}}}{z}=\f{Q}{\frac{v_1-s}{\sigma_n}}
\end{split}
\end{equation}

Ponendo la soglia $s=\frac{v_0+v_1}{2}$ pari al valor medio delle tensioni corrispondenti ai picchi del segnale all'uscita dei filtri adattati si ha che le due probabilità condizionate sono 
\begin{equation}p(\epsilon|0_\text{T})=p(\epsilon|1_\text{T})=\f{Q}{\frac{v_1-v_0}{2\sigma_n}}\end{equation}

Si ha pertanto che la probabilità di errore $p(\epsilon)=\f{Q}{\gamma}$ è funzione di $\gamma=\frac{v_1-v_0}{2\sigma_n}$, ovvero proporzionale alla distanza picco picco tra i segnali di uscita dei filtri adattati e inversamente proporzionale alla deviazione standard del rumore. La funzione $\f{Q}{x}$ è l'integrale rappresentativo dell'area della coda di una gaussiana di valor medio nullo e varianza unitaria.

La varianza del rumore considerato in $\gamma$ è quella del rumore campionato. Ricordando che la densità spettrale di un processo è la trasformata di Fourier della sua funzione di autocorrelazione, segue che il rumore in uscita dal filtro adattato ha funzione di autocorrelazione
\begin{equation}
R_n^\text{B}(\tau)=\E{n^\text{B}(t)\cdot n^\text{B}(t+\tau)}
\end{equation}
che è stazionario e dipende solo dalla differenza dei tempi $\tau$.
La funzione di autocorrelazione del processo rumore campionato è
\begin{equation}
R_n^\text{C}(k T)=\E{n^\text{C}(i T)\cdot n^\text{C}(i T+k T)}
\end{equation}
Essendo il processo stazionario si ha quindi 
\begin{equation}
R_n^\text{C}(k T)=R_n^\text{B}(k T)
\end{equation}
che la funzione di autocorrelazione dei campione di rumore all'uscita del campionatore è numericamente uguale al valore della funzione di autocorrelazione del rumore prima del campionamento, presa negli istanti di tempo multipli del periodo di campionamento. Pertanto per la stazionarietà del processo rumore la varianza del rumore in uscita del campionatore è 
\begin{equation}
\restrict{\sigma^2_n}{\text{C}}=R_n^\text{C}(0)=R_n^\text{B}(0)
\end{equation}
Per la ergodicità del processo si ha anche che la varianza è pari alla potenza media del rumore all'uscita del filtro di ricezione 
\begin{equation}
\restrict{\sigma^2_n}{\text{C}}=P_{n \text{medio}}^\text{B}
\end{equation}

Nell'ipotesi di \keyword[codifica!antipodale]{codifica antipodale} si ha a valle del campionatore l'uscita dei filtri adattati con la massima distanza tra le tensioni $v_0=-a$ e $v_1=+a$ con la soglia posta a $s=0$ e il valore $\gamma=a/\sigma$. A monte del campionatore alla tensione di picco $v_\text{picco}=\pm a$ corrisponde una potenza pari al quadrato della tensione (ipotizzando una impedenza unitaria).
\begin{equation}
\restrict{\frac{S}{N}}{B}=\gamma^2=\frac{a^2}{\sigma_n^2}=\frac{P_\text{S picco}^\text{B}}{P_\text{N media}^\text{B}}
\label{eq:rapporto_segnale_rumore_uscita_filtro_adattato}
\end{equation}

Il massimo rapporto segnale rumore all'uscita del filtro adattato è esprimibile in funzione dei parametri all'ingresso del filtro come rapporto tra l'energia della forma d'onda che rappresenta il simbolo ricevuto e la densità spettrale bilatera della potenza di rumore all'uscita del canale
\begin{equation}
\restrict{\frac{S}{N}}{B}=\frac{E_\text{R}}{h_n^\text{A}/2}=\frac{E_\text{R}/T}{h_n^\text{A}/2\cdot f_\text{S}}=\frac{P_\text{S media}^\text{A}}{h_n^\text{A} f_\text{S}/2}
\end{equation}

Si ottiene che il rapporto segnale rumore è al numeratore la potenza media del segnale ricevuto, a denominatore la potenza media del rumore all'ingresso misurata rapportata alla banda $f_\text{S}/2$.

Nell'ipotesi di \keyword[codifica!ortogonale]{codifica ortogonale} si ha a valle del campionatore l'uscita dei filtri adattati con la distanza tra le tensioni $v_0=0$ e $v_1=+a$ con la soglia posta a $s=a/2$ e il valore $\gamma=a/2\sigma$. 
\begin{equation}
\restrict{\frac{S}{N}}{B}=\gamma^2=\frac{a^2}{4\sigma_n^2}=\frac{P_\text{S picco}^\text{B}}{4 P_\text{N media}^\text{B}}=\frac{E_\text{R}}{4 h_n^\text{A}/2}=\frac{E_\text{R}/T}{2 h_n^\text{A}\cdot f_\text{S}}=\frac{P_\text{S media}^\text{A}}{h_n^\text{A} f_\text{S}}
\end{equation}
dove $E_R$ è l'energia ricevuta quando c'è la forma d'onda del simbolo “1”, pertanto $E_R/T$ rappresenta il doppio della potenza media del segnale ricevuto.

Dal confronto tra i rapporti segnale rumore della codifica ortogonale e codifica antipodale si evince che per ottenere da un sistema con condifica ortogonale la stessa probabilità di errore che si ottiene con codifica ortogonale, è necessario trasmettere una potenza media doppia, ovvero $\SI{3}{\decibel}$ superiore.

\begin{esercizio}
	Si voglia \textsc{progettare un sistema di trasmissione numerica digitale multitratta} con una velocità $\SI{100}{\mega\bit\per\second}$ con una probabilità di errore non superiore a $p(\epsilon)=\num{e-7}$ su un cavo di lunghezza $L=\SI{100}{\kilo\meter}$ che attenuazione specifica $\alpha_\text{sp}=\SI{1}{\decibel\per\kilo\meter}$ alla frequenza $f_\text{sp}=\SI{1}{\mega\hertz}$, con equalizzazione passiva e codifica antipodale.
\end{esercizio}

\`{E} necessario calcolare il rapporto segnale rumore all'uscita del campionatore che garantisca $p(\epsilon)=\num{e-7}$. Ipotizzando un rumore gaussiano bianco sovrapposto al segnale e una soglia nel valor medio delle tensioni all'uscita del filtro adattato si ha $p(\epsilon)=\f{Q}{\gamma}$. Ricordando mnemonicamente che per $p(\epsilon)=\num{e-6}$ si ha un valore di $\gamma=\SI{13.5}{\decibel}$ e che per diminuire di un fattore 10 la probabilità di errore è necessario aumentare $\gamma$ di un $\SI{1}{\decibel}$ si ha che per $p(\epsilon)=\num{e-7}$ si ottiene $\gamma=\SI{14.5}{\decibel}$.

Nell'ipotesi di codifica antipodale il rapporto all'uscita del filtro di ricezione della potenza di picco del segnale e la potenza media di rumore
\[\gamma^2=\frac{P_\text{S picco}^\text{B}}{P_\text{N media}^\text{B}}=\SI{14.5}{\decibel}\]
Nell'ipotesi di utilizzare un filtro adattato, sempre in codifica antipodale, si ha che il rapporto segnale rumore risulta pari a 
\[\frac{P_\text{S media}^\text{A}}{\frac{h_n^\text{A}}{2} f_\text{S}}=\SI{14.5}{\decibel}\]

Nota la rumorosità delle apparecchiature riceventi, per ipotesi un fattore di rumore $F=\SI{10}{\decibel}$, si può valutare la minima potenza media necessaria in ricezione per rispettare i requisiti di progetto (espressa in $\si{\dBm}$):
\[P_\text{S media}^\text{A}=\restrict{\gamma^2}{\si{\decibel}}+\restrict{h_n}{\si{\decibel}}+\restrict{F}{\si{\decibel}}+\restrict{\frac{f_s}{2}}{\si{\decibel}}=(\num{14.5}-174+10+77)\si{\dBm}=\SI{-72.5}{\dBm}\]

Nell'ipotesi di equalizzazione passiva si calcola l'attenuazione totale $\alpha_\text{TOT}$ determinando l'attenuazione specifica alla frequenza massima del segnale, che per forma d'onda di Nyquist con coefficiente di roll-off $\delta=1$ è $f_\text{max}=\frac{1}{T}=\SI{100}{\mega\hertz}$:
\[\alpha_{f_\text{max}}=\alpha_\text{rif}\sqrt{\frac{f_\text{max}}{f_\text{rif}}}=\SI{1}{\decibel\per\kilo\meter}\cdot\sqrt{\frac{\num{100e6}}{\num{e6}}}=\SI{10}{\decibel\per\kilo\meter}\to \alpha_\text{TOT}=L\cdot\alpha_{f_\text{max}}=\SI{1000}{\decibel}\]

Tale attenuazione è eccessiva per una trasmissione su una unica tratta. Si deve ricorrere ad un sistema multitratta analogico, dividendo la lunghezza del cavo in più sezioni e ritrasmettendo il segnale con apparecchiature di amplificazione.

Ipotizzando che ogni trasmettitore abbia una potenza massima di trasmissione pari a $P_T=\SI{100}{\milli\watt}$, che le tratte siano tutte uguali e che all'uscita dell'ultima tratta si voglia ottenere un rapporto segnale rumore $\restrict{S/N}{u}=\SI{14.5}{\decibel}$ si calcoli il numero di tratte.

L'attenuazione sulla singola tratta sarà $\alpha_\text{tot}/n$, mentre il rumore risulta moltiplicato per $n$: il rumore della prima tratta viene amplificato/attenuato e iniettato nella seconda tratta, e così via.

Sempre in ipotesi di codifica antipodale si ha che il rapporto segnale rumore è pari a 
\[\gamma^2=\frac{P_\text{S medio}^\text{B}}{P_\text{N}^\text{B}}=\frac{P_T\cdot\alpha_\text{tot}/n}{n\cdot\frac{h_n}{2}f_s}\]

\[\restrict{\gamma^2}{\si{\decibel}}\leq \restrict{P_T}{\si{\dBm}} - \restrict{\frac{\alpha_\text{tot}}{n}}{\si{\decibel}}-10\Log{n}-\restrict{F\frac{h_n f_s}{2}}{\si{\dBm}}\]

\[ \SI{14.5}{\decibel}\leq\SI{20}{\dBm} - \frac{\SI{1000}{\decibel}}{n}-10\Log{n}-(-\SI{174}{\dBm}+\SI{10}{\decibel}-\SI{3}{\decibel}+\SI{80}{\decibel})\]

Espressione che ha soluzione per $n\geq 13$.

\begin{figure}[ht!]\centering
\subfloat[Multitratta analogico]{
	\begin{tikzpicture}[scale=.8]
	\begin{axis}[yscale=.6,xlabel=$n$,ylabel=$\alpha_\text{tratta}$,samples at={1,...,15},extra x ticks={13},
	extra x tick style={grid=major}]
	\addplot[const plot]{14.5+10*log10(x)-20+1000/x-87};
	\end{axis}
	\end{tikzpicture}}\quad%
\subfloat[Multitratta digitale]{
	\begin{tikzpicture}[scale=.8]
	\begin{axis}[yscale=.6,xlabel=$n$,ylabel=$\alpha_\text{tratta}$,samples at={1,...,15},extra x ticks={11},
	extra x tick style={grid=major}]
	\addplot[const plot]{14.5+log10(x)-20+1000/x-87};
	\end{axis}
	\end{tikzpicture}}
\end{figure}

In alternativa al sistema multitratta analogico che ha il difetto di moltiplicare $n$ volte il rumore in uscita è possibile adottare il multitratta digitale in cui si rigenera il segnale numerico alla fine di ogni tratta. Ogni tratta è un sistema di trasmissione numerico a sé stante, non trasferisce il rumore da una tratta all'altra, è caratterizzato dalla propria probabilità di errore statisticamente indipendente dalle altre tratte, pertanto il numero medio di bit errati sarà la somma dei bit errati su ogni tratta. Si ha così che la probabilità di errore complessiva è pari alla somma delle probabilità di errore sulle singole tratte
\begin{equation}
P_\text{tot}(\epsilon)=\sum_{i}p_i(\epsilon)
\end{equation}

La probabilità di errore finale, con tratte uguali, è
\begin{equation}
p(\epsilon)=1-\prod_{i=1}^{n}[1-p_i(\epsilon)]=1-[1-p_\text{ST}(\epsilon)]^n\cong 1-[1-n\cdot p_\text{ST}(\epsilon)]\cong n\cdot p_\text{ST}(\epsilon)
\end{equation}

Se il sistema di trasmissione numerico deve garantire una probabilità di errore $p(\epsilon)$, ad esempio di $p(\epsilon)\leq\num{e-7}$, il dimensionamento di un sistema multitratta digitale richiede che ogni singola tratta abbia una probabilità di errore $n$ volte più piccola:
\begin{equation}p_\text{ST}(\epsilon)\leq\frac{p(\epsilon)}{n}\end{equation}

Nel caso di studio in esame il dimensionamento richiede che il rapporto segnale rumore sia superiore a $\restrict{\gamma^2}{\si{\decibel}}=\SI{14.5}{\decibel}+\Log{N}$. La correzione si ha perché se ad ogni $p(\epsilon)/10$ corrisponde un aumento del rapporto segnale rumore di $\SI{1}{\decibel}=10\Log{10}$, ad un errore per tratta $p(\epsilon)/n$ corrisponde un aumento di $\Log{n}\si{\decibel}$.

\[P_T-\frac{\alpha_\text{tot}}{n}-F\frac{h_n f_s}{2}\geq \num{14.5}+\Log{n}\]

\section{Estrazione del timing}
Il campionatore deve effettuare il campionamento del segnale all'uscita del filtro di ricezione quando questo raggiunge il suo valore massimo per massimizzare il rapporto segnale rumore in B pari al valore $\gamma^2$ (eq.~\ref{eq:rapporto_segnale_rumore_uscita_filtro_adattato}). Non è possibile utilizzare un oscillatore che comandi il campionatore e che oscilli esattamente alla frequenza desiderata pari a quella del trasmettitore, anche piccoli errori nel periodo sfaserebbero i circuiti, inoltre è impossibile conoscere la fase corretta a causa del ritardo di propagazione lungo il mezzo trasmissivo. L'unico modo per sincronizzare il campionatore è quello di ricavare il timing estraendo l'informazione dalla fase e la frequenza dello stesso segnale ricevuto.
\begin{figure}[!ht]
	\centering
	\resizebox{\textwidth}{!}{
		\begin{tikzpicture}[>=latex']
		\coordinate(c0);
		\node[block,right=1.5cm of c0](b0){GFO};
		\node[block,right=1.5cm of b0](b1){MT};
		\node[sum,right=1cm of b1](s0){$+$};
		\coordinate[above=1cm of s0](n);
		\coordinate[right=1cm of s0](n1);
		\node[block,right=1cm of n1](b2){FR};
		\node[block,below=1cm of b2](b4){TIMING};
		\node[campionatore,right=1cm of b2](q0){};
		\node[block,right=1cm of q0](b3){DEC};
		\coordinate[right=1.5cm of b3](c1);
		\draw[->](c0)--node[above,near start]{bit}(b0);
		\draw[dot=O](b0)--(b1);
		\draw[->](b1)--(s0);
		\draw[->](n)node[above]{$n(t)$}--(s0);
		\draw[->](n1)|-(b4);
		\draw[->](b4)-|(q0);
		\draw[->](s0)--(b2);
		\draw[dot=B](b2)--(q0);
		\draw[dot=C](q0)--(b3);
		\draw[->](b3)--node[above,near end]{bit}(c1);
		\node[fitted, fit=(b0),label=above:Trasmettitore]{};
		\node[fitted, fit=(b1)(s0),label=above:Canale]{};
		\node[fitted, fit=(b2)(q0)(b3)(b4),label=above:Ricevitore]{};
		\end{tikzpicture}
	}
	\caption{Schema sistema di trasmissione numerico con estrattore del timing}
	\label{fig:schema_sistema_trasmissione_numerico_timing}
\end{figure}

Nell'ipotesi di utilizzare forme d'onda rettangolari di periodo di cifra $T$, una sequenza di $a\rect{t/T}$ ritardati, non è possibile utilizzare un filtro passa banda molto stretta attorno alla frequenza di cifra $f_S=1/T$, essendo nullo il contributo spettrale del $a T\sinc{f T}$ in corrispondenza di tale frequenza e sue armoniche, alla quale si trova solo rumore.

Si studia come poter alterare le sequenze di impulsi rettangolari per codificare nel segnale trasmesso l'informazione di temporizzazione con i \keyword[codice di linea]{codici di linea}.

\section{Codici di linea}
\subsection{Codice \ac{NRZ}}
Una sequenza di impulsi rettangolari che mantengono costante il proprio valore nel periodo $T$. In caso di simboli ripetuti il segnale \keyword[codifica!NRZ]{non ritorna a zero} mantenendo il livello alto. Il contenuto spettrale alla frequenza $1/T$ è praticamente nullo.
L'informazione sulla temporizzazione è contenuta nei fronti di salita e discesa del segnale: per recuperare il timing è necessario derivare il segnale, raddrizzarlo e filtrare passa banda stretta sulla frequenza di cifra $1/T$ per estrarre gli impulsi di pilotaggio per il campionatore. Si ottiene un treno di impulsi quasi periodico: una lunga sequenza di “0” o “1” fa perdere il sincronismo.

\begin{figure}[ht!]
\centering
\subfloat[Codice \ac{NRZ} ortogonale]{
\begin{tikzpicture}
	\begin{axis}[xtick=\empty,ytick=\empty,yscale=.25,every extra x tick/.append style={major tick length=0pt},extra x ticks={.5,1.5,2.5,3.5,4.5,5.5},extra x tick labels={1,0,0,1,1,0},ymin=-1,ymax=1]
	\addplot [const plot,thick,draw=black]coordinates {(0,1)(1,0)(2,0)(3,1)(5,0)(6,0)};
	\end{axis}
\end{tikzpicture}}%
\subfloat[Impulsi di temporizzazione \ac{NRZ} ortogonale]{
	\begin{tikzpicture}
	\begin{axis}[xtick=\empty,ytick=\empty,yscale=.25]
	\addplot [quiver={u=0,v=1},-latex']coordinates {(0,0)(1,0)(3,0)(5,0)};
	\addplot [quiver={u=0,v=-1},-latex',dotted,gray]coordinates {(1,0)(5,0)};
	\end{axis}
	\end{tikzpicture}}

\subfloat[Codice \ac{NRZ} antipodale]{
	\begin{tikzpicture}
	\begin{axis}[xtick=\empty,ytick=\empty,yscale=.25,every extra x tick/.append style={major tick length=0pt},extra x ticks={.5,1.5,2.5,3.5,4.5,5.5},extra x tick labels={1,0,0,1,1,0},ymin=-1,ymax=1]
	\addplot [const plot,thick,draw=black]coordinates {(0,1)(1,-1)(2,-1)(3,1)(5,-1)(6,-1)};
	\end{axis}
	\end{tikzpicture}}%
\subfloat[Impulsi di temporizzazione \ac{NRZ} antipodale]{
	\begin{tikzpicture}
	\begin{axis}[xtick=\empty,ytick=\empty,yscale=.25]
	\addplot [quiver={u=0,v=1},-latex']coordinates {(0,0)(1,0)(3,0)(5,0)};
	\addplot [quiver={u=0,v=-1},-latex',dotted,gray]coordinates {(1,0)(5,0)};
	\end{axis}
	\end{tikzpicture}}%
\end{figure}

\subsection{Codice \ac{RZ}}
Una sequenza di impulsi rettangolari con codifica \ac{RZ} a valore costante di durata $\tau<T$. A parità di potenza di picco la potenza media del segnale con \keyword[codifica!RZ]{ritorno a zero} è inferiore a quella senza ritorno a zero (e rispetto al rumore). Lo spettro della forma d'onda è dato dalla sovrapposizione degli impulsi $a\tau\frac{sin{\pi f\tau}}{\pi f\tau}\e{-\jmath 2\pi f k\tau}$ che danno un contributo non nullo alla frequenza $1/T$, pari a $a\tau\frac{sin{\pi\tau/T}}{\pi\tau/T}$, consentendo l'estrazione del timing con un filtro molto stretto accordato alla frequenza $1/T$.

Utilizzando una codifica antipodale associando valori di tensione opposti alle forme d'onda che codificano i simboli si può ottenere dal circuito derivatore e raddrizzatore un maggior numero di impulsi di pilotaggio per il campionatore per una migliore temporizzazione.

\begin{figure}[ht!]
	\centering
	\subfloat[Codifica \ac{RZ} ortogonale]{
		\begin{tikzpicture}
		\begin{axis}[ybar,bar shift=.5,bar width=.5,xtick=\empty,ytick=\empty,yscale=.25,every extra x tick/.append style={major tick length=0pt},extra x ticks={0.5,1.5,2.5,3.5,4.5,5.5},extra x tick labels={1,0,0,1,1,0},ymin=-1,ymax=1]
		\addplot [thick,draw=black]coordinates {(0,1)(1,0)(2,0)(3,1)(4,1)(5,0)(6,0)};
		\end{axis}
		\end{tikzpicture}}
	\subfloat[Impulsi di temporizzazione \ac{RZ} ortogonale]{
		\begin{tikzpicture}
		\begin{axis}[xtick=\empty,ytick=\empty,yscale=.25,ymin=-1,ymax=1]
		\addplot [quiver={u=0,v=1},-latex']coordinates {(0.25,0)(.75,0)(3.25,0)(3.75,0)(5.25,0)(5.75,0)};
		\addplot [quiver={u=0,v=-1},-latex',dotted,gray]coordinates {(.75,0)(3.75,0)(5.75,0)};
		\end{axis}
		\end{tikzpicture}}
	
	\subfloat[Codifica \ac{RZ} antipodale]{
		\begin{tikzpicture}
		\begin{axis}[ybar,bar shift=.5,bar width=.5,xtick=\empty,ytick=\empty,yscale=.25,every extra x tick/.append style={major tick length=0pt},extra x ticks={0.5,1.5,2.5,3.5,4.5,5.5},extra x tick labels={1,0,0,1,1,0},ymin=-1,ymax=1]
		\addplot [thick,draw=black]coordinates {(0,1)(1,-1)(2,-1)(3,1)(4,1)(5,-1)};
		\end{axis}
		\end{tikzpicture}}
	\subfloat[Impulsi di temporizzazione \ac{RZ} antipodale]{
		\begin{tikzpicture}
		\begin{axis}[xtick=\empty,ytick=\empty,yscale=.25,ymin=-1,ymax=1]
		\addplot [quiver={u=0,v=1},-latex']coordinates {(0.25,0)(.75,0)(1.25,0)(1.75,0)(2.25,0)(2.75,0)(3.25,0)(3.75,0)(4.25,0)(4.75,0)(5.25,0)(5.75,0)};
		\addplot [quiver={u=0,v=-1},-latex',dotted,gray]coordinates {(.75,0)(1.25,0)(2.25,0)(3.75,0)(4.75,0)(5.25,0)};
		\end{axis}
		\end{tikzpicture}}%
\end{figure}

\subsection{Codice \ac{AMI}}
Una alternativa alla codifica con ritorno a zero che causa una perdita di energia del segnale per ottenere una migliore estrazione di segnale di temporizzazione è l'utilizzo del codice di linea \ac{AMI} di tipo non ritorno a zero con codifica ortogonale con i simboli “1” realizzati alternativamente con impulsi di tensione $\pm 1$. La sequenza ripetuta di simboli “1” darà luogo a discontinuità che saranno rilevate dal derivatore, incrementando il numero di impulsi al raddrizzatore. Lo spettro di potenza ha valor medio nullo e il segnale non ha informazione in componente continua consentendo l'utilizzo della codifica per apparati alimentati a distanza.

\begin{figure}[ht!]\centering
\subfloat[Codifica \ac{AMI}]{
	\begin{tikzpicture}
		\begin{axis}[xtick=\empty,ytick=\empty,yscale=.25,every extra x tick/.append style={major tick length=0pt},extra x ticks={.5,1.5,2.5,3.5,4.5,5.5},extra x tick labels={1,0,0,1,1,0},ymin=-1,ymax=1]
		\addplot [const plot,thick,draw=black]coordinates {(0,1)(1,0)(2,0)(3,-1)(4,1)(5,0)(6,0)};
		\end{axis}
	\end{tikzpicture}}
\subfloat[Impulsi di temporizzazione \ac{AMI}]{
	\begin{tikzpicture}
	\begin{axis}[xtick=\empty,ytick=\empty,yscale=.25,ymin=-1,ymax=1]
	\addplot [quiver={u=0,v=1},-latex']coordinates {(0,0)(1,0)(3,0)(4,0)(5,0)};
	\addplot [quiver={u=0,v=-1},-latex',dotted,gray]coordinates {(1,0)(3,0)(5,0)};
	\end{axis}
	\end{tikzpicture}}%
\end{figure}

\subsection{Codice \ac{HDB-3}}
Per risolvere l'analogo problema relativo alla trasmissione di più simboli “0” consecutivi si introduce il codice \ac{HDB-3} come evoluzione del codice \ac{AMI}: per evitare che non arrivino impulsi al circuito di estrazione del timing per più di 3 periodi di cifra in corrispondenza di un quarto zero consecutivo si invia un impulso rettangolare con la stessa tensione dell'“1” precedente. Violando l'alternanza del codice \ac{AMI} il simbolo sarà interpretato non come un “1” ma come uno “0”. 

\begin{figure}[ht!]\centering
	\subfloat[Codifica \ac{HDB-3}]{
		\begin{tikzpicture}
		\begin{axis}[xtick=\empty,ytick=\empty,yscale=.25,every extra x tick/.append style={major tick length=0pt},extra x ticks={.5,1.5,2.5,3.5,4.5,5.5},extra x tick labels={1,0,0,0,0,0},ymin=-1,ymax=1]
		\addplot [const plot,thick,draw=black]coordinates {(0,1)(1,0)(2,0)(3,0)(4,1)(5,0)};
		\end{axis}
		\end{tikzpicture}}
	\subfloat[Impulsi di temporizzazione \ac{HDB-3}]{
		\begin{tikzpicture}
		\begin{axis}[xtick=\empty,ytick=\empty,yscale=.25,ymin=-1,ymax=1]
		\addplot [quiver={u=0,v=1},-latex']coordinates {(0,0)(1,0)(4,0)(5,0)};
		\addplot [quiver={u=0,v=-1},-latex',dotted,gray]coordinates {(1,0)(5,0)};
		\end{axis}
		\end{tikzpicture}}%
\end{figure}

\subsection{Codice Manchester}
Il \keyword[codice di linea!Manchester]{codice Manchester} garantisce per entrambe i simboli la transizione in salita o discesa a metà periodo: questo garantisce la presenza dell'impulso di sincronizzazione per la migliore temporizzazione pur dimezzando la potenza media del segnale.


\begin{figure}[ht!]
	\centering
	\subfloat[Codice Manchester]{
		\begin{tikzpicture}
		\begin{axis}[xticklabels={},xmajorgrids,ytick=\empty,yscale=.25,ymin=-1,ymax=1]
		\addplot [const plot,thick,draw=black]coordinates {(0,1)(.5,0)(1,0)(1.5,1)(2,0)(2.5,1)(3,1)(3.5,0)(4,1)(4.5,0)(5,0)(5.5,1)(6,1)};
%		\foreach\bit[count=\ii]in{1,0,0,1,1,0}{\node at(axis cs:\ii,0){\bit};};
		\end{axis}
		\end{tikzpicture}}
	\subfloat[Impulsi di temporizzazione con codice Manchester]{
		\begin{tikzpicture}
		\begin{axis}[ytick=\empty,yscale=.25,ymin=-1,ymax=1]
		\addplot [quiver={u=0,v=1},-latex']coordinates {(0.5,0)(1.5,0)(2.5,0)(3.5,0)(4.5,0)(5.5,0)};
		\end{axis}
		\end{tikzpicture}}
\end{figure}

\clearpage
\section{Rinuncia al filtraggio adattato}\label{cap:rinuncia_filtro_adattato}
Il dimensionamento di un sistema di trasmissione numerico deve massimizzare il rapporto segnale rumore al campionatore a parità di potenza media ricevuta
\[\restrict{\frac{S}{N}}{C}=\frac{E_R}{h_n/2}\]

La progettazione di un sistema di trasmissione numerica basato su forme d'onda di Nyquist e filtro adattato massimizza il rapporto segnale rumore al campionatore perché si campiona il segnale nel suo picco ottenuto con un integratore ad azzeramento. La potenza di picco è proporzionale al quadrato dell'ampiezza massima del segnale campionato. Per la necessità di avere forme d'onda ad intersimbolo nullo all'uscita del filtro si scelgono segnali della famiglia di Nyquist, e il modulo dello spettro del segnale è legato al coefficiente di roll-off. La fase viene scelta in modo tale da ovviare alle limitazioni sulla potenza di picco dovute ai limiti di saturazione degli amplificatori. 
Un sistema di trasmissione basato su filtro adattato è fortemente vulnerabile ad ogni distorsione delle forme d'onda produce una riduzione del rapporto segnale rumore $\restrict{S/N}{C}$.

\begin{figure}[!ht]
	\centering
		\begin{tikzpicture}[>=latex']
		\coordinate(start);
		\node[block,right=1.5cm of start](ht){$H_T(f)$};
		\draw[->](start)--node[below,near start]{$S_T^\text{O}(f)$}(ht);
		\node[sum,right=1cm of ht](s0){$+$};
		\draw[dot=A](ht)--(s0);
		\coordinate[above=1cm of s0](n);
		\draw[->](n)node[above]{$n(t)$}--(s0);
		\node[block,right=1cm of s0](hr){$H_R(f)$};
		\draw[->](s0)--(hr);
		\node[campionatore,right=1cm of hr](q0){};
		\coordinate[right=1.5cm of q0](end);
		\draw[dot=B](hr)--(q0);
		\draw[dot=C](q0)--(end);
		\node[fitted, fit=(ht)(s0),label=above:Canale]{};
		\node[fitted, fit=(hr)(q0),label=above:Ricevitore]{};
		\end{tikzpicture}
	\caption{Schema filtro di ricezione}
	\label{fig:schema_filtro_ricezione}
\end{figure}

Per superare i limiti di tale sistema di trasmissione si rinuncia al filtro adattato che vincola la potenza media trasmessa e si ottimizzano le prestazioni utilizzando forme d'onda rettangolari in trasmissione che presentano la massima efficienza avendo la potenza media pari alla potenza di picco. 

Impulsi rettangolari di ampiezza $a$ e durata $\tau\leq T$ (con $\tau<T$ si ha codifica con ritorno a zero per facilitare il sincronismo) hanno espressione nel tempo e in frequenza, al trasmettitore:
\begin{equation}
\begin{split}
s_T^\text{O}(t)&=a\rect{\frac{t}{\tau}}\\
S_T^\text{O}(f)&=a\tau\frac{\sen{\pi f\tau}}{\pi f\tau}
\end{split}
\end{equation}

Si valuta il rapporto segnale rumore al campionatore del segnale ricevuto 
\begin{equation}
\restrict{\frac{S}{N}}{C}=\frac{P_\text{S picco}}{P_\text{N media}}
\end{equation}
e si confronta il risultato con quello ottenuto con il filtro adattato.

Date le funzioni di trasferimento del canale $H_T(f)$, del filtro generico $H_R(f)$, valutando la potenza del segnale nell'istante $t_m$ di picco, indicando con $h_n$ la densità spettrale monolatera di rumore bianco, valutando la potenza media di rumore a valle del filtro, si ha
\begin{equation}
\begin{split}
\restrict{\frac{S}{N}}{C}&=\frac{\abs{\intinf{ S_T(f) H_T(f)H_R(f)\e{\jmath 2\pi f t_m}}{f}}^2}{\intinf{\frac{h_n}{2}\abs{H_R(f)}^2}{f}}=\\
&=\frac{\abs{\intinf{ a\tau \frac{\sen{\pi f\tau}}{\pi f\tau} H_T(f)H_R(f)\e{\jmath 2\pi f t_m}}{f}}^2}{\frac{h_n}{2}\intinf{\abs{H_R(f)}^2}{f}}=\\
&=\frac{a^2}{\frac{h_n}{2}f_s}\underbrace{\frac{\tau^2}{T}\frac{\abs{\intinf{\frac{\sen{\pi f\tau}}{\pi f\tau} H_T(f)H_R(f)\e{\jmath 2\pi f t_m}}{f}}^2}{\intinf{\abs{H_R(f)}^2}{f}}}_{\text{rapporto adimensionale}}=\\
&=\frac{a^2}{\frac{h_n}{2}f_s}\e{-2\alpha_e}
\end{split}
\end{equation}

Si ottiene che il rapporto segnale rumore al campionatore è proporzionale alla potenza di picco del segnale trasmesso, dove essendo $a^2$ il quadrato dell'ampiezza del rettangolo trasmesso, affetto da una attenuazione esponenziale equivalente pari al rapporto adimensionale tra gli integrali. Tale rapporto dipende dalla forma d'onda trasmessa, dato dalla durata del rettangolo $\tau$ rispetto al periodo di cifra $T$, dal roll-off se tale forma d'onda appartiene alla famiglia di Nyquist, e dalle caratteristiche del mezzo trasmissivo.

Per confrontare il risultato con il rapporto segnale rumore ottenuto con filtro adattato che è pari a \[\restrict{\frac{S}{N}}{C}=\frac{P_\text{R media}}{\frac{h_n}{2}f_s}\]
è necessario valutare l'espressione dell'attenuazione equivalente essendo, nel caso di codifica \ac{NRZ}, con $\tau=T$ la potenza di picco pari alla potenza media trasmessa. 

Si deve progettare il filtro ipotizzando di dover equalizzare gli effetti di attenuazione di un mezzo di trasmissione ideale, con funzione di trasferimento $H_T(f)=\e{-\alpha_T}$, con $\alpha_T$ costante, e imponendo che l'impulso rettangolare in ricezione all'uscita del filtro abbia una forma d'onda di Nyquist per ottenere l'intersimbolo nullo:
\begin{equation}
a\tau S_R^\text{B}(f)=a\tau\frac{\sen{\pi f\tau}}{\pi f\tau} H_T(f) H_R(f)
\end{equation}
da cui si ricava, potendo invertire lo spettro della forma d'onda di Nyquist che non si annulla nelle frequenze $f\in[0,f_s]$,
\begin{equation}
H_R(f)=S_R^\text{B}(f) H_T^{-1}(f) \frac{\pi f\tau}{\sen{\pi f\tau}}
\end{equation}
Sostituendo tale risultato nel rapporto adimensionale tra integrali si ottiene che il rapporto segnale rumore ha espressione:
\begin{equation}
\begin{split}
\restrict{\frac{S}{N}}{C}&=\frac{a^2}{\frac{h_n}{2}f_s} \frac{\tau^2}{T}\frac{\abs{\intinf{S_R^\text{B}(f)\e{\jmath 2\pi f t_m}}{f}}^2}{\intinf{\abs{S_R^\text{B}(f) \e{\alpha_T}\frac{\pi f\tau}{\sen{\pi f\tau}}}^2}{f}}=\\
&=\frac{a^2}{\frac{h_n}{2}f_s}\e{-2\alpha_T}\e{-2\alpha_F}
\end{split}
\end{equation}
Avendo espresso il fattore di attenuazione del mezzo trasmissivo ideale nel restante fattore adimensionale confluisce l'attenuazione $\e{\alpha_F}$ dovuta al \keyword[fattore di forma]{fattore di forma} che dipende dalle forme d'onda trasmesse. A numeratore si ha quindi la potenza di picco ricevuta:
\begin{equation}
\restrict{\frac{S}{N}}{C}=\frac{P_\text{R picco}^\text{A}}{\frac{h_n}{2}f_s}\e{-2\alpha_F}
\end{equation}
Per codifiche di non ritorno a zero con $\tau=T$ si la potenza di picco in ricezione coincide con la potenza media ricevuta $P_\text{R picco}^\text{A}=P_\text{R media}^\text{A}$ pertanto
\begin{equation}
\restrict{\frac{S}{N}}{C}=\frac{P_\text{R media}^\text{A}}{\frac{h_n}{2}f_s}\e{-2\alpha_F}
\end{equation}
La formula con il filtro adattato ha rapporto segnale rumore 
\begin{equation}
\restrict{\frac{S}{N}}{C}=\frac{P_\text{R media}^\text{A}}{\frac{h_n}{2}f_s}
\end{equation}

In un bilancio di potenze l'entità dell'attenuazione data dal fattore di forma vale circa $\SI{0.5}{\decibel}$ ma il guadagno in termini di efficienza dell'amplificatore in trasmissione dell'ordine anche di $\SI{10}{\decibel}$ di potenza trasmessa (trasmettendo rettangoli si possono mandare gli amplificatori in saturazione).

Tale risultato continua a valere anche nel caso di mezzo trasmissivo non ideale: è necessario che l'attenuazione $\alpha_T$ sia costante solo nella banda di frequenze dello spettro del segnale all'uscita del filtro di ricezione ovvero delle forme d'onda di Nyquist, per le quali la frequenza massima è data da eq.\ref{eq:banda_nyquist} $B=(1+\delta)f_s/2$.

Nel dimensionamento di un sistema di trasmissione numerico con filtro generico, data una probabilità di errore tollerabile, si vuole determinare l'ampiezza del segnale rettangolare da trasmettere, ricordando che $\gamma=a/\sigma$ per codifica antipodale e $\gamma=a/2\sigma$ per codifica ortogonale, si impone che 
\begin{equation}
\frac{P_\text{T picco}^\text{A}}{\frac{h_n}{2}f_s}=\gamma^2\e{2\alpha_e}
\end{equation}
L'attenuazione equivalente $\alpha_e$ può essere calcolata integrando numericamente la funzione di trasferimento del mezzo trasmissivo. Nel caso di mezzo trasmissivo ideale con attenuazione costante $\alpha_T$ almeno nella banda delle forme d'onda di Nyquist, si impone che 
\begin{equation}
\frac{P_\text{R media}^\text{A}}{\frac{h_n}{2}f_s}=\gamma^2\e{2\alpha_F}
\end{equation}
Una volta calcolata l'ampiezza del segnale da trasmettere è possibile determinare la funzione di trasferimento del filtro di ricezione.

\section{Effetto di disturbi generici}
Si è visto come per evitare l'interferenza intersimbolica si imponga di ricevere all'uscita del filtro in ricezione delle forme d'onda di Nyquist a zeri equidistanti. L'effetto di errori di campionamento viene mitigato dall'ampiezza limitata della forma d'onda che si ottiene per roll-off maggiori di $\delta>0$ (idealmente $\delta=1$) al prezzo di un maggior utilizzo di banda e maggior potenza di rumore passante dal filtro.

Si valuta ora l'effetto di disturbi generici di altra natura rispetto al rumore termico gaussiano bianco. Le potenze medie dei segnali di disturbo si possono sommare a quelle del rumore solo nel caso il disturbo abbia statistica gaussiana; in generale nota la densità di probabilità $f_d$ del disturbo si ha
\begin{figure}[!ht]
	\centering
	\begin{tikzpicture}[>=latex']
	\coordinate(start);
	\node[sum,right=1cm of start](s0){$+$} edge[<-](start);
	\coordinate[above=1cm of s0](n1);
	\draw[->](n1)node[above]{$n(t)$}--(s0);
	\node[block,right=1cm of s0](hr){$H_R(f)$} edge[<-](s0);
	\node[sum,right=1cm of hr](s1){$+$} edge[<-](hr);
	\coordinate[above=1cm of s1](n2);
	\draw[->](n2)node[above]{$d(t)$}--(s1);
	\node[campionatore,right=1cm of s1](q0){} edge[<-](s1);
	\node[decisore,right=1cm of q0](dec){} edge[<-](q0);
	\coordinate[right=1.5cm of dec](end);
	\draw[dot=C](dec)--(end);
	\end{tikzpicture}
	\caption{Schema del ricevitore affetto da disturbo a monte del campionatore}
	\label{fig:schema_disturbo_ricevitore}
\end{figure}

Le tensioni misurate al campionatore per un sistema di trasmissione binario con codifica antipodale affetto da disturbo sono
\begin{equation}
v_{C\vert d}=\begin{cases}
0_T \to v_{C\vert 0_T}=v_0+n+d& \\ 1_T \to v_{C\vert 1_T}=v_1+n+d
\end{cases}
\end{equation}

La soglia tra i valori di tensione attesi $v_0$ e $v_1$ non può essere spostata essendo il disturbo una variabile casuale. Se la statistica del disturbo ha valor medio nullo si lascia la soglia fissa a zero.

La probabilità di errore condizionata al disturbo $d$ risulta
\begin{equation}
p(\epsilon|d)=p(0_T)p(\epsilon|0_T,d)+p(1_T)p(\epsilon|1_T,d)
\end{equation}
\begin{equation}
p(\epsilon|0_T,d)=\intd{S}{+\infty}{f_{v_{C|0_T,d}}(v)}{v}=\intd{S}{+\infty}{ \frac{1}{\sqrt{2\pi}\sigma_n}\e{-\frac{v-v_0-d}{2\sigma_n^2}}}{v}=\f{Q}{\frac{s-v_0-d}{\sigma_n}}
\end{equation}
\begin{equation}
p(\epsilon|1_T,d)=\intd{-\infty}{S}{f_{v_{C|1_T,d}}(v)}{v}=\intd{-\infty}{S}{ \frac{1}{\sqrt{2\pi}\sigma_n}\e{-\frac{v-v_1-d}{2\sigma_n^2}}}{v}=\f{Q}{\frac{v_1+d-s}{\sigma_n}}
\end{equation}

\begin{figure}[ht!]\centering
	\begin{tikzpicture}
	\begin{axis}[xlabel=$v_{C\vert d}$,axis x line=middle,xtick={-1.5,-1.0,1.5,2.0},ytick=\empty,xticklabels={$v_0$,$v_0+d$,$v_1$,$v_1+d$},extra x ticks={-1.0,0.0,2.0},extra x tick labels={,soglia,},extra x tick style={grid=major},samples=300,domain=-2:2.5]
	\addplot [help lines,dashed, name path=gaussv0] {gauss(x,-1.5,.707)};
	\addplot [help lines,dashed, name path=gaussv1] {gauss(x,1.5,.707)};
	\addplot [name path=gaussv0] {gauss(x,-1.0,.707)};
	\addplot [name path=gaussv1] {gauss(x,2.0,.707)};
	\path[name path=axis] (axis cs:-2.5,0)--node[above right,pos=0.6,pin=60:\footnotesize $P(1_R|0_T)$]{}node[above left,pos=0.5,pin=120:\footnotesize $P(0_R|1_T)$]{}(axis cs:2.5,0);
	\addplot[pattern=north west lines] fill between[of=gaussv0 and axis,soft clip={domain=0:2}];
	\addplot[pattern=north east lines] fill between[of=gaussv1 and axis,soft clip={domain=-2:0}];
	\end{axis}
	\end{tikzpicture}
	\caption{Modello statistico della decisione con disturbo di valore $+d$}
	\label{fig:trasmissione_numerica_modello_statistico_decisione_disturbo}
\end{figure}

Nel caso di codifica antipodale la soglia $s=(v_0+v_1)/2=0$ si ha una probabilità di errore condizionata
\begin{equation}
p(\epsilon|d)=\frac{1}{2}\left[\f{Q}{\frac{a-d}{\sigma_n}}+\f{Q}{\frac{a+d}{\sigma_n}}\right]
\end{equation}

In definitiva la probabilità di errore che è necessario soddisfare nel dimensionamento è
\begin{equation}
p(\epsilon)=\intd{}{}{p(\epsilon|d)f_d(v)}{v}
\end{equation}


\section{Sistema di trasmissione multilivello}
\`{E} possibile aumentare la velocità di trasmissione di un sistema di trasmissione numerica, a parità di frequenza di cifra e banda passante, incrementando la potenza del segnale trasmesso in codifica \acf{PAM} in modo tale da poter distinguere più livelli di ampiezza di una stessa forma d'onda. Ciascun livello rappresenta un simbolo a cui sono associate più cifre binarie: con un numero di livelli $M=2^n$ è possibile associare $n$ bit ad ogni simbolo.

Si ottiene a parità di frequenza di trasmissione dei simboli, \keyword[baud rate]{baud-rate}, un maggiore \keyword[bit rate]{bit-rate} pari al \keyword{baud-rate} moltiplicato per il numero di bit $n$ convogliati da ciascun simbolo. 

\begin{figure}[ht!]
\centering
\begin{tikzpicture}
	\begin{axis}[xtick=\empty,ytick=\empty,yscale=.5,every extra x tick/.append style={major tick length=0pt},extra x ticks={.5,1.5,2.5,3.5},extra x tick labels={00,01,10,11},ymin=-2,ymax=2]
	\addplot [const plot,thick,draw=black]coordinates {(0,-1)(1,1)(2,-2)(3,2)(4,0)};
	\end{axis}
\end{tikzpicture}
\caption{Trasmissione multilivello quadripodale}
\end{figure}

La maggiore potenza disponibile consente in un sistema binario antipodale di distanziare le gaussiane in uscita dal filtro di ricezione tanto da diminuire di ordini di grandezza la probabilità di errore. Essendo la probabilità di errore legata alla varianza del rumore e alla distanza tra i livelli, avendo potenza in esubero è possibile inserire altri livelli a spaziatura tale da rispettare i limiti di probabilità d'errore e aumentare il numero di bit trasmessi da ciascun simbolo.\footnote{I livelli sono equispaziati in quanto la varianza del rumore non varia con la forma d'onda del simbolo trasmesso.} 

\begin{figure}[ht!]\centering
	\def\var{.707}
	\begin{tikzpicture}
	\pgfplotsset{set layers}
	\begin{axis}[scale=.5,xscale=4,xlabel=$V_C$,hide y axis,axis x line=middle,xtick={-3,-2,-1,1,2,3},ytick=\empty,xticklabels={$-3C$,$-2C$,$-C$,$C$,$2C$,$3C$},extra x ticks={-2,0,+2},extra x tick labels={},extra x tick style={grid=major},samples=300,xmin=-5.5,xmax=+5.5,domain=-5.5:+5.5]
	\addplot [name path=gauss00] {gauss(x,-3,\var)};
	\addplot [name path=gauss01] {gauss(x,-1,\var)};
	\addplot [name path=gauss10] {gauss(x,+1,\var)};
	\addplot [name path=gauss11] {gauss(x,+3,\var)};
	\path[name path=axis] (axis cs:-4,0)--node[above right,pos=0.5,pin=60:\footnotesize $P(01_R|00_T)$]{}node[above left,pos=0.25,pin=120:\footnotesize $P(10_R|00_T)$]{}(axis cs:4,0);
	\addplot[pattern=north west lines] fill between[of=gauss01 and axis,soft clip={domain=-4:-2}];
	\addplot[pattern=north east lines] fill between[of=gauss01 and axis,soft clip={domain=0:2}];
	\end{axis}
	\begin{axis}[scale=.5,xscale=4,xlabel=simboli,
	yshift=-0.4cm,axis x line=bottom,hide y axis,
	xtick={-3,-1,1,3},xticklabels={$10$,$00$,$01$,$11$},extra x ticks={-2,0,+2},extra x tick labels={soglia,soglia,soglia},extra x tick style={grid=major,yshift=-0.4cm},xmin=-5.5,xmax=+5.5]%
	\addplot coordinates {(0,0)};
	\end{axis}
	\end{tikzpicture}
	\caption{Modello statistico della decisione per un sistema a 4 livelli}
	\label{fig:trasmissione_numerica_modello_statistico_decisione_multilivello}
\end{figure}

Al ricevitore si dovrà discriminare con affidabilità sufficiente quale tra i quattro simboli è stato trasmesso in ogni periodo di simbolo. Definite le soglie tra le funzioni densità di probabilità associate alle possibili forme d'onda il sistema può decidere con una certa probabilità d'errore quale forma d'onda è stata inviata, in modo tale da individuare la configurazione di due cifre binarie associata al livello.

La probabilità d'errore è la probabilità di ottenere in uscita un livello diverso da quello trasmesso. Ciascun livello può essere confuso con qualunque livello precedente o successivo, ma risulta trascurabile la probabilità di confondere livelli tra loro non contigui. Si ha quindi che la probabilità d'errore è data dalla somma delle probabilità dell'evento confusione del livello $i$-esimo con la coppia dei livelli adiacenti per i livelli interni e con l'unico livello attiguo per i livelli esterni. Tale probabilità è pari all'area tagliata dalle soglie sulle code della gaussiana del livello $i$-esimo.

Si ha che la probabilità di errare livello in ricezione con $M$ livelli con simboli equiprobabili
\begin{equation}
p(\epsilon)=\sum_{i=1}^{M}p(\epsilon|S_i)p(S_i)=\frac{1}{M}\sum_{i=1}^{M}p(\epsilon|S_i)=\frac{(M-2)2p+2p}{M}=2p\frac{M-1}{M}
\end{equation}

Nell'ipotesi di associare a ciascuna forma d'onda una configurazione binaria secondo un codice Gray a distanza unitaria, in cui cambia un solo bit della configurazione tra simboli adiacenti, si ha che la probabilità di errare livello corrisponde con l'evento errare un bit del simbolo ricevuto. Avendo configurazioni di $n=\log_2{M}$ bit si ha che il \keyword[bit error rate]{bit error rate} o probabilità d'errore per un sistema ad $M=2^n$ livelli è
\begin{equation}
\text{BER}=\frac{p(\epsilon)}{\log_2{M}}=2p\frac{M-1}{M\log_2{M}}
\end{equation}

Confrontando le prestazioni di un sistema di trasmissione binario antipodale ($M=2$)  con i sistemi multilivello si ha che al costo di una maggiore potenza di trasmissione \footnote{si passa da una potenza di trasmissione $P_B$ ad una potenza di picco $P_M^p=(M-1)^2 P_B$ e una potenza media $P_M^m=\frac{M^2-1}{3}P_B$} si ottiene un minore \keyword{bit error rate} e una velocità di trasmissione $\log_2{M}$ superiore. La maggiore potenza di trasmissione rende i sistemi multilivello sensibili all'interferenza degli echi riflessi che possono avere potenza comparabile a quella dei livelli interni.
\begin{table}[ht]
\centering
\begin{tabular}{ccc}
$M$ & $p(\epsilon)$ & BER \\ \hline
2	& $p$ & $p$ \\
4	& $\frac{3}{2}p$ & $\frac{3}{4}p$ \\
8	& $\frac{7}{4}p$ & $\frac{7}{12}p$ \\
16	& $\frac{15}{8}p$ & $\frac{15}{32}p$ \\
\end{tabular}
\end{table}

\section{Scrambling}
Nell'ipotesi che i bit prodotti dalla sorgente dell'informazione non siano equiprobabili\footnote{ad esempio nella codifica ASCII il bit più significativo uguale a zero per la maggioranza dei caratteri d'uso comune} è possibile ottenerla applicando una operazione di \keyword[scrambling]{scrambling}: in fase di trasmissione si altera la sequenza di bit generata dalla sorgente scambiandone i bit con quelli di una sequenza pseudocasuale nota al mittente e al destinatario applicando l'operatore logico binario \keyword{XOR}, tale operazione reversibile è ripetuta in fase di ricezione per riottenere la sequenza originale.

\begin{table}[!ht]
	\centering
	\begin{tabular}{c}
		$a \oplus b = c$ \\ \hline
		$0 \oplus 0 = 0$ \\
		$0 \oplus 1 = 1$ \\
		$1 \oplus 0 = 1$ \\
		$1 \oplus 1 = 0$ 
	\end{tabular}
	\caption{Tabella di verità dell'operatore logico \keyword[operatore logico!XOR]{XOR}}
\end{table}

\clearpage
\section{Esercizio}
\begin{esercizio}
Un segnale analogico è campionato, quantizzato e codificato in un segnale binario \ac{PCM} con rappresentazione a $M=128$ livelli. Un impulso di sincronizzazione è è inviato dopo ogni campione.

Il segnale \ac{PCM} risultante è trasmesso su un canale in banda base $B=\SI{13}{\kilo\hertz}$ con modulazione di ampiezza multilivello \ac{PAM} quaternario con spettro a coseno rialzato.

Si vogliono determinare
\begin{enumerate}
	\item la velocità in bit/secondo di trasmissione dell'informazione (\keyword[bit rate]{bitrate}).
	\item la velocità di campionamento e la massima frequenza ammessa dal segnale analogico.
\end{enumerate}

Il segnale analogico ha campioni quantizzati a 7 bit a cui è aggiunto un bit di sincronizzazione tale da formare un parola di codice a 8 bit.

Il sistema trasmette simboli di 2 bit (\ac{PAM} quaternario) alla frequenza di cifra (\keyword{baud rate}) data dalla massima frequenza in banda base con forme d'onda di Nyquist dello spettro a coseno rialzato con $\delta=1$:
\[ B=\frac{1}{2 T_b}(1+\delta) \implies \text{baud rate}=\frac{1}{T_b}=\frac{2}{1+\delta}B=B=\SI{13}{\kilo\hertz}\]
\end{esercizio}
per cui la velocità di trasmissione del canale (\keyword{bitrate}): $\SI{26}{\kilo\bit\per\second}$

La velocità di campionamento o \keyword[sampling rate]{sampling rate} per parole di $\SI{8}{\bit}$:
\[f_\text{sampling}=\frac{\SI{26}{\kilo\bit\per\second}}{\SI{8}{\bit\per sample}}=\SI{3250}{\hertz}\]
Campionando almeno ad una frequenza doppia della massima frequenza del segnale analogico si ha \[f_s^\text{max}=\frac{f_S}{2}=\SI{1625}{\hertz}\]
\chapter{Sistemi di trasmissione in modulazione numerica}
Un sistema di trasmissione di bit generati da una sorgente digitale può utilizzare un canale trasmissivo passa banda applicando le tecniche di modulazione analogica alle forme d'onda che codificano i simboli “0” e “1” in banda base. \`{E} necessario adattare tali segnali alla funzione di trasferimento del canale passa banda: lo spettro rettangolare di una forma d'onda di Nyquist con $\delta=0$ e banda $B$ viene modulato per rientrare nelle frequenze della banda passante del canale.

Al ricevitore è necessario operare inversamente: demodulare il segnale passa banda per riportarlo in banda base, per ottenere le forme d'onda ad intersimbolo nullo, campionare con stringenti requisiti di precisione dell'estrattore del timing e applicare il decisore per ricostruire la sequenza di bit trasmessa.

\begin{figure}[!ht]
	\centering
	\resizebox{\textwidth}{!}{
	\begin{tikzpicture}[>=latex',fitted/.style={draw,thick,dotted,inner sep=4mm,rounded corners}]
	\coordinate(c0);
	\node[block,right=1.5cm of c0](b0){GFO};
	\node[mult,right=1cm of b0](m0){}edge[<-](b0);
	\node[below=1cm of m0](t0){$\cos{\omega t}$}edge[->](m0);
	\node[block,right=1cm of m0](b1){MT}edge[<-](m0);
	\node[sum,right=1cm of b1](s0){$+$}edge[<-](b1);
	\node[above=1cm of s0](n0){$n(t)$}edge[->](s0);
%	\node[mult,right=1cm of s0](m1){}edge[<-](s0);
%	\node[below=1cm of m1](t1){$2 \cos{\omega t}$}edge[->](m1);
	\node[block,right=1cm of s0](b2){FR}edge[<-](s0);
	\node[block,right=1cm of b2](b3){DEM}edge[<-](b2);
	\node[block,right=1cm of b3](b4){$H_R$}edge[<-](b3);
	\node[campionatore,right=1cm of b4](q0){}edge[<-](b4);
	\node[block,right=1cm of q0](b5){DEC}edge[<-](q0);
	\coordinate[right=1.5cm of b5](c1);
	\draw[->](c0)--node[above,near start]{bit}(b0);
	\draw[->](b5)--node[above,near end]{bit}(c1);
	\node[fitted, fit=(b0)(m0),label=above:Trasmettitore]{};
	\node[fitted, fit=(b2)(b3)(b4)(b5),label=above:Ricevitore]{};
	\node[fitted, fit=(b1)(s0),label=above:Canale]{};
	\end{tikzpicture}
	}
	\caption{Schema di trasmissione in modulazione numerica}
\end{figure}

Tutto quanto studiato nel capitolo \ref{cap:sis_telcom_band_pass} sui sistemi di telecomunicazione su canale radio passa banda, con riferimento alle modulazioni d'ampiezza e angolare di segnali analogici, si può applicare alla trasmissione numerica.

\section{Amplitude Shift Keying}\index{modulazione di ampiezza!Amplitude Shift Keying}
La modulazione d'ampiezza in un mezzo trasmissivo passa banda modifica l'ampiezza della portante trasmessa in accordo con la forma d'onda da trasmettere. Un semplice esempio di modulazione \ac{ASK} si ottiene modulando una portante sinusoidale con una forma d'onda a codifica di linea ortogonale, costituita da rettangoli di ampiezza positiva unitaria per codificare il simbolo “1” e forma d'onda nulla per codificare lo “0” (fig.~\ref{fig:OOK_modulazione}). Tale modulazione del tipo “tutto o niente” \ac{OOK} può essere realizzata con un interruttore che pone a massa l'uscita di un oscillatore che genera la sinusoide portante quando si deve trasmettere il simbolo “0” (fig.~\ref{fig:OOK_circuito}).

Per un migliore sfruttamento della potenza trasmessa si utilizza la codifica antipodale che modula la portante con valori costanti positivi o negativi. Si realizza di fatto una modulazione 2-\ac{PSK} in cui la forma d'onda ha inviluppo costante e la portante inverte la fase ogni volta che il segnale modulante cambia segno (fig.~\ref{fig:2PSK_modulazione}).

Nel caso la portante sia modulata da forme d'onda con inviluppo non costante si parla propriamente di modulazione \ac{ASK} (fig.~\ref{fig:ASK_modulazione}).

\begin{figure}[ht]\centering
	\def\omegazero{2*pi*4}
	\subfloat[Modulazione \acf{OOK}]{
		\begin{tikzpicture}
		\begin{axis}[enlargelimits,yscale=0.6,xlabel=$t$,ylabel=$s_T(t)$,xtick=\empty,ytick={-1,1},black,samples=300]
		\addplot[domain=0:1] {sin((\omegazero*x))};
		\addplot[domain=2:4] {sin((\omegazero*x))};
		\addplot[thick,domain=0:4.1] {rect(x,0,1)+rect(x,2,4)};
		\end{axis}
		\end{tikzpicture}\label{fig:OOK_modulazione}
	}\quad\subfloat[Circuito \ac{OOK}]{
	\begin{circuitikz}
	\coordinate(c0);
	\coordinate[above=1cm of c0] (c1);
	\coordinate[below=1cm of c0] (c2);
	\coordinate[right=2cm of c0] (c3);
	\coordinate[above=.25cm of c3] (c4);
	\coordinate[below=.25cm of c3] (c5);
	\coordinate[right=of c3] (c6);
	\draw(c1)to[vco,l=$\sen{\omega_0 t}$]++(2,0);
	\draw[-o](c1)++(2,0)-|(c4);
	\draw(c2)node[ground]{}(c2);
	\draw[-o](c2)-|(c5);
	\draw[o-](c3)--node[above]{$1$}node[below]{$0$}(c6);
	\end{circuitikz}\label{fig:OOK_circuito}
	}\quad\subfloat[Modulazione ad inviluppo costante (2-\ac{PSK})]{
	\begin{tikzpicture}
	\begin{axis}[enlargelimits,yscale=0.6,xlabel=$t$,ylabel=$s_T(t)$,xtick=\empty,ytick={-1,1},black,samples=300]
	\addplot[domain=0:1] {sin((\omegazero*x))};
	\addplot[domain=1:2] {sin((\omegazero*x+pi))};
	\addplot[domain=2:3] {sin((\omegazero*x))};
	\addplot[domain=3:4] {sin((\omegazero*x+pi))};
	\addplot[thick,domain=0:4.1] {rect(x,0,1)-rect(x,1,2)+rect(x,2,3)-rect(x,3,4)};
	\end{axis}
	\end{tikzpicture}\label{fig:2PSK_modulazione}
	}\quad\subfloat[Circuito 2-\ac{PSK}]{
	\begin{circuitikz}
	\coordinate(c0);
	\coordinate[above=1cm of c0] (c1);
	\coordinate[below=1cm of c0] (c2);
	\coordinate[right=2cm of c0] (c3);
	\coordinate[above=.2cm of c3] (c4);
	\coordinate[below=.2cm of c3] (c5);
	\coordinate[right=of c3] (c6);
	\draw(c1)to[vco,l=$\sen{\omega_0 t+\pi}$]++(2,0);
	\draw[-o](c1)++(2,0)-|(c4);
	\draw(c2)to[vco,l=$\sen{\omega_0 t}$]++(2,0);
	\draw[-o](c2)++(2,0)-|(c5);
	\draw[o-](c3)--node[above]{$1$}node[below]{$0$}(c6);
	\end{circuitikz}\label{fig:2PSK_circuito}
	}\quad\subfloat[Modulazione ad inviluppo non costante (\ac{ASK})]{
	\begin{tikzpicture}
	\begin{axis}[enlargelimits,yscale=0.6,xlabel=$t$,ylabel=$s_T(t)$,xtick=\empty,ytick={-1,1},black,samples=300]
	\addplot[domain=0:1] {sin((\omegazero*x))*sin((2*pi*x))};
	\addplot[thick,domain=0:1] {sin((2*pi*x))};
	\addplot[domain=0:1,dashed] {-sin((2*pi*x))};
	\addplot[domain=1:2] {sin((\omegazero*x))*sin((2*pi*x+pi))};
	\addplot[thick,domain=1:2] {sin((2*pi*x+pi))};
	\addplot[domain=1:2,dashed] {-sin((2*pi*x+pi))};
	\end{axis}
	\end{tikzpicture}\label{fig:ASK_modulazione}
	}\quad\subfloat[Circuito \ac{ASK}]{
	\begin{circuitikz}
		\coordinate(c0);
		\coordinate[above=1cm of c0] (c1);
		\coordinate[below=1cm of c0] (c2);
		\node[mult,right=2cm of c0] (c3){};
		\coordinate[right=of c3] (c4);
		\draw(c1)to[vco,l=$\sen{\omega_0 t+b_m\cdot\pi}$]++(2,0)-|(c3);
		\draw(c2)to[vco,l=$\sen{\omega_0 t}$]++(2,0)-|(c3);
		\draw(c3)--(c4);
	\end{circuitikz}
}
\caption{Modulazione numerica \acf{ASK}}
\end{figure}

\section{Quadrature Amplitude Modulation}\index{modulazione di ampiezza!Quadrature Amplitude Modulation}
Per aumentare l'efficienza spettrale di un sistema di trasmissione numerica in modulazione di ampiezza \ac{ASK} si sfruttano le due bande laterali per trasmettere due segnali modulanti con portanti in quadratura, ad esempio dei bit pari sul canale in fase e dei bit dispari sul canale in quadratura.

Nella trasmissione analogica in \ac{DSB-SC} l'imperfetta ortogonalità delle portanti causata da un errore di fase nella modulazione in quadratura genera un accoppiamento tra i canali. La trasmissione numerica è più robusta all'interferenza tra canali finché è garantita una buona discriminazione dei livelli associati ai simboli in ricezione. L'operazione di decisione al ricevitore consiste nella verifica del sottospazio nel quale cade il punto identificato dalle misure ottenute dai due campionatori sul canale in fase e sul canale in quadratura.

Dati due segnali digitali in codifica antipodale di ampiezza $\pm a$ modulanti la portante in fase $\cos{\omega_0 t}$ e la portante in quadratura $\sen{\omega_0 t}$ si ottengono, utilizzando le formule trigonometriche: \[\sen{\alpha}=\cos{\alpha-\frac{\pi}{2}}\quad-\cos{\alpha}=\cos{\pi-\alpha}\] \[\Cos\alpha+\Cos\beta=2\cos{\frac{\alpha+\beta}{2}}\cos{\frac{\alpha-\beta}{2}}\qquad\Cos\alpha-\Cos\beta=-2\sen{\frac{\alpha+\beta}{2}}\sen{\frac{\alpha-\beta}{2}}\]
\begin{description}
	\footnotesize
	\item[Bit “11”] Portante in fase $c^\text{I}=+a$, portante in quadratura $c^\text{Q}=+a$:
	\[\begin{split}a\cos{\omega_0 t}+a\sen{\omega_0 t}&=a\cos{\omega_0 t}+a\cos{\omega_0 t-\frac{\pi}{2}}=2a\cos{\omega_0 t-\frac{\pi}{4}}\cos{\frac{\pi}{4}}=\sqrt{2}a\cos{\omega_0 t-\frac{\pi}{4}}\end{split}\]
	\item[Bit “10”] Portante in fase $c^\text{I}=+a$, portante in quadratura $c^\text{Q}=-a$:
	\[\begin{split}a\cos{\omega_0 t}-a\sen{\omega_0 t}&=a\cos{\omega_0 t}-a\cos{\omega_0 t-\frac{\pi}{2}}=-2a\sen{\omega_0 t-\frac{\pi}{4}}\sen{\frac{\pi}{4}}=\sqrt{2}a\cos{\omega_0 t-\frac{3}{4}\pi}\end{split}\]
	\item[Bit “00”] Portante in fase $c^\text{I}=-a$, portante in quadratura $c^\text{Q}=-a$:
	\[\begin{split}-a\cos{\omega_0 t}-a\sen{\omega_0 t}&=a\cos{\pi-\omega_0 t}-a\cos{\omega_0 t-\frac{\pi}{2}}=2a\sen{\omega_0 t-\frac{3}{4}\pi}\sen{\frac{\pi}{4}}=\sqrt{2}a\cos{\omega_0 t-\frac{5}{4}\pi}\end{split}\]
	\item[Bit “01”] Portante in fase $c^\text{I}=-a$, portante in quadratura $c^\text{Q}=+a$:
	\[\begin{split}-a\cos{\omega_0 t}+a\sen{\omega_0 t}&=-a\cos{\omega_0 t}+a\cos{\omega_0 t-\frac{\pi}{2}}=-2a\sen{\omega_0 t-\frac{\pi}{4}}\sen{\frac{\pi}{4}}=\sqrt{2}a\cos{\omega_0 t-\frac{7}{4}\pi}\end{split}\]	
\end{description}

\begin{figure}[ht]
\centering
\subfloat[Rappresentazione fasoriale portanti]{
	\begin{tikzpicture}[>=latex']
	\def\anglealpha{30}
	\def\anglebeta{\anglealpha-90}
	\def\anglegamma{\anglealpha-45}
	\def\angletheta{\anglealpha-135}
	\coordinate (b11) at (\anglegamma:1.414cm);
	\coordinate (b01) at (\angletheta:1.414cm);
	\coordinate (b00) at (\anglegamma:-1.414cm);
	\coordinate (b10) at (\angletheta:-1.414cm);
	\draw[->] (-1.5,0)--(1.6,0);
	\draw[->] (0,-1.5)--(0,1.6);
	\draw[gray](0,0) circle [radius=1.41cm];
	\filldraw[fill=gray!20,draw=gray!50!black] (0,0) -- (5mm,0mm) arc [start angle=0, end angle=\anglealpha, radius=5mm] -- cycle node[right]{\footnotesize$\omega_0 t$};
	\draw[dotted](\anglealpha:1.41cm)node{\footnotesize$\cos{\omega_0 t}$}--(0,0)--(\anglebeta:1.41cm)node{\footnotesize$\sen{\omega_0 t}$};
	\draw[thick] (b11) circle(2pt) node[below right]{$11$} --(0,0)--(b00) circle(2pt) node[above left]{$00$};
	\draw[thick](b01) circle(2pt) node[below left]{$01$}--(0,0)--(b10) circle(2pt) node[above right]{$
		10$};
	\draw[dashed](b11)--(b10)--(b00)--(b01)--cycle;
	\end{tikzpicture}}\quad%
\subfloat[Portanti modulate in quadratura]{
	\begin{tikzpicture}
	\begin{axis}[yshift=2cm,xscale=.6,yscale=.3,smooth,xtick=\empty,enlargelimits,ytick={-1,1},yticklabels={$-a$,$+a$},ylabel=$c^\text{I}$,extra x ticks={3.14,6.28,9.42,12.56},extra x tick labels=\empty,extra x tick style={grid=major}]
		\addplot [domain=0:2*pi]{cos(x)};
		\addplot [domain=2*pi:4*pi]{-cos(x)};
	\end{axis}
	\begin{axis}[xscale=.6,yscale=.3,smooth,xtick=\empty,enlargelimits,ylabel=$c^\text{Q}$,ytick={-1,1},yticklabels={$-a$,$+a$},extra x ticks={3.14,6.28,9.42,12.56},extra x tick labels=\empty,extra x tick style={grid=major}]
		\addplot [domain=0:pi]{sin(x)};
		\addplot [domain=pi:3*pi]{-sin(x)};
		\addplot [domain=3*pi:4*pi]{sin(x)};
	\end{axis}
	\end{tikzpicture}}
\quad\subfloat[Segnale modulato 4-\ac{PSK}]{
	\begin{tikzpicture}
		\begin{axis}[scale=.66,smooth,ylabel=$s^\text{T}(t)$,xtick=\empty,ytick={-1,1},yticklabels={$-a$,$+a$},xtick={1.57,4.71,7.85,10.99},xticklabels={$11$,$10$,$00$,$01$},extra x ticks={3.14,6.28,9.42,12.56},extra x tick labels=\empty,extra x tick style={grid=major}]
		\addplot [domain=0:pi]{cos(x)+sin(x)};
		\addplot [domain=pi:2*pi]{cos(x)-sin(x)};
		\addplot [domain=2*pi:3*pi]{-cos(x)-sin(x)};
		\addplot [domain=3*pi:4*pi]{-cos(x)+sin(x)};
		\end{axis}
	\end{tikzpicture}}
	\caption{Modulazione di ampiezza con portanti in quadratura (4-\ac{PSK})}
\end{figure}

La modulazione \ac{QAM} con portanti in quadratura di forme d'onda rettangolari risulta in una modulazione 4-\ac{PSK} con una sinusoide scalata di una ampiezza costante, pulsazione $\omega_0 t$, e fasi con quattro possibili valori equidistribuiti sull'angolo giro, pari a $\ang{45}$, $\ang{135}$, $\ang{225}$, $\ang{315}$ a seconda della combinazione delle ampiezze antipodali. \`{E} possibile assegnare in modo arbitrario le sequenze di bit alle varie configurazione di fase ma si predilige la numerazione di Grey in base al criterio di distanza minima di Hamming di 1 bit tra simboli adiacenti.

Utilizzando segnali modulanti rettangolari multilivello si può estendere la modulazione \ac{QAM} mantenendo la stessa occupazione spettrale con modulazioni di fase 8-\ac{PSK} (3 bit per simbolo) o 16-\ac{PSK} (4 bit per simbolo). Il dimensionamento di un tale sistema è limitato dal rumore sul canale trasmissivo che introduce una deviazione di fase tale da causare una errata decisione sul valore di soglia al demodulatore per simboli adiacenti sulla circonferenza.

\subsection{Dimensionamento}
Nel dimensionamento di un sistema di trasmissione numerico \ac{QAM} con portanti in quadratura valgono le considerazioni relative ad un sistema con una sola portante con modulazione \ac{ASK} binario o \ac{PSK} applicate separatamente alle due portanti, per cui il decisore a soglia al ricevitore verifica il livello del segnale ricevuto a cui è sommato un disturbo che causa errore.

Nel seguente schema è rappresentato, in un sistema di assi cartesiani solidali alle portanti in quadratura, la costellazione dei valori assunti da una modulazione \ac{QAM} a valle del filtro di ricezione affetto da rumore, con la separazione in settori circolari del piano in corrispondenza delle soglie tra i livelli
\begin{figure}[ht]
	\centering
	\begin{tikzpicture}[>=latex']
	\draw[->](-1.5,0)--(2.1,0)node[right]{\footnotesize$\sen{\omega_0 t}$};
	\draw[->](0,-1.5)--(0,1.8)node[above]{\footnotesize$\cos{\omega_0 t}$};
	\draw[gray](0,0)circle[radius=1.41cm];
	\foreach\anglealpha/\bits/\shade in {0/000/50,45/001/25,90/011/50,135/010/25,180/110/50,225/111/25,270/101/50,315/100/25} {
		\pgfmathparse{\anglealpha-22.5}\def\anglebeta{\pgfmathresult}
		\pgfmathparse{\anglealpha+22.5}\def\angletheta{\pgfmathresult}
		\draw[fill=gray!50!black,draw=gray!50!black,dotted] (0,0) -- (2cm,0mm) arc [start angle=\anglebeta, end angle=\angletheta, radius=2cm] -- cycle;
		\draw(\anglealpha:1.41cm)decorate[decoration={random steps,segment length=3pt,amplitude=2pt}]{circle(10pt)} node{\footnotesize$\bits$};
	}
	\end{tikzpicture}
	\caption{Schema spazio dei segnali nella modulazione 8-\ac{PSK}}
\end{figure}

Il ricevitore per decidere quale sia la configurazione ricevuta verifica in quale sottospazio ricadono le due misure ottenute dai due campionatori sul canale in fase e in quadratura applicando un criterio di distanza minima.

Al ricevitore le componenti indipendenti del rumore in fase $n^\text{I}(t)$ e quadratura $n^\text{Q}(t)$ si sommano ai segnali rettangolari antipodale ricevuti attenuati:

\[\begin{array}{ll}
11\colon\begin{cases}
+a\stackrel{I}{\longrightarrow}+k\cdot a+n^\text{I} \\
+a\stackrel{Q}{\longrightarrow}+k\cdot a+n^\text{Q}
\end{cases} &
10\colon\begin{cases}
+a\stackrel{I}{\longrightarrow}+k\cdot a+n^\text{I} \\
-a\stackrel{Q}{\longrightarrow}-k\cdot a+n^\text{Q}
\end{cases} \\\\
00\colon\begin{cases}
-a\stackrel{I}{\longrightarrow}-k\cdot a+n^\text{I} \\
-a\stackrel{Q}{\longrightarrow}-k\cdot a+n^\text{Q}
\end{cases}	&
01\colon\begin{cases}
-a\stackrel{I}{\longrightarrow}-k\cdot a+n^\text{I} \\
+a\stackrel{Q}{\longrightarrow}+k\cdot a+n^\text{Q}
\end{cases}
\end{array}\]

Il rumore passa banda sovrapposto alle portanti in quadratura rappresenta una variabile aleatoria gaussiana bidimensionale a simmetria circolare, ovvero un vettore con ampiezza e fase casuale (fig.~\ref{fig:var_aleatoria_gaussiana_bidimensionale}).

\begin{figure}[ht]\centering\def\amp{2.0}\def\var{0.5}
	\subfloat{
		\begin{tikzpicture}
		\begin{axis}[xlabel=$v^\text{I}$,ylabel=$v^\text{Q}$,small,domain=-6:6,domain y=-6:6,colormap name=graywhite]
		\addplot3[surf,z buffer=auto,faceted color=black,samples=32]{
			\amp/\var*exp(-((x-3)^2+(y-3)^2)/(2*\var))+
			\amp/\var*exp(-((x+3)^2+(y-3)^2)/(2*\var))+
			\amp/\var*exp(-((x-3)^2+(y+3)^2)/(2*\var))+
			\amp/\var*exp(-((x+3)^2+(y+3)^2)/(2*\var))};
		\end{axis}
		\end{tikzpicture}}\quad\def\amp{2.0}\def\var{0.5}
	\subfloat{
		\begin{tikzpicture}
		\begin{axis}[view={30}{60},xlabel=$v^\text{I}$,ylabel=$v^\text{Q}$,small,domain=-6:6,domain y=-6:6,colormap name=graywhite]
		\addplot3[surf,faceted color=black,z buffer=auto,samples=32] {
			\amp/\var*exp(-((x-4.24)^2+y^2)/(2*\var))+
			\amp/\var*exp(-((x-3)^2+(y-3)^2)/(2*\var))+
			\amp/\var*exp(-((x-3)^2+(y+3)^2)/(2*\var))+
			\amp/\var*exp(-(x^2+(y-4.24)^2)/(2*\var))+
			\amp/\var*exp(-(x^2+(y+4.24)^2)/(2*\var))+
			\amp/\var*exp(-((x+4.24)^2+y^2)/(2*\var))+
			\amp/\var*exp(-((x+3)^2+(y-3)^2)/(2*\var))+
			\amp/\var*exp(-((x+3)^2+(y+3)^2)/(2*\var))};
		\end{axis}
		\end{tikzpicture}}
	\caption{Rappresentazione delle densità di probabilità del segnale in ricezione modulato affetto da rumore espresso da variabile aleatoria gaussiana bidimensionale}
	\label{fig:var_aleatoria_gaussiana_bidimensionale}
\end{figure}

La probabilità di errore è valutabile come il volume della densità di probabilità condizionata esterna all'area delimitata dai valori di soglia tra simboli contigui.
\[p(\epsilon)=\sum_{i=0}^{M-1}p(\epsilon|s_i)p(s_i)=\frac{1}{M}\sum_{i=0}^{M-1}p(\epsilon|s_i)\]
\[p(\epsilon|s_i)=p(s_j^R|s_i^T)=p(0_R|1_T\vee 1_R|0_T)\quad i\neq j\]

Tale probabilità si può valutare estraendo il sistema binario equivalente ottenuto valutando il rapporto tra la distanza tra i vertici di due vettori contigui della costellazione \ac{QAM}:
\begin{equation}
p(\epsilon)=\f{Q}{\frac{d}{2\sigma_n}}
\end{equation}

Tale approssimazione per un sistema multilivello M-\ac{PSK} è lecita in quanto è possibile trascurare la probabilità di errore di confondere un livello con livelli più lontani di quelli contigui, ovvero i settori circolari in cui è diviso il piano.

Per un sistema di trasmissione numerico M-\ac{PSK} con codifica di Grey la probabilità di errore su un bit è pari alla probabilità di confondere il livello con i due adiacenti ovvero il doppio di quella del sistema binario equivalente:
\begin{equation}
\text{BER}=2\f{Q}{\frac{d}{2\sigma_n}}
\label{eq:qam_mpsk_biterrorrate}
\end{equation}

Essendo la potenza del rumore una caratteristica del mezzo trasmissivo e degli apparati di ricezione a parità di potenza della portante aumentare il numero di livelli fa diminuire la distanza tra i vertici della costellazione con la sovrapposizione delle code delle gaussiane, aumentando la probabilità di errore su bit.

Data una rumorosità del sistema se si vuole aumentare la velocità di trasmissione aumentando il numero di livelli rappresentati per ogni trasmissione di un simbolo, mantendo costante il \keyword{BER}, è necessario incrementare la potenza di trasmissione per mantenere la distanza tra i vertici dei vettori contigui.
Tale sistema risulta inefficiente in termini di potenza oltre la modulazione 16-\ac{PSK} risultando inutilizzato lo spazio all'interno del cerchio ove giacciono i vertici della costellazione M-\ac{PSK}.

A tal fine si introduce la modulazione di ampiezza multilivello M-\ac{QAM} con portanti in quadratura e inviluppo non costante: la costellazione dei vertici è disposta a reticolo imponendo la distanza minima tra vertici contigui tale da mantenere costante la probabilità d'errore. Ad esempio la 4-\ac{QAM} risulta in due portanti in quadratura modulate da segnali a due livelli. Nel caso 16-\ac{QAM} con 4 livelli per portante si hanno segnali trasmessi secondo la costellazione in fig.~\ref{fig:16-QAM} con un $\text{BER}=4\f{Q}{\frac{d}{2\sigma_n}}$. Raddoppiando ancora i livelli su entrambe le portanti si ottengono $M=4^4=256$ configurazioni.

Per ottimizzare ulteriormente il valore valore di picco di potenza dell'amplificatore in corrispondenza delle configurazioni ai vertici del quadrato si spostano tali configurazioni nei punti med\^{i} dei lati, mantenendo regolare il passo tra i punti del reticolo come in fig.~\ref{fig:256-QAM}. Altre tecniche di riempimento ottimizzano ulteriormente lo spazio tra i punti in modo che siano equidistanti al prezzo di una maggiore complessità.

\begin{figure}[ht]
	\centering
	\subfloat[16-\ac{QAM}]{
	\begin{tikzpicture}[>=latex']
	\draw[->](-2.5,0)--(2.6,0)node[right]{\footnotesize$\sen{\omega_0 t}$};
	\draw[->](0,-2.5)--(0,2.6)node[above]{\footnotesize$\cos{\omega_0 t}$};
	\draw[gray!50](0,0)circle[radius=2.25cm];
	\draw[->](0,0)--(1.5,1.5);
	\draw[dotted](-.5,.5)--(.5,.5)--node[below]{$d$}(1.5,.5)(.5,-.5)--(.5,1.5);
	\foreach \cx in {-1.5,...,1.5}
		\foreach \cy in {-1.5,...,1.5}
		{
			\draw (\cx,\cy) circle (2mm);
			\fill (\cx,\cy) circle (1mm);
		}
	\end{tikzpicture}\label{fig:16-QAM}}%
	\quad\subfloat[256-\ac{QAM}]{
		\begin{tikzpicture}[>=latex']
		\draw[->](-2.7,0)--(2.9,0)node[right]{\footnotesize$\sen{\omega_0 t}$};
		\draw[->](0,-2.7)--(0,2.9)node[above]{\footnotesize$\cos{\omega_0 t}$};
		\draw[gray!50](0,0)circle[radius=2.6cm]; %r=2.55+0.05
		\draw[gray!80,->](0,0)--(59:2.55cm);
		\newdimen\bailout
% 		creo un quadrato di lato 2R
% 		in cui visualizzare x punti interni alla circonferenza di raggio R
%		x : 4*R^2 = 256 : pi*R^2
%		lato = 51 mm  spazio tra i punti 3mm   18 punti per lato
		\foreach \cx in {-2.55,-2.25,...,2.85}  % 18
		\foreach \cy in {-2.55,-2.25,...,2.85}  % 18
		{
			\pgfmathparse{\cx*\cx+\cy*\cy}
			\bailout = \pgfmathresult cm
			\ifdim \bailout < 7.29 cm % (0.3*18/2)^2
				\draw (\cx,\cy) circle (1mm);
				\fill (\cx,\cy) circle (0.5mm);
			\fi
		}
		\end{tikzpicture}\label{fig:256-QAM}}
	\caption{Schema spazio dei segnali nella modulazione M-\ac{QAM} a 16 e 256 configurazioni}
\end{figure}

\subsection{Vantaggi dell'inviluppo costante}\index{inviluppo costante}
Nella modulazione \ac{PSK} la forma d'onda è a inviluppo costante: l'informazione è contenuta alla fase della portante pertanto è possibile mandare in saturazione gli amplificatori di trasmissione per il massimo guadagno di potenza. La sinusoide della portante viene squadrata e assume al limite la forma di un'onda quadra alla stessa frequenza della portante modulata, conservando gli istanti di attraversamento dello zero al cambio di segno. Se il segnale modulante è un treno di impulsi rettangolari all'uscita dell'amplificatore in saturazione si ottiene un segnale con spettro costituito dalle repliche dei seni cardinali centrati nelle armoniche della frequenza portante (fig.~\ref{fig:portante_squadrata_PSK}). \`{E} possibile limitare in banda lo spettro eliminando le code del $\sinc{f}$ mediante filtro passa banda, in quanto in ricezione si devono fornire forme d'onda ad intersimbolo nullo con termini spettrali contenuti all'interno del lobo principale. Maggiore attenzione è necessaria per segnali modulanti di ampiezza generica su portante squadrata: il segnale ricostruito viene distorto dalle non linearità. La portante squadrata filtrata passa-banda presenta spigoli smussati quindi un inviluppo non costante, amplificando e filtrando in ricezione si ottiene un segnale con modulazione di fase spuria.

\begin{figure}[ht!]\centering
	\begin{tikzpicture}
	\begin{axis}[xlabel=$f$,ylabel=$H(f)$,xtick={20,25,30},xticklabels={$f_0$,$2 f_0$,$3 f_0$},axis x discontinuity=crunch,xtickmin=20,ytick={1},yscale=.66,samples=500,domain=15:35]
	\addplot[gray] {(sinc(x-20,1)+sinc(x-25,1)+sinc(x-30,1)+sinc(x-40,1))^1.1};
	\addplot[black,thick,domain=19:21] {(sinc(x-20,1)+sinc(x-25,1)+sinc(x-30,1))};
	\addplot[gray,dashed] {rect(x,19,21)};
	\end{axis}
	\end{tikzpicture}
	\caption{Spettro portante modulata \ac{PSK} all'ingresso e all'uscita di un amplificatore spinto in saturazione}
	\label{fig:portante_squadrata_PSK}
\end{figure}

\subsection{Offset temporale tra canali in quadratura}
Confrontando le densità spettrali di potenza generate da diversi sistemi di trasmissione in modulazione numerica, a parità di prestazioni, si preferisce il segnale con la minore occupazione di banda sul mezzo trasmissivo.

Nelle modulazioni di fase \ac{PSK} sono le brusche variazioni di fase che incrementano l'occupazione spettrale del segnale. Per il \ac{PSK} binario sono gli indispensabili salti di fase di $\pi$, per il M-\ac{PSK} si ha il salto di fase massimo sempre di $\pi$ per configurazioni opposte.

Per ridurre la banda necessaria in trasmissione si introduce la modulazione in quadratura con offset temporale O-\ac{QAM}.
I simboli della sequenza numerica generati con una frequenza di cifra $f_s$ vengono inviati sul canale in quadratura con un ritardo di mezzo periodo di cifra $T_s/2$ rispetto al canale in fase: in tal modo non si può verificare una transizione contemporanea di polarità sulle due portanti, per cui la loro somma potrà differire al massimo di $\ang{90}$. Riducendo l'ampiezza delle discontinuità sfasando temporalmente le commutazioni si dimezza l'occupazione in banda.

\clearpage
\section{Frequency Shift Keying \ac{FSK}}\index{modulazione di frequenza!Frequency Shift Keying}
La modulazione di frequenza \ac{FSK} è una tecnica di modulazione per la trasmissione numerica digitale. Le forme d'onda trasmesse per codificare i simboli “0” e “1” sono sinusoidi a frequenze $f_0$ e $f_1$. Il generatore di forme d'onda nel modulatore \ac{FSK} può essere realizzato con due oscillatori alle frequenze $f_0$ o $f_1$ e uno switch commutatore comandato che seleziona la forma d'onda in base al bit da trasmettere (fig.~\ref{fig:FSK_modulatore_oscillatori}). In alternativa è possibile utilizzare un oscillatore controllato in tensione con in ingresso la forma d'onda rettangolare rappresentante i bit da trasmettere (fig.~\ref{fig:FSK_modulatore_VCO}).

La prima soluzione nella transizione tra oscillazioni a frequenze diverse si può presentare discontinuità nel segnale, data la difficoltà di commutare esattamente in corrispondenza degli zeri, introducendo componenti spettrali non legate all'informazione ma che causano una maggiore occupazione di banda.
Con un oscillatore controllato in tensione la variazione di frequenza nell'anello in retroazione varia con continuità, e  a meno di un transitorio in cui la forma d'onda risulta distorta sono assenti salti di tensione e componenti spettrali spurie.
Le due tecniche di realizzazione del modulatore \ac{FSK} portano oltre che ad una diversa occupazione di banda anche a diverse soluzioni in demodulazione.



\begin{figure}[!ht]
\centering\subfloat[\ac{GFO} con due oscillatori]{
\begin{tikzpicture}
\node[block](f0){$f_0$};
\coordinate[below=5mm of f0] (c0);
\coordinate[right=2cm of c0] (c1);
\coordinate[above=2mm of c1] (c2);
\coordinate[below=2mm of c1] (c3);
\coordinate[right=15mm of c1] (c4);
\node[block, below=5mm of c0](f1){$f_1$};
%\node[clock, below=1cm of c4](clk){}edge[->](c4);
\draw[-o](f0)-|(c2);
\draw[-o](f1)-|(c3);
\draw[o-](c1)--(c4);
\node[clock,below=5mm of c4](clk){};
\draw[decorate,decoration=snake](c1)+(2mm,5mm) -- +(12mm,5mm);
\draw[decorate,decoration={snake,segment length=5}](c1)+(2mm,-5mm) -- +(12mm,-5mm);
\end{tikzpicture}\label{fig:FSK_modulatore_oscillatori}}\quad\subfloat[\ac{GFO} con oscillatore controllato]{
\begin{tikzpicture}
\node[block](b0){$VCO$};
\coordinate[left=1cm of b0](c0){};
\node[clock,left=5mm of c0](clk){}edge[->](c0);
\coordinate[right=1cm of b0](c1);
\draw[-](b0)--(c1);
\draw[o-](c0)--node[above]{$0$}node[below]{$1$}(b0);
\draw[decorate,decoration=snake](c1)+(-5mm,5mm) -- +(5mm,5mm);
\draw[decorate,decoration={snake,segment length=5}](c1)+(-5mm,-5mm) -- +(5mm,-5mm);
\end{tikzpicture}\label{fig:FSK_modulatore_VCO}}
\caption{Modulatore \ac{FSK}}
\end{figure}

Il ricevitore è costituito da un circuito \keyword{demodulatore} molto semplice che non richiede il recupero di una \keyword{portante}. Un demodulatore incoerente si realizza confrontando l'uscita di due filtri passabanda adattati alle forme d'onda trasmesse centrati alle frequenze $f_0$ e $f_1$. Il decisore può così ricostruire la forma d'onda rettangolare in banda base che rappresenta la sequenza di simboli trasmessa.

\begin{figure}[!ht]\centering
\begin{tikzpicture}[>=latex',fitted/.style={draw,thick,dotted,inner sep=4mm,rounded corners}]
\node[passabanda,label=below:$f_0$](f0){};
\node[block, right=1cm of f0](i0){Rilevatore inviluppo} edge[<-](f0);
\coordinate[below=8mm of f0] (c2);
\coordinate[left=1.5cm of c2] (c1);
\coordinate[left=1.5cm of c1] (c0);
\draw[decorate,decoration=snake](c0)+(-5mm,5mm) -- +(5mm,5mm);
\draw[decorate,decoration={snake,segment length=5}](c0)+(-5mm,-5mm) -- +(5mm,-5mm);
\node[passabanda,label=below:$f_1$,below=8mm of c2](f1){};
\node[block, right=1cm of f1](i1){Rilevatore inviluppo}edge[<-](f1);
\node[block, right=55mm of c2](cc){Decisore $\gtrless$};
\coordinate[right=2cm of cc] (c3);
\draw[->](c0)--(c1)|-(f0);
\draw[->](c0)--(c1)|-(f1);
\draw[->](i0)-|(cc);
\draw[->](i1)-|(cc);
\draw[->](cc)--node[above]{$0/1$}(c3);
\node[fitted, fit=(c1)(f0)(f1)(i0)(i1)(cc),label=above:Demodulatore incoerente \ac{FSK}]{};
\end{tikzpicture}
\caption{Schema demodulatore incoerente \ac{FSK}}
\end{figure}

\clearpage
\section{Continuous Phase \ac{FSK}}\index{modulazione di frequenza!Continuous Phase FSK}
La codifica di linea della modulazione \ac{FSK} non trasmette forme d'onda in codifica antipodale il che causa una perdita di efficienza rispetto alla codifica antipodale della modulazione \ac{PSK} di almeno $\SI{3}{\decibel}$. Le discontinuità nella fase della portante della modulazione \ac{PSK} sono causa di inefficienza spettrale che è possibile superare con opportuni accorgimenti nella modulazione \ac{FSK}.

Al fine di ottenere una alta efficienza spettrale si definisce la modulazione \ac{CP-FSK} imponendo per costruzione l'ortogonalità delle forme d'onda trasmesse e la continuità di fase ad ogni periodo di simbolo $T$.

L'ortogonalità impone che le frequenze adottate per la trasmissione dei simboli “0” e “1” non possano essere svincolate dal tempo di simbolo, il periodo $T$, dovendo far rientrare un numero intero di oscillazioni nel periodo, e non possono essere svincolate tra loro, dovendo imporre ad ogni tempo di simbolo la continuità di fase del segnale.

\begin{figure}[ht!]
\label{fig:FSK_ortogonali}
\centering
\def\freqA{20}
\foreach\freqB[count=\c] in{40,45,50} {
\subfloat[$T=\frac{1}{10},f_0=\SI{\freqA}{\hertz},f_1=\SI{\freqB}{\hertz}$]{
 	\begin{tikzpicture}
 	\begin{axis}[xlabel=$f$,ylabel=$H_s(f)$,xtick={\freqA,\freqB},xticklabels={$f_0$,$f_1$},axis x discontinuity=crunch,xtickmin=10,ytick={1},yscale=.5,samples=500,domain=5:60]
 	\addplot[black] {sinc(x-\freqA,1)+sinc(x-\freqB,1)};
 	\addplot[red!50,densely dotted] {sinc(x-\freqA,1)};
 	\addplot[blue!50,densely dotted] {sinc(x-\freqB,1)};
 	\end{axis}
% 	\begin{scope}[xshift=-5cm]
% 	\begin{axis}[scale=.5,domain=0:.1,samples=256,ytick={-1,1},xtick={.1},xticklabels={$T$},xlabel=$t$,ylabel=$s(t)$]
% 	\addplot[red!50] {cos(2*pi*\freqA*x)};
% 	\addplot[blue!50]{cos(2*pi*\freqB*x)};
% 	\addplot[thick,black]{cos(2*pi*\freqA*x)*cos(2*pi*\freqB*x)};
% 	\end{axis}
% 	\end{scope}
 	\end{tikzpicture}}}
\end{figure}

L'\keyword[forme d'onda!ortogonalità]{ortogonalità} dalle forme d'onda trasmesse ha inoltre l'effetto di ridurre la densità spettrale di potenza\index{densità spettrale di potenza} dei lobi secondari dei $\sinc{}$ compresi tra le frequenze $f_0$ e $f_1$ (notare la differenza tra fig.~\ref{fig:FSK_ortogonali}b e fig.~\ref{fig:FSK_ortogonali}c).

Si ha l'ortogonalità imponendo che:
\begin{equation}
\intd{0}{T}{\cos{2\pi f_0 t}\cos{2\pi f_1 t}}{t}=0
\label{eq:sin_ortogonali}
\end{equation}
dove
\[ \cos{2\pi f_0 t}\cos{2\pi f_1 t}=\frac{1}{2}\cos{2\pi(f_0+f_1)t}+\frac{1}{2}\cos{2\pi(f_1-f_0)t} \]
Risolvendo l'integrale \ref{eq:sin_ortogonali} si ha
\[ \frac{1}{2}\bound{0}{T}{\frac{\sin{2\pi(f_0+f_1)t}}{2\pi(f_0+f_1)}}+\frac{1}{2}\bound{0}{T}{\frac{\sin{2\pi(f_1-f_0)t}}{2\pi(f_1-f_0)}}=0\]
dove i termini si annullano per
\[ f_0+f_1=\frac{n}{2 T} \qquad f_1-f_0=\frac{m}{2 T} \]

L'ortogonalità richiesta da \ac{CP-FSK} è garantita dall'integrale nullo quando $f_0$ e $f_1$ sono multipli interi differenti di $\frac{1}{2 T}$. Per \ac{FSK} è sufficiente che siano multipli interi differenti di $\frac{1}{T}$.

\section{Minimum Shift Keying \ac{MSK}}\index{modulazione di frequenza!Minimum Shift Keying}
Tra le possibili scelte consentite dalla modulazione \ac{CP-FSK} quella che garantisce la minima occupazione di banda si ha imponendo la minima differenza tra le frequenze $f_0=\frac{n}{2T}$ e $f_1=\frac{n+1}{2T}$ pari a $f_1-f_0=\frac{1}{2T}$.
Per tali frequenze i lobi principali della densità spettrale di potenza\index{densità spettrale di potenza} hanno distanza minima (fig.~\ref{fig:FSK_ortogonali}a).

La portante modulata assume la seguente espressione:
\begin{equation}
s_T(t)=\sum_{m}{\cos{2\pi f_P t + b_m\frac{2\pi}{4T}t}\rect{\frac{t-m T}{T}}}
\label{eq:FSK_portante_modulata}
\end{equation}
dove la frequenza della portante è $f_P=(f_0+f_1)/2$ e il coefficiente $b_m$ codifica i bit di informazione secondo la convenzione:
\[ b_m=\begin{cases}
+1 & \text{“1”}\\-1 & \text{“0”}
\end{cases} \]

Tale espressione garantisce l'ortogonalità tra le forme d'onda ma non soddisfa la continuità di fase della portante, perché la sinusoide a frequenza $f_1=f_0+\frac{1}{2T}$ ha mezzo periodo in più rispetto a quella a frequenza $f_0$, come in fig.~\ref{fig:FSK_forme_d'onda}.

\begin{figure}[ht]\centering
\def\freqP{1}
\begin{tikzpicture}
\begin{axis}[domain=0:8,samples=200,xscale=1.25,yscale=.66,xlabel=$t$,xtick={1,...,8},xticklabels=\empty,ytick={-1,1},xmajorgrids,extra x ticks={.5,1.5,2.5,3.5,4.5,5.5,6.5,7.5},extra x tick labels={0,1,1,0,0,1,0,1},extra x tick style={grid=none}]
\addplot[black,domain=0:1]{cos(2*pi*\freqP*x-pi*x/2)};
\addplot[black,domain=1:2]{cos(2*pi*\freqP*x+pi*x/2)};
\addplot[black,domain=2:3]{cos(2*pi*\freqP*x+pi*x/2)};
\addplot[black,domain=3:4]{cos(2*pi*\freqP*x-pi*x/2)};
\addplot[black,domain=4:5]{cos(2*pi*\freqP*x-pi*x/2)};
\addplot[black,domain=5:6]{cos(2*pi*\freqP*x+pi*x/2)};
\addplot[black,domain=6:7]{cos(2*pi*\freqP*x-pi*x/2)};
\addplot[black,domain=7:8]{cos(2*pi*\freqP*x+pi*x/2)};
\end{axis}
\end{tikzpicture}
\caption{Forme d'onda \ac{MSK}}
\label{fig:FSK_forme_d'onda}
\end{figure}

\`{E} possibile ottenere una codifica CP-\ac{MSK} eliminando le discontinuità di fase imponendo in presenza dei salti di fase il cambio del segno della forma d'onda, ovvero aggiungendo un contributo di fase pari a $\pi$. Tale modo di procedere non modifica l'informazione trasmessa in quanto tale informazione è legata alla frequenza e non alla fase della portante. Si modifica l'espressione della portante eq.~\ref{eq:FSK_portante_modulata} iniettando il contributo di fase $\phi_m$ che assume valore $0$ o $\pi$ tale da impedire i salti di fase (fig.~\ref{fig:FSK_forme_d'onda_fase_continua}):
\begin{equation}
s_T(t)=\sum_{m}{\cos{2\pi f_P t + b_m\frac{2\pi}{4T}t + \phi_m}\rect{\frac{t-m T}{T}}}
\label{eq:FSK_portante_modulata_fase_continua}
\end{equation}

Si ottiene così l'ortogonalità e la continuità di fase con il minimo valore di deviazione della frequenza, da cui il nome di modulazione CP-\ac{MSK} a spettro compatto.

\begin{figure}[ht]\centering
	\def\freqP{1}
	\begin{tikzpicture}
	\begin{axis}[domain=0:8,samples=200,xscale=1.25,yscale=.66,xlabel=$t$,xtick={1,...,8},xticklabels=\empty,ytick={-1,1},xmajorgrids,extra x ticks={.5,1.5,2.5,3.5,4.5,5.5,6.5,7.5},extra x tick labels={0,1,1,0,0,1,0,1},extra x tick style={grid=none}]
	\addplot[black,domain=0:1]{cos(2*pi*\freqP*x-pi*x/2+pi)};
	\addplot[black,domain=1:2]{cos(2*pi*\freqP*x+pi*x/2)};
	\addplot[black,domain=2:3]{cos(2*pi*\freqP*x+pi*x/2)};
	\addplot[black,domain=3:4]{cos(2*pi*\freqP*x-pi*x/2+pi)};
	\addplot[black,domain=4:5]{cos(2*pi*\freqP*x-pi*x/2+pi)};
	\addplot[black,domain=5:6]{cos(2*pi*\freqP*x+pi*x/2)};
	\addplot[black,domain=6:7]{cos(2*pi*\freqP*x-pi*x/2)};
	\addplot[black,domain=7:8]{cos(2*pi*\freqP*x+pi*x/2+pi)};
	\end{axis}
	\end{tikzpicture}
	\caption{Forme d'onda \ac{MSK} con continuità di fase}
	\label{fig:FSK_forme_d'onda_fase_continua}
\end{figure}

La modulazione \ac{M-FSK} è l'estensione a più livelli della modulazione \ac{FSK} con la trasmissione su $M$ portanti ortogonali. Un banco di filtri in ricezione farà passare solo un segnale alla frequenza trasmessa individuando il livello corrispondente e la configurazione di bit associata.

\begin{figure}[!ht]\centering
	\begin{tikzpicture}[scale=0.8,>=latex',fitted/.style={draw,thick,dotted,inner sep=4mm,rounded corners}]
	\node[passabanda,label=below:$f_0$](f0){};
	\node[block, right=1cm of f0](i0){Rilevatore inviluppo}edge[<-](f0);
	\node[mult,left=1cm of f0](p0){}edge[->](f0);
	\node[below=7mm of p0](m0){$2\cos{\omega_0 t}$}edge[<-](p0);
	\coordinate[below=1cm of p0] (c2);
	\coordinate[left=1cm of c2] (c1);
	\coordinate[left=15mm of c1] (c0);
	\draw[decorate,decoration=snake](c0)+(-5mm,5mm) -- +(5mm,5mm);
	\draw[decorate,decoration={snake,segment length=5}](c0)+(-5mm,-5mm) -- +(5mm,-5mm);
	\node[mult,below=1cm of c2](p1){};
	\node[passabanda,label=below:$f_1$,right=1cm of p1](f1){}edge[<-](p1);
	\node[below=7mm of p1](m1){$2\cos{\omega_1 t}$}edge[<-](p1);
	\node[block, right=1cm of f1](i1){Rilevatore inviluppo}edge[<-](f1);
	\node[block, right=75mm of c2](cc){Decisore $\gtrless$};
	\coordinate[right=15mm of cc] (c3);
	\draw[->](c0)--(c1)|-(p0);
	\draw[->](c0)--(c1)|-(p1);
	\draw[->](i0)-|(cc);
	\draw[->](i1)-|(cc);
	\draw[->](cc)--node[above]{$0/1$}(c3);
	\node[fitted, fit=(c1)(f0)(f1)(i0)(i1)(cc)(m0)(m1),label=above:Demodulatore coerente \ac{FSK}]{};
	\end{tikzpicture}
	\caption{Schema demodulatore coerente \ac{FSK}}
\end{figure}

Una ulteriore estensione consente la trasmissione in parallelo su $M$ portanti ortogonali ciascuna modulata \ac{QAM}. Ogni canale dello spettro ha il suo rapporto \ac{SNR} a seconda del quale si sceglie la migliore modulazione M-\ac{QAM}. Adattando la modulazione al rapporto segnale rumore variabile nel tempo si ottiene la modulazione \ac{OFDM} utilizzata nelle comunicazioni Wi-Fi e \ac{ADSL}.

\section{Demodulazione coerente}\index{demodulazione coerente}
L'utilizzo di portanti in quadratura nella modulazione \ac{QAM} e nella necessità di rilevare la fase nella modulazione \ac{PSK} richiede in ricezione una demodulazione coerente: è necessaria una oscillazione sincrona con la portante di riferimento per la fase locale al ricevitore. Tale informazione deve essere estratta direttamente dal segnale ricevuto.

Non è possibile pensare di filtrare ad esempio un segnale in modulazione \ac{PSK} binaria $\cos{\omega_0 t+a_i\pi}$ ($a_i=0$ se trasmetto “0” e $a_i=1$ se trasmetto “1”) (fig.~\ref{fig:2PSK_modulazione}) con un filtro molto stretto centrato attorno alla frequenza $f_0$: un filtro a banda molto stretta media i valori del segnale nel tempo su intervalli temporali lunghi fornendo un risultato a media nulla.

Per estrarre un riferimento di fase dalla portante è possibile utilizzare due procedure: raddrizzare il segnale ricevuto, a onda intera o a semionda, per ottenere oscillazioni risonanti alla frequenza $2f_0$; effettuare la quadratura del segnale ottenendo $\cos[2]{2\pi f_0 t+a_i\pi}=\frac{1}{2}+\frac{1}{2}\cos{4\pi f_0 t+a_i 2\pi}$ ed isolare la sinusoide a frequenza $2f_0$. Tali procedure si iterano per modulazioni di fase multilivello M-\ac{PSK}.

\begin{esempio}
Nella modulazione 4-\ac{PSK} si trasmette il segnale $\cos{\omega_0 t+a_i\frac{\pi}{4}}=\cos{\omega_0 t+\frac{\pi}{4}+b_i\frac{\pi}{2}}$ con $a_i=1,3,5,7$, $b_i=0,1,2,3$.

Effettuando la quadratura si ottengono sinusoidi a frequenza doppia e quadrupla della portante:
\[\begin{split}\cos[2]{\omega_0 t+\frac{\pi}{4}+b_i\frac{\pi}{2}}=\frac{1}{2}+\frac{1}{2}\cos{2\omega_0 t+\frac{\pi}{2}+b_i\pi} \\ \cos[2]{2\omega_0 t+\frac{\pi}{2}+b_i\pi}=\frac{1}{2}+\frac{1}{2}\cos{4\omega_0 t+\pi+2b_i\pi}\end{split}\]
Non è possibile risalire alla fase iniziale $b_i\frac{\pi}{2}+\frac{\pi}{4}$. $\square$
\end{esempio}

All'uscita del filtro è necessario quindi un divisore di frequenza che è possibile realizzare con un circuito flip-flop. Tale operazione rende indeterminata la fase del segnale M-\ac{PSK} che è un multiplo di $2\pi/M$, con il risultato di ricostruire la corretta sequenza di bit o la sequenza complementare. Per risolvere tale problema è possibile inviare delle configurazioni binarie note sia al trasmettitore che al ricevitore tali da poter verificare l'inversione di segno  nell'oscillazione locale e ottenere il corretto sincronismo.

\begin{figure}[!ht]
\centering
\subfloat[Segnale \ac{PSK} binario $\cos{\omega_0 t+a_i\pi}$]{
	\begin{tikzpicture}
	\begin{axis}[enlargelimits,yscale=0.5,xlabel=$t$,ylabel=$s_T(t)$,xtick={1},ytick={-1,1},xmajorgrids,black,samples=300]
	\addplot[domain=0:1] {sin((2*pi*3*x))};
	\addplot[domain=1:2] {sin((2*pi*3*x+pi))};
	\end{axis}
	\end{tikzpicture}}%
\quad%
\subfloat[Raddrizzamento ad onda intera]{
\begin{tikzpicture}
\begin{axis}[enlargelimits,yscale=0.5,xlabel=$t$,ylabel=$s_T(t)$,xtick={1},ytick={-1,1},xmajorgrids,black,samples=500]
\addplot[domain=0:2] {abs(sin((2*pi*3*x)))};
\end{axis}
\end{tikzpicture}
}

\subfloat[Raddrizzamento a semionda]{
	\begin{tikzpicture}
	\begin{axis}[enlargelimits,yscale=0.4,xlabel=$t$,ylabel=$s_T(t)$,xtick={1},ytick={-1,1},xmajorgrids,black,samples=300]
	\addplot[domain=0:1] {sin((2*pi*3*x))>0?sin((2*pi*3*x)):0};
	\addplot[domain=1:2] {sin((2*pi*3*x+pi))>0?sin((2*pi*3*x+pi)):0};
	\end{axis}
	\end{tikzpicture}}%
\quad%
\subfloat[Quadratura del segnale]{
	\begin{tikzpicture}
	\begin{axis}[enlargelimits,yscale=0.4,xlabel=$t$,ylabel=$s_T(t)$,xtick={1},ytick={0,1},xmajorgrids,black,samples=300]
	\addplot[domain=0:2] {(sin((2*pi*3*x)))^2};
	\end{axis}
	\end{tikzpicture}}
\caption{Estrazione della portante da segnale \ac{PSK} binario}
\end{figure}

\section{Modulazioni differenziali}\index{modulazione differenziale}
\`{E} possibile svincolarsi dalla necessità di estrarre la fase assoluta dalla portante mediante la codifica \ac{DPSK}: si associano i simboli “0” e “1” alle transizione di fase piuttosto che al valore assoluto della fase. Nel \ac{DPSK} binario in trasmissione si fa cambiare la fase della portante rispetto alla fase del periodo precedente per trasmettere “1”, si lascia invariata la fase per trasmettere “0”. Nella modulazione 4-\ac{DPSK} si lascia invariata la fase per trasmettere la coppia di bit “00” e si cambia la fase di multipli di $\pi/2$ per le successive configurazioni.

Per realizzare la modulazione \ac{DPSK} si può utilizzare il modulatore \ac{PSK} a patto di modificare la sequenza di binaria in ingresso, operando una complementazione del bit in funzione del bit inviato al modulatore nel periodo precedente, ottenibile con un circuito dotato di uno operatore logico XOR e un delay (fig.~\ref{fig:DPSK_circuiti}). Similmente in ricezione bisogna effettuare l'operazione duale ricostruendo la sequenza di bit di informazione, verificando la permanenza o la variazione della fase tra periodi successivi.
\begin{figure}[!ht]
	\centering
	\subfloat[Codificatore \ac{DPSK}]{
		\begin{circuitikz}[scale=.5, transform shape]
			\draw (0,1.5)node[american xor port](xor){}
			(xor.in 2)|-(-1.25,0) to[twoport,t=$\Delta T$] (0,0)-|(xor.out) to[short,-o]++(.5,0) (xor.in 1)to[short,-o]++(-.5,0);
		\end{circuitikz}}
		\quad\subfloat[Decodificatore \ac{DPSK}]{		
			\begin{circuitikz}[scale=.5, transform shape]
				\draw (0,0) node[american xor port](xor){}
				(xor.in 1)--++(-2,0)|- ++(0,-1.5)to[twoport,t=$\Delta T$]++(2,0)-|(xor.in 2) (xor.in 1)to[short,-o]++(-2.5,0) (xor.out)to[short,-o]++(.5,0);
			\end{circuitikz}}
			\caption{Circuiti per codifica e decodifica differenziale}
			\label{fig:DPSK_circuiti}
\end{figure}
Il vantaggio ottenuto con la codifica differenziale di non essere legata al valore assoluto della fase si paga in caso di errore su un bit al decisore: tale errore causa l'errata decodifica di un numero di bit almeno pari al numero di divisioni di frequenza.

\clearpage
\section{Esercizio dimensionamento ponte radio}
\begin{esercizio}
Si vuole dimensionare un ponte radio per la trasmissione numerica digitale alla velocità di $\SI{100}{\mega\bit\per\second}$ con una probabilità d'errore massima su bit pari a $p(\epsilon)=\num{e-7}$ su una distanza di $\SI{100}{\kilo\meter}$ e una probabilità di fuori servizio $p_\text{FS}=\num{e-3}$. Si consideri la possibilità di spezzare il collegamento in due tratte di $\SI{50}{\kilo\meter}$ per esigenze di visibilità.
\end{esercizio}

\begin{figure}[!ht]
	\centering
%	\resizebox{\textwidth}{!}{
		\begin{tikzpicture}[>=latex',fitted/.style={draw,thick,dotted,inner sep=4mm,rounded corners}]
		\coordinate(c0);
		\node[mult,right=1cm of c0](m0){};
		\node[below=1cm of m0](t0){$2\cos{\omega t}$}edge[->](m0);
		\node[passabasso,right=1cm of m0](b1){};
		\node[campionatore,right=1cm of b1](b2){};
		\node[decisore,right=1cm of b2](b3){};
		\coordinate[right=1.5cm of b3](c1);
		\draw[dot=A](c0)--(m0);
		\draw[dot=$\restrict{\frac{S}{N}}{\text{in}}$](m0)--(b1);
		\draw[dot=$\restrict{\frac{S}{N}}{\text{out}}$](b1)--(b2);
		\draw[dot=$\restrict{\frac{S}{N}}{\text{dec}}$](b2)--(b3);
		\draw[->](b3)--node[above,near end]{$p(\epsilon)$}(c1);
		\end{tikzpicture}
%	}
	\caption{Schema per il dimensionamento del ricevitore di sistema di trasm. numerica su ponte radio}
\end{figure}

Scopo del dimensionamento è determinare la potenza da impegnare in trasmissione per ottenere determinati rapporti segnale rumore nei diversi punti del sistema e rispettare i vincoli di progetto, quali la velocità di trasmissione nel numero di simboli o bit al secondo, la distanza da coprire, e i livelli minimi di qualità da garantire in ricezione quali la probabilità di errore su bit e la probabilità di fuori servizio.

Sono necessarie numerose ipotesi progettuali motivate dai vari aspetti dei fondamenti delle telecomunicazioni, che portano a scegliere il numero di tratte, la modulazione, la tipologia di codifica di linea, tenendo conto delle caratteristiche del mezzo trasmissivo.

\subsection{Risoluzione con singola tratta e modulazione \ac{PSK} binaria}
A partire dal requisito sulla probabilità di errore all'uscita del sistema, causata dal rumore termico gaussiano bianco, ipotizzando una codifica antipodale del segnale in modulazione \ac{PSK} binaria, resta individuato il rapporto segnale rumore al decisore pari a $\gamma^2$ (vedi eq.~\ref{eq:rapporto_segnale_rumore_uscita_filtro_adattato})
\[p(\epsilon)=\f{Q}{\gamma}=\f{Q}{\frac{2a}{2\sigma_n}}=\f{Q}{\frac{a}{\sigma_n}}=\num{e-7}\]
che corrisponde esprimendo in decibel $\restrict{\gamma^2}{\text{dB}}=10\Log\frac{a}{\sigma_n}=\SI{14.5}{\decibel}$.

Nell'ipotesi di modulazione \ac{PSK}, per la massima efficienza di amplificazione di potenza di forme d'onda rettangolari, si utilizzerà un filtro di ricezione non adattato, che causa una attenuazione per il fattore di forma di $\SI{0.5}{\decibel}$ (vedi cap.~\ref{cap:rinuncia_filtro_adattato}).

Risalendo all'ingresso del filtro di ricezione si ottiene un rapporto segnale rumore pari a $\restrict{S/N}{\text{in}}=\frac{a^2}{h_n f_s/2}=\SI{15}{\decibel}$ essendo la potenza di picco del segnale, prima e dopo il demodulatore, pari al quadrato dell'ampiezza $a^2$. In ingresso al ricevitore prima del demodulatore si ha un segnale sinusoidale di potenza media, sia in fase che quadratura, pari a $P_R=a^2/2$.

La potenza del rumore è calcolata moltiplicando la banda monolatera $f_s/2$ per la densità spettrale di rumore, tenendo conto che la densità spettrale di rumore nella banda passante del mezzo trasmissivo dopo la demodulazione si somma in banda base risultando in $h_n=2 h'_n$, e da cui risulta
\[\restrict{\frac{S}{N}}{\text{A}}=\frac{a^2/2}{h_n f_s/2}=\frac{1}{2}\frac{a^2}{h_n f_s/2}=\frac{1}{2}\restrict{\frac{S}{N}}{\text{in}}=\SI{15}{\decibel}-\SI{3}{\decibel}=\SI{12}{\decibel}\]
con l'ipotesi di un fattore di rumore degli apparati pari a $F=\SI{10}{\decibel}$ la densità spettrale di rumore equivalente $h_n$ (ovvero rapportata alla banda monolatera $f_s/2$) è pari prima del campionatore a $F k T_0$, ovvero in decibel
\[h_n \frac{f_s}{2}=\underbrace{-174}_{k T_0}+\underbrace{10}_{F}+\underbrace{3}_{\cdot 2}+\underbrace{10\Log{50\cdot 10^6}}_{\frac{f_s}{2}}=\SI{-84}{\dBm}\]
Resta determinata così la potenza minima del segnale portante da garantire in ingresso al ricevitore che dovrà essere pari a $P_\text{R}=12-84=\SI{-72}{\dBm}$.

Si passa a dimensionare il trasmettitore per determinare la potenza della portante da trasmettere che sia sufficiente a maggiorare l'attenuazione del mezzo trasmissivo (cap.~\ref{cap:canale_radio_attenuazione_spazio_libero}) e l'attenuazione dovuta ai cammini multipli (cap.~\ref{cap:canale_radio_effetti_cammini_multipli}):
\[p_T=p_R+\alpha_\text{SL}+\alpha_\text{suppl}\]

L'attenuazione del mezzo trasmissivo è dovuta alla divergenza sferica del segnale radio nello spazio libero tra le antenne, ed è funzione della distanza e dall'area efficace delle antenne (eq.~\ref{eq:radio_attenuazione_spazio_libero}). Supponendo che le antenne in ricezione e trasmissione siano uguali, con un guadagno $G=\SI{30}{\decibel}$ alla frequenza di cifra $f_s=\SI{1}{\giga\hertz}$ (data dalla velocità di trasmissione), resta determinata l'area efficace dell'antenna (per eq.~\ref{eq:radio_legame_area_efficace_guadagno_antenna}) $A=\frac{G c^2}{4\pi f^2}=\SI{7.16}{\meter\squared}=\SI{8.5}{\decibel\meter\squared}$, per cui l'attenuazione di spazio libero:
\[\alpha_\text{SL}=10\Log(4\pi(\SI{100}{\kilo\meter})^2)-\restrict{G}{dB}-\restrict{A}{dB}=111-30-8.5=\SI{72.5}{\decibel}\]

Va considerata l'attenuazione supplementare causata dai cammini multipli in radiofrequenza che fanno giungere al ricevitore segnali con fasi incoerenti, risultando in una potenza di picco descrivibile da una variabile aleatoria esponenziale. La probabilità di fuori servizio $p_\text{FS}=\num{e-3}$, data come requisito del sistema, coincide con la probabilità che l'ampiezza del segnale risultante sia inferiore ad una soglia ovvero che la potenza picco-picco del segnale ricevuto sia inferiore alla potenza di fuori servizio $P_\text{FS}$. Troncando al primo termine la serie di Taylor si ottiene la relazione tra probabilità di fuori servizio e il margine di potenza per superare l'attenuazione supplementare
\[ p_\text{FS}=p(P_R^\text{PP}<\gamma)=1-\e{\frac{\gamma}{P_0}}\cong\frac{\gamma}{P_0}=\frac{1}{\alpha_\text{suppl}}\implies\alpha_\text{suppl}=\frac{1}{p_\text{FS}}=\SI{30}{\decibel}\]

Il sistema di trasmissione numerico digitale dimensionato per un ponte radio su singola tratta da 100 km richiede una potenza di trasmissione
\[p_T=\SI{-72}{\dBm}+\SI{30}{\decibel}+\SI{72.5}{\decibel}=\SI{30.5}{\decibel}\]

\subsection{Risoluzione multitratta digitale}
\`{E} possibile realizzare lo stesso sistema di trasmissione sotto l'ipotesi di ponte radio multitratta digitale: si suddivide la distanza massima in più tratte in cascata, utilizzando apparecchiature intermedie che ricevono, amplificano e ritrasmettono il segnale da una tratta alla successiva, oppure di un sistema di rigenerazione numerico che ricostruito il segnale digitale lo ritrasmette alla tratta successiva. Nel secondo caso la probabilità d'errore dell'intero sistema si può approssimare con la probabilità d'errore della tratta maggiormente affetta da errore.
\[p_T(\epsilon)\cong p_S(\epsilon)\]

Il dimensionamento viene effettuato nell'ipotesi che la probabilità d'errore del sistema sia imputabile alla tratta peggiore. Questo modello riprende il concetto del fuori servizio: la tratta fuori servizio mette fuori servizio l'intero sistema di trasmissione.

Il dimensionamento del sistema di ricezione della singola tratta non cambia al variare della lunghezza della tratta: per garantire una certa probabilità di errore su bit, data la rumorosità del sistema di ricezione e il tipo di codifica e modulazione, si deve sempre ricevere una potenza di ricezione minima di $P_R=\SI{-72}{\dBm}$.

Il dimensionamento del sistema di trasmissione considera una attenuazione di spazio libero su una tratta di 50km che risulta essere $1/4$ di quella da 100 km, pertanto  \[\alpha_\text{SL}=10\Log(4\pi R^2)-\restrict{G}{dB}-\restrict{A}{dB}=105-30-8.5=\SI{66.5}{\decibel}\]
mentre l'attenuazione supplementare cambia, non per una diversa statistica dei cammini multipli, ma perché è dimezzata la probabilità di fuori servizio sulla singola tratta (infatti considerando trascurabile la probabilità che siano fuori servizio più tratte si ha $p_{\text{FS}}=p_{\text{fs}1}(1-p_{\text{fs}2})+(1-p_{\text{fs}1})p_{\text{fs}2}=1-(1-p_{\text{fs}1})(1-p_{\text{fs}2})=p_{\text{fs}1}+p_{\text{fs}2}$),
avendo tratte uguali $p_{\text{fs}1}=p_{\text{fs}2}=p_{\text{FS}}/2$ per cui aumenta l'attenuazione supplementare $\alpha_\text{suppl}=\SI{33}{\decibel}$.

La potenza di trasmissione sulla singola tratta sarà
\[P_T=-72+66.5+33=\SI{27.5}{\decibel}\]
che consente di risparmiare $\SI{3}{\decibel}$ per tratta, con una potenza trasmessa complessiva inalterata.

\subsection{Risoluzione con modulazione 4-\ac{PSK}}
In alternativa è possibile dimezzare la frequenza di simbolo portandola da $\SI{100}{\mega\bit}$ a $\SI{50}{\mega\bit}$ adottando la modulazione 4-\ac{PSK} con la trasmissione e ricezione su due canali con portanti in quadratura.

\begin{figure}[!ht]\centering
	\begin{tikzpicture}[>=latex',fitted/.style={draw,thick,dotted,inner sep=4mm,rounded corners}]
	\node[passabanda,label=below:$f_0$](f0){};
	\node[mult,left=1cm of f0](p0){}edge[->](f0);
	\node[below=7mm of p0](m0){$2\cos{\omega_0 t}$}edge[->](p0);
	\node[campionatore,right=1cm of f0](q0){};
	\coordinate[below=1cm of p0] (c2);
	\coordinate[left=1cm of c2] (c1);
	\coordinate[left=15mm of c1] (c0);
	\node[mult,below=1cm of c2](p1){};
	\node[passabanda,label=below:$f_0$,right=1cm of p1](f1){}edge[<-](p1);
	\node[below=7mm of p1](m1){$2\sen{\omega_0 t}$}edge[->](p1);
	\node[campionatore,right=1cm of f1](q1){};
	\coordinate[right=10mm of q0] (c4);
	\coordinate[right=10mm of q1] (c5);
	\node[decisore, right=55mm of c2](cc){};
	\coordinate[right=15mm of cc] (c3);
	\draw[dot=$\restrict{\frac{S}{N}}{\text{in}}$](p0)--(f0);
	\draw[dot=$\restrict{\frac{S}{N}}{\text{in}}$](p1)--(f1);
	\draw[dot=$\restrict{\frac{S}{N}}{\text{out}}$](f0)--(q0);
	\draw[dot=$\restrict{\frac{S}{N}}{\text{out}}$](f1)--(q1);
	\draw[dot=$\restrict{\frac{S}{N}}{\text{dec}}$](q0)--(c4);
	\draw[dot=$\restrict{\frac{S}{N}}{\text{dec}}$](q1)--(c5);
	\draw[->](c0)--(c1)|-(p0);
	\draw[->](c0)--(c1)|-(p1);
	\draw[->](c4)-|(cc);
	\draw[->](c5)-|(cc);
	\draw[->](cc)--node[above]{$0/1$}(c3);
	\node[fitted, fit=(c1)(f0)(f1)(i0)(i1)(q0)(q1)(cc)(m0)(m1),label=above:Demodulatore coerente \ac{FSK}]{};
	\end{tikzpicture}
	\caption{Schema demodulatore coerente 4-\ac{PSK}}
\end{figure}

Data la probabilità di errore $p(\epsilon)=\f{Q}{\gamma}=\num{e-7}$ bisogna dimensionare la potenza di picco del segnale delle componenti in fase e quadratura in modo tale da distanziare dai $N=2$ simboli adiacenti (eq.~\ref{eq:qam_mpsk_biterrorrate}),
\[p(\epsilon)=N\f{Q}{\frac{d}{2\sigma_n}}\]
Resta individuato così il rapporto segnale rumore al decisore è \[\restrict{\frac{S}{N}}{\text{dec}}=\gamma^2=10\Log{p(\epsilon)}+\Log N=\SI{14.5}{\decibel}+\SI{0.3}{\decibel}=\SI{14.8}{\decibel}\]
Il rapporto segnale rumore in ingresso al filtro di ricezione è
\[\restrict{\frac{S}{N}}{\text{in}}=\frac{(d/4)^2}{h_n f_s/2}=\SI{15.3}{\decibel}\]
prima del demodulatore perdo $\SI{3}{\decibel}$
\[\restrict{\frac{S}{N}}{\text{A}}=\frac{1}{2}\restrict{\frac{S}{N}}{\text{in}}=\SI{12.3}{\decibel}\]
La densità spettrale di rumore equivalente per una banda monolatera a frequenza di simbolo dimezzata su due canali è
\[h_n \frac{f_s}{2}=\underbrace{-174}_{k T_0}+\underbrace{10}_{F}+\underbrace{3}_{\cdot 2}+\underbrace{10\Log{25\cdot 10^6}}_{\frac{f_s}{2}}=\SI{-87}{\dBm}\]
Si è determinata così la potenza minima di ingresso al ricevitore che dovrà essere pari a $P_R=\SI{12.3}{\dB}-\SI{87}{\dBm}=\SI{-74.7}{\dBm}$
\chapter{Sistemi di telecomunicazione su canale in fibra ottica}
\section{Introduzione}
I sistemi di telecomunicazione in fibra ottica sono basati sulla trasmissione di radiazione luminosa convogliata in un mezzo trasmissivo con banda passante a frequenze dell'ordine di $\SI{e14}{\hertz}$. La fibra è composta da un \keyword[fibra ottica!nucleo]{nucleo} circondato da un involucro che forma il \keyword[fibra ottica!mantello]{mantello}, realizzati con materiali dielettrici ottenuti con differenti tipi di drogaggio della silice per ottenere diversi indici di rifrazione, e da una \keyword[fibra ottica!guaina]{guaina} protettiva esterna che fornisce robustezza fisica, resistenza meccanica, isolamento termico e protezione dall'umidità.

Un raggio monocromatico di luce che incide con un angolo $\theta_1$ rispetto alla normale alla superficie di separazione di materiali a diverso indice di rifrazione emerge con un angolo $\theta_2$ tale da rispettare la \keyword[fibra ottica!legge di Snell]{legge di Snell}:
\begin{equation}
n_1\Sen{\theta_1}=n_2\Sen{\theta_2}
\end{equation}

\begin{figure}[!ht]
\centering
\subfloat{
\begin{tikzpicture}[>=latex',scale=.8]
\fill[pattern=north west lines](-6,-1)rectangle(6,-1.5);
\fill[pattern=north west lines](-6,1)rectangle(6,1.5);
\draw(-6,-1)--(6,-1)(-6,-1.5)--(6,-1.5)(-6,1)--(6,1)(-6,1.5)--(6,1.5);
\draw[dashed](-6,0)--(6,0);
\draw[->](-6,0)--(-3,1);
\draw[->](-3,1)--(3,-1);
\draw[->](3,-1)--(6,0);
\node at(0,1.25){$n_2$};
\node at(0,-1.25){$n_2$};
\node at(0,0.5){$n_1$};
\end{tikzpicture}}\quad\subfloat{
\begin{tikzpicture}
\begin{axis}[xscale=.5,scale=.5,xlabel=$n$,ylabel=$r$,xtick=\empty,ytick=\empty,xmin=0,xmax=2.5]
\addplot[black] coordinates {(0,-1.5)(1,-1.5)(1,-1)(2,-1)(2,1)(1,1)(1,1.5)(0,1.5)};
\end{axis}
\end{tikzpicture}
}
\caption{Fibra ottica \keyword[fibra ottica!step index]{step index}}
\end{figure}

Si definisce \keyword[fibra ottica!angolo critico]{angolo critico} $\theta_c$ l'angolo in corrispondenza del quale la radiazione che incide su una superficie con indice di rifrazione $n_2>n_1$ non viene trasmessa nel mantello ma si propaga lungo la superficie di discontinuità dei due materiali (blu in fig.~\ref{fig:legge_Snell}). Se l'angolo di incidenza è maggiore dell'angolo critico (rosso in fig.~\ref{fig:legge_Snell}), per $\theta_1>\theta_c=\f{\arcsen}{n_2/n_1}$, il raggio è deviato al punto da risultare completamente riflesso dalla superficie di separazione rimanendo intrappolato nel nucleo.
\begin{figure}[!ht]
\centering
\begin{tikzpicture}[>=latex',scale=0.8]
\draw(-3,0)--(3,0);
\draw[dashed](0,-2)--(0,2.5);
\draw[<-](0,0)--(120:2.5);
\draw[<-,blue](0,0)--(135:2.5);
\draw[<-,red](0,0)--(150:2.5);
\draw[->](0,0)--(-30:2.5);
\draw[->,blue](0,0)--(0:2.5);
\draw[->,red](0,0)--(30:2.5);
\node at(3,2.5){$n_1$};
\node at(3,-2){$n_2$};
\draw pic{carc=90:120:1:$\theta_1$};
\draw pic{carc=-90:-30:1:$\theta_2$};
\end{tikzpicture}
\caption{Legge di Snell}
\label{fig:legge_Snell}
\end{figure}

La fibra ottica è in grado di comportarsi in modo quasi ideale fino ad alte frequenze in quanto i fenomeni ottici non sono affetti da attenuazioni tipiche dei trasportatori di carica nei materiali conduttori, come le collisioni degli elettroni che causano una resistenza elettrica con dissipazione di energia termica (effetto Joule), e il rallentamento dei portatori per addensamento sulla superficie esterna dei conduttori in presenza di forti campi elettrici (effetto pelle).

La capacità trasmissiva delle fibre ottiche è limitata dall'effetto di \keyword[fibra ottica!dispersione modale]{dispersione modale}. I raggi che si propagano con velocità costante $v$ in un mezzo trasmissivo con indice di rifrazione costante $n=c/v$ percorrono una distanza che cambia in funzione dell'angolo di incidenza: ogni raggio uscirà dal nucleo con un ritardo di propagazione diverso in funzione del numero di riflessioni che ha subito. Per minimizzare la dispersione modale, proporzionale alla lunghezza della fibra, è possibile utilizzare fibre ottiche \keyword[fibra ottica!monomodale]{monomodali} con un nucleo ridotto ad un diametro inferiore ai $\SI{10}{\micro\meter}$ oppure utilizzare fibre \keyword[fibra ottica!graded index]{graded index} con indice di rifrazione che varia con continuità. I raggi vengono curvati nel nucleo piuttosto che essere riflessi da una superficie di separazione netta. Si cerca di compensare il percorso più lungo dei raggi che viaggiano per le regioni esterne della fibra aumentando la velocità di propagazione con un minore drogaggio del materiale. Essendo la velocità di propagazione inversamente proporzionale alla radice quadrata del'indice di rifrazione, è possibile stabilire con precisione la curva di drogaggio.

\begin{figure}[!ht]
	\centering
	\subfloat{
		\begin{tikzpicture}[>=latex',scale=.8,ray/.style={decoration={markings,mark=between positions 0 and 1 step 2cm with {\arrow{stealth}}}}]
		\fill[pattern=north west lines](-6,-1)rectangle(6,-1.5);
		\fill[pattern=north west lines](-6,1)rectangle(6,1.5);
		\draw(-6,-1)--(6,-1)(-6,-1.5)--(6,-1.5)(-6,1)--(6,1)(-6,1.5)--(6,1.5);
		\foreach\i[evaluate=\i as \x using {1/sqrt(1-(\i/10-1)^2)-1.}] in{1,...,9} {
			\draw[dotted](-6,\x)--(6,\x);
			\draw[dotted](-6,-\x)--(6,-\x);
		}
		\draw[ray,postaction={decorate}](-6,0).. controls(-3,1)and(-3,1)..(0,0)
		..controls(3,-1)and(3,-1)..(6,0);
		\draw[ray,postaction={decorate}](-6,0).. controls(-4,1)and(-4,1)..(-2,0)
		..controls(0,-1)and(0,-1)..(2,0) .. controls(4,1)and(4,1)..(6,0);
		\end{tikzpicture}}\quad\subfloat{
		\begin{tikzpicture}[yshift=-4mm]
		\begin{axis}[xscale=0.5,scale=.5,xlabel=$n$,ylabel=$r$,xtick=\empty,ytick=\empty]
		\addplot[black] coordinates {(0,-1.5)(1,-1.5)(1,-1)};
		\addplot[smooth,domain=1:2]{sqrt(1-(x-1)^2)};
		\addplot[smooth,domain=1:2]{-sqrt(1-(x-1)^2)};
		\addplot[black] coordinates {(1,1)(1,1.5)(0,1.5)};
		\end{axis}
		\end{tikzpicture}
	}\caption{Fibra ottica \keyword[fibra ottica!graded index]{graded index}}
\end{figure}

Altri fenomeni ottici comportano problemi come l'efficienza di iniezione, in cui va considerato che solo una quota parte della radiazione emessa dalla sorgente penetra nel nucleo con angolo che porta alla propagazione, che una parte viene persa per assorbimento nel mantello, e viene riflessa o si perde in presenza delle giunzioni. Inoltre si verifica dispersione cromatica per cui diverse componenti spettrali del segnali si propagano a diverse velocità rendendo difficoltosa la ricomposizione del segnale in ricezione.

Bisogna tener conto dei fenomeni di attenuazione in corrispondenza delle lunghezze d'onda alle quali avvengono fenomeni di assorbimento dovuti alle molecole del materiale. In particolare i gruppi ossidrile OH dovuti al residuo di acqua contenuto nella silice causano assorbimento alle lunghezze d'onda di $\SI{1.24}{\micro\meter}$ e $\SI{1.38}{\micro\meter}$.
Tenendo conto dei diversi fenomeni di attenuazione, dispersione, e di ulteriori perdite per diffusione (\emph{scattering di Rayleigh}), restano individuate tre finestre trasmissive adatte all'uso nelle telecomunicazioni, con prestazioni e costi crescenti, senza apparecchiature intermedie per rigenerare il segnale per collegamenti a lunga distanza:
\begin{itemize}
\item la prima finestra a $\SI{850}{\nano\meter}$ nel campo del visibile, utilizzata con fotodiodi e LED al silicio e i primi laser economici a luce multimodale, che consentono collegamenti inferiori al km e una attenuazione $\cong\SI{1}{\decibel\per\kilo\meter}$.
\item la seconda finestra a $\SI{1.33}{\micro\meter}$ utilizzata con laser multimodali e monomodali e collegamenti inferiori ai $\SI{10}{\kilo\meter}$ con attenuazione $\cong\SI{.5}{\decibel\per\kilo\meter}$.
\item la terza finestra a $\SI{1.55}{\micro\meter}$ usata con laser monomodali può raggiungere collegamenti di $\SI{100}{\kilo\meter}$ grazie alla bassissima attenuazione di $\SI{0.2}{\decibel\per\kilo\meter}$.
\end{itemize}

\begin{figure}[!ht]\centering
\begin{tikzpicture}
\begin{axis}[xlabel={$\lambda [\si{\micro\meter}]$},ylabel={$\alpha [\si{\decibel\per\kilo\meter}]$},xtick=\empty,ytick=\empty,extra x ticks={.8,1.33,1.55},extra y ticks={.2,.5,1}]
\addplot[smooth,name path=f1] coordinates {(0.3,2.5) (0.8,1)(1.24,1.25)};
\addplot[smooth,name path=f2] coordinates {(1.24,1.25) (1.33,0.5)(1.38,1.0)};
\addplot[smooth,name path=f3] coordinates {(1.38,1) (1.56,0.22) (2,2.5)};
\draw[help lines,dashed](axis cs:0,1)-|(axis cs:.8,0) (axis cs:0,.5)-|(axis cs:1.33,0) (axis cs:0,.2)-|(axis cs:1.55,0);
\fill[fill=gray!20,fill opacity=.5] (axis cs:.78,0)rectangle(axis cs:1,2.5) node[midway]{$I$};
\fill[fill=gray!20,fill opacity=.5] (axis cs:1.3,0)rectangle(axis cs:1.36,2.5) node[midway]{$II$};
\fill[fill=gray!20,fill opacity=.5] (axis cs:1.49,0)rectangle(axis cs:1.58,2.5) node[midway]{$III$};
\end{axis}
\end{tikzpicture}
\end{figure}

Le sorgenti di radiazione del sistema di trasmissione possono operare a diverse lunghezze d'onda e potenza. Un \ac{LED} ha densità spettrale di potenza del segnale emesso qualitativamente mostrato in figura \ref{fig:densita_potenza_LED} con una larghezza di banda di circa $\SI{100}{\mega\hertz}$.
Una sorgente \ac{LASER} genera la radiazione luminosa in una cavità risonante chiusa tra specchi riflettenti: la radiazione emessa acquista potenza prima di emergere dalla sorgente in modo accentuato alle frequenze di risonanza della cavità (fig.~\ref{fig:densita_potenza_LASER}).Ad un'analisi spettrale risultano ravvicinate tra loro con una larghezza di banda di circa $\SI{10}{\mega\hertz}$ e una banda relativa molto ristretta.

Le velocità di trasmissione su fibre ottiche è limitata dalla capacità delle sorgenti, in particolare i \ac{LED}, di variare il loro stato con transizioni molto rapide.
\begin{figure}[!ht]\centering
	\subfloat[Sorgente LED]{
	\begin{tikzpicture}
	\begin{axis}[hide y axis,axis x line=middle,ymax=1.2,xlabel={$\lambda$},xtick={0},xticklabels={$\lambda_0$},ytick=\empty]
	\addplot[domain=-3:3,samples=200] {gauss(x,0,1)};
	\end{axis}
	\end{tikzpicture}
	\label{fig:densita_potenza_LED}
	}\quad\subfloat[Sorgente laser]{
	\begin{tikzpicture}
	\begin{axis}[hide y axis,axis x line=middle,xlabel={$\lambda$},xtick={0},xticklabels={$\lambda_0$},ytick=\empty]
	\addplot[samples=400] {gauss(x,0,1)/(.2+(sin(5*x))^2)};
	\end{axis}
	\end{tikzpicture}
	\label{fig:densita_potenza_LASER}
	}
	\caption{Densità spettrale di potenza del segnale emesso da sorgenti di radiazione \ac{LED} e \ac{LASER}}
\end{figure}

\section{Dimensionamento}
Un sistema di trasmissione su fibra ottica è costituito da un trasmettitore con una sorgente di radiazione luminosa con modulazione \ac{OOK} e un ricevitore dotato di un fotorivelatore che trasforma la potenza ottica
\begin{figure}[!ht]
	\centering
	\resizebox{\textwidth}{!}{
		\begin{tikzpicture}[>=latex',fitted/.style={draw,thick,dotted,inner sep=4mm,rounded corners}]
		\coordinate(c0);
		\node[block,right=1.5cm of c0](b0){OOK};
		\node[block,right=1.5cm of b0,minimum height=1em](f0){fibra}edge[<-](b0);
		\node[block,right=1cm of f0](b1){PIN}edge[<-](f0);
		\node[draw,right=1cm of b1,thick,isosceles triangle,minimum height=1cm](b2){}edge[<-](b1);
		\node[block,right=1cm of b2](b3){DEM}edge[<-](b2);
		\node[block,right=1cm of b3](b4){$H_R$}edge[<-](b3);
		\node[campionatore,right=1cm of b4](q0){}edge[<-](b4);
		\node[block,right=1cm of q0](b5){DEC}edge[<-](q0);
		\coordinate[right=1.5cm of b5](c1);
		\draw[->](c0)--node[above,near start]{bit}(b0);
		\draw[->](b5)--node[above,near end]{bit}(c1);
		\node[fitted, fit=(b0),label=above:Trasmettitore]{};
		\node[fitted, fit=(b1)(b2)(b3)(b4)(b5),label=above:Ricevitore]{};
		\node[fitted, fit=(f0),label=above:Canale]{};
		\draw [section={$P_T$}] (b0)--(f0);
		\end{tikzpicture}
	}
	\caption{Schema di trasmissione in fibra ottica}
\end{figure}

\phantomsection
\addcontentsline{toc}{chapter}{Acronimi}
\chapter*{Acronimi}
\begin{acronym}[smaller,withpage,printonlyused]
	\acro{ADSL}{\emph{Asymmetric Digital Subscriber Line}}
	\acro{AMI}{\emph{Alternated Mark Inversion}}
	\acro{ASK}{\emph{Amplitude Shift Keying}}
	\acro{BER}{\emph{Bit Error Rate}}
	\acro{CP-FSK}{\emph{Continuous Phase Frequency Shift Keying}}		\acro{DPSK}{\emph{Differential Phase Shift Keying}}
	\acro{DSB-SC}{\emph{Double Side Band Suppressed Carrier}}
	\acro{FSK}{\emph{Frequency Shift Keying}}
	\acro{GFO}{Generatore Forme d'Onda}
	\acro{HDB-3}{\emph{High Density Binary Code}}
	\acro{LASER}{\emph{Light Amplification by Stimulated Emission of Radiation}}
	\acro{LED}{\emph{Light Emitting Diode}}
	\acro{M-FSK}{\emph{Multilevel Frequency Shift Keying}}
	\acro{MSK}{\emph{Minimum Shift Keying}}
	\acro{MTI}{\emph{Moving Target Indicator}}
	\acro{NRZ}{Non Ritorno a Zero}
	\acro{OFDM}{\emph{Orthogonal Frequency Division Multiplexing}}
	\acro{OOK}{\emph{On-Off Keying}}
	\acro{PAM}{\emph{Pulse Amplitude Modulation}}
	\acro{PCM}{\emph{Pulse Code Modulation}}
	\acro{PSK}{\emph{Phase Shift Keying}}
	\acro{QAM}{\emph{Quadrature Amplitude Modulation}}
	\acro{RZ}{Ritorno a Zero}
	\acro{RADAR}{\emph{RAdio Detection And Ranging}}
	\acro{SNR}{\emph{Signal Noise Ratio}}
\end{acronym}
\phantomsection
\addcontentsline{toc}{chapter}{\indexname}
\printindex
\phantomsection
\addcontentsline{toc}{chapter}{\listfigurename}
\listoffigures
\phantomsection
\addcontentsline{toc}{chapter}{Elenco dei contributori}
\chapter*{Elenco contributori}
Si ringrazia per il contributo alla revisione del documento:

% un rigo alla fine dell'elenco per ogni nuovo contributore con il solo nome e cognome, separati da virgola
[2016/2017]
Giovanni Grieco, 
Roberto Artuso

\end{document}